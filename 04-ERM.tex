\chapter{Empirische Risikominimierung (ERM)}
\index{ERM}

\begin{bem*}[Motivation.]
Sei $D$ das empirische Wahrscheinlichkeitsmaß,
\begin{align*}
\RR_{L,D}(f) \defl \frac{1}{n}\sum_{i=1}^n L(x_i,y_i,f(x_i))
\overset{n\to\infty}{\longrightarrow} \RR_{L,P}(f).
\end{align*}
Ziel des Lernens: Finde $f$ mit $\RR_{L,P}(f)$ klein.\\
Heuristik von ERM: Minimiere Schätzung $\RR_{L,D}(\cdot)$ von
$\RR_{L,P}(\cdot)$.\maphere
\end{bem*}

\section{ERM über endliche Funktionenmengen}

\begin{defn}
\label{defn:4.1.1}
Sei $L$ eine Verlustfunktion, $\varnothing\neq \FF\subset \LL_0(X)$. Dann heißt
eine Lernmethode $\LL$ \emph{ERM} über $\FF$, wenn
\begin{align*}
\RR_{L,D}(f_D) = \inf_{f\in\FF} \RR_{L,D}(f),\qquad \forall n\ge 1,\; D\in
(X\times Y)^n \text{ und } f_D\in\FF.\fishhere
\end{align*}
\end{defn}

\begin{bem*}[Bemerkungen.]
\begin{bemenum}
\item Im Allgemeinen gibt es keine ERM.
\item Im Allgemeinen ist ERM nicht eindeutig.
\item Für $\FF=\LL_0(X)$ oder $\FF=\LL_\infty(X)$ erhält man im Allgemeinen
overfitted ERM.
\item Für "`kleine"' $\FF$ erhält man im Allgemeinen underfitted ERM.
\end{bemenum}
\end{bem*}

Im Folgenden sei $\FF$ endlich.

\begin{prop}[Orakelungleichung I]
\index{Orakelungleichung!ERM}
\label{prop:4.1.2}
Sei $L$ eine Verlustfunktion und $\FF\subset\LL_0(X)$ nicht leer und endlich.
Ferner sei $B>0$ mit
\begin{align*}
L(x,y,f(x)) \le B,\qquad \forall (x,y)\in X\times Y,\quad f\in\FF.
\end{align*}
Für den Wert des Orakels
\begin{align*}
\RR_{L,P,\FF}^* \defl \inf_{f\in\FF} \RR_{L,P}(f)
\end{align*}
und ERM gilt für $n\ge 1$ und $\tau > 0$,
\begin{align*}
&P^n\left(\setdef{D\in (X\times Y)^n}{\RR_{L,P}(f_D) \le \RR_{L,P,\FF}^* +
B\sqrt{\frac{2\tau + 2\log(2\abs{\FF})}{n}} } \right)\\ 
&\qquad\ge 1-e^{-\tau}.\fishhere
\end{align*}
\end{prop}
\begin{proof}
Für $\delta > 0$ existiert ein $f_\delta \in\FF$ mit $\RR_{L,P}(f_\delta)\le
\RR_{L,P,\FF}^* + \delta$. Damit ist
\begin{align*}
\RR_{L,P}(f_D) - \RR_{L,P,\FF}^*
&\le \RR_{L,P}(f_D)-\RR_{L,D}(f_D) + \RR_{L,D}(f_D)
- \RR_{L,P}(f_\delta)+ \delta\\
&\le \abs{\RR_{L,P}(f_D)-\RR_{L,D}(f_D)} + 
\abs{\RR_{L,P}(f_\delta)-\RR_{L,D}(f_\delta)} + \delta\\
&\le 2 \sup_{f\in\FF} \abs{\RR_{L,P}(f)-\RR_{L,D}(f)} + \delta.
\end{align*}
Für "`$\delta\to 0$"' folgt
\begin{align*}
\RR_{L,P}(f_D) - \RR_{L,P,\FF}^* \le
2 \sup_{f\in\FF} \abs{\RR_{L,P}(f)-\RR_{L,D}(f)}
\end{align*}
Damit folgt
\begin{align*}
&P^n\left(\setdef{D}{\RR_{L,P}(f_D)- \RR_{L,P,\FF}^* \ge
B\sqrt{\frac{2\tau}{n}}} \right)\\
\le\;\; 
&P^n\left(\setdef{D}{\sup_{f\in\FF}\abs{\RR_{L,P}(f)- \RR_{L,D}} \ge
B\sqrt{\frac{\tau}{2n}}}\right)\\
\overset{\atop{\text{union}}{\text{bound}}}{\le}
&\sum_{f\in\FF}
P^n\left(\setdef{D}{\abs{\RR_{L,P}(f)- \RR_{L,D}} \ge
B\sqrt{\frac{\tau}{2n}}}\right).
\end{align*}
Setzen wir $\E_P\xi_1 =\RR_{L,P}(f)$ und $\xi_i = L(x_i,y_i,f(x_i))$, so ist
$\RR_{L,D}(f) = \frac{1}{n}\sum_{i=1}^n \xi_i$ und mit der 2-seitigen
Hoeffding Ungleichung folgt
\begin{align*}
\ldots \le 
\sum_{f\in\FF} 2e^{-\tau} = 2\abs{\FF}e^{-\tau} = e^{\log(2\abs{\FF})-\tau}.
\end{align*}
Mit $-t' = \log(2\abs{\FF})-\tau$ folgt die Behauptung.\qedhere
\end{proof}

Idee: Falls $\abs{\FF} = \infty$ finde $\FF_\ep \subset\FF$ endlich, das $\FF$
gut "`approximiert"'.

\section{ERM über unendlichen Funktionenmengen}
\label{sec:4.2}

\begin{defn}
\label{defn:4.2.1}
Sei $(T,d)$ ein metrischer Raum (MR) und $\ep > 0$.
\begin{defnenum}
\item $S\subset T$ heißt \emph{$\ep$-Netz}\index{$\ep$-Netz} falls
\begin{align*}
\forall t\in T \exists s\in S : d(t,s) \le \ep.
\end{align*}
\item Die \emph{Überdeckungszahl}\index{Überdeckungszahl} (covering number) ist
definiert als
\begin{align*}
\NN(T,d,\ep) \defl \inf\setdef{n\ge 1}{\exists s_1,\ldots,s_n\in T : T\subset
\bigcup_{i=1}^n B_d(s_i,\ep)}.
\end{align*}
D.h. $S=\setd{s_1,\ldots,s_n}$ ist $\ep$-Netz. Damit entspricht $\NN(T,d,\ep)$
der Größe des kleinsten $\ep$-Netzes von $T$.
\item Sei $T\subset E$, $E$ normierter Raum. Dann ist
\begin{align*}
\NN(T,\norm{\cdot}_E,\ep) \defl \NN(T,d_E,\ep),\qquad d_E(x,x') =
\norm{x-x'}_E.\fishhere
\end{align*}
\end{defnenum}
\end{defn}

\begin{bem*}[Interpretation.]
Ist $T$ kompakt, dann ist $\NN(T,d,\ep) < \infty$ für jedes $\ep > 0$. Genauer
ist $T$ genau dann präkompakt, wenn $\NN(T,d,\ep) < \infty$ für jedes $\ep > 0$.

Da für $\ep < \ep'$ gilt $\NN(T,d,\ep')<\NN(T,d,\ep)$, stellt das
Wachstumsverhalten von $\NN(T,d,\ep)$ für $\ep \downarrow 0$ ein
quantitatives Maß für die Kompaktheit dar.

$\NN(T,d,\ep)\to \infty$ für $\ep \to 0$ genau dann, wenn $T$ unendlich.\maphere
\end{bem*}

\begin{defn}
\label{defn:4.2.2}
Sei $(T,d)$ ein metrischer Raum und $n\ge 1$.
\begin{defnenum}
\item Die $n$-te (b-a-dische) \emph{Entropiezahl}\index{Entropiezahl} von $T$
ist
\begin{align*}
e_n(T,d) = \inf\setdef{\ep >0}{\exists s_1,\ldots,s_{2^{n-1}}\in T :
T\subset\bigcup_{i=1}^{2^{n-1}} B_d(s_i,\ep)}.
\end{align*}
\item $e_n(T,\norm{\cdot}_E)$ für $T\subset (E,\norm{\cdot}_E)$ ist wie üblich
definiert.
\item Sei $S: E\to F$ ein stetige linearer Operator zwischen den normierten
Räumen $E$ und $F$ und es sei $B_E \defl \setdef{x\in E}{\norm{x}_E \le 1}$, dann
schreibe
\begin{align*}
e_n(S) = e_n(S: E\to F) \defl e_n(SB_E,\norm{\cdot}_F).\fishhere
\end{align*}
\end{defnenum}
\end{defn}

\begin{bem*}[Bemerkungen (ohne Beweis).]
\begin{bemenum}
\item $e_1(S) = \norm{S}$.
\item $e_{n+1}(S)\le e_n(S)$ für $n\ge 1$.
\item $\lim\limits_{n\to\infty} e_n(S) = 0$ genau dann, wenn $S$ kompakt.
\item $e_{n+k-1}(S+T) \le e_{n}(S) + e_k(T)$.
\item $e_{n+k-1}(S\circ T) \le e_n(S)e_k(T)$.
\item Sei $d=\rank S = \dim SE$. Dann ist $d<\infty$ genau dann, wenn
\begin{align*}
\exists c_1,c_2 : c_1 2^{-\frac{n-1}{d}} \le e_n(S) \le c_22^{-\frac{n-1}{d}}.
\end{align*}
\item Sei $X\subset \R^d$ offen und beschränkt und
\begin{align*}
C_b^m(X) \defl \setdef{f \in C^m(X\to \R)}{\max_{\abs{\alpha}\le m}
\norm{\partial^\alpha f}_\infty < \infty}
\end{align*}
versehen mit der Norm
\begin{align*}
\norm{f}_{C_b^m(X)} = \max_{\abs{\alpha}\le m} \norm{\partial^\alpha f}_\infty.
\end{align*}
Dann ist $e_n(\id: C_b^m(X)\to C_b(X)) \le C_{m,d}(X)n^{-\frac{m}{d}}$ und die
Abschätzung ist scharf, d.h. es gibt ein $\tilde{C}_{m,d}(X)$ mit $e_n(\ldots)
\ge \tilde{C}_{m,d}(x)n^{-\frac{m}{d}}$. Weiterhin ist
\begin{align*}
C_{m,d}(r\cdot X) = r^m C_{m,d}(X),\qquad r\ge 1.\maphere
\end{align*}
\end{bemenum}
\end{bem*}

\begin{lem}
\label{prop:4.2.3}
Sei $(T,d)$ MR und $a,q >0$ mit
\begin{align*}
e_n(T,d) \le an^{-\frac{1}{q}},\qquad n\ge 1.
\end{align*}
Dann gilt
\begin{align*}
\log(\NN(T,d,\ep)) \le \log(4)\left(\frac{a}{\ep}\right)^q,\qquad \ep >
0.\fishhere
\end{align*}
\end{lem}
\begin{proof}
Für $\ep > a$ ist $\NN(T,d,\ep)=1$, d.h. $\log(\NN(T,d,\ep)) = 0$ und es ist
nichts zu beweisen. Sei $\delta > 0$ und $\ep \in (0,a]$, dann existiert ein
$n\ge 1$ mit
\begin{align*}
a(1+\delta)(n+1)^{-\frac{1}{q}} \le \ep \le a(1+\delta)n^{-\frac{1}{q}}.\tag{*}
\end{align*}
Da $e_n(T,d) < a(1+\delta)n^{-\frac{1}{q}}$, gibt es ein
$a(1+\delta)n^{-\frac{1}{q}}$-Netz der Größe $2^{n-1}$, d.h.
\begin{align*}
\NN(T,d,a(1+\delta)n^{-\frac{1}{q}}) \le 2^{n-1}.
\end{align*}
Nun ist
\begin{align*}
a(1+\delta)n^{-\frac{1}{q}} &=
a(1+\delta)(n+1)^{-\frac{1}{q}}\underbrace{\left(\frac{n+1}{n}\right)^{\frac{1}{q}}}_{\le
2}
\le 2^{\frac{1}{q}}a(1+\delta)(n+1)^{-\frac{1}{q}}\\
&\overset{(*)}{\le} 2^{\frac{1}{q}}\ep.
\end{align*}
Und ferner folgt mit der rechten Seite von (*),
\begin{align*}
n \le \left(\frac{a(1+\delta)}{\ep}\right)^q.
\end{align*}
Somit ist
\begin{align*}
\log(\NN(T,d,2^{1/q}\ep))&\le 
\log(\NN(T,d,a(1+\delta)n^{-1/q}))
\le \log(2^{n-1})\\
& \le n \log 2
\le (\log2)\left(\frac{a(1+\delta)}{\ep}\right)^q 
\end{align*}
Für $\delta \to 0$ und $2^{1/q}\ep = \tilde{\ep}$ folgt die Behauptung.\qedhere
\end{proof}

\begin{prop}[Orakelungleichung II]
\index{Orakelungleichung!ERM (2)}
\label{prop:4.2.4}
Sei $L:X\times Y\times\R\to [0,\infty)$ eine lokal lipschitz-stetige
Verlustfunktion und $P$ ein W-Maß auf $(X\times Y)$. Ferner sei
\begin{align*}
\FF\subset\LL_\infty(X) = \setdef{f: X\to\R}{f\text{ messbar und beschränkt}}
\end{align*}
für die $B>0$ und $M>0$ existieren, so dass
\begin{align*}
&\norm{f}_\infty  < M, && \text{für alle }f\in\FF,\\
&L(x,y,f(x)) \le B, && \text{für alle }(x,y)\in (X\times Y),\; f\in\FF.
\end{align*}
Dann gilt für ERM über $f$,
\begin{align*}
&P^n\left( \setdef{D}{\RR_{L,P}(f_D) < \RR_{L,P,\FF}^* +
B\sqrt{\frac{2\tau+2\log\left(2\NN(\FF,\norm{\cdot}_\infty,\ep)\right)
}{n}}+4\ep\abs{L}_{M,1} } \right)\\
&\ge 1-e^{-\tau},\qquad \text{für alle }n\ge1, \ep > 0\text{ und }\tau >
0.\fishhere
\end{align*}
\end{prop}

Beachte, dass falls $\log(\NN(\FF,\norm{\cdot}_\infty,\ep)) \le \mu(\ep)$ für
$\ep > 0$, der Ausdruck nach $\ep$ optimiert werden kann. Somit erhält man eine
Abschätzung in $n\ge 1$ und $\tau > 0$.

\begin{proof}
Fixiere $\ep > 0$ und ein minimales $\ep$-Netz $\FF_\ep$ von $\FF$ bezüglich
$\norm{\cdot}_\infty$, d.h.
\begin{align*}
\NN(\FF,\norm{\cdot}_\infty,\ep) = \abs{\FF_\ep}
\end{align*}
und
\begin{align*}
\forall f\in\FF \exists g\in \FF_\ep \text{ mit } \norm{f-g}_\infty \le \ep.
\end{align*}
Somit ist für solches $f,g$
\begin{align*}
&\abs{\RR_{L,P}(f)-\RR_{L,D}(f)}\\
\le &\abs{\RR_{L,P}(f)-\RR_{L,P}(g)}
+ \abs{\RR_{L,P}(g)-\RR_{L,D}(g)}
+ \abs{\RR_{L,D}(g)-\RR_{L,D}(f)}\\
\overset{\ref{defn:2.1.9}}{\le} &2 \ep \abs{L}_{M,1} + 
\abs{\RR_{L,P}(g)-\RR_{L,D}(g)}.
\end{align*}
D.h.
\begin{align*}
\sup_{f\in\FF} \abs{\RR_{L,P}(f)-\RR_{L,D}(f)}
\le 2\ep \abs{L}_{M,1} + 
\sup_{g\in\FF_\ep}\abs{\RR_{L,P}(g)-\RR_{L,D}(g)}.
\end{align*}
Der Beweis von \ref{prop:4.1.2} zeigte außerdem, dass
\begin{align*}
\RR_{L,P}(f_D)-\RR_{L,P,\FF}^*
&\le 2 \sup_{f\in\FF} \abs{\RR_{L,P}(f)-\RR_{L,D}(f)}\\
&\le 4\ep \abs{L}_{M,1} + 2
\sup_{g\in\FF_\ep} \abs{\RR_{L,P}(g)-\RR_{L,D}(g)}. 
\end{align*}
Anwendung der Hoeffding Ungleichung und es union bounds liefert, 
\begin{align*}
&P^n\left(\setdef{D}{\RR_{L,P}(f_D)-\RR_{L,P,F}^*\ge B\sqrt{\frac{2\tau}{n}}+
4\ep \abs{L}_{M,1}} \right)\\
\le &P^n\left( \setdef{D}{\sup_{g\in\FF_\ep} \abs{\RR_{L,P}(g) -
\RR_{L,D}(g)}\ge B\sqrt{\frac{\tau}{2n}}} \right)\\
\le &2\NN\left(\FF,\norm{\cdot}_\infty,\ep \right)e^{-\tau}
= \exp\left(\log(2 \NN(\FF,\norm{\cdot}_\infty, \ep))-\tau \right).
\end{align*}
Die Behauptung folgt nun mit einer Variablentransformation
\begin{align*}
-\tau' = \log(2
\NN(\FF,\norm{\cdot}_\infty, \ep))-\tau.\qedhere
\end{align*}
\end{proof}