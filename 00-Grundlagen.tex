\chapter{Grundlagen}

\section{Statistische Lerntheorie (SLT)}

Die statistische Lerntheorie ist Teil der nichtparametrischen Statistik sowie
der Informatik. Sie bildet den mathematischen Arm des maschinellen
Lernens und befasst sich sowohl mit den theoretischen Grundlagen wie auch
Fragen der Implementierung und besitzt zahlreiche Anwendungen.

\subsection{Inhalt der Vorlesung}

\begin{itemize}
  \item Was ist Lernen?
  \item Klassische Verfahren
  \item Empirische Risikominimierung (ERM)
  \item Regularisierte ERMs
  \item Kernbasierte Verfahren (support vector machines, SVMs)
  \item Verlustfunktionen
\end{itemize}

\section{Lernszenarien und -ziele}

\subsection{Überwachtes Lernen}
\index{Überwachtes Lernen}
Beim überwachten Lernen wird der Lernprozess durch einen \emph{Lehrer} geführt.
Den \emph{Eingaberaum} bezeichnen wir mit $X$, den \emph{Ausgaberaum} mit $Y$.
Weiterhin sind \emph{Trainingsdaten} in der From
\begin{align*}
((x_1,y_1),\ldots,(x_n,y_n))\in (X\times Y)^n
\end{align*} 
gegeben. Das Ziel ist es eine Funktion
\begin{align*}
f_D : X\to Y
\end{align*}
zu so finden, dass $f_D(x)\approx y$ für neue Paare $(x,y)$. Es geht also
nicht darum, die Daten einfach auswendig zu lernen, sondern $y$ für zukünftiges $x$
vorherzusagen. Was ``$\approx$'' genau bedeutet werden wir später erklären.

Beim überwachten Lernen, erhält der Lehrer ein $x_i$ und ordnet diesem ein
\emph{Label} $y_i$ zu. Der Lernende erhält dann vom Lehrer das Tupel
$(x_i,y_i)$ und soll darauf $f_D$ ableiten.

\subsubsection{Klassifikation}
\index{Klassifikation}

Gegeben sei der Ausgaberaum $Y=\setd{-1,1}$ und Ziel ist $f_D$ so zu bestimmen,
dass $f_D(x) = y$ "`möglichst häufig"'.

Es ist nun grundsätzlich zu entscheiden, ob man den Trainingsdaten traut oder
eher ein Rauschen vermutet.

\begin{figure}[!htpb]
\centering
\begin{pspicture}(0,-0.94)(3.6364853,0.94)
\psdots[linecolor=gdarkgray](0.06,0.5)
\psdots[linecolor=gdarkgray](0.86,0.76)
\psdots[linecolor=gdarkgray](0.62,-0.38)
\psdots[linecolor=gdarkgray](0.62,0.2)
\psdots[linecolor=gdarkgray](2.22,-0.14)
\psdots[linecolor=gdarkgray](1.44,-0.4)
\psdots[linecolor=gdarkgray](1.48,0.38)
\psdots[linecolor=gdarkgray](1.88,0.76)
\psdots[linecolor=gdarkgray,dotstyle=x](2.3,0.6)
\psdots[linecolor=gdarkgray,dotstyle=x](1.88,-0.68)
\psdots[linecolor=gdarkgray,dotstyle=x](3.16,-0.16)
\psdots[linecolor=gdarkgray,dotstyle=x](2.92,0.24)
\psdots[linecolor=gdarkgray,dotstyle=x](2.52,-0.74)
\psdots[linecolor=gdarkgray,dotstyle=x](3.56,0.32)
\psline[linecolor=darkblue](2.76,0.86)(1.38,-0.92)
\psbezier[linecolor=purple](2.32,0.92)(1.64,0.2)(2.54,0.16)(2.42,-0.24)(2.3,-0.64)(1.42,-0.34)(1.08,-0.86)
\end{pspicture} 
\caption{Zwei Möglichkeiten zur Klassifikation von Trainingsdaten.}
\end{figure}

Geht man davon aus, dass auf den Trainingsdaten keine Fehler sind, besteht die
Gefahr des sogenannten \emph{overfitting}, d.h. man traut den Daten zu sehr.
Vermutet man andererseits ein Rauschen besteht die Gefahr des
\emph{underfitting}, d.h. die Annahmen an die Zielfunktion sind unzureichend.

Over- und underfitting konkurrieren miteinander. Gute Lernverfahren finden
einen guten bzw. den ``besten'' Kompromiss.

\subsubsection{Anwendungen der Klassifikation}

\begin{itemize}
  \item Spamfilter
  \item Fraud Detection (Betrugsvorfälle bei Kreditkartenunternehmen)
  \item Schriftenerkennung
  \item Diagnoseverfahren
\end{itemize}

\subsubsection{Regression}
\index{Regression}

Sei $Y=[-M,M]$ oder $Y=\R$. Auch hier ist das Ziel $f_D(x) \approx y$ möglichst
häufig und es besteht ebenfalls die Gefahr des over- bzw. underfittings.

\begin{figure}[!htpb]
\centering
\begin{pspicture}(0,-1.3)(4.4,1.3)
\psline{->}(0.18,-1.28)(0.18,1.28)
\psline{->}(0.0,-1.06)(4.38,-1.06)
\psdots[dotstyle=x](0.36,0.44)
\psdots[dotstyle=x](0.8,0.94)
\psdots[dotstyle=x](1.32,0.0)
\psdots[dotstyle=x](2.1,0.62)
\psdots[dotstyle=x](2.96,-0.1)
\psdots[dotstyle=x](4.12,0.72)
\psbezier[linecolor=darkblue](0.26,0.34)(0.54,0.5)(0.42,0.96)(0.8,0.94)(1.18,0.92)(0.78,0.06)(1.32,0.0)(1.86,-0.06)(1.72,0.64)(2.26,0.66)(2.8,0.68)(2.74,-0.16)(3.24,-0.28)(3.74,-0.4)(3.56,0.68)(4.36,0.8)
\psline[linecolor=purple](0.18,0.56)(4.32,0.56)
\end{pspicture}
\caption{Zwei Möglichkeiten zur Regression (datentreu oder linear).}
\end{figure}

Was man genau unter ``$\approx$'' versteht, fällt unter den Terminus
\emph{Verlustfunktion}. Je nachdem wie man ``$\approx$'' mathematisch
präzesiert, erreicht man unterschiedliche Lernziele.

\subsubsection{Anwendung der Regression}
\begin{itemize}
  \item Finanzindustrie: Kursvorhersage von Aktien, Ausfallrisiken von
  Krediten, \ldots
  \item Data Mining
  \item Klassifikation $P(Y=y\mid x) =\ ?$
  \item Schätzungen von Restlebenszeiten von Teilen
\end{itemize}

\subsection{Halbüberwachtes Lernen}
\index{Halbüberwachtes Lernen}

Hier sind $X$ und $Y$ wie beim überwachten Lernen definiert. Die Trainingsdaten
sind jedoch nur noch zum Teil mit einem Label versehen. Sie bestehen also aus
einer "`kleinen"' Menge an
\begin{align*}
((x_1,y_1),\ldots,(x_n,y_n))\in (X,Y)^n,
\end{align*}
wobei $n=2$ den Extremfall bildet, sowie einer "`großen"' Menge an
\begin{align*}
(x_{n+1},\ldots,x_m)\in X^{m-n},\qquad m>> 1.
\end{align*}
Nun kann man ebenfalls Klassifikation oder Regression betreiben.

Die Motivation hierbei ist, dass die Beschaffung der $y_i$ entweder ``teuer''
oder gar unmöglich ist. Beispiele dafür sind medizinische Studien oder
Diagnoseverfahren zur Überwachung von Brücken.

Die Methoden für das halbüberwachte Lernen sind noch weitgehend unausgereift.
Es gibt noch sehr viele Fragestellungen an denen man arbeiten kann.

\subsection{Unüberwachtes Lernen}
\index{Unüberwachtes Lernen}

Beim unüberwachten Lernen sind nur Trainingsdaten ohne Label gegeben
\begin{align*}
x_1,\ldots,x_n\in X.
\end{align*}

Es lassen sich nun verschiedene Lernziele definieren. Wir wollen hier nur einen
kurzen Überblick geben; ausführlich werden wir uns lediglich mit der
Ausreißeridentifikation beschäftigen.

\subsubsection{Ausreißeridentifikation}
\index{Ausreißeridentifikation}

Es geht darum ``untypische'' Daten zu identifizieren und von "`typischen"' zu
unterscheiden.

\begin{figure}[!htpb]
\centering
\begin{pspicture}(0,-1.24)(5.08,1.24)
\psdots[linecolor=darkblue](0.3,0.66)
\psdots[linecolor=darkblue](0.06,-0.26)
\psdots[linecolor=darkblue](1.24,0.44)
\psdots[linecolor=darkblue](0.62,-0.04)
\psdots[linecolor=darkblue](1.18,-0.28)
\psdots[linecolor=darkblue](2.22,0.28)
\psdots[linecolor=darkblue](0.98,0.86)
\psdots[linecolor=darkblue](1.88,-0.2)

\psdots[linecolor=darkblue,dotstyle=x](1.62,0.34)
\psdots[linecolor=purple,dotstyle=x](3.32,-0.34)
\psbezier[linecolor=darkblue](1.64,0.54)(1.86,1.1)(2.02,0.16)(2.32,0.9)
\psbezier[linecolor=purple](3.44,-0.36)(3.8,-0.54)(3.04,-0.78)(3.6,-0.84)

\rput(2.73,1.045){\color{gdarkgray}neu typisch}
\rput(3.89,-1.015){\color{gdarkgray}neu untypisch}
\end{pspicture} 
\caption{Zwei Möglichkeiten zur Klassifikation von Trainingsdaten.}
\end{figure}

Die Ausreißeridentifikation hat viele Ähnlichkeiten mit der Klassifikation und
bereits gut verstanden.

\subsubsection{Clusteranalyse}
\index{Clusteranalyse}

Hier ist zunächst festzustellen, ob die Daten Cluster bilden, und diese
gegebenenfalls zu ermitteln.

\begin{figure}[!htpb]
\centering
\begin{pspicture}(0,-1.02)(4.22,0.98)
\psdots[linecolor=darkblue](0.42,0.5)
\psdots[linecolor=darkblue](1.06,0.54)
\psdots[linecolor=darkblue](0.64,0.22)
\psdots[linecolor=darkblue](1.08,0.14)
\psdots[linecolor=darkblue](0.44,-0.14)
\psdots[linecolor=darkblue](3.42,0.44)
\psdots[linecolor=darkblue](3.2,0.12)
\psdots[linecolor=darkblue](3.6,0.26)
\psdots[linecolor=darkblue](2.98,0.6)
\psdots[linecolor=darkblue](3.54,0.84)
\psdots[linecolor=darkblue](3.8,0.7)
\psellipse[linestyle=dotted](0.76,0.21)(0.76,0.61)
\psellipse[linestyle=dotted](3.43,0.42)(0.79,0.56)
\psdots[linecolor=purple,dotstyle=x](2.84,-0.86)
\rput(3.08,-0.855){\color{gdarkgray}?}
\end{pspicture} 
\caption{Clusterbildung.}
\end{figure}

Im Gegensatz zur Ausreißeridentifikation bietet die Clusteranalyse noch sehr
viele offene Fragen.

\subsubsection{Dimensionsreduzierung}
\index{Dimensionsreduzierung}

Viele Modelle erzeugen sehr hochdimensionale Daten. Hochdimensionale
Daten sind schlecht numerisch zu handhaben und außerdem schlecht zu
visualisieren.

\begin{figure}[!htpb]
\centering
\begin{pspicture}(0,-0.57)(5.48,0.57)
\psdots(0.16,-0.23)
\psdots(0.5,0.49)
\psdots(0.82,-0.13)
\psdots(1.08,0.31)
\psdots(1.58,-0.37)
\psdots(2.04,0.39)
\psdots(2.44,-0.01)
\psdots(3.74,0.15)
\psdots(3.2,0.35)
\psdots(3.2,-0.19)
\psdots(5.4,0.19)
\psdots(4.72,0.07)
\psdots(4.38,-0.11)
\psdots(1.44,0.11)
\psdots(1.28,-0.13)
\psdots(1.2,-0.43)
\psdots(1.74,-0.05)
\psdots(1.72,0.27)
\psdots(1.18,0.15)
\psbezier[linecolor=purple](0.0,-0.55)(0.14,0.19)(0.78,-0.05)(1.42,-0.01)(2.06,0.03)(2.08,0.45)(2.48,0.29)(2.88,0.13)(3.16,-0.17)(3.78,-0.01)(4.4,0.15)(4.72,-0.33)(5.28,0.33)
\end{pspicture} 
\caption{Dimensionsreduktion durch Approximation der Daten durch eine
Funktion.}
\end{figure}

\begin{bsp*}
Eine Kaffeekanne wird unter Rotation fotografiert, um so ein dreidimensionales
Bild zu erhalten. Jedes Foto besteht aus $n$ Pixeln, die man als Punkte im
$\R^n$ ansehen kann. Diese gewaltige Datenmenge lässt sich jedoch oft auf den
$\R^3$ reduzieren.\bsphere
\end{bsp*}

Auch die Dimensionsreduzierung ist noch weitgehend unerforscht.

\subsubsection{Dichteschätzung}
\index{Dichteschätzung}

Die Dichteschätzung ist mit klassischen Verfahren gut handhabbar und bereits
sehr ausgereift.

\begin{figure}[!htpb]
\centering
\begin{pspicture}(0,-0.91)(2.46,0.91)
\psdots(0.06,0.41)
\psdots(0.38,0.51)
\psdots(0.22,0.07)
\psdots(0.94,0.39)
\psdots(0.18,0.79)
\psdots(0.5,0.25)
\psdots(0.7,0.83)
\psdots(0.66,0.55)
\psdots(0.4,-0.83)
\psdots(2.26,-0.31)
\psdots(1.48,-0.45)
\psdots(2.38,0.77)
\end{pspicture} 
\caption{Zur Dichteschätzung.}
\end{figure}

\subsection{Weitere Lernszenarien}

\begin{itemize}
  \item\textit{Reinforcement Learning}
  \index{Reinforcement Learning}

 Reinforcement Learning hat das Lernen von Handlungen zum Ziel. Beispielsweise
 misst ein Roboter Umwelteinflüsse und entscheidet sich dannach für
 Handlungen. Welche Entscheidung ``gut'' war ist jedoch nur mit einer gewissen
 Verzögerung feststellbar.
 
 
 \item\textit{Multi-armed Bandit Learning}
 \index{Multi-armed Bandit Learning}
 
 Man hat $n$ Optionen, wovon jede einen Gewinn ausschüttet. Aufgrund
 beschränkter Ressourcen muss man eine Priorisierung vornehmen, um den
 größtmöglichen Gewinn zu erzielen.
 \end{itemize}

 \subsection{Datengenerierung}
 
\begin{itemize}
\item\textit{Batch Learning}
\index{Batch Learning}
 
 Hier liegen alle Trainingsdaten vor Trainingsbeginn vor.
 
 \item\textit{Online Learning}
 \index{Online Learning}
 
Das Online Learning ist bei den Informatikern sehr beliebt, da sich damit sehr
gut ``worst case''-Szenarien realisieren lassen.

\noindent
Ein möglicher Algorithmus wäre\\
a) Habe bereites $(x_1,y_1),\ldots,(x_i,y_i)$\\
b) Finde $f_i : X\to Y$\\
c) Erhalte $x_{i+1}$\\
d) Mache Vorhersage $f_i(x_{i+1})$\\
e) Bekomme $y_{i+1}$\\
f) Zurück zu a)

Eine ``billige" Generierung von $f_i$ ist wesentlich für den Erfolg des Online
Learning.
\item\textit{Active Learning}
\index{Active Learning}

Bestimme $x_1,\ldots,x_n$ selbst und erhalte anschließend die zugehörigen
Labels $y_1,\ldots,y_n$.
\end{itemize}

\subsubsection{Natur der Daten}

\begin{itemize}
\item \textit{Zufällige Daten}. Dies ist typisch für batch learning. Wir werden
uns später damit ausführlich beschäftigen.
\item \textit{Pseudozufällige Daten}. Die Daten werden eigentlich von einem
deterministischen System generiert, dieses ist aber chaotisch.
\item \textit{``worst case'' Daten}. Typisch für online learning.
\end{itemize}

\section{Klassische Verfahren zur Klassifikation}

Sei $X=\R^d$ und $Y=\setd{-1,1}$.

\begin{itemize}
\item\textit{Histogrammregel}.
\index{Histogrammregel}

Sei $(Q_j)_{j\ge1}$ eine Zerlegung von $X$ in Würfel. Es gilt dann
offenbar
\begin{align*}
\forall x\in X \exists j\ge 1 : x\in Q_j.
\end{align*}

\begin{figure}[!htpb]
\centering
\begin{pspicture}(0,-0.91)(3.42,0.91)
\psline(0.0,0.63)(3.4,0.63)
\psline(0.0,0.03)(3.4,0.03)
\psline(0.0,-0.57)(3.4,-0.57)
\psline(0.4,0.89)(0.4,-0.87)
\psline(1.2,0.87)(1.2,-0.89)
\psline(2.0,0.87)(2.0,-0.89)
\psline(2.82,0.87)(2.82,-0.89)
\psdots[linecolor=darkblue](0.56,0.51)
\psdots[linecolor=darkblue](0.68,0.17)
\psdots[linecolor=darkblue](0.98,0.53)
\psdots[linecolor=darkblue](1.58,-0.17)
\psdots[linecolor=darkblue](2.4,0.55)
\psdots[linecolor=darkblue](2.68,0.31)
\psdots[linecolor=purple](1.34,-0.27)
\psdots[linecolor=purple](1.66,-0.43)
\psdots[linecolor=purple](2.2,0.37)
\psdots[linecolor=purple](2.34,0.19)
\psdots[linecolor=purple](1.56,0.43)
\psdots[linecolor=purple](0.9,0.29)
\end{pspicture} 
\caption{Zerlegung von $X$.}
\end{figure}

Wir definieren nun
\begin{align*}
f_D(x) \defl \sign \left(\sum_{x_i \in Q_i(x)} y_i\right).
\end{align*}
\item\textit{Nearest Neighbor}.
\index{Nearest Neighbor}

Sei $(X,d)$ ein metrischer Raum (z. B. $X=\R^d$). Fixiere die Anzahl $k$ der
betrachteten Nachbarn. Für $x\in X$ suche die $k$ Beispiele
\begin{align*}
(x_{i_1},y_{i_1}),\ldots,(x_{i_k},y_{i_k})
\end{align*}
aus $D$, die am nächsten liegen.
\begin{align*}
f_D(x) = \sign\left(\sum_{j=1}^k y_{i_j}\right).
\end{align*}

\begin{figure}[!htpb]
\centering

\begin{pspicture}(0,-0.76824266)(3.6364853,0.76824266)
\psline(1.2,0.5482426)(1.42,0.42824265)
\psline(1.34,0.22824265)(1.42,0.42824265)
\psline(1.4,0.42824265)(1.64,0.6282427)
\psline(2.28,-0.31175736)(2.4,-0.45175734)
\psline(2.4,-0.45175734)(2.58,-0.37175736)
\psline(2.4,-0.45175734)(2.44,-0.61175734)

\psdots[linecolor=darkblue](0.06,0.5482426)
\psdots[linecolor=darkblue](0.84,0.028242646)
\psdots[linecolor=darkblue](0.26,0.008242645)
\psdots[linecolor=darkblue](0.76,0.52824265)
\psdots[linecolor=darkblue](1.7,0.008242645)
\psdots[linecolor=darkblue](1.68,0.6882427)
\psdots[linecolor=darkblue](1.06,-0.65175736)
\psdots[linecolor=darkblue](1.28,0.12824264)
\psdots[linecolor=darkblue](0.32,-0.39175737)
\psdots[linecolor=purple,dotstyle=x](1.14,0.5482426)
\psdots[linecolor=purple,dotstyle=x](1.54,-0.31175736)
\psdots[linecolor=purple,dotstyle=x](2.5,-0.6917574)
\psdots[linecolor=purple,dotstyle=x](2.64,-0.33175737)
\psdots[linecolor=purple,dotstyle=x](2.24,-0.25175735)
\psdots[linecolor=purple,dotstyle=x](2.48,0.28824264)
\psdots[linecolor=purple,dotstyle=x](3.56,0.20824264)
\psdots[linecolor=purple,dotstyle=x](2.98,-0.13175735)
\psdots[linecolor=purple,dotstyle=x](3.32,-0.41175735)
\end{pspicture} 
\caption{Nearest Neighbor Identifikation.}
\end{figure}

\item\textit{Moving window / kernel rules}.
\index{Moving window / kernel rules}

Sei $X=\R^d$ ein euklidischer Raum und
\begin{align*}
K: \R\to [0,\infty)
\end{align*}
symmetrisch und auf $[0,\infty)$ monoton fallend gegeben.

\begin{figure}[!htpb]
\centering
\begin{pspicture}(0,-0.64)(3.56,1.8)
\psline{<-}(1.7,1.78)(1.7,-0.62)
\psline{->}(0.0,-0.44)(3.54,-0.44)
\psarc[linecolor=darkblue](1.7,-0.42){1.36}{0.0}{180.0}
\end{pspicture}
\begin{pspicture}(0,-1.22)(3.56,1.22)
\psline{<-}(1.7,1.2)(1.7,-1.2)
\psline{->}(0.0,-1.02)(3.54,-1.02)
\psbezier[linecolor=darkblue](0.06,-1.0)(1.08,-1.0)(1.0801282,0.23601015)(1.7,0.24)(2.319872,0.24398986)(2.24,-0.96)(3.24,-1.0)
\end{pspicture}
\begin{pspicture}(0,-1.22)(3.58,1.22)
\psline{<-}(1.72,1.2)(1.72,-1.2)
\psline{->}(0.02,-1.02)(3.56,-1.02)
\psline[linecolor=darkblue]{|-|}(0.72,0.38)(2.72,0.38)
\psline[linecolor=purple]{-|}(0.0,-1.02)(0.74,-1.02)
\psline[linecolor=purple]{-|}(3.32,-1.02)(2.68,-1.02)
\end{pspicture} 
\caption{Beispiele für Funktionen $K:\R\to[0,\infty)$.}
\end{figure}

Fixiere $h>0$ und setze
\begin{align*}
f_D(x) = \sign\left( \sum_{i=1}^k y_i\ K(h^{-1}\norm{x-x_i})\right)
\end{align*}

\begin{figure}[!htpb]
\centering
\begin{pspicture}(0,-0.61)(2.7164853,0.61)
\psdots[linecolor=darkblue](0.056485273,0.23)
\psdots[linecolor=darkblue](0.79648525,0.31)
\psdots[linecolor=darkblue](0.27648526,-0.47)
\psdots[linecolor=darkblue](1.2164853,-0.43)
\psdots[linecolor=darkblue](0.83648527,-0.11)
\psdots[linecolor=darkblue](1.9164853,0.35)
\psdots[linecolor=darkblue](1.3964853,0.15)
\psdots[linecolor=purple](0.47648528,0.17)
\psdots[linecolor=purple](1.6164852,-0.23)
\psdots[linecolor=purple](2.0564852,0.05)
\psdots[linecolor=purple](2.3764853,0.41)
\psdots[linecolor=purple](2.2564852,-0.53)
\psdots[linecolor=purple](2.6364853,-0.03)
\psdots[linecolor=purple](2.1564853,-0.17)
\psbezier[linestyle=dotted](1.0164852,0.05)(0.9364853,-0.49)(0.6964853,-0.03)(0.39648527,-0.21)(0.09648527,-0.39)(0.03648527,0.11)(0.3164853,0.29)(0.59648526,0.47)(1.0964853,0.59)(1.0164852,0.05)
\end{pspicture} 
\caption{Identifikation mittels moving window.}
\end{figure}

\end{itemize}
