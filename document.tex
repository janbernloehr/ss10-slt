% =============================================================================
% Titel:	SLT - Mitschrieb
% Erstellt:	SS 10
% Dozent:	Prof. Dr. I. Steinwart
% Autor:	Jan-Cornelius Molnar
% =============================================================================
\documentclass[%
	paper=a5,%
	fleqn,%
% ===================================================
%  10pt aktivieren, falls *nicht* die Lucida Schrift verwendet wird.
% =================================================== 
%	10pt,%
	DIV=14,%
	BCOR=0mm,
	titlepage=false,%
	twoside=true]%
	{scrbook}

% =============================================================================
% 					Benötigte Pakete
% =============================================================================

% ===================================================
%  Option nolucida aktivieren, falls *nicht* die Lucida Schrift verwendet wird.
% =================================================== 
\usepackage%[nolucida]%
		   {janmcommon}
\usepackage{janmscript}
\usepackage{fancyhdr}
\usepackage{marginnote}
\usepackage{makeidx}
\usepackage{yhmath}

\makeindex

% =============================================================================
% 					Theorem-Style
% =============================================================================
% Theorem Umgebungen *MIT* Numerierung
\theoremstyle{graymarginwithblueheader}
\theorembodyfont{\itshape}
\theoremseparator{}
\theoremsymbol{}

\newtheorem{prop}{Satz}[section]
\newtheorem{thm}[prop]{Theorem}
\newtheorem{lem}[prop]{Lemma}
\newtheorem{defn}[prop]{Definition}
\newtheorem{cor}[prop]{Korollar}

\theoremstyle{graymarginwithyellowheader}
\theorembodyfont{\normalfont}
\theoremseparator{}
\theoremsymbol{}

\newtheorem{bsp}[prop]{Bsp}

\theoremstyle{graymarginwithitblackheader}
\theorembodyfont{\normalfont}
\theoremseparator{}
\theoremsymbol{}

\newtheorem{bem}[prop]{Bemerkung.}

% Theorem Umgebungen *OHNE* Numerierung
\theoremstyle{graymarginwithblueheadern}
\theorembodyfont{\itshape}
\theoremseparator{}
\theoremsymbol{}

\renewtheorem{defn*}{Definition}
\renewtheorem{prop*}{Satz}
\renewtheorem{lem*}{Lemma}
\renewtheorem{cor*}{Korollar}

\theoremstyle{graymarginwithyellowheadern}
\theorembodyfont{\normalfont}
\theoremseparator{}
\theoremsymbol{}

\renewtheorem{bsp*}{Bsp}

\theoremstyle{graymarginwithitblackheadern}
\theorembodyfont{\normalfont}
\theoremseparator{}
\theoremsymbol{}

\renewtheorem{bem*}{Bemerkung.}

% =============================================================================
% 					Überschriften-Style
% =============================================================================
\renewcommand\thesection{\arabic{chapter}-\arabic{section}}
\renewcommand\thesubsection{\arabic{chapter}-\arabic{section}-\arabic{subsection}}
\renewcommand\thesubsubsection{\small\ensuremath{\blacksquare}\normalsize}
\renewcommand\thebsp{\arabic{bsp}}

\setkomafont{chapter}{\normalfont\bfseries\Huge\boldmath\color{darkblue}}
\setkomafont{section}{\normalfont\bfseries\Large\boldmath\color{darkblue}}
\setkomafont{subsection}{\normalfont\bfseries\large\boldmath\color{darkblue}}
\setkomafont{subsubsection}{\normalfont\bfseries\boldmath\color{darkblue}}

% =============================================================================
% 					PSTricks-Standards
% =============================================================================
\psset{linecolor=gdarkgray}
\psset{tickcolor=gdarkgray}
\psset{fillcolor=glightgray}

% Startkapitel 0
\setcounter{chapter}{-1}


% =============================================================================
% 					Header/Footer-Style
% =============================================================================
% Select page style
\pagestyle{fancyplain}

% Reset header & footer
\fancyhf{}

% Reset chapter & sectionmark
\renewcommand{\chaptermark}[1]{\markboth{\textsc{#1}}{}}
\renewcommand{\sectionmark}[1]{\markright{\textsc{#1}}{}}

% Clear headrule
\renewcommand{\headrule}{}

% =============================================================================
% 					Listen-Style
% =============================================================================
\newenvironment{bemenum}%
	{\begin{enumerate}[label=\textsc{\alph{*}}.,leftmargin=17pt]}{\end{enumerate}}
\newenvironment{defnenum}%
	{\begin{enumerate}[label={\rm(\alph{*})}]}{\end{enumerate}}
\newenvironment{propenum}%
	{\begin{enumerate}[label=\arabic{*})]}{\end{enumerate}}
\newenvironment{equivenum}%
	{\begin{enumerate}[label=(\roman{*})]}{\end{enumerate}}
\newenvironment{bspenum}%
	{\begin{enumerate}[label=\alph{*}),leftmargin=17pt]}{\end{enumerate}}
\newenvironment{proofenumarabicbr}%
	{\begin{enumerate}[label=(\arabic{*}),leftmargin=17pt]}{\end{enumerate}}
\newenvironment{proofenumroman}%
	{\begin{enumerate}[label=(\roman{*}),leftmargin=17pt]}{\end{enumerate}}
\newenvironment{proofenum}%
	{\begin{enumerate}[label=\arabic{*}),leftmargin=17pt]}{\end{enumerate}}
\newenvironment{proofenuma}%
	{\begin{enumerate}[label=\alph{*}.,leftmargin=2pt]}{\end{enumerate}}

% =============================================================================
% 					Eigene Operatoren
% =============================================================================
\renewcommand{\labelenumi}{{\normalfont(\alph{enumi})}}

\renewcommand{\card}{\#}

\newcommand{\fs}{\ \text{f.s.}}
\newcommand{\Pfs}{\ P\text{-f.s.}}

\newcommand{\Pto}{\overset{P}{\to}}

\DeclareMathOperator{\rank}{rank}

\renewcommand{\bar}[1]{{\overline{#1}}}
\newcommand{\ocirc}[1]{\overset{\circ}{#1}}
\newcommand{\cut}[1]{\wideparen{#1}}

\DeclareMathOperator*{\argmin}{argmin}

\newcommand{\pre}{{\mathrm{pre}}}

\newcommand{\defl}{\coloneq}
\newcommand{\defr}{\eqcolon}
\renewcommand{\le}{\leqslant}
\renewcommand{\ge}{\geqslant}


% =============================================================================
% 					Document-Body
% =============================================================================
\begin{document}

% Titel
\begin{titlepage}
\vspace*{2mm}
\noindent\bf\color{gdarkgray}
Jan-Cornelius Molnar

\begin{center}
\vspace*{10mm}
{\noindent\huge\textbf\textsc\color{darkblue} Nichtparametrische Statistik
und statistische Lerntheorie}

\vspace*{10mm}
Aufbauend auf einer Vorlesung von Prof. Dr. I. Steinwart

\vspace*{4mm}

Stuttgart, Sommersemester 2010
\end{center}

\vspace*{\fill}

\begin{flushright}
\small
Version vom \today
\vspace*{5mm}

Diese Rohfassung enthält sicher noch zahlreiche Druckfehler.
Für Hinweise auf solche und Kommentare jeder Art bin ich stets
dankbar.\footnote{\color{gdarkgray} Jan-Cornelius Molnar
\href{mailto:jan.molnar@studentpartners.de}{jan.molnar@studentpartners.de}}
Viel Spaß!
\end{flushright}
\end{titlepage}


% Inhaltsverzeichnis
\tableofcontents

\fancyhead[RO]{\footnotesize\color{gdarkgray}%
	\marginnote{$\Big|$\;\textbf{\thesection}}\rightmark}
\fancyhead[LE]{\footnotesize\color{gdarkgray}%
	\marginnote{\;\textbf{\thechapter}$\Big|$}\leftmark}

% Zweiseitig
\fancyfoot[LE]{\footnotesize\color{gdarkgray}%
 	\thepage}%
 \fancyfoot[RO]{\footnotesize\color{gdarkgray}%
 	\thepage}%
 \fancyfoot[RE,LO]{\tiny\color{gdarkgray}\today\; \thistime}

% Inhalt
\chapter{Grundlagen}

\section{Statistische Lerntheorie (SLT)}

Die statistische Lerntheorie ist Teil der nichtparametrischen Statistik sowie
der Informatik. Sie bildet den mathematischen Arm des maschinellen
Lernens und befasst sich sowohl mit den theoretischen Grundlagen wie auch
Fragen der Implementierung und besitzt zahlreiche Anwendungen.

\subsection{Inhalt der Vorlesung}

\begin{itemize}
  \item Was ist Lernen?
  \item Klassische Verfahren
  \item Empirische Risikominimierung (ERM)
  \item Regularisierte ERMs
  \item Kernbasierte Verfahren (support vector machines, SVMs)
  \item Verlustfunktionen
\end{itemize}

\section{Lernszenarien und -ziele}

\subsection{Überwachtes Lernen}
\index{Überwachtes Lernen}
Beim überwachten Lernen wird der Lernprozess durch einen \emph{Lehrer} geführt.
Den \emph{Eingaberaum} bezeichnen wir mit $X$, den \emph{Ausgaberaum} mit $Y$.
Weiterhin sind \emph{Trainingsdaten} in der From
\begin{align*}
((x_1,y_1),\ldots,(x_n,y_n))\in (X\times Y)^n
\end{align*} 
gegeben. Das Ziel ist es eine Funktion
\begin{align*}
f_D : X\to Y
\end{align*}
zu so finden, dass $f_D(x)\approx y$ für neue Paare $(x,y)$. Es geht also
nicht darum, die Daten einfach auswendig zu lernen, sondern $y$ für zukünftiges $x$
vorherzusagen. Was ``$\approx$'' genau bedeutet werden wir später erklären.

Beim überwachten Lernen, erhält der Lehrer ein $x_i$ und ordnet diesem ein
\emph{Label} $y_i$ zu. Der Lernende erhält dann vom Lehrer das Tupel
$(x_i,y_i)$ und soll darauf $f_D$ ableiten.

\subsubsection{Klassifikation}
\index{Klassifikation}

Gegeben sei der Ausgaberaum $Y=\setd{-1,1}$ und Ziel ist $f_D$ so zu bestimmen,
dass $f_D(x) = y$ "`möglichst häufig"'.

Es ist nun grundsätzlich zu entscheiden, ob man den Trainingsdaten traut oder
eher ein Rauschen vermutet.

\begin{figure}[!htpb]
\centering
\begin{pspicture}(0,-0.94)(3.6364853,0.94)
\psdots[linecolor=gdarkgray](0.06,0.5)
\psdots[linecolor=gdarkgray](0.86,0.76)
\psdots[linecolor=gdarkgray](0.62,-0.38)
\psdots[linecolor=gdarkgray](0.62,0.2)
\psdots[linecolor=gdarkgray](2.22,-0.14)
\psdots[linecolor=gdarkgray](1.44,-0.4)
\psdots[linecolor=gdarkgray](1.48,0.38)
\psdots[linecolor=gdarkgray](1.88,0.76)
\psdots[linecolor=gdarkgray,dotstyle=x](2.3,0.6)
\psdots[linecolor=gdarkgray,dotstyle=x](1.88,-0.68)
\psdots[linecolor=gdarkgray,dotstyle=x](3.16,-0.16)
\psdots[linecolor=gdarkgray,dotstyle=x](2.92,0.24)
\psdots[linecolor=gdarkgray,dotstyle=x](2.52,-0.74)
\psdots[linecolor=gdarkgray,dotstyle=x](3.56,0.32)
\psline[linecolor=darkblue](2.76,0.86)(1.38,-0.92)
\psbezier[linecolor=purple](2.32,0.92)(1.64,0.2)(2.54,0.16)(2.42,-0.24)(2.3,-0.64)(1.42,-0.34)(1.08,-0.86)
\end{pspicture} 
\caption{Zwei Möglichkeiten zur Klassifikation von Trainingsdaten.}
\end{figure}

Geht man davon aus, dass auf den Trainingsdaten keine Fehler sind, besteht die
Gefahr des sogenannten \emph{overfitting}, d.h. man traut den Daten zu sehr.
Vermutet man andererseits ein Rauschen besteht die Gefahr des
\emph{underfitting}, d.h. die Annahmen an die Zielfunktion sind unzureichend.

Over- und underfitting konkurrieren miteinander. Gute Lernverfahren finden
einen guten bzw. den ``besten'' Kompromiss.

\subsubsection{Anwendungen der Klassifikation}

\begin{itemize}
  \item Spamfilter
  \item Fraud Detection (Betrugsvorfälle bei Kreditkartenunternehmen)
  \item Schriftenerkennung
  \item Diagnoseverfahren
\end{itemize}

\subsubsection{Regression}
\index{Regression}

Sei $Y=[-M,M]$ oder $Y=\R$. Auch hier ist das Ziel $f_D(x) \approx y$ möglichst
häufig und es besteht ebenfalls die Gefahr des over- bzw. underfittings.

\begin{figure}[!htpb]
\centering
\begin{pspicture}(0,-1.3)(4.4,1.3)
\psline{->}(0.18,-1.28)(0.18,1.28)
\psline{->}(0.0,-1.06)(4.38,-1.06)
\psdots[dotstyle=x](0.36,0.44)
\psdots[dotstyle=x](0.8,0.94)
\psdots[dotstyle=x](1.32,0.0)
\psdots[dotstyle=x](2.1,0.62)
\psdots[dotstyle=x](2.96,-0.1)
\psdots[dotstyle=x](4.12,0.72)
\psbezier[linecolor=darkblue](0.26,0.34)(0.54,0.5)(0.42,0.96)(0.8,0.94)(1.18,0.92)(0.78,0.06)(1.32,0.0)(1.86,-0.06)(1.72,0.64)(2.26,0.66)(2.8,0.68)(2.74,-0.16)(3.24,-0.28)(3.74,-0.4)(3.56,0.68)(4.36,0.8)
\psline[linecolor=purple](0.18,0.56)(4.32,0.56)
\end{pspicture}
\caption{Zwei Möglichkeiten zur Regression (datentreu oder linear).}
\end{figure}

Was man genau unter ``$\approx$'' versteht, fällt unter den Terminus
\emph{Verlustfunktion}. Je nachdem wie man ``$\approx$'' mathematisch
präzesiert, erreicht man unterschiedliche Lernziele.

\subsubsection{Anwendung der Regression}
\begin{itemize}
  \item Finanzindustrie: Kursvorhersage von Aktien, Ausfallrisiken von
  Krediten, \ldots
  \item Data Mining
  \item Klassifikation $P(Y=y\mid x) =\ ?$
  \item Schätzungen von Restlebenszeiten von Teilen
\end{itemize}

\subsection{Halbüberwachtes Lernen}
\index{Halbüberwachtes Lernen}

Hier sind $X$ und $Y$ wie beim überwachten Lernen definiert. Die Trainingsdaten
sind jedoch nur noch zum Teil mit einem Label versehen. Sie bestehen also aus
einer "`kleinen"' Menge an
\begin{align*}
((x_1,y_1),\ldots,(x_n,y_n))\in (X,Y)^n,
\end{align*}
wobei $n=2$ den Extremfall bildet, sowie einer "`großen"' Menge an
\begin{align*}
(x_{n+1},\ldots,x_m)\in X^{m-n},\qquad m>> 1.
\end{align*}
Nun kann man ebenfalls Klassifikation oder Regression betreiben.

Die Motivation hierbei ist, dass die Beschaffung der $y_i$ entweder ``teuer''
oder gar unmöglich ist. Beispiele dafür sind medizinische Studien oder
Diagnoseverfahren zur Überwachung von Brücken.

Die Methoden für das halbüberwachte Lernen sind noch weitgehend unausgereift.
Es gibt noch sehr viele Fragestellungen an denen man arbeiten kann.

\subsection{Unüberwachtes Lernen}
\index{Unüberwachtes Lernen}

Beim unüberwachten Lernen sind nur Trainingsdaten ohne Label gegeben
\begin{align*}
x_1,\ldots,x_n\in X.
\end{align*}

Es lassen sich nun verschiedene Lernziele definieren. Wir wollen hier nur einen
kurzen Überblick geben; ausführlich werden wir uns lediglich mit der
Ausreißeridentifikation beschäftigen.

\subsubsection{Ausreißeridentifikation}
\index{Ausreißeridentifikation}

Es geht darum ``untypische'' Daten zu identifizieren und von "`typischen"' zu
unterscheiden.

\begin{figure}[!htpb]
\centering
\begin{pspicture}(0,-1.24)(5.08,1.24)
\psdots[linecolor=darkblue](0.3,0.66)
\psdots[linecolor=darkblue](0.06,-0.26)
\psdots[linecolor=darkblue](1.24,0.44)
\psdots[linecolor=darkblue](0.62,-0.04)
\psdots[linecolor=darkblue](1.18,-0.28)
\psdots[linecolor=darkblue](2.22,0.28)
\psdots[linecolor=darkblue](0.98,0.86)
\psdots[linecolor=darkblue](1.88,-0.2)

\psdots[linecolor=darkblue,dotstyle=x](1.62,0.34)
\psdots[linecolor=purple,dotstyle=x](3.32,-0.34)
\psbezier[linecolor=darkblue](1.64,0.54)(1.86,1.1)(2.02,0.16)(2.32,0.9)
\psbezier[linecolor=purple](3.44,-0.36)(3.8,-0.54)(3.04,-0.78)(3.6,-0.84)

\rput(2.73,1.045){\color{gdarkgray}neu typisch}
\rput(3.89,-1.015){\color{gdarkgray}neu untypisch}
\end{pspicture} 
\caption{Zwei Möglichkeiten zur Klassifikation von Trainingsdaten.}
\end{figure}

Die Ausreißeridentifikation hat viele Ähnlichkeiten mit der Klassifikation und
bereits gut verstanden.

\subsubsection{Clusteranalyse}
\index{Clusteranalyse}

Hier ist zunächst festzustellen, ob die Daten Cluster bilden, und diese
gegebenenfalls zu ermitteln.

\begin{figure}[!htpb]
\centering
\begin{pspicture}(0,-1.02)(4.22,0.98)
\psdots[linecolor=darkblue](0.42,0.5)
\psdots[linecolor=darkblue](1.06,0.54)
\psdots[linecolor=darkblue](0.64,0.22)
\psdots[linecolor=darkblue](1.08,0.14)
\psdots[linecolor=darkblue](0.44,-0.14)
\psdots[linecolor=darkblue](3.42,0.44)
\psdots[linecolor=darkblue](3.2,0.12)
\psdots[linecolor=darkblue](3.6,0.26)
\psdots[linecolor=darkblue](2.98,0.6)
\psdots[linecolor=darkblue](3.54,0.84)
\psdots[linecolor=darkblue](3.8,0.7)
\psellipse[linestyle=dotted](0.76,0.21)(0.76,0.61)
\psellipse[linestyle=dotted](3.43,0.42)(0.79,0.56)
\psdots[linecolor=purple,dotstyle=x](2.84,-0.86)
\rput(3.08,-0.855){\color{gdarkgray}?}
\end{pspicture} 
\caption{Clusterbildung.}
\end{figure}

Im Gegensatz zur Ausreißeridentifikation bietet die Clusteranalyse noch sehr
viele offene Fragen.

\subsubsection{Dimensionsreduzierung}
\index{Dimensionsreduzierung}

Viele Modelle erzeugen sehr hochdimensionale Daten. Hochdimensionale
Daten sind schlecht numerisch zu handhaben und außerdem schlecht zu
visualisieren.

\begin{figure}[!htpb]
\centering
\begin{pspicture}(0,-0.57)(5.48,0.57)
\psdots(0.16,-0.23)
\psdots(0.5,0.49)
\psdots(0.82,-0.13)
\psdots(1.08,0.31)
\psdots(1.58,-0.37)
\psdots(2.04,0.39)
\psdots(2.44,-0.01)
\psdots(3.74,0.15)
\psdots(3.2,0.35)
\psdots(3.2,-0.19)
\psdots(5.4,0.19)
\psdots(4.72,0.07)
\psdots(4.38,-0.11)
\psdots(1.44,0.11)
\psdots(1.28,-0.13)
\psdots(1.2,-0.43)
\psdots(1.74,-0.05)
\psdots(1.72,0.27)
\psdots(1.18,0.15)
\psbezier[linecolor=purple](0.0,-0.55)(0.14,0.19)(0.78,-0.05)(1.42,-0.01)(2.06,0.03)(2.08,0.45)(2.48,0.29)(2.88,0.13)(3.16,-0.17)(3.78,-0.01)(4.4,0.15)(4.72,-0.33)(5.28,0.33)
\end{pspicture} 
\caption{Dimensionsreduktion durch Approximation der Daten durch eine
Funktion.}
\end{figure}

\begin{bsp*}
Eine Kaffeekanne wird unter Rotation fotografiert, um so ein dreidimensionales
Bild zu erhalten. Jedes Foto besteht aus $n$ Pixeln, die man als Punkte im
$\R^n$ ansehen kann. Diese gewaltige Datenmenge lässt sich jedoch oft auf den
$\R^3$ reduzieren.\bsphere
\end{bsp*}

Auch die Dimensionsreduzierung ist noch weitgehend unerforscht.

\subsubsection{Dichteschätzung}
\index{Dichteschätzung}

Die Dichteschätzung ist mit klassischen Verfahren gut handhabbar und bereits
sehr ausgereift.

\begin{figure}[!htpb]
\centering
\begin{pspicture}(0,-0.91)(2.46,0.91)
\psdots(0.06,0.41)
\psdots(0.38,0.51)
\psdots(0.22,0.07)
\psdots(0.94,0.39)
\psdots(0.18,0.79)
\psdots(0.5,0.25)
\psdots(0.7,0.83)
\psdots(0.66,0.55)
\psdots(0.4,-0.83)
\psdots(2.26,-0.31)
\psdots(1.48,-0.45)
\psdots(2.38,0.77)
\end{pspicture} 
\caption{Zur Dichteschätzung.}
\end{figure}

\subsection{Weitere Lernszenarien}

\begin{itemize}
  \item\textit{Reinforcement Learning}
  \index{Reinforcement Learning}

 Reinforcement Learning hat das Lernen von Handlungen zum Ziel. Beispielsweise
 misst ein Roboter Umwelteinflüsse und entscheidet sich dannach für
 Handlungen. Welche Entscheidung ``gut'' war ist jedoch nur mit einer gewissen
 Verzögerung feststellbar.
 
 
 \item\textit{Multi-armed Bandit Learning}
 \index{Multi-armed Bandit Learning}
 
 Man hat $n$ Optionen, wovon jede einen Gewinn ausschüttet. Aufgrund
 beschränkter Ressourcen muss man eine Priorisierung vornehmen, um den
 größtmöglichen Gewinn zu erzielen.
 \end{itemize}

 \subsection{Datengenerierung}
 
\begin{itemize}
\item\textit{Batch Learning}
\index{Batch Learning}
 
 Hier liegen alle Trainingsdaten vor Trainingsbeginn vor.
 
 \item\textit{Online Learning}
 \index{Online Learning}
 
Das Online Learning ist bei den Informatikern sehr beliebt, da sich damit sehr
gut ``worst case''-Szenarien realisieren lassen.

\noindent
Ein möglicher Algorithmus wäre\\
a) Habe bereites $(x_1,y_1),\ldots,(x_i,y_i)$\\
b) Finde $f_i : X\to Y$\\
c) Erhalte $x_{i+1}$\\
d) Mache Vorhersage $f_i(x_{i+1})$\\
e) Bekomme $y_{i+1}$\\
f) Zurück zu a)

Eine ``billige" Generierung von $f_i$ ist wesentlich für den Erfolg des Online
Learning.
\item\textit{Active Learning}
\index{Active Learning}

Bestimme $x_1,\ldots,x_n$ selbst und erhalte anschließend die zugehörigen
Labels $y_1,\ldots,y_n$.
\end{itemize}

\subsubsection{Natur der Daten}

\begin{itemize}
\item \textit{Zufällige Daten}. Dies ist typisch für batch learning. Wir werden
uns später damit ausführlich beschäftigen.
\item \textit{Pseudozufällige Daten}. Die Daten werden eigentlich von einem
deterministischen System generiert, dieses ist aber chaotisch.
\item \textit{``worst case'' Daten}. Typisch für online learning.
\end{itemize}

\section{Klassische Verfahren zur Klassifikation}

Sei $X=\R^d$ und $Y=\setd{-1,1}$.

\begin{itemize}
\item\textit{Histogrammregel}.
\index{Histogrammregel}

Sei $(Q_j)_{j\ge1}$ eine Zerlegung von $X$ in Würfel. Es gilt dann
offenbar
\begin{align*}
\forall x\in X \exists j\ge 1 : x\in Q_j.
\end{align*}

\begin{figure}[!htpb]
\centering
\begin{pspicture}(0,-0.91)(3.42,0.91)
\psline(0.0,0.63)(3.4,0.63)
\psline(0.0,0.03)(3.4,0.03)
\psline(0.0,-0.57)(3.4,-0.57)
\psline(0.4,0.89)(0.4,-0.87)
\psline(1.2,0.87)(1.2,-0.89)
\psline(2.0,0.87)(2.0,-0.89)
\psline(2.82,0.87)(2.82,-0.89)
\psdots[linecolor=darkblue](0.56,0.51)
\psdots[linecolor=darkblue](0.68,0.17)
\psdots[linecolor=darkblue](0.98,0.53)
\psdots[linecolor=darkblue](1.58,-0.17)
\psdots[linecolor=darkblue](2.4,0.55)
\psdots[linecolor=darkblue](2.68,0.31)
\psdots[linecolor=purple](1.34,-0.27)
\psdots[linecolor=purple](1.66,-0.43)
\psdots[linecolor=purple](2.2,0.37)
\psdots[linecolor=purple](2.34,0.19)
\psdots[linecolor=purple](1.56,0.43)
\psdots[linecolor=purple](0.9,0.29)
\end{pspicture} 
\caption{Zerlegung von $X$.}
\end{figure}

Wir definieren nun
\begin{align*}
f_D(x) \defl \sign \left(\sum_{x_i \in Q_i(x)} y_i\right).
\end{align*}
\item\textit{Nearest Neighbor}.
\index{Nearest Neighbor}

Sei $(X,d)$ ein metrischer Raum (z. B. $X=\R^d$). Fixiere die Anzahl $k$ der
betrachteten Nachbarn. Für $x\in X$ suche die $k$ Beispiele
\begin{align*}
(x_{i_1},y_{i_1}),\ldots,(x_{i_k},y_{i_k})
\end{align*}
aus $D$, die am nächsten liegen.
\begin{align*}
f_D(x) = \sign\left(\sum_{j=1}^k y_{i_j}\right).
\end{align*}

\begin{figure}[!htpb]
\centering

\begin{pspicture}(0,-0.76824266)(3.6364853,0.76824266)
\psline(1.2,0.5482426)(1.42,0.42824265)
\psline(1.34,0.22824265)(1.42,0.42824265)
\psline(1.4,0.42824265)(1.64,0.6282427)
\psline(2.28,-0.31175736)(2.4,-0.45175734)
\psline(2.4,-0.45175734)(2.58,-0.37175736)
\psline(2.4,-0.45175734)(2.44,-0.61175734)

\psdots[linecolor=darkblue](0.06,0.5482426)
\psdots[linecolor=darkblue](0.84,0.028242646)
\psdots[linecolor=darkblue](0.26,0.008242645)
\psdots[linecolor=darkblue](0.76,0.52824265)
\psdots[linecolor=darkblue](1.7,0.008242645)
\psdots[linecolor=darkblue](1.68,0.6882427)
\psdots[linecolor=darkblue](1.06,-0.65175736)
\psdots[linecolor=darkblue](1.28,0.12824264)
\psdots[linecolor=darkblue](0.32,-0.39175737)
\psdots[linecolor=purple,dotstyle=x](1.14,0.5482426)
\psdots[linecolor=purple,dotstyle=x](1.54,-0.31175736)
\psdots[linecolor=purple,dotstyle=x](2.5,-0.6917574)
\psdots[linecolor=purple,dotstyle=x](2.64,-0.33175737)
\psdots[linecolor=purple,dotstyle=x](2.24,-0.25175735)
\psdots[linecolor=purple,dotstyle=x](2.48,0.28824264)
\psdots[linecolor=purple,dotstyle=x](3.56,0.20824264)
\psdots[linecolor=purple,dotstyle=x](2.98,-0.13175735)
\psdots[linecolor=purple,dotstyle=x](3.32,-0.41175735)
\end{pspicture} 
\caption{Nearest Neighbor Identifikation.}
\end{figure}

\item\textit{Moving window / kernel rules}.
\index{Moving window / kernel rules}

Sei $X=\R^d$ ein euklidischer Raum und
\begin{align*}
K: \R\to [0,\infty)
\end{align*}
symmetrisch und auf $[0,\infty)$ monoton fallend gegeben.

\begin{figure}[!htpb]
\centering
\begin{pspicture}(0,-0.64)(3.56,1.8)
\psline{<-}(1.7,1.78)(1.7,-0.62)
\psline{->}(0.0,-0.44)(3.54,-0.44)
\psarc[linecolor=darkblue](1.7,-0.42){1.36}{0.0}{180.0}
\end{pspicture}
\begin{pspicture}(0,-1.22)(3.56,1.22)
\psline{<-}(1.7,1.2)(1.7,-1.2)
\psline{->}(0.0,-1.02)(3.54,-1.02)
\psbezier[linecolor=darkblue](0.06,-1.0)(1.08,-1.0)(1.0801282,0.23601015)(1.7,0.24)(2.319872,0.24398986)(2.24,-0.96)(3.24,-1.0)
\end{pspicture}
\begin{pspicture}(0,-1.22)(3.58,1.22)
\psline{<-}(1.72,1.2)(1.72,-1.2)
\psline{->}(0.02,-1.02)(3.56,-1.02)
\psline[linecolor=darkblue]{|-|}(0.72,0.38)(2.72,0.38)
\psline[linecolor=purple]{-|}(0.0,-1.02)(0.74,-1.02)
\psline[linecolor=purple]{-|}(3.32,-1.02)(2.68,-1.02)
\end{pspicture} 
\caption{Beispiele für Funktionen $K:\R\to[0,\infty)$.}
\end{figure}

Fixiere $h>0$ und setze
\begin{align*}
f_D(x) = \sign\left( \sum_{i=1}^k y_i\ K(h^{-1}\norm{x-x_i})\right)
\end{align*}

\begin{figure}[!htpb]
\centering
\begin{pspicture}(0,-0.61)(2.7164853,0.61)
\psdots[linecolor=darkblue](0.056485273,0.23)
\psdots[linecolor=darkblue](0.79648525,0.31)
\psdots[linecolor=darkblue](0.27648526,-0.47)
\psdots[linecolor=darkblue](1.2164853,-0.43)
\psdots[linecolor=darkblue](0.83648527,-0.11)
\psdots[linecolor=darkblue](1.9164853,0.35)
\psdots[linecolor=darkblue](1.3964853,0.15)
\psdots[linecolor=purple](0.47648528,0.17)
\psdots[linecolor=purple](1.6164852,-0.23)
\psdots[linecolor=purple](2.0564852,0.05)
\psdots[linecolor=purple](2.3764853,0.41)
\psdots[linecolor=purple](2.2564852,-0.53)
\psdots[linecolor=purple](2.6364853,-0.03)
\psdots[linecolor=purple](2.1564853,-0.17)
\psbezier[linestyle=dotted](1.0164852,0.05)(0.9364853,-0.49)(0.6964853,-0.03)(0.39648527,-0.21)(0.09648527,-0.39)(0.03648527,0.11)(0.3164853,0.29)(0.59648526,0.47)(1.0964853,0.59)(1.0164852,0.05)
\end{pspicture} 
\caption{Identifikation mittels moving window.}
\end{figure}

\end{itemize}


\chapter{Grundkonzepte der statistischen Lerntheorie}
\newcommand{\class}{\mathrm{class}}
\newcommand{\diam}{\mathrm{diam}}

\section{Grundannahmen}

Folgende Notationen werden durchgehend verwendet.
\begin{itemize}
  \item $(X,\AA)$ sei ein metrischer Raum.
  \item $Y\subset\R$ sei abgeschlossen mit $Y=\setd{-1,1}$, $Y=[-M,M]$ oder
  $Y=\R$. $Y$ sei stets mit der Borel-$\sigma$-Algebra versehen.
  \item $P$ sei ein W-Maß auf $X\times Y$.
  \item $D=((x_1,y_1),\ldots,(x_n,y_n))\in(X\times Y)^n$.
  \item $\LL_0(X)\defl\setdef{f:X\to \R}{f\text{ messbar}}$.
\end{itemize}

Wir benötigen folgenden Satz zu regulären bedingten Wahrscheinlichkeiten.
\begin{prop*}
Es existiert eine Abbildung
\begin{align*}
P(\cdot\mid\cdot) : \BB\times X\to [0,1]
\end{align*}
mit folgenden Eigenschaften.
\begin{propenum}
\item $P(\cdot\mid x): \BB \to [0,1]$ ist ein W-Maß auf $\BB$ für alle $x\in X$.
\item $x\mapsto P(B\mid x)$ ist messbar für alle $B\in\BB$.
\item Für alle messbaren $A\in\AA$ und $B\in\BB$ gilt
\begin{align*}
P(A\times B) = \int_A P(B\mid x)\dP_X(x)\tag{*}
\end{align*}
wobei $P_X$ ein W-Maß auf $(X,\AA)$ ist, das durch $P_X(A) \defl P(A\times Y)$ für
$A\in\AA$ definiert ist.
\item $P(\cdot\mid\cdot)$ ist $P_X$-f.s. eindeutig.\fishhere
\end{propenum}
\end{prop*}

\begin{bem*}[Interpretation.]
Falls $P=P_Y\otimes P_X$, so ist $P(B\mid x) = P_Y(B)$ und daher
\begin{align*}
\text{(*)} = P(A\times B) = P_Y(B)P_X(A).
\end{align*}
$P(\cdot\mid\cdot)$ ermöglicht es, $P$ im Sinne von (*) aufzuspalten
(``desintegrate'').\maphere
\end{bem*}

Als Folgerung erhalten wir eine Verallgemeinerung des Satzes von Fubini von
Produktmaßen auf solche $P$.

\begin{cor*}
Sei $f: X\times Y\to \R$ $P$-integrierbar, so gilt
\begin{align*}
\E_P f\defl \int_{X\times Y} f\dP = \int_X \int_Y f(x,y)
P(\dy\mid x)\dP_X(x).\fishhere
\end{align*}
\end{cor*}

\begin{bem*}
Für allgemeine messbare Räume ($Y,\AA'$) ist dies \textit{nicht} möglich.
(Wesentlich ist hier, dass $Y\subset \R$ abgeschlossen).\maphere
\end{bem*}

\noindent
Weitere Grundannahmen sind.
\begin{itemize}
  \item $P$ ist uns völlig unbekannt. Wir wissen lediglich, dass $P$ existiert.
  \item Die Trainingsdaten $D$ sind Beobachtungen einer Folge $Z=(Z_i)_{i=1}^n$
  mit $Z_i=(X_i,Y_i)$ von Zufallsvariablen, die $X\times Y$-wertig und
  unabhängig sind und alle die Verteilung $P$ haben.

\noindent
\textit{Interpretation}.\\
$X_i\sim P_X$, dies gibt eine Beobachtung $x_i\in X$.\\
$Y_i\sim P(\cdot\mid x_i)$, dies gibt eine Beobachtung $y_i\in Y$.

\noindent
\textit{Datengenerierungsalgorithmus}.\\
\begin{itemize}
  \item $i=1$.
  \item ``Würfle'' $x$ gemäß der Verteilung $P_X$ und unabhängig von
  $(x_1,y_1)$, \ldots, $(x_{i-1},y_{i-1})$.
  \item ``Würfle'' $y$ gemäß der Verteilung $P(\cdot\mid x_i)$.
  \item Wiederhole bis $i=n$.
\end{itemize}
\item Alle zukünftigen Daten $(x,y)$ sind Beobachtungen von einer
Zufallsvariable $\tilde{Z} = (\tilde{X},\tilde{Y})$, die $X\times Y$-wertig
ist, Verteilung $P$ hat und unabhängig von der Folge $(Z_i)_{i=1}^n$ ist.
\item Für eine große Anzahl zukünftiger Beobachtungen, betrachte die Folge
$(Z_i)_{i=n+1}^\infty$ von $(X\times Y)$-wertigen Zufallsvariablen, die
untereinander und von $Z_1,\ldots,Z_n$ unabhängig sind und Verteilung $P$ haben.
\end{itemize}

\begin{defn}
\label{defn:1.1.1}
Eine \emph{Lernmethode} oder auch \emph{Lernverfahren} ist eine Folge
$(L_n)_{n\ge 1}$ von Abbildungen
\index{Lernmethode}
\index{Lernverfahren}
\begin{align*}
L_n : (X\times Y)^n\to \LL_0(X),\qquad D\mapsto f_D.\fishhere
\end{align*}
\end{defn}

\begin{bsp*}
\begin{bspenum}
\item Histogrammregel.
\item Nearest Neighbor.
\item Moving windows, kernel rules.\bsphere
\end{bspenum}
\end{bsp*}

\begin{bem}
Wir werden später benötigen, dass die Abbildungen
\begin{align*}
(X\times Y)^n\times X \to \R,\qquad (D,x) \mapsto f_D(x)
\end{align*}
messbar sind.\maphere
\end{bem}

\section{Verlustfunktionen und Risiken}
\index{Verlustfunktion}\index{Risiko}
\label{sec:1.2}

Ziel dieses Abschnitts ist es, genauer zu beschreiben, was $f_D(x) \approx y$
bedeutet.

\begin{defn}
\label{defn:1.2.1}
Eine messbare Abbildung
\begin{align*}
L: X\times Y\times\R\to [0,\infty)
\end{align*}
heißt \emph{Verlustfunktion}\index{Verlustfunktion}. Selbiges gilt für
Funktionen
\begin{align*}
L: X\times \R\to [0,\infty),\quad \text{oder}\quad L:Y\times \R\to
[0,\infty).\fishhere
\end{align*}
\end{defn}

\begin{bem*}[Interpretation.]
$L(x,y,f(x))\in [0,\infty)$ beschreibt den ``Verlust'' bei der Entscheidung
$f_D(x)$, falls $(x,y)$ beobachtet wird. Generell sind kleine Verluste besser
als große!\maphere
\end{bem*}

\begin{defn}
\label{defn:1.2.2}
Sei $L$ eine Verlustfunktion. Dann ist das \emph{Risiko} einer messbaren
Funktion $f: X\to\R$ durch\index{Risiko}
\begin{align*}
\RR_{L,P}(f) &= \int_{X\times Y} L(x,y,f(y))\dP(x,y)\\
&= \int_X \int_Y L(x,y,f(x))P(\dy\mid x)\dP_X(x)
\end{align*}
definiert.\fishhere
\end{defn}
\begin{bem*}[Interpretation.]
Ist $Z=(Z_i)_{i=1}^n$ eine Folge von $X\times Y$-wertigen, unabhängigen und
$P$-verteilten Zufallsvariablen und $\RR_{L,P}(f) < \infty$, so besagt das
starke Gesetz der großen Zahlen, dass
\begin{align*}
\RR_{L,P}(f) = \lim\limits_{m\to\infty} \frac{1}{m-n-1}\sum_{i=n+1}^\infty
L(x_i,y_i,f(x_i))\Pfs
\end{align*}
$\RR_{L,P}(f)$ beschreibt daher den mittleren zukünftigen Verlust.\maphere
\end{bem*}

Wir wollen jetzt genauer spezifizieren, was wir unter ``kleinen'' Risiken
zu verstehen ist.
\begin{defn}
\label{defn:1.2.3}
Sei $L$ eine Verlustfunktion, dann heißt das kleinstmögliche Risiko
\begin{align*}
\RR_{L,P}^* \defl \inf\setdef{\RR_{L,P}(f)}{f:X\to\R\text{ messbar}}
\end{align*}
\emph{Bayes-Risiko}.\index{Bayes!Risiko}
Eine messbare Funktion $f_{L,P}^* : X \to \R$ für die gilt
\begin{align*}
\RR_{L,P}(f_{L,P}^*) = \RR_{L,P}^*
\end{align*}
heißt
\emph{Bayes-Entscheidungsfunktion}.\index{Bayes!Entscheidungsfunktion}\fishhere
\end{defn}

$f_{L,P}^*$ ist im Allgemeinen nicht $\Pfs$ eindeutig.

Unser informelles Lernziel besteht nun darin, ein Lernverfahren $\LL$ zu
finden, für das das \emph{Überschussrisiko}\index{Überschussrisiko}
\begin{align*}
\RR_{L,P}(f_D) - \RR_{L,P}^*,\qquad f_D\in\LL
\end{align*}
mit hoher Wahrscheinlichkeit klein ist. In Abschnitt \ref{chap:1.3} werden wir
präzisieren was wir unter ``hoher Wahrscheinlichkeit'' und ``klein''
verstehen.

\begin{bsp*}[Bsp Klassifikation]

Hier ist unser Ausgaberaum $Y=\setd{-1,1}$ und unser
Ziel, $y$ ``richtig vorauszusagen''. Mithilfe der vorangegangen Definitionen
lässt sich dies nun mathematisch formulieren.

Wir definieren eine Verlustfunktion wie folgt
\begin{align*}
&L_\class : Y\times \R\to[0,\infty),\quad (y,t)\mapsto
\Id_{(-\infty,0]}(y\sign t),\\
&L_\class(y,t) =
\begin{cases}
0, & y = -1,\; t<0,\\
1, & y = -1,\; t\ge0,\\
1, & y = 1,\; t<0,\\
0, & y = 1,\; t\ge0.
\end{cases} 
\end{align*}
$L_\class$ bestraft somit Vorhersagen, für die $y\neq \sign t$. Wir können nun
das Risiko angeben,
\begin{align*}
\RR_{L_\class,P}(f) &=
\int_{X\times Y} \Id_{(-\infty,0]}(y\sign f(x)) \dP(x,y)\\
&= P\setdef{(x,y)}{y\neq \sign f(x)}\\
&= \int_X\int_Y L_\class(y,f(x))P(\dy\mid x)\dP_X(x).
\end{align*}
Setzen wir nun $\eta(x) = P(y=1\mid x)$ so können wir das Integral schreiben als
\begin{align*}
\int_X\int_Y \eta(x)\Id_{(-\infty,0)}(f(x)) +
(1-\eta(x))\Id_{[0,\infty)}(f(x))\dP_X(x).
\end{align*}
Damit der Integrand minimal wird muss gelten,
\begin{align*}
&\eta(x) > \frac{1}{2}\Rightarrow f(x)\ge 0,\\
&\eta(x) <\frac{1}{2}\Rightarrow f(x)< 0.
\end{align*}
Daher minimiert $f$ das Risiko genau dann, wenn
\begin{align*}
&f(x)\ge 0, && \text{auf } [\eta> \frac{1}{2}],\\ 
&f(x)< 0, && \text{auf } [\eta< \frac{1}{2}].
\end{align*}
In diesem Fall ist $f=f_{L_\class,P}^*$ und folglich
\begin{align*}
\RR_{L,P}^* = \int_X \min\setd{\eta,1-\eta}\dP_X.
\end{align*}
Wir wollen noch zeigen, dass
\begin{align*}
\RR_{L,P}(f) - \RR_{L,P}^* =
\int_X \abs{2\eta-1}\Id_{(-\infty,0]}((2\eta-1)\sign f(x))\dP_X.
\end{align*}
\begin{proof}[Beweisskizze.]
Dazu betrachtet man einfach die 6 möglichen Kombinationen von
\begin{align*}
&\eta(x) > \frac{1}{2} & \eta(x) < \frac{1}{2} && \eta(x) = \frac{1}{2}\\
& f(x) \ge 0 & f(x) < 0. 
\end{align*}
z.B. $\eta(x) > \frac{1}{2}$ und $f(x)\ge 0$, so ist
\begin{align*}
\abs{2\eta(x)-1}\Id_{(-\infty,0]}(\underbrace{(2\eta-1)\sign f(x)}_{>0})= 0
\end{align*}
und
\begin{align*}
\eta(x)\underbrace{\Id_{(-\infty,0)}(f(x))}_{=0} +
(1-\eta(x))\underbrace{\Id_{[0,\infty)}(f(x))}_{=1}
- \underbrace{\min\setd{\eta(x),1-\eta(x)}}_{1-\eta(x)} = 0.
\end{align*}
oder $\eta(x)>\frac{1}{2}$ und $f(x) < 0$, so ist
\begin{align*}
\abs{2\eta(x)-1}\Id_{(-\infty,0]}(\underbrace{(2\eta-1)\sign f(x)}_{<0}) =
2\eta(x)-1
\end{align*}
und
\begin{align*}
&\eta(x)\underbrace{\Id_{(-\infty,0)}(f(x))}_{=1} +
(1-\eta(x))\underbrace{\Id_{[0,\infty)}(f(x))}_{=0}
- \underbrace{\min\setd{\eta(x),1-\eta(x)}}_{1-\eta(x)} \\ &\quad = 2\eta(x)-1.
\end{align*}
Die übrigen Fälle folgen analog.\qedhere
\end{proof}

In unserer Definition von $L_\class$ haben wir alle Fehler gleich gewichtet.
Oft nimmt man eine unterschiedliche Gewichtung vor (Übungsaufgabe).\bsphere
\end{bsp*}

\begin{bsp*}[Regression mit kleinsten Fehlerquadraten]
\newcommand{\LS}{\mathrm{LS}}
Sei $Y\subset\R$ ein Intervall oder ganz $\R$. Unser Ziel ist es $f$ zu finden
mit $f(x)\approx y$. Dies wollen wir nun mathematisch fassen.

Dazu definieren wir folgende Verlustfunktion
\begin{align*}
L_\LS : Y\times \R\to [0,\infty),\qquad (y,t)\mapsto (y-t)^2.
\end{align*}
LS steht hier für least squares. Das Risiko ist dann gegeben durch
\begin{align*}
\RR_{L,P}(f) &= \int_X\int_Y (y-f(x))^2 P(\dy\mid x)\dP_X(x)\\
&= \int_X \int_Y y^2-2yf(x) + f^2(x) P(\dy\mid x)\dP_X(x)\\
&= \int_X \int_Y y^2P(\dy\mid x) - 2 f(x)\int_Y yP(\dy\mid x)\\
&\qquad\qquad+ f^2(x)\dP_X(x).
\end{align*}
Setzen wir $t=f(x)$ und $h(t) = -2t \int_Y y P(\dy\mid x) + t^2$, so erhalten
wir ein Minimum durch
\begin{align*}
0 = h'(t) = -2\int_Y yP(\dy\mid x) + 2t \Rightarrow t = \int_Y yP(\dy\mid x).
\end{align*}
Somit ist $f: X\to\R$ eine Bayes-Entscheidungsfunktion, wenn
\begin{align*}
f(x) = \int_Y yP(\dy\mid x) \defr \E_P(Y\mid x) \Pfs
\end{align*}
$f$ ist in diesem Fall tatsächlich eindeutig.

Das alles funktioniert jedoch nur, wenn $\RR_{L,S}^* < \infty$, denn sonst ist
jede Funktion eine Bayes-Entscheidungsfunktion.

Setzen wir nun in $\RR_{L,S}(\cdot)$ ein, erhalten wir
\begin{align*}
\RR_{L,S}(f^*_{L,S}) &=
\int_X \int_Y y^2P(\dy\mid x) - 2(\E_P(Y\mid x))^2 + (\E_P(Y\mid x))^2\dP_X(x)\\
&= 
\int_X \int_Y y^2P(\dy\mid x) - (\E_P(Y\mid x))^2\dP_X(x)\\
&= \text{mittlere Varianz der Label }y.
\end{align*}
Somit ist das Überschussrisiko
\begin{align*}
\RR_{L,P}(f) - \RR_{L,P}^* &=
\int_X\int_Y y^2 - 2f(x)y + f^2 - y^2 \\ &\qquad\qquad - (\E_P(Y\mid x))^2
P(\dy\mid x) \dP_X(x)\\
&= \int_X (f(x)-f^*_{L,P}(x))^2\dP_X(x) = \norm{f-f_{L,S}^*}_{L^2(P_X)}^2.
\end{align*}
Minimierung des Risikos ist also äquivalent zur $L^2$-Approximation von
$f^*_{L,S}$.\bsphere
\end{bsp*}

Bis jetzt waren alle konkreten Verlustfunktionen von $x$ unabhängig.

\begin{defn}
\label{defn:1.2.4}
Eine messbare Funktion $L:Y\times\R\to [0,\infty)$ heißt \emph{strikt überwachte
Verlustfunktion}\index{Verlustfunktion!strikt überwachte}.\fishhere
\end{defn}

\begin{defn}
\label{defn:1.2.5}
Eine messbare Funktion $L:X\times\R\to [0,\infty)$ heißt \emph{unüberwachte
Verlustfunktion}\index{Verlustfunktion!unüberwachte}.\fishhere
\end{defn}

Es liegt die Vermutung nahe, dass unüberwachte Verlustfunktionen nutzlos für
überwachtes Lernen ist. Dem ist aber nicht so, wie folgendes Beispiel zeigen
wird.

\begin{bsp*}[Bsp Medianregression]
$Y\subset\R$ ist hier typischerweise ein Intervall oder $\R$.

Für $x\in X$ ist der \emph{bedingte Median}\index{bedingter Median} (Median von
der bedingten Erwartung in $x$ $P(\cdot\mid x)$) gegeben durch
\begin{align*}
F_{1/2}^* \defl \setdef{t^*\in\R}{\frac{1}{2}\le P((-\infty,t^*]\mid x)\text{ und
}\frac{1}{2}\le P([t^*,\infty)\mid x)}.
\end{align*}
Im Allgemeinen ist dieser nicht eindeutig, wir nehmen nun aber an, dass dem so
ist.

Das Ziel ist es nun $f^*$ bei bekanntem
$D=((x_1,y_1),\ldots,(x_n,y_n))$ abzuschätzen. Wie genau das gemacht wird
behandeln wir später, jetzt soll es nur darum gehen, die Verluste zu bewerten.

Betrachte die Verlustfunktion
\begin{align*}
L: X\times\R \to [0,\infty),\qquad (x,t)\mapsto \abs{f^*(x)-t}.
\end{align*}
Das Risiko ist dann
\begin{align*}
&\RR_{L,P}(f) \defl \int_X \abs{f^*(x)-f(x)}\dP_X(x),\\
&\RR^*_{L,P} = \RR_{L,P}(f^*) = 0.
\end{align*}
Somit ist das Überschussrisiko
\begin{align*}
\RR_{L,P}(f) - \RR_{L,P}^* = \norm{f^*-f}_{L^1(P_X)}.
\end{align*}
Die Verlustfunktion ist also tatsächlich unabhängig von den Daten, obwohl wir
überwachte Labels haben.

Problematisch hierbei ist, dass $L$ auf dem bedingten Median beruht, den wir
nicht kennen und so $L$ überhaupt nicht berechnen können. Man muss also einen
anderen Ansatz wählen.

Dies ist auch ein gutes Beispiel dafür, dass ein Unterschied darin besteht,
eine Verlustfunktion zu finden, die das Lernziel gut beschreibt, und eine, die 
in der Praxis gut verwendbar ist.\bsphere
\end{bsp*}

\section{Universelle Konsistenz}
\label{chap:1.3}

Ziel dieses Abschnittes ist es die Formulierung
\begin{align*}
\RR_{L,P}(f)-\RR_{L,P}^*
\end{align*}
ist mit hoher Wahrscheinlichkeit klein, mathematisch zu präzisieren.

\begin{defn}
\label{defn:1.3.1}
Sei $L:X\times Y\times \R\to [0,\infty)$ eine Verlustfunktion. Eine Lernmethode
$\LL$ heißt \emph{$P$-konsistent}\index{$P$-konsistent} bezüglich $L$, falls für
alle $\ep > 0$ gilt
\begin{align*}
\lim\limits_{n\to\infty} P^n\setdef{D\in (X\times Y)^n}{\RR_{L,P}(f_D) -
\RR_{L,P}^* < \ep} = 1.
\end{align*}
$\LL$ heißt \emph{universell konsistent}\index{universell konsistent} bezüglich
$L$, falls $\LL$ $P$-konsistent bezüglich $L$ für W-Maße $P$ auf $X\times Y$ für die
$\RR_{L,P}^* < \infty$.\fishhere
\end{defn}

Die universelle Konsistenz ist einerseits eine schwache Forderung, da sie eine
rein asymptotische Aussage macht. Sie ist andererseits aber auch eine sehr
starke Forderung wie das folgende Beispiel zeigen wird.

\begin{bsp*}
Wir betrachten Klassifikation auf $X=[0,1]$. Seien dazu
\begin{align*}
&X_1 = C\subset[0,1],\qquad \text{``Cantor Menge''},\\
&X_{-1} = (\Q\cap[0,1])\setminus C
\end{align*}
Nun existiert ein W-Maß $\mu$ auf $X_1$ (z.B. das Hausdorffmaß) und ein W-Maß
$\nu$ auf $X_{-1}$ (z.B. Zählmaß+Dichte), so dass für
 $P_X\defl\frac{1}{2}(\mu+\nu)$ gilt
\begin{align*}
P(Y=1\mid x) =
\begin{cases}
0.51, & x\in X_1,\\
0.49 & x\in X_{-1}.\bsphere
\end{cases}
\end{align*}
\end{bsp*}

\subsection{Plug-in-rules für Klassifikation}

Wir wollen nun für ein spezielles Verfahren nachweisen, dass es universell
konsistent ist.

Sei $Y=\setd{-1,1}$ und $\eta(x) = P(Y=1\mid x)$ die Wahrscheinlichkeit dafür,
dass $x$ ein positives Label besitzt. Gegeben sei eine Lernmethode $\LL$, die
die Entscheidungsfunktion $f_D$ konstruiert und $f_D\approx \eta$.

Wir haben bereits gezeigt, dass für die
Bayes-Klassifikationsentscheidungsfunktion gilt
\begin{align*}
f_{L_\class,P}^*(x) =
\begin{cases}
1, & \eta(x)> \frac{1}{2},\\
-1, & \eta(x) < \frac{1}{2}.
\end{cases}
\end{align*}

\begin{defn}
\label{defn:1.3.2}
Das zu $\LL$ gehörige \emph{plug-in-Verfahren} ist durch
\begin{align*}
\hat{f}_D(x) \defl
\begin{cases}
1, & f_D(x) > \frac{1}{2},\\
-1, & f_D(x) < \frac{1}{2},
\end{cases}
\end{align*}
gegeben.\fishhere
\end{defn}

Wie gut das Verfahren arbeitet, beschreibt das folgende
\begin{lem}
\label{prop:1.3.3}
Sei $P$ ein W-Maß auf $X\times Y$, $Y=\setd{-1,1}$, $\eta(x) = P(Y=1\mid x)$
und $x\in X$. Für $h:X\to\R$ gilt dann
\begin{align*}
\RR_{L_\class}(2h-1) - \RR_{L_\class,P}^*
\le 2 \int_X \abs{\eta-h}\dP_X.\fishhere
\end{align*}
\end{lem}

\textit{Interpretation}. Ist $h$ eine ``gute'' Schätzung von $\eta$, dann ist
$2h-1$ (bzw. $\sign (2h-1)$) eine ``gute'' Klassifikationsentscheidungsfunktion.

\begin{proof}[Beweis des Lemmas.]
Sei $f\defl 2h-1$. Wir haben bereits gezeigt, dass
\begin{align*}
\RR_{L_\class}(f) - \RR_{L_\class,P}^*
= \int_X \underbrace{\abs{2\eta-1}\Id_{(-\infty,0]}((2\eta-1)\sign
f)}_{\text{(1)}}\dP_X.
\end{align*}

Falls $(2\eta-1)\sign f > 0$, so ist der Integrand Null und daher
\begin{align*}
(1) \le 2\abs{\eta-h}.
\end{align*}

Falls $(2\eta-1)\sign f < 0$, so wertet die Indikatorfunktion zu 1 aus. Ist nun
$\eta > \frac{1}{2}$, so ist $f < 0$ und somit $-f-1 > -1$, also
\begin{align*}
\abs{2\eta-1} &= 2\eta -1 \le 2\eta-1-f = 2\eta -1 -(2h-1) = 2(\eta - h)\\
&\le 2\abs{\eta -h}.
\end{align*}
Ist dagegen $\eta < \frac{1}{2}$, so ist $f\ge 0$ und somit $f+1 \ge 1$, also
\begin{align*}
\abs{2\eta-1} &= 1-2\eta \le f+1-2\eta = 2\abs{\eta-h}.
\end{align*}
Ist $\eta=\frac{1}{2}$, so ist (1)$=0$.\qedhere
\end{proof}

Betrachten wir die Histogrammregel mit $Y=\setd{-1,1}$, $X=\R^d$ und
\begin{align*}
D=((x_1,y_1),\ldots,(x_n,y_n))\in (X\times Y)^n.
\end{align*}
Sei nun $\AA=(A_i)_{i\ge 1}$ mit $A_i$ messbar eine Partition von $X$, d.h.
\begin{align*}
X= \bigcup_{i\ge 1} A_i,\qquad A_i\cap A_j = \varnothing,\quad i\neq j. 
\end{align*}
Für jedes $x\in X$ existiert ein eindeutiger Index $j\ge 1$, so dass $x\in
A_j$. Setzen wir
\begin{align*}
N_D(x) \defl \sum_{i=1}^n \Id_{A_i(x)}(x_i) = 
\card\setd{\text{Samples }(x_i,y_i)\text{ für die }x_i\in A(x)}, 
\end{align*}
so erhalten wir
\begin{align*}
\hat{n}_D(x)
=
\begin{cases}
\frac{1}{N_D(x)}\sum_{\setdef{i}{y_i=1}} \Id_{A(x)}(x_i), & N_D(x)\neq 0,\\
0, & \text{sonst}.
\end{cases}
\end{align*}
Damit können wir nun eine plug-in-Regel definieren.
\begin{align*}
f_D^\AA(x) \defl
\begin{cases}
-1, & \hat{n}_D(x) < \frac{1}{2},\\
1 & \hat{n}_D(x) \ge \frac{1}{2}.  
\end{cases}
\end{align*}
Für $A\subset\R^d$ ist der \emph{Durchmesser} definiert als
\begin{align*}
\diam A \defl \sup_{x,x'\in A} \norm{x-x'}.
\end{align*}
Für $n\ge 1$ betrachten wir nun eine weitere Partition $\AA_n = (A_{n,j})_{j\ge
1}$. Das Lernverfahren $\LL$ sei nun durch
\begin{align*}
(X\times Y)\ni D \mapsto f_D^{\AA_n} \defr f_D\tag{*}
\end{align*}
gegeben.

\begin{prop}
\label{prop:1.3.4}
Sei $P$ ein W-Maß auf $X\times Y$, $Y=\setd{-1,1}$ und $X=\R^d$. Ferner gelten
\begin{propenum}
\item $\lim\limits_{n\to\infty} \sup_j \diam A_{n,j} = 0$.
\item $\lim\limits_{n\to\infty} P^n\otimes P_X(\setdef{(D,x)}{N_D(x)\le l}) =
0\fs$ für $l\ge 1$.
\end{propenum}
Dann ist die Lernmethode (*) $P$-konsistent bezüglich $L_\class$.\fishhere
\end{prop}
\begin{proof}
Nach Lemma \ref{prop:1.3.3} gilt für $h=\hat{\eta}_D$ und
$f_D=\sign(2\hat{\eta}_D-1)$,
\begin{align*}
\RR_{L_\class}(f_D) - \RR_{L_\class,P}^*
\le 2\int_X \abs{\eta-\hat\eta_D}\dP_X.
\end{align*}
Definiere nun $\bar{\eta}:X\to[0,1]$,
\begin{align*}
\bar{\eta}(x) =
\begin{cases}
\frac{1}{P_X(A(x))}\int_{A(x)} \eta \dP_X, & P_X(A(x)) > 0,\\
0, & \text{sonst},
\end{cases}
\end{align*}
als ``mittleres $\eta$ auf $A(x)$''. $\bar{\eta}$ ist somit auf jedem $A(x)$
konstant.
Dann gilt
\begin{align*}
\int \abs{\eta-\hat{\eta}_D}\dP_X \le
\underbrace{\int\abs{\eta-\bar{\eta}}\dP_X}_{(1)}
+ \underbrace{\int \abs{\bar{\eta}-\hat{\eta}_D}\dP_X}_{(2)}.
\end{align*}

Wir betrachten zunächst den von $D$ abhängigen Term (2),
\begin{align*}
&\int_{X\times Y} \int_X \abs{\bar{\eta}(x)-\hat{\eta}_D(x)}\dP_X(x)\dP^n(D)\\
&\quad =
 \int_X\int_{X\times Y} \abs{\bar{\eta}(x)-\hat{\eta}_D(x)}\dP^n(D)\dP_X(x)\\
&\quad=
 \int_X\sum_{k=0}^n \int_{\setdef{D}{N_D(x)=k}}
 \abs{\bar{\eta}(x)-\hat{\eta}_D(x)}\dP^n(D)\dP_X(x)
\end{align*}
Sei $N_D(x)=k$, dann existieren eindeutige $x_{i_1},\ldots,x_{i_k}\in A(x)$.
Die zugehörigen $y_{i_1},\ldots,y_{i_k}$ sind unabhängige Zufallsvariablen,
wovon jede einzelne mit der Wahrscheinlichkeit $\bar{\eta}(x)$  positiv ist.

Somit ist $\sum_{\setdef{j}{Y_{i_j}=1}} y_{i_j}$ binomialverteilt, genauer
$B(k,\bar{\eta}(x))$-verteilt und
\begin{align*}
\sum_{\setdef{i}{y_i=1}}\Id_{A(x)}(x_i) = N_D(x)\hat{\eta}_D(x) =
k\cdot\hat{\eta}_D(x).
\end{align*}

Sei $\xi: \Omega\to\R$ eine $B(k,\bar{\eta}(x))$-verteilte Zufallsvariable. Dann
gilt $\xi/k \sim \hat{\eta}_D(x)$ für $k\ge 1$ und folglich
\begin{align*}
&\frac{1}{P^n(\setdef{D}{N_D(x)=k})}\int_{\setdef{D}{N_D(x)=k}}
\!\!\!\!\!\!\!\!\!\!\!\!\!\!\!\!\!\!\!\!\!\!\!\!\!\!
\abs{\bar{\eta}(x)-\hat{\eta}_D(x)}\dP^n(D)\\
&\qquad=
\E \abs{\bar{\eta}(x)-\frac{1}{k}\xi}\\
&\qquad\le
\frac{1}{k}\left(\E\left((k\bar{\eta}(x)-\xi)^2 \right)\right)^{1/2}\\
&\qquad= \frac{1}{k}\left(\E(\E\xi - \xi)^2 \right)^{1/2},
\end{align*}
nach Hölders Ungleichung. Die Varianz einer $B(k,\bar{\eta}(x))$-verteilten
Zufallsvariablen ist $k\bar{\eta}(x)(1-\bar{\eta}(x))\le k/4$ und damit erhalten
wir für obigen Ausdruck die Abschätzung
\begin{align*}
\frac{1}{k}\left(\E(\E\xi - \xi)^2 \right)^{1/2} = 
 \frac{1}{k}\left(k\bar{\eta}(x)(1-\bar{\eta}(x))\right)^{1/2}
\le \frac{1}{2\sqrt{k}}.
\end{align*}

Für $k=0$ gilt auf $\setdef{D}{N_D(x)=k}$ nach Definition $\hat{\eta}\equiv0$
sowie $\abs{\bar{\eta}(x)}\le 1$. Deshalb ist
\begin{align*}
\int_{\setdef{D}{N_D(x)=k}}
\!\!\!\!\!\!\!\!\!\!\!\!\!\!\!\!\!\!\!\!\!\!\!\!\!\!
\abs{\bar{\eta}(x)-\hat{\eta}_D(x)}\dP^n(x) \le P^n(\setdef{D}{N_D(x)=k}).
\end{align*} 
Damit folgt
\begin{align*}
&\int_{(X\times Y)^n} \int_X
\abs{\bar{\eta}(x)-\hat{\eta}_D(x)}\dP_X(x)\dP^n(D)\\
&\quad\le
\int_X \sum_{k=1}^n \frac{1}{2\sqrt{k}} P^n(\setdef{D}{N_D(x)=k})
+ P^n(\setdef{D}{N_D(x)=0})\dP_X(x)\\
&\quad\le
\int_X \sum_{k=1}^n \int_{\setdef{D}{N_D(x)=k}}
\frac{1}{2\sqrt{k}} \dP^n(D) + P^n(\setdef{D}{N_D(x)=0})\dP_X(x)\\
&\quad\le
\frac{1}{2}\int_X \int_{\setdef{D}{N_D(x)>0}}
\frac{\dP^n(D)}{\sqrt{N_D(x)}}\dP_X(x)  +
P^n\otimes P(\setdef{(D,x)}{N_D(x)=0})\tag{**}
\end{align*}
Für $l\ge 1$ gilt dann
\begin{align*}
&\int_X  \int_{\setdef{D}{N_D(x)>0}}
\frac{\dP^n(D)}{\sqrt{N_D(x)}}\dP_X(x)\\
&\quad \le
\int_X  \int_{\setdef{D}{0<N_D(x)\le l}}
\underbrace{\frac{\dP^n(D)}{\sqrt{N_D(x)}}}_{\le1}\dP_X(x)
+
\int_X  \int_{\setdef{D}{l<N_D(x)}}
\underbrace{\frac{\dP^n(D)}{\sqrt{N_D(x)}}}_{\le \sqrt{l}^{-1}}\dP_X(x)\\
&\quad \le
P^n\otimes P(\setdef{(D,x)}{N_D(x)\le l}) + \frac{1}{\sqrt{l}}.\tag{***}
\end{align*}
Insgesamt gilt also
\begin{align*}
\text{(**)} \le \frac{3}{2}P^n\otimes P(\setdef{(D,x)}{N_D(x)\le l}) +
\frac{1}{2\sqrt{l}}.
\end{align*}
Sei nun $\ep >0$ und $l> \ep^{-2}$, so ist $\frac{1}{\sqrt{l}} < \ep$ und nach
Voraussetzung (ii) gilt für hinreichend große $n$,
\begin{align*}
P^n\otimes P(\setdef{(D,x)}{N_D(x)\le l}) \le \ep
\end{align*}
und somit (***)$\le \frac{5}{2}\ep$.\qedhere
\end{proof}

\begin{prop}
\label{prop:1.3.5}
Sei $\mu$ ein endliches Borelmaß auf $\R^d$ und
\begin{align*}
f: \R^d\to \R,\qquad \mu\text{-integrierbar}.
\end{align*}
Dann gibt es für jedes $\ep > 0$ eine gleichmäßig stetige Abbildung $h:\R^d\to
\R$, die $\mu$-integrierbar ist und für die gilt
\begin{align*}
\int \abs{f-h}\dmu \le \ep.\fishhere
\end{align*}
\end{prop}
\begin{proof}
Der Beweis findet sich in \cite{bauerI}, oder ergibt sich direkt, wenn das
Lebesgue-Integral nach der Methode von Daniel Stone eingeführt wird.\qedhere
\end{proof}

$P_X$ ist als W-Maß insbesondere endlich und
$\eta=P(Y=1\mid\cdot)\in[0,1]$ ist beschränkt und damit $P_X$-integrierbar. Für
$\ep  > 0$ gibt es nun eine gleichmäßig stetige Abbildung $\eta_\ep :
\R^d\to\R$, die ebenfalls $P_X$-integrierbar ist und
\begin{align*}
\int\abs{\eta-\eta_\ep}\dP_X < \ep 
\end{align*}
erfüllt. Setzen wir also
\begin{align*}
\bar{\eta}_\ep(x) \defl
\begin{cases}
\frac{1}{P_X(A(x))}\int_{A(x)}\eta_\ep\dP_X,& P_X(A(x)) > 0,\\
0, & \text{sonst},
\end{cases}
\end{align*}
so gilt
\begin{align*}
\int \abs{\eta-\bar{\eta}}\dP_X
\le
\underbrace{\int \abs{\eta-\eta_\ep}\dP_X}_{\text{(I)}}
+\underbrace{\int \abs{\eta_\ep-\bar{\eta}_\ep}\dP_X}_{\text{(II)}}
+\underbrace{\int \abs{\bar{\eta}_\ep-\bar{\eta}}\dP_X}_{\text{(III)}}
\end{align*}

(I) ist nach Konstruktion $<\ep$. Analog folgt
\begin{align*}
\text{(III)} &= \int \abs{\bar{\eta}_\ep-\bar{\eta}}\dP_X\\
&\overset{\frac{0}{0}\defl0}{=} \int_X \frac{1}{P_X(A(x))} \int_{A(x)}
\abs{\eta_\ep(x')-\eta(x')} \dP_X(x')\dP_X(x)\\
&=
\int_X\int_X \frac{1}{P_X(A(x))} \Id_{A(x)}(x) \abs{\eta_\ep(x')-\eta(x')}
\dP_X(x)\dP_X(x')\\
&= \int_X \abs{\eta_\ep(x')-\eta(x')} \dP_X(x') \le \ep.
\end{align*}

Da $\eta_\ep$ gleichmäßig stetig, gibt es ein $\delta > 0$, so dass
\begin{align*}
\norm{x-x'}< \delta \Rightarrow \abs{\eta_\ep(x)-\eta_\ep(x')} < \ep
\end{align*}
und nach Voraussetzung (i) des Satzes gilt für große Zahlen $n$,
\begin{align*}
\sup_{j\ge 1} \diam A_{n,j} \le \delta.
\end{align*}
Außerdem erhalten wir
\begin{align*}
\text{(II)} &= 
\int_X \abs{\eta_\ep - \bar{\eta}_\ep}\dP_X\\
&=
\int_X \abs{\eta_\ep(x) - \frac{1}{P_X(A(x))}\int_{A(x)}
\eta_\ep(x')\dP_X(x')}\dP_X(x)\\
&\le
\int_X \frac{1}{P_X(A(x))}\int_{A(x)} \underbrace{\abs{\eta_\ep(x) -
\eta_\ep(x')}}_{<\ep}\dP_X(x')\dP_X(x)\le \ep. 
\end{align*}
Insgesamt haben wir somit gezeigt,
\begin{align*}
\forall \ep > 0 \exists N_0 \in\N \forall n\ge N : 
\int \abs{\eta-\bar{\eta}} \dP_X < \ep.
\end{align*}
Damit haben wir gezeigt, dass
\begin{align*}
\E_{D\sim P^n} \int \abs{\eta-\hat{\eta}_D} \dP_X \to 0,\qquad n\to \infty.
\end{align*}
Jetzt müssen wir uns noch davon überzeugen, dass dies auch für die universelle
Konsistenz genügt.

Schreibe $h_D \defl \RR_{L_\class,P}(f_D) - \RR_{L_\class,P}^* \ge 0$, so gilt
nach Lemma \ref{prop:1.3.3} und dem soeben gezeigten, dass
\begin{align*}
\E_{D\sim P^n} h_D \to 0,\qquad n\to\infty. 
\end{align*}
Ferner gilt
\begin{align*}
\E_{D\sim P^n} h_D = \E_{D\sim P^n} \abs{h_D} =
\int_{[h_D< \ep]} \abs{h_D}\dP^n(D)
+
\underbrace{\int_{[h_D\ge \ep]} \abs{h_D}\dP^n(D)}_{\ge \ep
P^n(\setdef{D}{\abs{h_D} \ge \ep})}.
\end{align*}
Das heißt
\begin{align*}
\lim\limits_{n\to\infty}
P^n\left(\setdef{D}{\RR_{L_\class,P}(f_D) - \RR_{L_\class,P}^* \ge \ep}\right) =
0,
\end{align*}
da $\E_{D\sim P^n} h_D \to 0$.

Wir wollen jetzt zeigen, dass es Lernmethoden gibt, die universell konsistent
sind.

\begin{prop}[Universelle Konsistenz der Histogrammregel]
\index{Histogrammregel!universelle Konsistenz}
\label{prop:1.3.6}
Sei $\AA_n \defl (A_{n,j})_{j\ge1}$ eine Partition von $\R^d$, so dass jedes
$A_{n,j}$ ein Hyperwürfel der Kantenlänge $h_n$ ist, d.h. wir haben zu
$x_{n,j}\in\R^d$,
\begin{align*}
A_{n,j} = x_{n,j} + [0,h_n]^d.
\end{align*}
Es gelten
\begin{propenum}
\item $\lim\limits_{n\to\infty} h_n  = 0$,
\item $\lim\limits_{n\to\infty} n\cdot h_n  = \infty$.
\end{propenum}
Dann ist die zugehörige Histogrammregel universell konsistent.\fishhere
\end{prop}
\begin{proof}
Wir weisen die Voraussetzungen von Satz \ref{prop:1.3.4} nach.\\
``1)'': $\lim\limits_{n\to\infty} \sup_{j\ge 1} \diam A_{n,j} = 0$ ist
erfüllt.\\
``2)'': Dazu führen wir $S>0$ ein. Dann gibt es höchstens $c_1+c_2
h_n^{-d}$ Hyperwürfel $A_{n,j}$ mit $A_{n,j}\cap B_S(0)\neq \varnothing$ und somit
\begin{align*}
&P^n\otimes P_X\left(\setdef{(D,x)}{N_D(x)\le l} \right)
\le P^n\otimes P_X\left(\setdef{(D,x)}{x\notin B_S(0)} \right)\\
&\le
\sum_{j:A_{n,j}\cap B_S(0)\neq \varnothing} P^n\otimes P_X
\left(\setdef{(D,x)}{N_D(x)\le l}\cap (X\times Y)^n\times A_{n,j}\right)
\end{align*}
Dann ist $P^n\otimes P_X\left(\setdef{(D,x)}{x\notin B_S(0)} \right) =
P_X(\R^d\setminus B_S(0))$ also
\begin{align*}
&\sum_{j:A_{n,j}\cap B_S(0)\neq \varnothing} P^n\otimes P_X
\left(\setdef{(D,x)}{N_D(x)\le l,\; x\in A_{n,j}}\right)\\
&\le 
\sum_{\atop{j:A_{n,j}\cap B_S(0)\neq \varnothing}{P_X(A_{n,j})\le \frac{2l}{n}}}
P^n\otimes P_X \left(\setdef{(D,x)}{N_D(x)\le l,\; x\in
A_{n,j}}\right)\tag{1}\\ &+
\sum_{\atop{j:A_{n,j}\cap B_S(0)\neq \varnothing}{P_X(A_{n,j})> \frac{2l}{n}}}
\left(\setdef{(D,x)}{N_D(x)\le l,\; x\in A_{n,j}}\right)\tag{2}
\end{align*}
 Zu (1): $\setdef{(D,x)}{N_D(x)\le l,\; x\in A_{n,j}}\subset (X\times
 Y)^n\times A_{n,j}$ und daher
 \begin{align*}
 &P^n\otimes P_X\left(\setdef{(D,x)}{N_D(x)\le l,\; x\in A_{n,j}}\right)
 \le P^n\otimes P_X((X\times Y)^n\times A_{n,j})\\
 &= P_X(A_{n,j})
 \end{align*}
Folglich ist
\begin{align*}
\text{(1)} \le
\sum_{\atop{j:A_{n,j}\cap B_S(0)\neq \varnothing}{P_X(A_{n,j})\le \frac{2l}{n}}}
P_X(A_{n,j})
\le
\frac{2l}{n}
(c_1+c_2h_n^{-d}) \to 0,
\end{align*}
nach Voraussetzung 2).

Zu (2): Setze $\mu_{D_X} \defl n^{-1} \sum_{i=1}^n \delta_{\setd{x_i}}$, wobei
\begin{align*}
\delta_{\setd{x_i}}(A) = 
\begin{cases}
1, & x_i\in A,\\
0, & \text{sonst}.
\end{cases}
\end{align*}
$\mu_{D_X}$ heißt \emph{empirisches Maß}
\index{empirisches Maß}
bezüglich des Datensatzes
$D_X=(x_1,\ldots,x_n)$.

Für $x\in A_{n,j}$ gilt
\begin{align*}
N_D(x) = \sum_{i=1}^n \Id_{A(x)}(x_i) = n\cdot\mu_{D_X}(A(x)).
\end{align*}
Daher folgt
\begin{align*}
&P^n\otimes P_X\left(\setdef{(D,x)}{N_D(x)\le l,\; x\in A_{n,j}} \right)\\
&\quad \le P_X(A_{n,j})P^n\left(\setdef{D}{n\cdot \mu_{D_X}(A_{n,j})\le l}
\right)
\end{align*}
Es folgt wegen $P_X(A_{n,j}) > \frac{2l}{n}$
\begin{align*}
&P^n\left(\setdef{D}{\mu_{D_X}(A_{n,j})\le \frac{l}{n}} \right)\\
&\quad =
P^n\left(\setdef{D}{\mu_{D_X}(A_{n,j}) - P_X(A_{n,j})\le \frac{l}{n}-
P_X(A_{n,j})} \right)\\
&\quad \le
P^n\left(\setdef{D}{\mu_{D_X}(A_{n,j}) - P_X(A_{n,j})\le -\frac{1}{2}
P_X(A_{n,j})} \right).
\end{align*}
Zur Fortsetzung des Beweises benötigen wir folgendes Lemma.\qedhere
\end{proof}

\begin{lem}
\label{lem:1.3.7}
Sei $(\Omega,\AA,P)$ ein W-Raum und $h:\Omega\to\R$ messbar mit
$\int h^2 \dP < \infty$. Dann gilt für alle $n\ge 1$ und $t > 0$,
\begin{align*}
P^n\left(\setdef{\omega\in\Omega}{\frac{1}{n}\sum_{i=1}^n h(\omega_i) - \E_P h
\ge t } \right) \le
\frac{\E_P h^2}{t^2\cdot n}.\fishhere
\end{align*}
\end{lem}
\begin{proof}
Die Tschebyscheff-Ungleichung besagt für quadratintegrierbares $f$,
\begin{align*}
P\left[\abs{f}\ge t\right] \le \frac{\E_P f^2}{t^2}.
\end{align*}
Wenden wir dies auf $f=\frac{1}{n}\sum_{i=1}^n h\circ\pi_i - \E_P h$, wobei
\begin{align*}
\pi_i : \Omega^n\to \Omega
\end{align*}
die $i$-te Projektion.
\begin{align*}
&P^n\left(\setdef{\omega\in\Omega^n}{\frac{1}{n}\sum_{i=1}^n h(\omega_i) - \E_P
h \ge t } \right)\\ &\quad \le
\frac{\E_{P^n}\left(\frac{1}{n} \sum_{i=1}^n h\circ\pi_i - \E_P
h\right)^2}{t^2}
\end{align*}
und mit $\eta_i \defl h\circ\pi_i - \E_P h$ gilt
\begin{align*}
&\E_{P^n}\left(\frac{1}{n}\sum_{i=1}^n (h\circ\pi_i -\E_P h) \right)^2
= \frac{1}{n^2} \E_{P^n} \left(\sum_{i=1}^n \eta_i\right)^2\\
&\qquad= \frac{1}{n^2}
\sum_{i=1}^n \E_{P^n} h^2\circ \pi_i + \frac{1}{n^2}\sum_{i\neq j} 
\underbrace{\E_{P^n} \eta_i\eta_j}_{=0}\\
&\qquad=
 \frac{1}{n^2}
\sum_{i=1}^n \E_{P^n} h^2\circ \pi_i
=
 \frac{1}{n^2}
\sum_{i=1}^n \E_{P} h^2 = \frac{1}{n} \E_{P} h^2.\qedhere
\end{align*}
\end{proof}

\begin{proof}[Fortsetzung des Beweises von Satz \ref{prop:1.3.6}.]
Setze nun in Lemma \ref{prop:1.3.5}, $h=\Id_{A_{n,j}}$ und
$t=\frac{1}{2}P_X(A_{n,j})$, so gilt
\begin{align*}
&P^n\left(\setdef{D}{\mu_{D_X}}(A_{n,j}) -P(A_{n,j}) \le
-\frac{1}{2}P_X(A_{n,j}) \right)
\le \frac{\E_P \Id^2_{A_{n,j}}}{4P_X^2(A_{n,j})n}\\
&\qquad= \frac{1}{4n\cdot P_X(A_{n,j})}.
\end{align*}
Somit gilt schließlich
\begin{align*}
\text{(2)} &= \sum_{\atop{j:A_{n,j}\cap B_S(0)\neq \varnothing}{P_X(A_{n,j})>
\frac{2l}{n}}} \left(\setdef{(D,x)}{N_D(x)\le l,\; x\in A_{n,j}}\right)\\
&\le
\sum_{\atop{j:A_{n,j}\cap B_S(0)\neq \varnothing}{P_X(A_{n,j})> \frac{2l}{n}}}
\frac{1}{4n}
\le \frac{c_1+c_2h_n^{-d}}{4n} \overset{n\to\infty}{\longrightarrow} 0.
\end{align*}

Zum Abschluss des Beweises, müssen wir noch $S$ passend wählen. Sei also $\ep >
0$, dann existiert ein $\delta > 0$, so dass $P_X(\R^d\setminus B_S(0)) \le
\ep$ und für hinreichend große $n$ gilt ferner
\begin{align*}
&\frac{2l}{n}\left(c_1 + c_2h_n^{-d}\right) \le \ep,\\
&\left(c_1 + c_2h_n^{-d}\right)\frac{1}{4n} \le \ep,
\end{align*}
und letztlich
\begin{align*}
P^n\otimes P_X\left(\setdef{(D,x)}{N_D(x)\le \ep} \right)\le 3\ep.\qedhere
\end{align*}
\end{proof}

\begin{bem*}[Schlussbemerkungen]
\begin{bemenum}
\item Es gibt universell konsistente Klassifizierungsmethoden. Stone hat 1977
gezeigt, dass das Nearest-Neighbour-Verfahren universell konsistent ist, falls
$K_n\to \infty$ und $\frac{K_n}{n}\to 0$.
\item Gibt es ein $S>0$, so dass $P_X(\R^d\setminus B_S(0)) = 0$, dann ist die
Konvergenz von
\begin{align*}
P^n\otimes P_X\left(\setdef{(D,x)}{N_D(x)\le l} \right)\tag{*}
\end{align*}
für $n\to\infty$ von $P$ unabhängig. Der wesentliche Grund dafür ist, dass
Lemma \ref{prop:1.3.7} von $P$ unabhängig ist.
\item Die Konvergenzgeschwindigkeit von
\begin{align*}
\int_{(X\times Y)^n}\int_X \abs{\bar{\eta}-\hat{\eta}_D}\dP_X\dP^n
\overset{n\to\infty}{\to} 0
\end{align*}
hängt lediglich von (*) ab und ist daher von $P$ unabhängig.
\item In Satz \ref{prop:1.3.4} benötigen wir eine gleichmäßig stetige
Approximation von $\eta$ und $\bar{\eta}$ und können daher keine Aussage zur
Konvergenzrate von
\begin{align*}
\RR_{L_\class,P}(f_D) - \RR_{L_\class,P}^*
\le \int \abs{\eta-\bar{\eta}}\dP_X \to 0
\end{align*}
machen.
\item Ist $\eta$ lipschitz stetig mit Konstante $\le 1$, so können wir eine
Konvergenzrate von
\begin{align*}
\RR_{L_\class,P}(f_D) - \RR_{L_\class,P}^*
\le \int \abs{\eta-\bar{\eta}}\dP_X \to 0
\end{align*}
berechnen.
Diese ist unabhängig von $P$, in dem Sinn, dass man nur die $P$
betrachtet für die $\eta$ lipschitz ist.\maphere 
\end{bemenum}
\end{bem*}

\section{Lernraten}
\index{Lernraten}

Ziel dieses Abschnitts ist es, die Konvergenzraten von
\begin{align*}
\RR_{L_\class,P}(f_D) - \RR_{L_\class,P}^*\to 0
\end{align*}
zu untersuchen.

\begin{defn}
\label{defn:1.4.1}
Sei $L:X\times Y\times \R\to[0,\infty)$ eine Verlustfunktion, $P$ ein W-Maß auf
$X\times Y$, $\LL$ eine Lernmethode und $(a_n)$ eine Folge echt positiver
Zahlen mit $a_n\downarrow 0$.

Wir sagen, dass $\LL$ \emph{im Mittel mit Rate $(a_n)$ lernt},
\index{Lernrate!im Mittel}
falls es ein $c>0$ gibt, so dass
\begin{align*}
\E_{D\sim P^n} \left(\RR_{L,P}(f_D)-\RR_{L,P}^*\right)
\le ca_n,\qquad n\ge 1.
\end{align*}
Ferner sagen wir, dass $\LL$ \emph{in Verteilung mit Rate $(a_n)$ lernt},
\index{Lernrate!in Verteilung}
falls es eine Familie $(c_\tau)_\tau\subset [1,\infty)$ und $\tau\in [0,1]$
gibt, so dass gilt
\begin{align*}
P^n\left(\setdef{D}{\RR_{L,P}(f_D) - \RR_{L,P}^* \le c_\tau a_n} \right) \ge
1-\tau,\qquad n\ge 1,\; \tau\in[0,1].\fishhere
\end{align*}
\end{defn}

Bevor wir klären, ob es überhaupt Lernraten gibt, die für alle $P$ gelten,
wollen wir erstmal beide Begriffe vergleichen.

\begin{bem*}[Bemerkungen.]
\begin{bemenum}
\item Ist $\LL$ konsistent, so kann man zeigen, dass $\LL$ im Mittel mit einer
von $P$ abhängigen Rate im Mittel lernt.
\item Im Beweis von Satz \ref{prop:1.3.6} haben wir benutzt,
\begin{align*}
\E_{D\sim P^n}\left(\RR_{L,P}(f_D) - \RR_{L,P}^* \right)
\ge \ep P^n\left(\setdef{D}{\RR_{L,P}(f_D)- \RR_{L,P}^* > \ep}\right)
\end{align*}
Ist nun $\E_{D\sim P^n}\left(\RR_{L,P}(f_D) - \RR_{L,P}^*\right) \le c a_n$,
dann ist
\begin{align*}
P^n\left(\setdef{D}{\RR_{L,P}(f_D)- \RR_{L,P}^* > \ep}\right)
\ge \frac{c a_n}{\ep}.
\end{align*}
D.h. für $\tau=\frac{c a_n}{\ep} \Leftrightarrow \ep = \frac{c}{\tau}a_n$ und
$c_\tau \defl c\tau^{-1}$ ist
\begin{align*}
P^n\left(\setdef{D}{\RR_{L,P}(f_D)- \RR_{L,P}^* \le c_\tau a_n}\right)
\le 1-\tau.
\end{align*}
Lernraten im Mittel implizieren somit Lernraten in Verteilung.
\item Später werden wir häufig Lernraten in Verteilung beweisen, die
dann aufgrund der Struktur der $c_\tau$ auch im Mittel gelten.\maphere
\end{bemenum}
\end{bem*}

\begin{prop*}[Offene Frage]
Gibt es Verfahren $\LL$, die mit Rate $(a_n)$ für alle $P$ lernen?\fishhere
\end{prop*}

Wir werden im Folgenden sehen, dass dies ``normalerweise'' nicht so ist.

\begin{defn}
\label{defn:1.4.2}
Sei $(X,\AA,\mu)$ ein endlicher Maßraum. Ein $A\in\AA$ heißt
\emph{Atom}\index{Atom}, wenn
\begin{align*}
\forall B\subset A,\quad B\in\AA
\text{ gilt entweder }\mu(B)=0\text{ oder }\mu(A\setminus B) = 0.
\end{align*}
$(X,\AA,\mu)$ heißt \emph{atomfrei}, falls kein $A\in\AA$ ein Atom ist.\fishhere
\end{defn}

\begin{bsp*}
\begin{bspenum}
\item Versehen wir die natürlichen Zahlen versehen mit dem Zählmaß $\nu$ mit
$\nu(\setd{n})=1$, so ist jede einelementige Menge $\setd{n}$ ein Atom.
\item Das Intervall $[0,1]$ versehen mit dem Lebesgue-Maß ist atomfrei.\bsphere 
\end{bspenum}
\end{bsp*}

\begin{prop}[Satz von Lyapunov]
\index{Satz!von Lyapunov}
\label{prop:1.4.3}
Ist $(X,\AA,\mu)$ ein endlicher atomfreier Maßraum, so ist
\begin{align*}
\setdef{\mu(A)}{A\in\AA} = [0,\mu(X)].\fishhere
\end{align*}
\end{prop}
\begin{proof}
Ein Beweis findet sich in \cite{Werner07}.\qedhere
\end{proof}

\begin{prop}[No-free-lunch Theorem (Devroye 1982)]
\label{prop:1.4.4}
\index{No-free-lunch Theorem}
Sei $(a_n)\subset [0,1/32]$ eine fallende Nullfolge, $(X,\AA,\mu)$ ein
atomfreier W-Raum und $Y\defl\setd{-1,1}$. Dann gibt es zu jeder Lernmethode $\LL$
ein W-Maß $P$ auf $X\times Y$ mit
\begin{propenum}
\item $P_X = \mu$.
\item $\RR_{L_\class,P}^* = 0$.
\item $\E_{D\sim P^n} \RR_{L_\class,P}(f_D) - \RR_{L_\class,P}^* \ge a_n$ für
$n\ge 1$.\fishhere
\end{propenum}
\end{prop}

\begin{bem*}[Bemerkungen.]
\begin{bemenum}
\item Das NFL-Theorem besagt, dass es im Allgemeinen keine universellen
mittleren Raten gibt. Dies gilt auch für Raten in Verteilung.
\item Das NFL-Theorem gilt für \textit{praktisch jede} Verlustfunktion.
\item Für $X=\N$ gilt das NFL-Theorem, falls $\mu=P_X$ nicht vorher festgelegt
wird.
\item Für \textit{endliche} $X$ ist das NFL-Theorem falsch.\maphere
\end{bemenum}
\end{bem*}

\begin{lem}
\label{prop:1.4.5}
Sei $(a_n) \subset (0,1/16]$ mit $a_n\downarrow 0$, dann existiert eine Folge
$(p_n)\subset (0,1)$ mit $p_1\ge p_2 \ge \ldots$ und $\sum_{n=1}^\infty p_n =
1$ und
\begin{align*}
\sum_{i=n+1}^\infty p_i \ge \max\setd{8a_n,32n p_{n+1}}.\fishhere
\end{align*}
\end{lem}
\begin{proof}
Da $p_n\ge p_{n+1}$ gezeigt wird, genügt es zu zeigen, dass
\begin{align*}
\sum_{i=n+1}^\infty p_i \ge \max\setd{8a_n,32n p_{n}}.\fishhere
\end{align*}
Für $l\le m$ definiere
\begin{align*}
H(m,l) \defl \sum_{i=l}^{m-1} \frac{1}{i}.
\end{align*}
Sei nun $n_1\defl 1$, so gilt
\begin{align*}
8a_{n_1} = 8a_1 \le 8\frac{1}{16}\le 2^{-1}. 
\end{align*}
Ferner gibt es ein $n_2>n_1$ mit $H(n_2,n_1) \ge 32$, da $\sum_{i\ge 1} i^{-1}
= \infty$. Außerdem ist $H(\cdot,n_1)$ wachsend und $(a_n)$ fallend nach
Voraussetzung, also können wir $8a_{n_2}\le 2^{-2}$ für $n_2$ hinreichend groß
garantieren. Sukzessive finden wir $n_k$ für $k\ge 3$, so dass
\begin{align*}
n_k > n_{k-1},\qquad H(n_k,n_{k-1}) \ge H(n_{k-1},n_{k-2}),\qquad
8a_{n_k} \le 2^{-k}.
\end{align*}
Definiere nun
\begin{align*}
c_k \defl \frac{32}{2^k H(n_{k+1},n_k)},\qquad k\ge 1,
\end{align*}
so ist $c_k$ fallend und
\begin{align*}
\frac{1}{32}\sum_{k\ge 1} c_k H(n_{k+1},n_k) = \sum_{k\ge 1}\frac{1}{2^k} = 1.
\end{align*}
Für $n\in [n_k,n_{k+1})$ definiere $p_n \defl \frac{c_k}{32 n}$, dann fällt $p_n$
monoton und
\begin{align*}
\sum_{n\ge 1} p_n = \sum_{k\ge 1} \sum_{i=n_k}^{n_{k+1}-1} \frac{c_k}{32 i} = 
\sum_{k\ge 1} \frac{c_k}{32}
\underbrace{\sum_{i=n_k}^{n_{k+1}-1} \frac{1}{i}}_{H(n_{k+1},n_k)} = 1.
\end{align*}
Ferner gilt für $n\in[n_k,n_{k+1})$
\begin{align*}
\sum_{i\ge n+1} p_i &\ge \sum_{i\ge n_{k+1}} p_i
= \sum_{i\ge k+1}\sum_{j=n_i}^{n_{i+1}-1} \frac{c_i}{32 j}
= \sum_{i\ge k+1} \frac{c_i}{32} H(n_{i+1},n_i)\\
&= \sum_{i\ge k+1} 2^{-i} = 2^{-k}.
\end{align*}
Da außerdem $H(n_{k+1},n_k) \ge H(n_2,n_1) \ge 32$ folgt
\begin{align*}
\frac{1}{2^k} \ge \frac{32}{2^k H(n_{k+1},n_k)} = c_k = 32 np_n.
\end{align*}
Zudem ist
\begin{align*}
\frac{1}{2^k} \ge 8a_{n_k} \ge 8a_n.\qedhere
\end{align*}
\end{proof}

\begin{proof}[Beweis des No-free-lunch Theorems.]
Ohne Einschränkung sei $f_D \in \setd{-1,1}$, ansonsten betrachte $\sign f_D$.
Fixiere eine Folge $(p_i)\downarrow$ gemäß dem vorangegangen Lemma und zerlege
$X$ in Partitionen $(A_j)_{j\ge 1}$ mit $\mu(A_j)=p_j$ (Anwendung des Satzes
von Lyapunov).
Sei nun $\overline{\nu}$ ein W-Maß auf $\setd{0,1}$ mit
$\overline{\nu}(\setd{0}) = 1/2$, so ist
\begin{align*}
\nu \defl \bigotimes_{i=1}^\infty \overline{\nu},
\end{align*}
ein W-Maß auf $\Omega\defl \setd{0,1}^\infty$. Für $\omega = (\omega_j)\in\Omega$
schreibe
\begin{align*}
\eta_\omega(x) \defl \sum_{j=1}^\infty \omega_j \Id_{A_j}(x),\quad \in\setd{0,1}.
\end{align*}
Nun sei $P_\omega$ das W-Maß auf $X\times Y$, das durch
\begin{align*}
&P_X = \mu,\\
&P(Y=1\mid x) = \eta_\omega(x),\quad \Pfs
\end{align*}
charakterisiert wird, so gilt
\begin{align*}
\RR_{L,P_\omega}^* = \int_X
\underbrace{\min\setd{\eta_\omega,1-\eta_\omega}}_{=0} \dP_X = 0.
\end{align*}
Wir zeigen nun, dass
\begin{align*}
\int_\Omega \inf_{n\ge 1} \frac{1}{a_n} \int_{(X\times Y)^n}
\RR_{L,P_\omega}(f_D) \dP_\omega^n(D) \dnu(\omega) \ge \frac{1}{2}\tag{*},
\end{align*}
denn dann existiert ein $\omega$, so dass auch
\begin{align*}
\inf_{n\ge 1} \frac{1}{a_n} \int_{(X\times Y)^n}
\RR_{L,P_\omega}(f_D) \dP_\omega^n(D) \ge \frac{1}{2},
\end{align*}
und für die Folge $(2a_n)_{n\ge 1}$ folgt die Behauptung.

Idee: Im besten Fall genügt ein Punkt, um alle Informationen auf dem
zugehörigen $A_j$ zu liefern, denn $\eta_\omega$ ist dort konstant. Haben wir
beispielsweise 4 Datenpunkte mit je einem in $A_1,\ldots,A_4$, dann können wir
auf diesen ``richtig'' entscheiden, jedoch haben wir keine Möglichkeit über
$A_4$ hinaus noch ``richtig'' zu entscheiden, denn egal wie entschieden wird,
existieren immer $\omega$, für die die Entscheidung schlecht ist. Wir können
dann also nur noch raten\ldots
%TODO: Bild zur Erklärung.

Der folgende Beweis ist jedoch sehr technisch und daher ist es schwer, die
anschauliche Idee dort wieder zu finden.

Sei $D=(X\times Y)^\infty$ und schreibe $D_n \defl ((x_1,y_1),\ldots,(x_n,y_n))$,
so gilt
\begin{align*}
(*) &= \int_\Omega \inf_{n\ge 1} \frac{1}{a_n} \int_{(X\times Y)^n}
\RR_{L,P_\omega}(f_D) \dP_\omega^n(D) \dnu(\omega)\\
&\ge
\int_\Omega \int_{(X\times Y)^n} \inf_{n\ge 1} \frac{1}{a_n} 
\RR_{L,P_\omega}(f_D) \dP_\omega^n(D) \dnu(\omega).
\end{align*}
Ist $\RR_{L,P_\omega}(f_{D}) \ge a_n$, so ist auch
$\frac{1}{a_n}\RR_{L,P_\omega}(f_D)\ge 1$ und daher
\begin{align*}
\ldots \ge
\int_\Omega \int_{(X\times Y)^n} \Id_{\bigcap_{n=1}^\infty
\setd{\RR_{L,P_\omega}(f_D) \ge a_n}}
\dP_\omega^n(D) \dnu(\omega).
\end{align*}
Setzen wir $B_n\defl\setd{\RR_{L,P_\omega}(f_D) \ge a_n}$, so ist $\bigcap_{n\ge
1} B_n = X\setminus \left(\bigcup_{n\ge 1} X\setminus B_n\right)$ und für das
Maß $Q=P_\omega^\infty$ gilt folglich
\begin{align*}
Q\left(\bigcap_{n\ge 1} B_n\right) &= 1-Q\left(\bigcup_{n\ge 1} X\setminus
B_n\right) \ge 1-\sum_{n\ge 1} Q(X\setminus B_n)\\
&\ge 
1- \underbrace{\sum_{n\ge 1} \int_\Omega \int_{(X\times Y)^\infty}
\Id_{\setd{\RR_{L,P}(f_D)\le a_n}} \dP_\omega^n(D)\dnu(\omega)}_{\text{(**)}}.
\end{align*}
Im Folgenden wollen wir (**) nach oben abschätzen. Für $n\ge 1$ definiere
\begin{align*}
\overline{f}_{D_n} \defl
\begin{cases}
1, & \mu\left(\setd{f_D=1} \cap A_j \right) \ge \mu\left(\setd{f_D = -1}\cap A_j
\right),\\
0, & \text{sonst},
\end{cases}
\end{align*}
und schreibe ferner
\begin{align*}
E_{\omega,j}(f_{D_n}) \defl
A_j \cap \setd{f_D \neq 2\eta_\omega -1}
\end{align*}
für die Menge der Fehlentscheidungen von $f_{D_n}$ auf $A_j$ für das Maß
$P_\omega$. Auf $A_j$ ist $\eta_\omega(x)\equiv \omega_j$ und daher gilt
\begin{align*}
\overline{f}_{D_n}(j) \neq 2\omega -1 \Rightarrow
\mu(E_{\omega,j}(f_{D_n})) \ge \frac{p_j}{2}.
\end{align*}
Um dies einzusehen sei z.B. $\omega_j = 1$, dann ist
$\overline{f}_{D_n}(j)=-1$ und daher nach Definition von $\overline{f}_{D_n}$
\begin{align*}
\mu\left(\setd{f_{D_n}=1}\cap A_j\right) = \mu\left(\setd{f_{D_n} = -1}\cap
A_j\right)
\end{align*}
also
\begin{align*}
p_j &= \mu(A_j) = \mu\left(\setd{f_{D_n}=1}\cap A_j \right)
+ \mu\left(\setd{f_{D_n}=-1}\cap A_j \right)\\
&\le 2\mu\left(\underbrace{\setd{f_{D_n}=-1} \cap
A_j}_{E_{\omega,j}(f_{D_n})}\right)
\end{align*}
Damit folgt
$
\Id_{\setd{\mu(E_{\omega,j}(f_{D_n})) \ge \frac{p_j}{2}}} \ge
\Id_{\setd{\overline{f}_{D_n} \neq 2\omega_j -1}}
$
und wegen
\begin{align*}
\mu\left(E_{\omega,j}(f_{D_n}) \right) \ge
\frac{p_j}{2}\Id_{\setd{\mu(E_{\omega,j}(f_{D_n}))\ge \frac{p_j}{2}}}
\end{align*}
ist
\begin{align*}
\RR_{L,P_\omega}(f_{D}) &= \RR_{L,P_\omega}(f_D) -
\underbrace{\RR_{L,P_\omega}^*}_{=0}\\
&= \int_X \abs{2\eta_\omega -1}\Id_{(-\infty,0]} \left((2\eta_\omega - 1)\sign
f_{D_n} \right)\dmu\\
&=\mu\left(\setd{f_{D_n} \neq (2\eta_\omega -1)}\right)
= \sum_{j=1}^\infty \mu\left(E_{\omega,j}(f_{D_n}) \right)\\
&\ge \frac{1}{2}
\sum_{j=1}^\infty p_j \Id_{\setd{f_{D_n} \neq 2\omega_j -1}}
\ge
\frac{1}{2} \sum_{\atop{j : j\le n}{x_i\neq A_j}} p_j
\Id_{\setd{\overline{f}_{D_n}(j) \neq 2\omega_j -1}}
\end{align*}
Daraus folgt
\begin{align*}
&\int_\Omega\int_{(X\times Y)^\infty}
\Id_{\setd{\RR_{L,P_\omega}(f_{D_n})\le a_n}} \dP_\omega^\infty(D)
\dnu(\omega)\\ &\le
\int_\Omega \int_{(X\times Y)^\infty}
\Id_{\setd{\sum_{\atop{j: j\le n}{x_i\in A_j}} p_j
\Id_{\setd{\overline{f}_{D_n} \neq 2\omega_j-1}}\le 2a_n}}\dP_\omega^\infty(D)
\dnu(\omega).
\end{align*}
Für alles Weitere fixieren wir $\JJ\defl\setdef{j}{j\le n,\; x_i\in A_j}$.

Zu jedem $x\in X$ existiert genau ein $j_x\ge 1$ mit $x\in A_j$, wir schreiben
dafür $j(x) \defl j_x$. Nach Konstruktion sind die $x_i$ von $\omega$ unabhängig
und es gilt
\begin{align*}
y_i = 2\omega_j(x_i)-1.
\end{align*}
Die Labels $y_i$ lassen sich somit explizit berechnen
und daher hängt $\overline{f}_D$ nur von $D_X = (x_1,\ldots,x_n)$ und
$\omega_{D_X}=(\omega_j(x_1),\ldots,\omega_j(x_n))$ ab. Wir schreiben daher
$\overline{f}_{D_X,\omega}$ für $\overline{f}_D$ und erhalten,
\begin{align*}
&\int_\Omega \int_{(X\times Y)^\infty} \Id_{\setd{\sum_{j\in\JJ} p_j
\Id_{\setd{\overline{f}_D(j)\neq 2\omega_j-1}}\le
2a_n}}\dP_\omega^n(D)\dnu(\omega)\\ &=
\int_{X^n}\int_\Omega \Id_{\setd{\sum_{j\in\JJ} p_j
\Id_{\setd{\overline{f}_{D_X,\omega}(j)\neq 2\omega_j -1}}\le
2a_n}}\dnu(\omega)\dmu^n(D_X).
\end{align*}

Sei weiterhin $\Omega_{D_X}$ das Kreuzprodukt, das durch die Koordinaten
$j(x_1),\ldots,j(x_n)$ beschrieben wird, und $\Omega_{\not
D_X}$ stehe für die ``übrigen'' Koordinaten. So ist $\Omega =
\Omega_{D_X}\times \Omega_{\not D_X}$, in selber Weise definieren wir
$\nu_{D_X}$ und $\nu_{\not D_X}$. $\overline{f}_{D_X,\omega}$ ändert sich nur
in den Koordinaten von $\omega$, die zu $\Omega_{D_X}$ gehören, also ist
\begin{align*}
&\int_\Omega \Id_{\setd{\sum_{j\in\JJ} p_j
\Id_{\setd{\overline{f}_{D_X,\omega}(j)\neq 2\omega_j -1}}\le
2a_n}}\dnu(\omega)\\
&=\int_{\Omega_{D_X}} \int_{\Omega_{\not D_X}} \Id_{\setd{\sum_{j\in\JJ} p_j
\Id_{\setd{\overline{f}_{D_X,\omega}(j)\neq 2\omega_j -1}}\le
2a_n}}\dnu_{\not D_X}(\omega_{\not D_X})\dnu_{D_X}(\omega_{D_X}).
\end{align*}
Außer dem ist $\overline{f}_{D_X,\omega}(j)\equiv 1$ oder $-1$ für alle
$\omega_{\not D_X}$ bei festem $D_X$ und $\omega_{D_X}$, wobei $w_j=0$ oder $1$
jeweils mit Wahrscheinlichkeit $\frac{1}{2}$, also
\begin{align*}
\ldots &=
\int_{\Omega_{D_X}} \int_{\Omega_{\not D_X}} \Id_{\setd{\sum_{j\in\JJ} p_j
\underbrace{\Id_{\setd{\omega_j=1}}}_{=\omega_j}\le
2a_n}}\dnu_{\not D_X}(\omega_{\not D_X})\dnu_{D_X}(\omega_{D_X})\\
&\ge
\int_{\Omega} \Id_{\setd{\sum_{j=n+1}^\infty p_j
\omega_j \le 2a_n}}\dnu(\omega).\tag{**}
\end{align*}
Da $p_i$ monoton fällt und (**) unabhängig von $D_X$, ist
\begin{align*}
\text{(**)} &=\nu\left(\setdef{\omega}{-\sum_{j=n+1}^\infty p_j\omega_j > -2a_n}
\right) \\ &= \nu\left(\setdef{\omega}{\exp\left(2sa_n-s\sum_{j=n+1}^\infty
p_j\omega_j\right) > 1} \right) 
\end{align*}
für ein $s>0$, das wir später wählen. Anwendung der Markov-Ungleichung ergibt,
\begin{align*}
\ldots &\le \E_{\omega\sim \nu} \exp\left(2sa_n-s\sum_{j=n+1}^\infty
p_j\omega_j\right)
= \exp(2sa_n)\prod_{j=n+1}^\infty \E_{\omega\sim \nu} \exp (-sp_j \omega_j)\\
&= \exp(2sa_n)\prod_{j=n+1}^\infty \left(\frac{1}{2}+\frac{1}{2}\exp(-sp_j)
\right)
\end{align*}
Verwenden wir nun, dass $e^{-t} \le 1 - t + \frac{t^2}{2}$ für $t\ge 0$, so ist
\begin{align*}
\ldots \le
\exp(2sa_n)\prod_{j=n+1}^\infty
\frac{1}{2}\left(2 - sp_j + \frac{s^2p_j^2}{2}\right) 
\end{align*}
und, da $1-t \le e^{-t}$ ist
\begin{align*}
\ldots &\le
\exp(2sa_n)\prod_{j=n+1}^\infty
\left(\exp\left(\frac{-sp_j}{2} + \frac{s^2p_j^2}{4}\right)\right)\\
&=
\exp\left(2sa_n-s\sum_{j=n+1}^\infty p_j +
\frac{s^2}{4}\sum_{j=n+1}^\infty p_j^2\right)\\
&\le
\exp\left(2sa_n-s\sum_{j=n+1}^\infty p_j +
\frac{s^2p_{n+1}}{4}\sum_{j=n+1}^\infty p_j\right)\\
&=\exp\left(2sa_n+\left(\frac{s^2p_{n+1}}{4}-s\right)\sum_{j=n+1}^\infty p_j\right)
\end{align*}
Setzen wir nun $A\defl\sum_{j=n+1}^\infty p_j$ und
\begin{align*}
s = \frac{\sum_{j=n+1}^\infty p_j - 4a_n}{p_{n+1}\sum_{j=n+1}^\infty p_j} =
\frac{A-4a_n}{Ap_{n+1}},
\end{align*}
so ist $s>0$ nach Konstruktion der $p_j$. Eine längere aber elementare Rechnung
zeigt,
\begin{align*}
\exp\left(2sa_n+\left(\frac{s^2p_{n+1}}{4}-s\right)\sum_{j=n+1}^\infty p_j\right)
=
\exp\left(-\frac{1}{4}\frac{(A-4a_n)^2}{Ap_{n+1}}\right)
\end{align*}
Die Funktion $x\mapsto (A-4x)^2$ hat ein globales Minimum bei $x=\frac{A}{4}$
und ist auf $(-\infty, \frac{A}{4}]$ monoton fallend. Folglich ist
$(A-4x)^2\ge \left(A-\frac{A}{2}\right)^2 = \frac{A^2}{4}$ für 
$x\le\frac{A}{8}$ und daher
\begin{align*}
\ldots \le
\exp\left(-\frac{A}{16 p_{n+1}}\right) \le \exp(-2n),
\end{align*}
denn $A\ge 32 p_{n+1} n$.

Schließlich folgt
\begin{align*}
&\int_\Omega \inf_{n\ge 1} \frac{1}{a_n}\int_{(X\times Y)^n}
\RR_{L,P}(f_D)\dP_\omega^n(D)\dnu(\omega)\\
&\ge 1- \sum_{n=1}^\infty \int_\Omega \int_{(X\times Y)^\infty}
\Id_{\setd{\RR_{L,P}(f_{D_n}) \le a_n}}\dP_\omega^\infty(D) \dnu(\omega)\\
&\ge 1-\sum_{n=1}^\infty e^{-2n} = \frac{e^2-2}{e^2-1}\ge \frac{1}{2}.\qedhere
\end{align*}
\end{proof}

\begin{bem*}[Abschlussbemerkungen.]
\begin{bemenum}
\item Das NFL gilt auch in den folgenden Situationen:
\begin{itemize}
  \item $X=\R^d$ und $\eta\in C^\infty([0,1))$.
  \item $X=\R^2$ und $\eta$ ist \emph{unimodal}, d.h. es gibt ein $x_0\in X$
  mit $\lambda\mapsto \eta(\lambda x_0)$ fallend für wachsendes $\lambda > 0$.
  \item $\eta\in\setd{0,1}$, $X\subset\R^2$ und $\setd{\eta = 1}$ ist kompakt,
   konvex und $0\in \setd{\eta=1}$.
\end{itemize}
\item Es gibt keine Super-Klassifikationsmethoden. Ist $\LL$ eine
Klassifikationsmethode, so gibt es eine universell konsistente
Klassifikationsmethode $\LL'$ und ein W-Maß $P$ auf $X\times Y$ mit
\begin{align*}
\E_{D\sim P^n} \RR_{L_\class,P}(f_D) > \E_{D\sim
P^n}\RR_{L_\class,P}(f_D'),\qquad n\ge 1.
\end{align*}
Dies erscheint auf den ersten Blick natürlich negativ, es hat jedoch auch
positive Auswirkungen, denn dadurch wird das Feld stetig neu belebt\ldots
\item Für jede Methode $\LL$, die $\RR_{L_\class,P}^*$ abschätzt und jedes
$n\ge 1$, $\ep > 0$ gibt es ein $P$ mit
\begin{align*}
\E_{D\sim P^n} \abs{\RR_{L_\class,P}(f_D) - \RR_{L_\class,P}^*} \ge
\frac{1}{4}-\ep.\maphere
\end{align*}
\end{bemenum}
\end{bem*}

Wir wollen dieses Kapitel mit einer offenen Frage beenden. Dazu die folgende

\begin{defn*}
Eine Klassifikationsmethode $\LL$ heißt \emph{smart}\index{smart}, wenn
$\E_{D\sim P^n} \RR_{L_\class,P}(f_D)$
monoton fallend für $n\to\infty$ und alle $P$.\fishhere
\end{defn*}

\begin{prop*}[Offene Frage]
Gibt es eine universell konsistente und smarte Klassifikationsmethode?\fishhere
\end{prop*}

Bis jetzt ist noch keine solche Lernmethode bekannt. Dabei ist die universelle
Konsistenz entscheidend. Lässt man diese Voraussetzung fallen, so gibt es
natürlich zahllose smarte Lernmethoden. 


\chapter{Verlustfunktionen}
\newcommand{\hinge}{\mathrm{hinge}}

Sei $L$ eine Verlustfunktion, dann ist
\begin{align*}
\RR_{L,D}(f) \defl \frac{1}{n}\sum_{i=1}^n L(x_i,y_i,f(x_i))
\end{align*} 
sinnvoll, wenn man $D$ mit dem empirischen Maß
\begin{align*}
\mu=n^{-1} \sum_{i=1}^n \delta_{\setd{(x_i,y_i)}}
\end{align*}
identifiziert. Sei weiterhin $\FF$ eine Menge von Funktionen $f: X\to \R$, dann
suchen wir $(f,D)\in \FF\times (X\times Y)^n$, so dass das Infimum
\begin{align*}
\inf_{f\in\FF} \RR_{L,D}(f)+\gamma(f)
\end{align*}
für einen \emph{Strafterm} $\gamma$ angenommen wird.

Um diese Frage zu klären, wollen wir nach Eigenschaften der Verlustfunktion
suchen, die sich auf das Risiko übertragen.

\section{Eigenschaften von Verlustfunktionen}

Wir betrachten im Folgenden Verlustfunktionen
\begin{align*}
L : X\times Y \times \R \to [0,\infty).
\end{align*}
Während $X$ und $Y$ relativ ``strukturlos'' sind, hat $\R$ sehr viel
``Struktur''. Wir suchen nun nach Eigenschaften von $\R$ wie
Topologie, Metrik, Konvexität, \ldots die sich auf $L$ übertragen.

\begin{defn}
\label{defn:2.1.1}
Eine Verlustfunktion $L:X\times Y\times \R\to [0,\infty)$ heißt \emph{(strikt)
konvex}\index{Verlustfunktion!konvexe}, falls $L(x,y,\cdot):\R\to[0,\infty)$
(strikt) konvex.\fishhere
\end{defn}

\begin{bem*}[Erinnerung.]
$\LL_0(X) \defl \setdef{f: X\to\R}{f\text{ messbar}}$.\maphere
\end{bem*}

\begin{lem}
\label{prop:2.1.2}
Sei $L$ eine (strikt) konvexe Verlustfunktion, dann ist auch
\begin{align*}
\RR_{L,P}(\cdot) : \LL_0(X)\to [0,\infty]
\end{align*}
(strikt) konvex.\fishhere
\end{lem}
\begin{proof}
Übung.\qedhere
\end{proof}


\begin{defn}
\label{defn:2.1.3}
Eine Verlustfunktion $L: X\times Y\times \R\to[0,\infty)$ heißt
\emph{stetig}\index{Verlustfunktion!stetige}, falls 
%\begin{align*}
$L(x,y,\cdot) : \R\to[0,\infty)$
%\end{align*}
stetig für alle $(x,y)\in (X\times Y)$. $L$ heißt \emph{lokal lipschitz-stetig},
falls
\begin{align*}
\forall a > 0 \exists c_a : \abs{L(x,y,t)-L(x,y,t')} \le c_a\abs{t-t'} 
\end{align*}
für alle $(x,y)\in (X\times Y)$ und $t,t'\in [-a,a]$. Die kleinste Konstante
$c_a$ wird mit $\abs{L}_{a,1}$ bezeichnet.

$L$ heißt \emph{lipschitz-stetig}, falls $\abs{L}_1 \defl \sup_{a\ge 0}
\abs{L}_{a,1}< \infty$.\fishhere
\end{defn}

\begin{lem}
\label{prop:2.1.4}
Sei $L$ eine stetige Verlustfunktion und $P$ ein W-Maß auf $X\times Y$, so gilt
für alle Folgen $(f_n)$ in $\LL_0(X)$ und $f\in \LL_0(X)$ mit $f_n\Pto f$,
\begin{align*}
\RR_{L,P}(f) \le \liminf_{n\to\infty} \RR_{L,P}(f_n).\fishhere
\end{align*}
\end{lem}
Das Risiko ist also ``halbstetig von unten''. 

\begin{proof}
Da $f_n\Pto f$ existiert eine Teilfolge mit $f_{n_k} \to f\Pfs$ und daher
existiert auch eine weitere Teilfolge $(f_{n_{k_l}})$ mit
\begin{align*}
\lim\limits_{l\to\infty} \RR_{L,P}(f_{n_{k_l}}) = 
\liminf\limits_{n\to\infty} \RR_{L,P}(f_n).
\end{align*}
Wir schreiben nun kürzer $(f_{n_k})$ für die Teilfolge mit beiden Eigenschaften.
Da $L$ stetig gilt $L(x,y,f_{n_k}(x))\to L(x,y,f(x))$ $P_X\text{-f.s.}$ und 
mit dem Lemma von Fatou folgt,
\begin{align*}
\RR_{L,P}(f) &= \int_{X\times Y} L(x,y,f(x))\dP(x,y)
= \int_{X\times Y} \lim\limits_{k\to\infty }L(x,y,f_{n_k}(x)) \dP(x,y)\\
&\le \liminf_{k\to\infty} \int_{X\times Y} L(x,y,f_{n_k}(x)) \dP(x,y)
= \lim\limits_{k\to\infty} \RR_{L,P}(f_{n_k})\\
&= \liminf_{n\to\infty} \RR_{L,P}(f_n).\qedhere
\end{align*}
\end{proof}

Optimal für unsere Zwecke wäre ``$=$'' anstat ``$\le$'' im Lemma
\ref{prop:2.1.4}. Es stellt sich jedoch heraus, dass dies nicht ohne
zusätzliche Voraussetzungen an $L$ möglich ist.

\begin{defn}
\label{defn:2.1.5}
Eine Verlustfunktion $L:X\times Y\times \R\to[0,\infty)$ heißt
\emph{Nemitski-Verlustfunktion (NVF)},\index{Verlustfunktion!Nemitski-}
falls eine messbare Funktion $b: X\times Y\to [0,\infty)$ und eine messbare und monoton wachsende Funktion $h: \R\to
[0,\infty)$ existiert, so dass
\begin{align*}
L(x,y,t) \le b(x,y) + h(\abs{t}),\qquad \forall x,y,t.
\end{align*}
$L$ heißt \emph{NVF der Ordnung $p\in(0,\infty)$}, falls ein $c>0$ existiert, so
dass
\begin{align*}
L(x,y,t) \le b(x,y) + c\cdot\abs{t}^p,\qquad \forall x,y,t.
\end{align*}
Ist $P$ ein W-Maß auf $(X\times Y)$ und $L$ eine NVF, so heißt $L$
\emph{$P$-integrierbar}\index{Verlustfunktion!$P$-integrierbar}, falls $b$
$P$-integrierbar.\fishhere
\end{defn}

\begin{lem}
\label{prop:2.1.6}
Sei $P$ ein W-Maß auf $X\times Y$ und $L$ eine stetige, $P$-integrierbare NVF,
dann gelten.
\begin{propenum}
\item\label{prop:2.1.6:1} Sei $(f_n)$ eine gleichmäßig beschränkte Folge in
$\LL_0(P_X)$ und $f\in \LL_\infty(P_X)$ mit $f_n\to f\Pfs$ Dann folgt
\begin{align*}
\RR_{L,P}(f) = \lim\limits_{n\to\infty} \RR_{L,P}(f_n).
\end{align*}
\item\label{prop:2.1.6:2} $\RR_{L,P}(\cdot) : \LL_\infty(P_X) \to [0,\infty)$
ist stetig.
\item\label{prop:2.1.6:3} Ist $L$ außerdem von der Ordnung $p\in[1,\infty)$, so
ist
\begin{align*}
\RR_{L,P}(f) : \LL_p(P_X) \to [0,\infty)
\end{align*}
wohldefiniert und stetig.\fishhere
\end{propenum}
\end{lem}
\begin{proof}
``\ref{prop:2.1.6:1}'': $f$ ist beschränkt mit $\norm{f}_\infty \le B$. Ferner
gilt
\begin{align*}
\lim\limits_{n\to\infty} \underbrace{\abs{L(x,y,f_n(x))-L(x,y,f(x))}}_{g_n(x)}
= 0\Pfs
\end{align*}
und
\begin{align*}
g_n(x) &\le L(x,y,f_n(x)) + L(x,y,f(x))\\
&\le b(x,y) + h(\abs{f_n(x)}) + b(x,y)+ h(\abs{f(x)})\\
&\le 2(b(x,y) + h(B))\Pfs
\end{align*}
wobei die rechte Seite als Funktion in $(x,y)$ $P$-integrierbar ist. Mit dem
Satz von Lebesgue folgt nun,
\begin{align*}
\abs{\RR_{L,P}(f_n)-\RR_{L,P}(f)} \le \int_{X\times Y} g_n(x) \dP(x,y) \to
0,\qquad n\to\infty.
\end{align*}

``\ref{prop:2.1.6:2}'': Sei $(f_n)$ Folge in $\LL_\infty(X)$ und
$f\in\LL_\infty(X)$ mit $\norm{f_n-f}_\infty \to 0$. Dann ist $f_n$ gleichmäßig
beschränkt und $f_n\to f$ $P_X$-f.s., also $\RR_{L,P}(f_n)\to
\RR_{L,P}(f)$.

``\ref{prop:2.1.6:3}'': Es gilt für $f\in\LL_p(P_X)$,
\begin{align*}
R_{L,P}(f) = \int L(x,y,f(x))\dP(x,y)
\le \int b(x,y) + c\abs{f(x)}^p \dP(x,y) < \infty.
\end{align*}
Somit ist $\RR_{L,P}(\cdot): \LL_p(P_X)\to [0,\infty)$ wohldefiniert.

Sei nun $(f_n)\subset \LL_p(P_X)$ und $f\in\LL_p(P_X)$ mit $\norm{f_n-f}_p\to
0$, dann gilt auch $f_n\Pto f$. Mit Lemma \ref{prop:2.1.4} folgt
\begin{align*}
\RR_{L,P}(f) \le \liminf_{n\to\infty} \RR_{L,P}(f_n).\tag{*}
\end{align*}
Definiere nun $\bar{L}(x,y,t)\defl b(x,y) + c\abs{t}^p - L(x,y,t) \ge 0$, so ist
$\bar{L}:X\times Y\times \R \to [0,\infty)$ eine stetige Verlustfunktion.
Erneute Anwendung von \ref{prop:2.1.4} ergibt
\begin{align*}
\norm{b}_1 +c\norm{f}_p^p
- \RR_{L,P}(f) &= 
\RR_{\bar{L},P}(f) \le \liminf_{n\to\infty} \RR_{\bar{L},P}(f_n)\\
&= \norm{b}_1 + \liminf_{n\to\infty} c\norm{f_n}_p^p - \RR_{L,P}(f_n).
\end{align*}
Da $\norm{f_n}_p\to \norm{f}_p$ ist $-\RR_{L,P}(f)\le -
\limsup\limits_{n\to\infty} \RR_{L,P}(f_n)$, d.h. mit (*) folgt
\begin{align*}
\limsup_{n\to\infty} \RR_{L,P}(f_n) \le \liminf_{n\to\infty} \RR_{L,P}(f_n)
\end{align*}
also $\lim\limits_{n\to\infty} \RR_{L,P}(f_n) = \RR_{L,P}(f)$.\qedhere
\end{proof}

\begin{lem}
\label{prop:2.1.7}
Ist $L: Y\times \R\to [0,\infty)$ strikt überwacht und konvex und $Y$ endlich.
Dann ist $L$ lokal lipschitz.\fishhere
\end{lem}
\begin{proof}
Wir benutzen, dass jede konvexe Abbildung $g:[-a,a]\to [0,\infty)$ lokal
lipschitz ist. Nach dieser Aussage ist $L(y,\cdot) : \R\to [0,\infty)$ lokal
lipschitz für alle $y\in Y$. Da $Y$ endlich, folgt die Aussage.\qedhere
\end{proof}

\begin{lem}
\label{lem:2.1.8}
\begin{propenum}
\item Ist $L$ lokal lipschitz, so ist $L$ NVF.
\item Ist $L$ lokal lipschitz und $\RR_{L,P}(0) < \infty$, so ist $L$
$P$-integrierbare NVF.
\item Ist $L$ lipschitz stetig, so ist $L$ NVF der Ordnung $p=1$.\fishhere
\end{propenum}
\end{lem}
\begin{proof}
\begin{proofenum}
\item $\abs{L(x,y,t)-L(x,y,0)} \le \abs{L}_{\abs{t},1}\abs{t}$. Somit ist
\begin{align*}
L(x,y,t) \le \underbrace{\abs{L}_{\abs{t},1}\abs{t}}_{h(t)} +
\underbrace{L(x,y,0)}_{b(x,y)}.
\end{align*}
\item Falls $\RR_{L,P}(0) < \infty$ ist $b(x,y)$ $P$-integrierbar.
\item Falls $L$ lipschitz, ist $\abs{L}_{\abs{t},1} \le \abs{L}_1$ und damit
\begin{align*}
L(x,y,t) \le L(x,y,0) + \abs{L}_1\abs{t}.\qedhere
\end{align*}
\end{proofenum}
\end{proof}

\begin{lem}
\label{prop:2.1.9}
Sei $L$ lokal lipschitz, $B\ge 0$ und $f,g\in\LL_\infty(P_X)$
mit $\norm{f}_\infty,\norm{g}_\infty \le B$. Dann gilt
\begin{align*}
\abs{\RR_{L,P}(f)-\RR_{L,P}(g)} \le
\abs{L}_{B,1}\norm{f-g}_{\LL_1(P_X)}.\fishhere
\end{align*}
\end{lem}
\begin{proof}
Übung.\qedhere
\end{proof}

\begin{defn}
\label{defn:2.1.10}
Eine Verlustfunktion $L$ \emph{kann bei $M>0$ abgeschnitten werden},
wenn für alle $(x,y)\in X\times Y$ und $t\in\R$ gilt
\begin{align*}
L(x,y,\cut{t}) \le L(x,y,t),
\end{align*}
wobei 
\begin{align*}
\cut{t} = 
\begin{cases}
-M, & t \le -M,\\
t, & t\in (-M,M)\\
M, & t\ge M.\fishhere
\end{cases}
\end{align*}
\end{defn}
Man kann dies so interpretieren, dass Abschneiden den Verlust \textit{nicht}
erhöht.

\begin{lem}
\label{prop:2.1.11}
Sei $L$ eine konvexe Verlustfunktion und $M>0$. Dann sind folgende Aussagen
äquivalent:
\begin{equivenum}
\item\label{prop:2.1.11:1} $L$ kann bei $M$ abgeschnitten werden.
\item\label{prop:2.1.11:2} Für alle $(x,y)\in X\times Y$ hat die Funktion
$L(x,y,\cdot): \R\to[0,\infty)$ mindestens ein globales Minimum in $[-M,M]$.\fishhere
\end{equivenum}
\end{lem}
\begin{proof}
Schreibe $M_{x,y}\defl \setdef{t^*\in \R}{L(x,y,t^*)=\inf_{t\in\R} L(x,y,t)}$. Da
$L$ konvex ist, ist $M_{x,y}$ ein Intervall. 

``\ref{prop:2.1.11:1}$\Rightarrow$\ref{prop:2.1.11:2}": Angenommen es gibt ein
$(x,y)\in X\times Y$ mit $M_{x,y}\cap [-M,M] = \varnothing$.

\textit{1. Fall $M_{x,y}=\varnothing$}. $L$ ist konvex, also ist $L(x,y,\cdot)$
strikt monoton, denn falls $L(x,y,\cdot)\in C^2(\R)$, so ist aufgrund der
Konvexität $L''(x,y,\cdot) \ge 0$. Da aber $M_{x,y}=\varnothing$ folgt
$L'(x,y,t)\neq 0$ für alle $t$. Für allgemeines $L(x,y,\cdot)$ folgt die strikte
Monotonie aus der Betrachtung von Subdifferenzialen.

Aber da $L(x,y,\cdot)$ strikt monoton, kann $L$ nicht abgeschnitten
werden.\dipper

\textit{2. Fall $M_{x,y}\neq \varnothing$}. Da $M_{x,y}$ ein abgeschlossenes
Intervall, folgt ohne Einschränkung $t\defl\inf M_{x,y}$ erfüllt $M< t<\infty$,
d.h. $M_{x,y}$  liegt rechts von $[-M,M]$.

Somit ist $L(x,y,\cut{t}) = L(x,y,M) > L(x,y,t)$ da $t>M$ und $L(x,y,\cdot)$
aufgrund der Konvexität strikt fallend links von $M_{x,y}$.\dipper

``\ref{prop:2.1.11:2}$\Rightarrow$\ref{prop:2.1.11:1}": Es gilt $M_{x,y}\cap
[-M,M] \neq \varnothing$ und daher ist $\inf M_{x,y} \le M$ und $\sup M_{x,y}\ge
M$.

$L(x,y,\cdot)$ ist strikt konvex, also ist $L(x,y,\cdot)$ auf $[\sup
M_{x,y},\infty)$ wachsend und auf $(-\infty,\inf M_{x,y}]$ fallend.
Somit kann $L$ abgeschnitten werden.\qedhere
\end{proof}

\section{Margin basierte Verlustfunktionen}

Sei $Y=\setd{-1,1}$ und $\eta(x) = P(Y=1\mid x)$.

\begin{defn}
\label{defn:2.2.1}
Eine strikt überwachte Verlustfunktion $L: Y\times \R \to [0,\infty)$ heißt
\emph{margin-basiert}, falls eine repräsentative Funktion $\ph: \R\to[0,\infty)$
existiert, d.h. $L(y,t) = \ph(y\cdot t)$.\fishhere
\end{defn}

\begin{lem}
\label{lem:2.2.2}
Sei $L$ margin-basiert und $\ph$ die repräsentative Funktion. Dann gelten:
\begin{propenum}
\item $L$ ist genau dann (strikt) konvex, wenn $\ph$ (strikt) konvex.
\item $L$ ist genau dann stetig, wenn $\ph$ stetig.
\item $L$ ist genau dann (lokal) lipschitz, wenn $\ph$ (lokal) lipschitz.
\item Ist $L$ konvex, so ist $L$ (lokal) lipschitz.
\item $L$ ist $P$-integrierbare NVF.
\item Ist $L$ lipschitz stetig, so ist $L$ $P$-integrierbare NVF der Ordnung
$p=1$.\fishhere
\end{propenum}
\end{lem}
\begin{proof}
Übung.\qedhere
\end{proof}

\begin{bsp}
\label{bsp:2.2.3}
\newcommand{\LS}{\mathrm{LS}}
\textit{Kleinste Quadrate}. Scharfes Hinsehen ergibt,
\begin{align*}
L_\LS(y,t) = (y-t)^2 = (1-yt)^2, \quad\Rightarrow\quad \ph(t) = (1-t)^2.
\end{align*}
Damit ist $L_\LS$ strikt konvex, da $\ph$ strikt
konvex. $\abs{L_\LS}_{a,1} = 2a+2$ für $a\ge 0$. $L_\LS$ kann bei $M=1$
abgeschnitten werden ($\ph$ hat ein globales Minimum bei $t=1$ und damit
$L(1,\cdot)$ bei $t=1$ und $L(-1,\cdot)$ bei $t=-1$).\bsphere
\end{bsp}

\begin{bsp}
\label{bsp:2.2.4}
\newcommand{\Hinge}{\mathrm{Hinge}}
\textit{Hinge loss}.
\begin{align*}
L_\Hinge(y,t) = \max\setd{0,1-yt},\quad \Rightarrow\quad \ph(t) =
\max\setd{0,1-t}.
\end{align*}
$L_\Hinge$ ist konvex, lipschitz und $\abs{L_\Hinge}_1 = 1$. Sie ist
\textit{nicht} strikt konvex und kann bei $M=1$ abgeschnitten werden.\bsphere
\end{bsp}

\begin{bsp}
\label{bsp:2.2.5}
\textit{Quadrierter Hinge-Loss}.
\begin{align*}
L(y,t) = \left(\max\setd{0,1-yt}\right)^2,\quad\Rightarrow\quad
\ph(t) = \left(\max\setd{0,1-t}\right)^2.
\end{align*}
Sie stellt eine Mischung aus LS und HL dar. $L$ ist konvex (nicht strikt), lokal
lipschitz, $\abs{L}_{a,1} = 2a+2$ und kann bei $M=1$ abgeschnitten
werden.\bsphere
\end{bsp}

\begin{bsp}
\label{bsp:2.2.6}
\textit{Logistische Verlustfunktion für Klassifikation}.
\begin{align*}
L_{\log}(y,t) = \log(1+\exp(-yt)),\quad\Rightarrow\quad
\ph(t) = \log(1+\exp(-yt)).
\end{align*}
$L_{\log}$ ist strikt konvex, lipschitz $\abs{L_{\log}}_1 = 1$, kann aber
\textit{nicht} abgeschnitten werden.~\bsphere
\end{bsp}

\begin{figure}[!htpb]
\centering
\psset{unit=0.5cm}
\begin{pspicture}(-5,-0.8)(5,5) 
 \psaxes[labels=none,ticks=none,linecolor=gdarkgray,tickcolor=gdarkgray]{->}%
 (0,0)(-4.8,-0.5)(4.8,4.8)[\color{gdarkgray}$x$,-90][,0]

 \psplot[linewidth=1.2pt,linecolor=darkblue,algebraic=true]{-4.5}{1}%
 {1-x}
 \psline[linewidth=1.2pt,linecolor=darkblue](1,0)(4.5,0)	
\end{pspicture}
\begin{pspicture}(-5,-0.8)(5,5) 
 \psaxes[labels=none,ticks=none,linecolor=gdarkgray,tickcolor=gdarkgray]{->}%
 (0,0)(-4.8,-0.5)(4.8,4.8)[\color{gdarkgray}$x$,-90][,0]

 \psplot[linewidth=1.2pt,linecolor=purple,algebraic=true]{-4.5}{1}%
 {(1-x)^2}
 \psline[linewidth=1.2pt,linecolor=purple](1,0)(4.5,0)	
\end{pspicture}
\caption{Hinge loss und quadrierte hinge loss Verlustfunktion}
\end{figure}

\begin{figure}[!htpb]
\centering
\psset{unit=0.5cm}
\begin{pspicture}(-5,-0.8)(5,5) 
 \psaxes[labels=none,ticks=none,linecolor=gdarkgray,tickcolor=gdarkgray]{->}%
 (0,0)(-4.8,-0.5)(4.8,4.8)[\color{gdarkgray}$x$,-90][,0]

 \psplot[linewidth=1.2pt,linecolor=darkblue,algebraic=true]{-4.5}{4.5}%
 {ln(1+EXP(-x))}
 	
\end{pspicture}
\caption{Logistische Verlustfunktion}
\end{figure}

In Übung 6 wurde gezeigt
\begin{align*}
\RR_{L_\class,P}(f) - \RR_{L_\class,P}^* \le 
\sqrt{\RR_{L_\mathrm{LS},P}(f)-\RR_{L_\mathrm{LS},P}^*}
\end{align*}

\begin{lem}
\label{prop:2.2.7}
Für $\eta\in[0,1]$ und $t\in[-1,1]$ gilt
\begin{align*}
\abs{2\eta-1}\Id_{(-\infty,0]}((2\eta-1)\sign t)\le
\abs{2\eta-1}\abs{t-\sign(2\eta-1)}.\fishhere
\end{align*}
\end{lem}
\begin{proof}
\textit{1. Fall} $\eta =\frac{1}{2}$ ist klar.

\textit{2. Fall} $\eta < \frac{1}{2}$.
Für $t\in[-1,0)$ ist
\begin{align*}
\underbrace{(2\eta-1)}_{<0}\underbrace{\sign t}_{<0} > 0,
\end{align*}
somit verschwindet die linke Seite und die rechte ist $\ge 0$.\\
Für $t\in [0,1)$ gilt umgekehrt,
\begin{align*}
\underbrace{(2\eta-1)}_{<0}\underbrace{\sign t}_{=1} < 0,
\end{align*}
somit ist die linke Seite gleich $\abs{2\eta-1}< 1$ und $\abs{t-\sign(2\eta-1)}
= \abs{t+1} \ge 1$.\qedhere
\end{proof}

\begin{prop}[Zhang's (Un-)Gleichung]
\label{prop:2.2.8}
\index{Ungleichung!Zhang's-}
Sei $f^*_{L_\class}(x) \defl \sign(2\eta(x)-1)$ für $x\in X$. Dann gilt für $f:
X\to [-1,1]$:
\begin{align*}
\RR_{L_\hinge,P}(f) - \RR_{L_\hinge,P}^* =
\int_X \abs{f(x)-f_{L_\class,P}^*(x)}\abs{2\eta(x)-1}\dP_X(x)
\end{align*}
"`Überschuss-$L_\hinge$-Risiko = gewichtete $L^1$-Norm von
$f-f_{L_\class,P}^*$"'.\\
Und für $h: X\to\R$ gilt
\begin{align*}
\RR_{L_\class,P}(f) - \RR_{L_\class,P}^* \le
\RR_{L_\hinge,P}(f) - \RR_{L_\hinge,P}^*.\fishhere
\end{align*}
\end{prop}
\begin{proof}
$L_\hinge(y,t) = \max\setd{0,1-yt} \overset{t\in[-1,1]}{=} 1-yt$. Für $f:
X\to[-1,1]$ gilt daher
\begin{align*}
\RR_{L_\hinge,P}(f) &= \int_X \eta(x)(1-f(x))+(1-\eta(x))(1+f(x))\dP_X\\
&= \int_X 1+f(x)(1-2\eta(x))\dP_X.
\end{align*}
$f(x)(1-2\eta(x))$ ist minimal genau dann, wenn
\begin{align*}
\begin{rcases}
f = 1\text{ auf } \setd{\eta > \frac{1}{2}}\\
f = -1 \text{ auf }  \setd{\eta < \frac{1}{2}}
\end{rcases}
\text{d.h. } f=f^*_{L_\class,P}\text{ auf } \setd{\eta \neq \frac{1}{2}}
\end{align*}
Da $L_\hinge$ bei $M=1$ abgeschnitten werden kann folgt somit
\begin{align*}
\RR_{L_\hinge,P}^* = \inf_{f: X\to\R} \RR_{L_\hinge,P}(f) 
= \inf_{f: X\to [-1,1]} \RR_{L_\hinge,P}(f). 
\end{align*}
Damit ist
\begin{align*}
\RR_{L_\hinge,P}(f)- \RR_{L_\hinge,P}^* &= 
\int_X 1+ f\cdot(1-2\eta) - 1-f_{L_\class,P}^*\cdot(1-2\eta)\dP_X\\
&= \int_X \underbrace{(f-f_{L_\class,P}^*)\cdot(1-2\eta)}_{\ge 0}\dP_X,
\end{align*}
wobei der Integrand positiv ist, da $f_{L_\class,P}^*$ punktweise das
Überschussrisiko minimiert und dieses ist positiv, also
\begin{align*}
\RR_{L_\hinge,P(f)}- \RR_{L_\hinge,P}^* =
\int_X \abs{f-f_{L_\class,P}^*}\abs{1-2\eta}\dP_X.
\end{align*}
\textit{Ungleichung}. Da $L_\hinge$ bei $M=1$ abgeschnitten werden kann, folgt
\begin{align*}
\RR_{L_\hinge,P}(\cut{f}) - \RR_{L_\hinge,P}^* \le
\RR_{L_\hinge,P}(f)-\RR_{L_\hinge,P}^*
\end{align*}
und
\begin{align*}
\RR_{L_\class,P}(\cut{f})-\RR_{L_\class,P}^* =
\RR_{L_\class,P}(f) - \RR_{L_\class,P}^*. 
\end{align*}
Daher ist ohne Einschränkung $f: X\to [-1,1]$. In Kapitel \ref{sec:1.2} haben
wir gezeigt
\begin{align*}
\RR_{L_\class,P}(f) - \RR_{L_\class,P}^* &= \int_X
\abs{2\eta-1}\Id_{(-\infty,0]}((2\eta-1)\sign f) \dP_X\\
&\overset{\ref{prop:2.2.7}}{\le}
\int_X
\abs{2\eta-1}\abs{f-\underbrace{\sign(2\eta-1)}_{=f_{L_\class,P}^*}}\dP_X\\
&\overset{\text{Zhang}}{=} \RR_{L_\hinge}(f)-\RR_{L_\hinge,P}^*.\qedhere
\end{align*}
\end{proof}

\section{Distanzbasierte Verlustfunktionen}

\begin{defn}
\label{defn:2.3.1}
Eine Verlustfunktion $L: Y \times \R\to[0,\infty)$ heiß
\emph{distanzbasiert}\index{Verlustfunktion!distanzbasiert}, falls eine Funktion
$\psi: \R\to[0,\infty)$ existiert mit
\begin{align*}
L(y,t) = \psi(y-t),\qquad y\in Y,\quad t\in\R.
\end{align*}
Eine distanzbasierte Verlustfunktion heißt \emph{symmetrisch}, falls $\psi(r) =
\psi(-r)$ für alle $r\in\R$ heißt \index{Verlustfunktion!symmetrisch}.\fishhere
\end{defn}

\begin{lem}
\label{prop:2.3.2}
Sei $L$ distanzbasiert, so gelten:
\begin{propenum}
\item $L$ ist genau dann (strikt) konvex, wenn $\psi$ (strikt) konvex.
\item $L$ ist genau dann stetig, wenn $\psi$ stetig.
\item $L$ ist genau dann lipschitz, wenn $\psi$ lipschitz. (Gilt im Allgemeinen
nicht für lokal lipschitz, siehe $L_\mathrm{LS}$)\\
\item\label{prop:2.3.2:4} Ist ferner $Y\subset[-M,M]$, so ist $L$ genau
dann lokal lipschitz, wenn $\psi$ lokal lipschitz.
\item\label{prop:2.3.2:5} Ist $L$ konvex, so ist $\psi$ lokal lipschitz.
\item\label{prop:2.3.2:6} $L$ ist eine P-integrierbare NVF.\fishhere 
\end{propenum}
\end{lem}
\begin{proof}
\ref{prop:2.3.2:4} "`$\Leftarrow$"': Sei $t\in[-a,a]$ und $a > 0$, so gilt
\begin{align*}
\abs{L(y,t)-L(y,t')} = \abs{\psi(y-t)-\psi(y-t')} \le
\abs{\psi}_{1+M,1}\abs{t-t'}.
\end{align*}
"`$\Rightarrow$"': Analog folgt $\abs{L}_{a,1} \le \abs{\psi}_{a+M,1}$.

\ref{prop:2.3.2:5} Sei $L$ konvex, dann folgt für $y=0$, dass auch
\begin{align*}
t\mapsto \psi(-t) = \psi(0-t) = L(0,t)
\end{align*}
konvex und daher ist $\psi$ konvex und folglich auch lokal lipschitz. Mit
\ref{prop:2.3.2:4} folgt nun, dass $L$ lokal lipschitz.

\ref{prop:2.3.2:6} folgt aus Lemma \ref{prop:2.1.9}.\qedhere
\end{proof}

\begin{bsp}
\label{bsp:2.3.3}
\textit{Kleinste Quadrate}
\begin{align*}
L_{\mathrm{LS}}(y,t) \defl (y-t)^2 \Rightarrow \psi(r) = r^2.
\end{align*}
$\psi$ ist strikt konvex, nicht lipschitz aber lokal lipschitz und
symmetrisch.\bsphere
\end{bsp}

\begin{bsp}
\label{bsp:2.3.4}
\textit{Betragsfunktion}.
\begin{align*}
L_{\mathrm{abs}}(y,t) \defl \abs{y-t} \Rightarrow \psi(r) = \abs{r}.
\end{align*}
$\psi$ ist konvex (nicht strikt), symmetrisch und Lipschitz mit $\abs{L}_1 = 1$.
\end{bsp}


\begin{bsp}
\label{bsp:2.3.5}
\textit{Pinball für $\ep \in(0,1)$}.
\begin{align*}
\psi(r)\defl 
\begin{cases}
-1(1-\ep)r, & r < 0,\\
\ep r, & r\ge 0
\end{cases}
\end{align*}
$\psi$ ist konvex (nicht strikt), symmetrisch genau dann, wenn $\ep
=\frac{1}{2}$ und lipschitz mit $\abs{L}_1 = \min\setd{\ep,1-\ep}$.\bsphere
%TODO: Bild Pinball.
\end{bsp}

\chapter{Konzentrationsungleichungen}

Im Lemma \ref{lem:1.3.7} hatten wir gezeigt:
\begin{align*}
\mu^n\left(\setdef{\omega\in\Omega^n}{\frac{1}{n}\sum_{i=1}^n
h(\omega_i)-\E_\mu h \ge t} \right) \le \frac{\E_\mu h^2}{t^2n}
\end{align*}
für alle Maße $\mu$, $n\ge 1$, $t>0$ und $h: \Omega\to\R$.

\begin{prop}
\label{prop:3.1}
Sei $(\Omega,\AA,\mu)$ ein Wahrscheinlichkeitsraum, dann gilt für alle $f:
\Omega\to\R$ mit $\E_\mu \abs{f}< \infty$ und $t>0$,
\begin{align*}
\mu\left(\setdef{\omega\in\Omega}{f(\omega)\ge t} \right)
\le \frac{\E_\mu \abs{f}}{t}.\fishhere
\end{align*}
\end{prop}

\begin{lem}
\label{prop:3.2}
Für $x>-1$ gilt
\begin{align*}
(1+x)\ln(1+x) -x \ge \frac{3}{2}\frac{x^2}{x+3}.\fishhere
\end{align*}
\end{lem}
\begin{proof}
Seien $f(x) = (1+x)\ln(1+x)-x$ und $g(x) = \frac{3}{2}\frac{x^2}{x+3}$. Dann
folgt
\begin{align*}
&f'(x) = \ln(1+x), && g'(x) = \frac{3}{2}\frac{x^2+6x}{(x+3)^2},\\
&f''(x) = \frac{1}{1+x}, && g''(x) = \frac{27}{(x+3)^3}, 
\end{align*}
und es gelten
\begin{align*}
&f'(0) = 0,\quad f(0) = 0,\\
&f'(0) = 0, \quad g(0) = 0. 
\end{align*}
Wir zeigen nun $f''(x)\ge g''(x)$ für alle $x>-1$. Es ist
\begin{align*}
&0 \le x^2(x+9) = x^3+9x^2\\
\Rightarrow\;
&x^3+9x^2+27x+27 = (x+3)^3 \ge  27(1+x) = 27x + 27\\
\Rightarrow\;
&\frac{27}{(x+3)^3} \le \frac{1}{1+x}.
\end{align*}
Für $x\ge 0$ folgt mit dem Hauptsatz der Differential- und Integralrechnung,
\begin{align*}
f'(x) = f'(x)-f'(0) = \int_0^x f''(t)\dt
\ge \int_0^x g''(t)\dt = g'(x)-g'(0) = g'(x).
\end{align*}
Das selbe Argument angewandt auf $f$ liefert, $f(x) \ge g(x)$ und für $x\in
(-1,0]$ folgt $-f(x) \ge -g(x)$.\qedhere
\end{proof}

\begin{prop}[Bernsteins Ungleichung]
\index{Bernsteins Ungleichung}
\label{prop:3.3}
Sei $(\Omega,\AA,P)$ ein Wahrscheinlichkeitsraum und $B>0$, $\sigma > 0$ und
$n\ge 1$. Ferner seien $\xi_1,\ldots,\xi_n: \Omega\to\R$ i.i.d. (unabhängige und
identisch verteilte) Zufallsvariablen, so dass gilt
\begin{align*}
\begin{rcases}
\E_P \xi_i &= 0,\\
\norm{\xi_i}_\infty &\le B,\\
\E_P \xi^2 &\le \sigma^2
\end{rcases}
\text{ für alle } i = 1,\ldots,n.
\end{align*}
Dann gilt
\begin{align*}
P\left(\frac{1}{n} \sum_{i=1}^n \xi_i \ge \sqrt{\frac{2\sigma^2\tau}{n}} +
\frac{2 B \tau}{3n}\right) \le e^{-\tau},\quad \text{für alle } \tau >
0.\fishhere
\end{align*}
\end{prop}
\begin{proof}
Fixiere $t \ge 0$ und $\ep > 0$, dann gilt
\begin{align*}
P\left(\sum_{i=1}^n \xi_i \ge \ep n \right)
&= 
P\left(\exp\left(t\sum_{i=1}^n \xi_i\right) \ge \exp(t\ep n) \right)\\
&\overset{\text{Markov}}{\le}
e^{-t\ep n}\E_P \exp\left(t\sum_{i=1}^n \xi_i \right)\\
&\overset{\text{Unabh.}}{=}
e^{-t\ep n} \prod_{i=1}^n \E_P e^{t\xi_i}.
\end{align*}
Folglich ist
\begin{align*}
\E_P e^{t\xi_i} = \E_P \sum_{k=0}^\infty \frac{t^k}{k!}\xi_i^k
= \sum_{k=0}^\infty \frac{t^k}{k!}\E_P\xi^k
\end{align*}
und da $\E_P \xi_i = 0$ und $\E_P \xi^k \le \sigma^2 B^{k-2}$ folgt
\begin{align*}
\E_P e^{t\xi} &= 1 + \sum_{k=2}^\infty \frac{t^k}{k!}\E_P \xi_i^k
\le 1 + \sum_{k=2}^\infty \frac{t^k}{k!}\sigma^k B^{k-2}
= 1 + \frac{\sigma^2}{B^2}\sum_{k=2}^\infty \frac{t^k}{k!}B^k\\
&= 1 + \frac{\sigma^2}{B^2}\left(e^{tB} -tB-1\right).
\end{align*}
Damit gilt
\begin{align*}
P\left(\sum_{i=1}^n \xi_i \ge \ep n \right)
&\le e^{-tn \ep}
\left( 1+ \frac{\sigma^2}{B^2}\left(e^{tB}-tB-1\right) \right)^n\\
&\le
e^{-tn \ep}
\exp\left( n\frac{\sigma^2}{B^2}\left(e^{tB}-tB-1\right) \right)
\end{align*}
Sei $h(t)\defl - t n \ep + \frac{n\sigma^2}{B^2}\left(e^{tB}-tB-1\right)$, dann
suchen wir kritische Punkte
\begin{align*}
h'(t) = -n \ep + \frac{n\sigma^2}{B^2}\left(Be^{tB}-B\right)
= -n \ep + \frac{n\sigma^2}{B}\left(e^{tB}-1\right)
\overset{!}{=} 0.
\end{align*}
$t$ ist genau dann kritisch, wenn
\begin{align*}
\frac{\sigma^2n}{B} e^{tB} = \ep n + \frac{\sigma^2 n}{B}
\Leftrightarrow
e^{tB} = \frac{\ep B}{\sigma^2} + 1
\Leftrightarrow
t^* = \frac{1}{B}\log\left( 1+ \frac{\ep B}{\sigma^2}\right).
\end{align*}
$t^*$ ist der einzige Kandidat für ein Optimum. Weiterhin ist
\begin{align*}
\lim\limits_{t\to\pm\infty} h(t) = \infty
\end{align*}
und folglich hat $h$ ein globales Minimum bei $t^*$. Setzen wir $y=\frac{\ep
B}{\sigma^2}$, dann gilt $t^*=\frac{1}{B}\log(1+y)$ und somit
\begin{align*}
&-t^* \ep n + \frac{\sigma^2 n}{B^2}\left(e^{t^*B}-t^*B-1\right)\\
= &-\frac{\ep n}{B}\log(1+y) + \frac{\sigma^2 n}{B^2}(1+y-\log(1+y)-1)\\
= &\frac{\sigma^2 n}{B^2}
\left(-y\log(1+y) + y - \log(1+y) \right)\\
= &-\frac{\sigma^2n}{B^2}\left((1+y)\log(1+y)-y \right)\\
\overset{\ref{prop:3.2}}{\le}
&-\frac{\sigma^2 n}{B^2} \frac{3}{2}\frac{y^2}{y+3}
\overset{\text{Rechnung}}{=}
- \frac{3n\ep^2}{2\ep B + 6\sigma^2}.
\end{align*}
Damit ist
\begin{align*}
P\left(\frac{1}{n}\sum_{i=1}^n \xi_i \ge \ep \right)
\le \exp \left(-\frac{3n\ep^2}{2\ep B+6\sigma^2} \right).
\end{align*}
Wähle nun
\begin{align*}
\tau \defl \frac{3n \ep^2}{2\ep B + 6\sigma^2},
\end{align*}
dann folgt
\begin{align*}
\ep = \sqrt{\frac{2\sigma^2\tau}{n} + \frac{B^2\tau^2}{9n^2}} + \frac{B\tau}{3n}
\le \sqrt{\frac{2\sigma^2\tau}{n}} + \frac{2B\tau}{3n}.\qedhere
\end{align*}
\end{proof}

\begin{prop}[Hoeffdings Ungleichung]
\index{Hoeffdings Ungleichung}
\label{prop:3.4}
Sei $(\Omega,\AA,P)$ ein Wahrscheinlichkeitsraum, $a<b$, $1\le n\in\N$ und
$\xi_1,\ldots,\xi_n : \Omega\to[a,b]$ i.i.d. Dann gilt für $\tau >0$,
\begin{align*}
P \left(\frac{1}{n}\sum_{i=1}^n \left(\xi_i - \E_P \xi_i\right)\ge
(b-a)\sqrt{\frac{\tau}{2n}} \right) \le e^{-\tau}.\fishhere
\end{align*}
\end{prop}
\begin{bem*}[Beobachtung.]
Für $a=-b$ betrachte $\eta_i = \xi_i - \E_P\xi_i$.\\
Dann ist $\E_P\eta_i = 0$, $\E_P\eta_i^2 = \E_P\xi_i^2 - (\E_P\xi_i)^2 \le b^2
\defr \sigma^2$ und $\norm{\eta_i}_\infty \le 2b \defr B$. Mit Bernstein folgt,
\begin{align*}
P\left(\frac{1}{n}\sum_{i=1}^n \eta_i \ge \sqrt{\frac{2b^2\tau}{n}} +
\frac{4b\tau}{3n} \right) \le e^{-\tau},
\end{align*}
wobei $\sqrt{\frac{2b^2\tau}{n}} = 2b\sqrt{\frac{\tau}{2n}} =
(b-a)\sqrt{\frac{\tau}{2n}}$.\maphere
\end{bem*}

\begin{lem}[Lemma (Union bound)]
\index{Union bound}
\label{prop:3.5}
Sei $(\Omega,\AA,P)$ ein Wahrscheinlichkeitsraum, $f_1,\ldots,f_n: \Omega\to\R$
messbar. Dann gilt für alle $t\in\R$
\begin{align*}
\mu\left(\sup_{i=1,\ldots,n} f_i \ge t\right) \le
\sum_{i=1}^n \mu(f_i\ge t).\fishhere
\end{align*}
\end{lem}
\begin{proof}
$\setd{\sup_{i=1,\ldots,n} f_i \ge t} = \bigcup_{i=1}^n \setd{f_i\ge
t}$.\qedhere
\end{proof}

\begin{bem*}
Betrachte $\xi_1,\ldots,\xi_n,-\xi_1,\ldots,-\xi_n$ und das union bound für die
Summe über  1. Block $(\defr f_1)$ + 2. Block $(\defr f_2)$. Somit ist
\begin{align*}
\abs{\frac{1}{n}\sum_{i=1}^n \xi_i} = \sup\setd{f_1,f_2} 
\end{align*}
und folglich gilt
\begin{align*}
P \left(\abs{\frac{1}{n} \sum_{i=1}^n\xi_i } \ge \sqrt{\frac{2\sigma^2\tau}{n}}
+ \frac{2B\tau}{3 n} \right) \le 2e^{-\tau}.\maphere
\end{align*}
\end{bem*}
"`Zweiseitige Bernstein-Ungleichung"' (Hoeffdings analog!)

\chapter{Empirische Risikominimierung (ERM)}
\index{ERM}

\begin{bem*}[Motivation.]
Sei $D$ das empirische Wahrscheinlichkeitsmaß,
\begin{align*}
\RR_{L,D}(f) \defl \frac{1}{n}\sum_{i=1}^n L(x_i,y_i,f(x_i))
\overset{n\to\infty}{\longrightarrow} \RR_{L,P}(f).
\end{align*}
Ziel des Lernens: Finde $f$ mit $\RR_{L,P}(f)$ klein.\\
Heuristik von ERM: Minimiere Schätzung $\RR_{L,D}(\cdot)$ von
$\RR_{L,P}(\cdot)$.\maphere
\end{bem*}

\section{ERM über endliche Funktionenmengen}

\begin{defn}
\label{defn:4.1.1}
Sei $L$ eine Verlustfunktion, $\varnothing\neq \FF\subset \LL_0(X)$. Dann heißt
eine Lernmethode $\LL$ \emph{ERM} über $\FF$, wenn
\begin{align*}
\RR_{L,D}(f_D) = \inf_{f\in\FF} \RR_{L,D}(f),\qquad \forall n\ge 1,\; D\in
(X\times Y)^n \text{ und } f_D\in\FF.\fishhere
\end{align*}
\end{defn}

\begin{bem*}[Bemerkungen.]
\begin{bemenum}
\item Im Allgemeinen gibt es keine ERM.
\item Im Allgemeinen ist ERM nicht eindeutig.
\item Für $\FF=\LL_0(X)$ oder $\FF=\LL_\infty(X)$ erhält man im Allgemeinen
overfitted ERM.
\item Für "`kleine"' $\FF$ erhält man im Allgemeinen underfitted ERM.
\end{bemenum}
\end{bem*}

Im Folgenden sei $\FF$ endlich.

\begin{prop}[Orakelungleichung I]
\index{Orakelungleichung!ERM}
\label{prop:4.1.2}
Sei $L$ eine Verlustfunktion und $\FF\subset\LL_0(X)$ nicht leer und endlich.
Ferner sei $B>0$ mit
\begin{align*}
L(x,y,f(x)) \le B,\qquad \forall (x,y)\in X\times Y,\quad f\in\FF.
\end{align*}
Für den Wert des Orakels
\begin{align*}
\RR_{L,P,\FF}^* \defl \inf_{f\in\FF} \RR_{L,P}(f)
\end{align*}
und ERM gilt für $n\ge 1$ und $\tau > 0$,
\begin{align*}
&P^n\left(\setdef{D\in (X\times Y)^n}{\RR_{L,P}(f_D) \le \RR_{L,P,\FF}^* +
B\sqrt{\frac{2\tau + 2\log(2\abs{\FF})}{n}} } \right)\\ 
&\qquad\ge 1-e^{-\tau}.\fishhere
\end{align*}
\end{prop}
\begin{proof}
Für $\delta > 0$ existiert ein $f_\delta \in\FF$ mit $\RR_{L,P}(f_\delta)\le
\RR_{L,P,\FF}^* + \delta$. Damit ist
\begin{align*}
\RR_{L,P}(f_D) - \RR_{L,P,\FF}^*
&\le \RR_{L,P}(f_D)-\RR_{L,D}(f_D) + \RR_{L,D}(f_D)
- \RR_{L,P}(f_\delta)+ \delta\\
&\le \abs{\RR_{L,P}(f_D)-\RR_{L,D}(f_D)} + 
\abs{\RR_{L,P}(f_\delta)-\RR_{L,D}(f_\delta)} + \delta\\
&\le 2 \sup_{f\in\FF} \abs{\RR_{L,P}(f)-\RR_{L,D}(f)} + \delta.
\end{align*}
Für "`$\delta\to 0$"' folgt
\begin{align*}
\RR_{L,P}(f_D) - \RR_{L,P,\FF}^* \le
2 \sup_{f\in\FF} \abs{\RR_{L,P}(f)-\RR_{L,D}(f)}
\end{align*}
Damit folgt
\begin{align*}
&P^n\left(\setdef{D}{\RR_{L,P}(f_D)- \RR_{L,P,\FF}^* \ge
B\sqrt{\frac{2\tau}{n}}} \right)\\
\le\;\; 
&P^n\left(\setdef{D}{\sup_{f\in\FF}\abs{\RR_{L,P}(f)- \RR_{L,D}} \ge
B\sqrt{\frac{\tau}{2n}}}\right)\\
\overset{\atop{\text{union}}{\text{bound}}}{\le}
&\sum_{f\in\FF}
P^n\left(\setdef{D}{\abs{\RR_{L,P}(f)- \RR_{L,D}} \ge
B\sqrt{\frac{\tau}{2n}}}\right).
\end{align*}
Setzen wir $\E_P\xi_1 =\RR_{L,P}(f)$ und $\xi_i = L(x_i,y_i,f(x_i))$, so ist
$\RR_{L,D}(f) = \frac{1}{n}\sum_{i=1}^n \xi_i$ und mit der 2-seitigen
Hoeffding Ungleichung folgt
\begin{align*}
\ldots \le 
\sum_{f\in\FF} 2e^{-\tau} = 2\abs{\FF}e^{-\tau} = e^{\log(2\abs{\FF})-\tau}.
\end{align*}
Mit $-t' = \log(2\abs{\FF})-\tau$ folgt die Behauptung.\qedhere
\end{proof}

Idee: Falls $\abs{\FF} = \infty$ finde $\FF_\ep \subset\FF$ endlich, das $\FF$
gut "`approximiert"'.

\section{ERM über unendlichen Funktionenmengen}
\label{sec:4.2}

\begin{defn}
\label{defn:4.2.1}
Sei $(T,d)$ ein metrischer Raum (MR) und $\ep > 0$.
\begin{defnenum}
\item $S\subset T$ heißt \emph{$\ep$-Netz}\index{$\ep$-Netz} falls
\begin{align*}
\forall t\in T \exists s\in S : d(t,s) \le \ep.
\end{align*}
\item Die \emph{Überdeckungszahl}\index{Überdeckungszahl} (covering number) ist
definiert als
\begin{align*}
\NN(T,d,\ep) \defl \inf\setdef{n\ge 1}{\exists s_1,\ldots,s_n\in T : T\subset
\bigcup_{i=1}^n B_d(s_i,\ep)}.
\end{align*}
D.h. $S=\setd{s_1,\ldots,s_n}$ ist $\ep$-Netz. Damit entspricht $\NN(T,d,\ep)$
der Größe des kleinsten $\ep$-Netzes von $T$.
\item Sei $T\subset E$, $E$ normierter Raum. Dann ist
\begin{align*}
\NN(T,\norm{\cdot}_E,\ep) \defl \NN(T,d_E,\ep),\qquad d_E(x,x') =
\norm{x-x'}_E.\fishhere
\end{align*}
\end{defnenum}
\end{defn}

\begin{bem*}[Interpretation.]
Ist $T$ kompakt, dann ist $\NN(T,d,\ep) < \infty$ für jedes $\ep > 0$. Genauer
ist $T$ genau dann präkompakt, wenn $\NN(T,d,\ep) < \infty$ für jedes $\ep > 0$.

Da für $\ep < \ep'$ gilt $\NN(T,d,\ep')<\NN(T,d,\ep)$, stellt das
Wachstumsverhalten von $\NN(T,d,\ep)$ für $\ep \downarrow 0$ ein
quantitatives Maß für die Kompaktheit dar.

$\NN(T,d,\ep)\to \infty$ für $\ep \to 0$ genau dann, wenn $T$ unendlich.\maphere
\end{bem*}

\begin{defn}
\label{defn:4.2.2}
Sei $(T,d)$ ein metrischer Raum und $n\ge 1$.
\begin{defnenum}
\item Die $n$-te (b-a-dische) \emph{Entropiezahl}\index{Entropiezahl} von $T$
ist
\begin{align*}
e_n(T,d) = \inf\setdef{\ep >0}{\exists s_1,\ldots,s_{2^{n-1}}\in T :
T\subset\bigcup_{i=1}^{2^{n-1}} B_d(s_i,\ep)}.
\end{align*}
\item $e_n(T,\norm{\cdot}_E)$ für $T\subset (E,\norm{\cdot}_E)$ ist wie üblich
definiert.
\item Sei $S: E\to F$ ein stetige linearer Operator zwischen den normierten
Räumen $E$ und $F$ und es sei $B_E \defl \setdef{x\in E}{\norm{x}_E \le 1}$, dann
schreibe
\begin{align*}
e_n(S) = e_n(S: E\to F) \defl e_n(SB_E,\norm{\cdot}_F).\fishhere
\end{align*}
\end{defnenum}
\end{defn}

\begin{bem*}[Bemerkungen (ohne Beweis).]
\begin{bemenum}
\item $e_1(S) = \norm{S}$.
\item $e_{n+1}(S)\le e_n(S)$ für $n\ge 1$.
\item $\lim\limits_{n\to\infty} e_n(S) = 0$ genau dann, wenn $S$ kompakt.
\item $e_{n+k-1}(S+T) \le e_{n}(S) + e_k(T)$.
\item $e_{n+k-1}(S\circ T) \le e_n(S)e_k(T)$.
\item Sei $d=\rank S = \dim SE$. Dann ist $d<\infty$ genau dann, wenn
\begin{align*}
\exists c_1,c_2 : c_1 2^{-\frac{n-1}{d}} \le e_n(S) \le c_22^{-\frac{n-1}{d}}.
\end{align*}
\item Sei $X\subset \R^d$ offen und beschränkt und
\begin{align*}
C_b^m(X) \defl \setdef{f \in C^m(X\to \R)}{\max_{\abs{\alpha}\le m}
\norm{\partial^\alpha f}_\infty < \infty}
\end{align*}
versehen mit der Norm
\begin{align*}
\norm{f}_{C_b^m(X)} = \max_{\abs{\alpha}\le m} \norm{\partial^\alpha f}_\infty.
\end{align*}
Dann ist $e_n(\id: C_b^m(X)\to C_b(X)) \le C_{m,d}(X)n^{-\frac{m}{d}}$ und die
Abschätzung ist scharf, d.h. es gibt ein $\tilde{C}_{m,d}(X)$ mit $e_n(\ldots)
\ge \tilde{C}_{m,d}(x)n^{-\frac{m}{d}}$. Weiterhin ist
\begin{align*}
C_{m,d}(r\cdot X) = r^m C_{m,d}(X),\qquad r\ge 1.\maphere
\end{align*}
\end{bemenum}
\end{bem*}

\begin{lem}
\label{prop:4.2.3}
Sei $(T,d)$ MR und $a,q >0$ mit
\begin{align*}
e_n(T,d) \le an^{-\frac{1}{q}},\qquad n\ge 1.
\end{align*}
Dann gilt
\begin{align*}
\log(\NN(T,d,\ep)) \le \log(4)\left(\frac{a}{\ep}\right)^q,\qquad \ep >
0.\fishhere
\end{align*}
\end{lem}
\begin{proof}
Für $\ep > a$ ist $\NN(T,d,\ep)=1$, d.h. $\log(\NN(T,d,\ep)) = 0$ und es ist
nichts zu beweisen. Sei $\delta > 0$ und $\ep \in (0,a]$, dann existiert ein
$n\ge 1$ mit
\begin{align*}
a(1+\delta)(n+1)^{-\frac{1}{q}} \le \ep \le a(1+\delta)n^{-\frac{1}{q}}.\tag{*}
\end{align*}
Da $e_n(T,d) < a(1+\delta)n^{-\frac{1}{q}}$, gibt es ein
$a(1+\delta)n^{-\frac{1}{q}}$-Netz der Größe $2^{n-1}$, d.h.
\begin{align*}
\NN(T,d,a(1+\delta)n^{-\frac{1}{q}}) \le 2^{n-1}.
\end{align*}
Nun ist
\begin{align*}
a(1+\delta)n^{-\frac{1}{q}} &=
a(1+\delta)(n+1)^{-\frac{1}{q}}\underbrace{\left(\frac{n+1}{n}\right)^{\frac{1}{q}}}_{\le
2}
\le 2^{\frac{1}{q}}a(1+\delta)(n+1)^{-\frac{1}{q}}\\
&\overset{(*)}{\le} 2^{\frac{1}{q}}\ep.
\end{align*}
Und ferner folgt mit der rechten Seite von (*),
\begin{align*}
n \le \left(\frac{a(1+\delta)}{\ep}\right)^q.
\end{align*}
Somit ist
\begin{align*}
\log(\NN(T,d,2^{1/q}\ep))&\le 
\log(\NN(T,d,a(1+\delta)n^{-1/q}))
\le \log(2^{n-1})\\
& \le n \log 2
\le (\log2)\left(\frac{a(1+\delta)}{\ep}\right)^q 
\end{align*}
Für $\delta \to 0$ und $2^{1/q}\ep = \tilde{\ep}$ folgt die Behauptung.\qedhere
\end{proof}

\begin{prop}[Orakelungleichung II]
\index{Orakelungleichung!ERM (2)}
\label{prop:4.2.4}
Sei $L:X\times Y\times\R\to [0,\infty)$ eine lokal lipschitz-stetige
Verlustfunktion und $P$ ein W-Maß auf $(X\times Y)$. Ferner sei
\begin{align*}
\FF\subset\LL_\infty(X) = \setdef{f: X\to\R}{f\text{ messbar und beschränkt}}
\end{align*}
für die $B>0$ und $M>0$ existieren, so dass
\begin{align*}
&\norm{f}_\infty  < M, && \text{für alle }f\in\FF,\\
&L(x,y,f(x)) \le B, && \text{für alle }(x,y)\in (X\times Y),\; f\in\FF.
\end{align*}
Dann gilt für ERM über $f$,
\begin{align*}
&P^n\left( \setdef{D}{\RR_{L,P}(f_D) < \RR_{L,P,\FF}^* +
B\sqrt{\frac{2\tau+2\log\left(2\NN(\FF,\norm{\cdot}_\infty,\ep)\right)
}{n}}+4\ep\abs{L}_{M,1} } \right)\\
&\ge 1-e^{-\tau},\qquad \text{für alle }n\ge1, \ep > 0\text{ und }\tau >
0.\fishhere
\end{align*}
\end{prop}

Beachte, dass falls $\log(\NN(\FF,\norm{\cdot}_\infty,\ep)) \le \mu(\ep)$ für
$\ep > 0$, der Ausdruck nach $\ep$ optimiert werden kann. Somit erhält man eine
Abschätzung in $n\ge 1$ und $\tau > 0$.

\begin{proof}
Fixiere $\ep > 0$ und ein minimales $\ep$-Netz $\FF_\ep$ von $\FF$ bezüglich
$\norm{\cdot}_\infty$, d.h.
\begin{align*}
\NN(\FF,\norm{\cdot}_\infty,\ep) = \abs{\FF_\ep}
\end{align*}
und
\begin{align*}
\forall f\in\FF \exists g\in \FF_\ep \text{ mit } \norm{f-g}_\infty \le \ep.
\end{align*}
Somit ist für solches $f,g$
\begin{align*}
&\abs{\RR_{L,P}(f)-\RR_{L,D}(f)}\\
\le &\abs{\RR_{L,P}(f)-\RR_{L,P}(g)}
+ \abs{\RR_{L,P}(g)-\RR_{L,D}(g)}
+ \abs{\RR_{L,D}(g)-\RR_{L,D}(f)}\\
\overset{\ref{defn:2.1.9}}{\le} &2 \ep \abs{L}_{M,1} + 
\abs{\RR_{L,P}(g)-\RR_{L,D}(g)}.
\end{align*}
D.h.
\begin{align*}
\sup_{f\in\FF} \abs{\RR_{L,P}(f)-\RR_{L,D}(f)}
\le 2\ep \abs{L}_{M,1} + 
\sup_{g\in\FF_\ep}\abs{\RR_{L,P}(g)-\RR_{L,D}(g)}.
\end{align*}
Der Beweis von \ref{prop:4.1.2} zeigte außerdem, dass
\begin{align*}
\RR_{L,P}(f_D)-\RR_{L,P,\FF}^*
&\le 2 \sup_{f\in\FF} \abs{\RR_{L,P}(f)-\RR_{L,D}(f)}\\
&\le 4\ep \abs{L}_{M,1} + 2
\sup_{g\in\FF_\ep} \abs{\RR_{L,P}(g)-\RR_{L,D}(g)}. 
\end{align*}
Anwendung der Hoeffding Ungleichung und es union bounds liefert, 
\begin{align*}
&P^n\left(\setdef{D}{\RR_{L,P}(f_D)-\RR_{L,P,F}^*\ge B\sqrt{\frac{2\tau}{n}}+
4\ep \abs{L}_{M,1}} \right)\\
\le &P^n\left( \setdef{D}{\sup_{g\in\FF_\ep} \abs{\RR_{L,P}(g) -
\RR_{L,D}(g)}\ge B\sqrt{\frac{\tau}{2n}}} \right)\\
\le &2\NN\left(\FF,\norm{\cdot}_\infty,\ep \right)e^{-\tau}
= \exp\left(\log(2 \NN(\FF,\norm{\cdot}_\infty, \ep))-\tau \right).
\end{align*}
Die Behauptung folgt nun mit einer Variablentransformation
\begin{align*}
-\tau' = \log(2
\NN(\FF,\norm{\cdot}_\infty, \ep))-\tau.\qedhere
\end{align*}
\end{proof}

\chapter{Reproduzierende Kernhilberträume (RKHS)}



Es gibt zwei schöne Klassen von Banachräumen
\begin{itemize}
  \item Funktionenräume.
  \item Hilberträume.
\end{itemize}

RKHS sind beides, der $L^2([0,1])$ übrigens \textit{kein} Funktionenraum!

\section{Kerne und Beispiele}

\begin{defn}
\label{defn:5.1.1}
Eine Funktion $k: X\times X\to\R$ für die ein Hilbertraum $H$ und ein $\Phi:
X\to H$ existieren, so dass
\begin{align*}
k(x,x') = \lin{\Phi(x),\Phi(x')}_H,\qquad \forall x,x'\in X,
\end{align*}
heißt \emph{Kern}.\fishhere\index{Kern}
\end{defn}

\begin{bem*}[Bemerkungen.]
\begin{bemenum}
\item Im Allgemeinen sind $H$ und $\Phi$ nicht eindeutig.\\
Betrachte z.B. $k(x,x') = x\cdot x'$ für $x,x'$, so ist $k$ ein Kern, denn
\begin{align*}
&H_1 = \R,\quad \Phi_1 = \id_\R,\quad \lin{\Phi_1(x),\Phi_1(x')} = x\cdot x'.\\
&H_2 = \R^2,\quad \Phi_2(x) = \frac{1}{\sqrt{2}}\left(x,x\right)
\end{align*}
\item $H$ heißt \emph{Featurespace} (Merkmalraum), $\Phi$ heißt
Featuremap.\maphere
\end{bemenum}
\end{bem*}

\begin{lem}
\label{prop:5.1.2}
Seien $f_n:X\to\R$, so dass $(f_n(x))_{n\ge 1}\in l_2$ für alle $x\in X$, so
definiert
\begin{align*}
k(x,x') = \sum_{n\ge 1} f_n(x)f_n(x'),\qquad x,x'\in X
\end{align*}
einen Kern.\fishhere
\end{lem}
\begin{proof}
$k$ ist wohl definiert, denn nach Cauchy-Schwarz ist
\begin{align*}
\sum_{n\ge 1} \abs{f_n(x)f_n(x')}
\le \norm{(f_n(x))}_{l_2}\norm{(f_n(x'))}_{l_2} < \infty.
\end{align*}
Sei $H=l^2$ und $\Phi: X\to H$ gegeben durch
\begin{align*}
\Phi(x) = (f_n(x))_{n\ge 1},\qquad x\in X.
\end{align*}
Dann ist $\lin{\Phi(x),\Phi(x')} = k(x,x')$ und daher $k$ ein Kern.\qedhere
\end{proof}

\begin{lem}
\label{prop:5.1.3}
Ist $k$ ein Kern auf $X$ und $A:X'\to X$ eine Abbildung, dann definiert
\begin{align*}
k'(x,x') = k(A(x),A(x')),\qquad x,x'\in X
\end{align*}
einen Kern auf $X'$.\fishhere
\end{lem}
\begin{proof}
Sei $H$ ein Featurespace von $k$ und $\Phi: X\to H$ die zugehörige Featuremap.
Setze
\begin{align*}
\Psi(x) = \Phi(A(x)), 
\end{align*}
so gilt
\begin{align*}
k'(x,x') = k(A(x),A(x')) = \lin{\Phi(A(x)),\Phi(A(x'))}_H
= \lin{\Psi(x),\Psi(x')}_H,
\end{align*}
d.h. $\Psi$ ist Featuremap von $k'$.\qedhere
\end{proof}

\begin{lem}
\label{prop:5.1.4}
Sind $k_1$ und $k_2$ Kerne auf $X$ und $\alpha\ge 0$. Dann sind auch $k_1+k_2$
und $\alpha \cdot k_1$ Kerne auf $X$.\fishhere
\end{lem}
\begin{proof}
Der Beweis ist eine leichte Übung.\qedhere
\end{proof}

\begin{lem}
\label{prop:5.1.5}
Sind $k_1$ und $k_2$ Kerne auf $X$, so ist auch $k_1\cdot k_2$ ein Kern auf
$X$.\fishhere
\end{lem}
\begin{proof}
Der Beweis erfolgt unter Verwendung des Tensorprodukts von Hilberträumen. Wir
überspringen diesen, da wir die Aussage nicht benötigen.\qedhere
\end{proof}

\begin{lem}[Lemma für Polynom-Kerne]
\label{prop:5.1.6}
\index{Lemma!für Polynom-Kerne}
Seien $m\ge 0$, $d\ge 1$ natürliche Zahlen und $0 \le c \in\R$. Dann definiert
\begin{align*}
k(x,x') = \left(\lin{x,x'}_{\R^d} + c\right)^m,\qquad x,x'\in \R^d
\end{align*}
einen Kern auf $\R^d$.\fishhere
\end{lem}
\begin{proof}
Nach Konstruktion existiert ein Polynom $p$ der Ordnung $m$ mit
\begin{align*}
k(x,x') = p(\lin{x,x'}),\qquad x,x'\in \R^d,
\end{align*}
wobei $p$ nur nichtnegative Koeffizienten besitzt. Es genügt daher zu zeigen,
dass die Monome
\begin{align*}
(x,x') \mapsto \lin{x,x'}^m
\end{align*}
einen Kern definieren. Nun ist
\begin{align*}
\lin{x,x'}_{\R^d}^m = \left(\sum_{i=1}^d x_ix_i' \right)^m
\end{align*}
also können wir Lemma \ref{prop:5.1.2} anwenden mit $f_i(x) = \pi_i(x)$ der
Projektion auf die $i$-te Koordinate und deren Potenzen.\qedhere
\end{proof}

\begin{lem}[Lemma für Taylor-Kerne]
\label{prop:5.1.7}
\index{Lemma!für Taylor-Kerne}
%Sei $\ocirc{B}_{\R^d}\defl \setdef{x\in\R^d}{\norm{x}_2 < 1}$ und 
Sei $f: (-r,r)\to \R$
mit $r\in (0,\infty]$,
\begin{align*}
f(x) = \sum_{n=0}^\infty a_n t^n,\qquad t\in (-r,r).
\end{align*}
Falls $a_n\ge 0$ für alle $n\ge 0$, definiert
\begin{align*}
k(x,x') = f(\lin{x,x'}_{\R^d}) = \sum_{n=0}^\infty a_n \lin{x,x'}_{\R^d}^n
\end{align*}
einen Kern auf $\sqrt{r}\ocirc{B}_{\R^d}=\setdef{x\in\R^d}{\norm{x}_2 <
\sqrt{r}}$.\fishhere
\end{lem}
\begin{proof}
Für $x,x'\in \sqrt{r}\ocirc{B}_{\R^d}$ ist
\begin{align*}
\abs{\lin{x,x'}_{\R^d}} \le \norm{x}_2\norm{x'}_2 < r
\end{align*}
und daher die Konstruktion sinnvoll. Ferner seien
\begin{align*}
c_{j_1,\ldots,j_d} =
\frac{n!}{\prod_{i=1}^d j_i!},\qquad\qquad 
\sum_{j=1}^d j_i = n,
\end{align*}
dann gilt
\begin{align*}
k(x,x') &= \sum_{n=0}^\infty a_n \left( \sum_{i=1}^d x_i x_i'\right)^n
=
\sum_{n=0}^\infty a_n \sum_{\atop{j_1,\ldots,j_d\ge 0}{j_1+\ldots+j_d =n}}
c_{j_1,\ldots,j_d} \prod_{i=1}^d x_i^{j_i}(x_i')^{j_i}\\
&=
\sum_{n=0}^\infty a_n \sum_{\atop{j_1,\ldots,j_d\ge 0}{j_1+\ldots+j_d =n}}
c_{j_1,\ldots,j_d} \left(\prod_{i=1}^d x_i^{j_i}\right)\left(\prod_{i=1}^d
(x_i')^{j_i}\right).
\end{align*}
Setze $\Phi: X\to l_2(\N_0^d)$, wobei $l_2(I)\defl
\setdef{(a_i)_{i\in\II}}{\norm{(a_i)}_2 < \infty}$, dann ist
\begin{align*}
\Phi(x) = \left(\sqrt{a_{j_1+\ldots+j_d}c_{j_1,\ldots,j_d}}\prod_{i=1}^d
x_i^{j_i} \right)_{(j_1,\ldots,j_d)\in\N_0^d}
\end{align*}
eine Featuremap, denn
\begin{align*}
&\sum_{n=0}^\infty a_n \sum_{\atop{j_1,\ldots,j_d\ge 0}{j_1+\ldots+j_d =n}}
c_{j_1,\ldots,j_d} \left(\prod_{i=1}^d x_i^{j_i}\right)\left(\prod_{i=1}^d
(x_i')^{j_i}\right)\\
&=
\sum_{n=0}^\infty \sum_{\atop{j_1,\ldots,j_d\ge 0}{j_1+\ldots+j_d =n}}
\sqrt{a_{j_1+\ldots+j_d}c_{j_1,\ldots,j_d}}\left(\prod_{i=1}^d x_i^{j_i}\right)
\sqrt{a_{j_1+\ldots+j_d}c_{j_1,\ldots,j_d}}\left(\prod_{i=1}^d
(x_i')^{j_i}\right)\\
&= \lin{\Phi(x),\Phi(x')}_{l^2(\N_0^d)}.\qedhere
\end{align*}
\end{proof}


\begin{bsp}
\label{bsp:5.1.8}
\index{Exponential Kern}
\textit{Exponential Kern}. Die Abbildung
\begin{align*}
k(x,x') = \exp\left(\lin{x,x'}_{\R^d} \right),\qquad x,x'\in\R^d
\end{align*}
ist ein Kern.\bsphere
\end{bsp}

\begin{bsp}
\label{bsp:5.1.9}
\index{Gauß Kern}
\textit{Gauß Kern} oder \textit{Gauß'scher RBF-Kern}. Die Abbildung
\begin{align*}
k_\sigma(x,x') = \exp\left(-\sigma^2\norm{x-x'}_{\R^d}^2\right),\qquad
x,x'\in\R^d
\end{align*}
ist ein Kern.
\begin{proof}
Eine einfache Rechnung zeigt
\begin{align*}
\norm{x-x'}^2 &= \lin{x-x',x-x'} = 
\lin{x,x} - 2\lin{x,x'} + \lin{x',x'}\\
&= \norm{x}^2 + \norm{x'}^2 - 2\lin{x,x'}.
\end{align*}
Setzen wir $h(x) = \exp\left(\sigma^2\norm{x}_{\R^d}^2\right)$, so können wir
den Gauß-Kern schreiben als
\begin{align*}
\exp\left(-\sigma^2\norm{x-x'}_{\R^d}^2\right)
=
\frac{\exp\left(2\sigma^2\lin{x,x'}_{\R^d}\right)}
{h(x)
h(x')}.
\end{align*}
Ferner ist $(x,x')\mapsto \exp(2\sigma^2\lin{x,x'})$ ein Kern nach BSP
\ref{bsp:5.1.8} und Lemma \ref{prop:5.1.3}. Betrachten wir einen zugehörigen
Featurespace $H$ und eine Featuremap $\Phi$ und setzen
\begin{align*}
\Phi_\sigma = \frac{1}{h}\Phi : X\to H,
\end{align*}
so gilt
\begin{align*}
\lin{\Phi_\sigma(x),\Phi_\sigma(x')} &= 
\lin{\frac{\Phi(x)}{h(x)},\frac{\Phi(x')}{h(x')}}
= \frac{1}{h(x)h(x')} \lin{\Phi(x),\Phi(x')}\\
&= \frac{\exp(2\sigma^2\lin{x,x'})}{h(x)h(x')}.\qedhere\bsphere
\end{align*}
\end{proof}
\end{bsp}

\begin{defn}
\label{defn:5.1.10}
Eine Funktion $k: X\times X\to\R$ heißt \emph{positiv definit}\index{positiv
definit}, falls
\begin{align*}
\sum_{i,j=1}^n \alpha_i\alpha_j k(x_i,x_j) \ge 0
\end{align*}
für alle $n\ge 1$ und $\alpha_1,\ldots,\alpha_n\in\R$, d.h. die Grammatritzen
$(k(x_i,x_j))_{i,j=1}^n$ sind positiv definit.

$k$ heißt \emph{strikt positiv
definit}\index{positiv definit!strikt}, falls für alle paarweise verschiedenen
$x_1,\ldots,x_n$ und $\alpha_1,\ldots,\alpha_n\in\R$ gilt
\begin{align*}
\sum_{i,j=1}^n \alpha_i\alpha_j k(x_i,x_j) = 0 \Rightarrow
\alpha_1=\ldots=\alpha_n = 0.
\end{align*}

$k$ heißt \emph{symmetrisch}\index{symmetrisch} falls $k(x,x')=k(x',x)$ für alle
$x,x'\in X$.\fishhere
\end{defn}

\begin{prop}
\label{prop:5.1.11}
$k: X\times X\to\R$ ist genau dann ein Kern, wenn $k$ positiv definit und
symmetrisch.\fishhere
\end{prop}
\begin{proof}
"`$\Rightarrow$"': Sei $\Phi: X\to H$ eine Featuremap von $k$, dann gilt
\begin{align*}
k(x,x') = \lin{\Phi(x),\Phi(x')}
\end{align*}
und jedes Skalarprodukt ist positiv definit und symmetrisch
\begin{align*}
0 &\le \lin{h,h} = \lin{\sum_{i=1}^n \alpha_i\Phi(x_i),\sum_{j=1}^n \alpha_j
\Phi(x_j) }\\
&=
\sum_{i,j=1}^n \alpha_i \alpha_j \lin{\Phi(x_i),\Phi(x_j)}
= 
\sum_{i,j=1}^n \alpha_i \alpha_j k(x_i,x_j).
\end{align*}
"`$\Leftarrow$"':
Wir müssen Featurespace und -map konstruieren. Dazu setzen wir
\begin{align*}
H_\mathrm{pre} = \setdef{\sum_{i=1}^n \alpha_i k(x_i,\cdot)}{n\ge 1,\;
\alpha_1,\ldots,\alpha_n\in\R,\; x_1,\ldots,x_n\in X}
\end{align*}
und für $f=\sum_{i=1}^n \alpha_i k(x_i,\cdot)$ und $g=\sum_{j=1}^n \beta_j
k(x_j,\cdot)$ setze in $H_\mathrm{pre}$,
\begin{align*}
\lin{f,g}_{H_\mathrm{pre}} = 
\sum_{i=1}^n\sum_{j=1}^n \alpha_i \beta_j k(x_i,x_j).
\end{align*}
Weiterhin ist
\begin{align*}
&\lin{f,g}_{H_\mathrm{pre}} = 
\sum_{j=1}^m \beta_j f(x_j'),\\
&\lin{f,g}_{H_\mathrm{pre}} = 
\sum_{i=1}^m \alpha_i g(x_i),
\end{align*}
also ist $\lin{\cdot,\cdot}_{H_\mathrm{pre}}$ von der speziellen Darstellung von
$f$ oder $g$ unabhängig und daher wohldefiniert.

Offensichtlich ist $\lin{\cdot,\cdot}_{H_\mathrm{pre}}$ bilinear und
symmetrisch, da $k$ symmetrisch.

Zu zeigen ist also noch, dass $\lin{f,f}=0\Rightarrow f=0$. Wir zeigen dazu,
dass $\lin{\cdot,\cdot}_{H_\mathrm{pre}}$ die Cauchy-Schwarz'sche Ungleichung
erfüllt.

\textit{1. Fall}. Ist $\lin{f,f}=0$ und $\lin{g,g} = 0$, dann ist
\begin{align*}
&0\le \lin{f+g,f+g} = \lin{f,f} + 2\lin{f,g} + \lin{g,g} = 2\lin{f,g},\\
&0\le \lin{f-g,f-g} = \lin{f,f} - 2\lin{f,g} + \lin{g,g} = -2\lin{f,g}.
\end{align*}
Folglich ist $\lin{f,g}=0$ und die Cauchy-Schwarz'sche Ungleichung folgt. 

\textit{2. Fall}. Ohne Einschränkung ist $\lin{g,g}>0$, setzen wir also $\alpha
= - \frac{\lin{f,g}}{\lin{g,g}}$. Dann folgt
\begin{align*}
0\le \lin{f+\alpha g,f+\alpha g} = \ldots = \lin{f,f} -
\frac{\lin{f,g}^2}{\lin{g,g}}.
\end{align*}

% Somit ist $\lin{\cdot,\cdot}_{H_\mathrm{pre}}$ ein Skalarprodukt auf
% $H_\mathrm{pre}$. 

Sei nun $f\in H_\mathrm{pre}$ mit $\lin{f,f} = 0$ und
$
f= \sum_{i=1}^n \alpha_i k(x_i,\cdot).
$
Dann ist
\begin{align*}
\abs{f(x)}^2 &= \abs{\sum_{i=1}^n \alpha_i k(x_i,\cdot)}^2
=
\abs{\lin{f,k(x,\cdot)}}^2\\
&\le \lin{f,f}\lin{k(x,\cdot),k(x,\cdot)} = 0.
\end{align*}
Folglich ist $\lin{\cdot,\cdot}_{H_\mathrm{pre}}$ ein Skalarprodukt und damit
$H_\mathrm{pre}$ ein Prähilbertraum.

Zu $H_\mathrm{pre}$ existiert eine Vervollständigung $H$ mit einer isometrischen
Einbettung $J: H_\mathrm{pre}\to H$.
Setze nun $\Phi: X \to H$, $x\mapsto JK(x,\cdot)$, dann gilt
\begin{align*}
\lin{\Phi(x),\Phi(x')}_H &= \lin{Jk(x,\cdot),Jk(x',\cdot)}_H =
\lin{k(x,\cdot),k(x',\cdot)}_{H_\mathrm{pre}} \\ &= k(x,x').\qedhere
\end{align*}
\end{proof}

\begin{cor}
\label{prop:5.1.12}
Sind $k_n$, $n\ge 1$ Kerne auf $X$ und gibt es eine Funktion
$k: X\times X \to \R$ mit $k_n(x,x')\to k(x,x')$ für alle $x,x'\in X$, so ist
$k$ ein Kern.\fishhere
\end{cor}
\begin{proof}
Alle $k_n$ sind symmetrisch und daher ist es die Grenzfunktion $k$ auch.
Weiterhin gilt
\begin{align*}
0 \le \sum_{i,j=1}^d \alpha_i\alpha_j k_n(x_i,x_j) \to
\sum_{i,j=1}^n \alpha_i\alpha_j k(x_i,x_j)
\end{align*}
und daher ist $k$ positiv definit.\qedhere
\end{proof}

\section{RKHS (Reproduzierende Kern-Hilbert-Räume)}

Bisher waren Featurespace und -map nicht eindeutig, es gibt aber eine kanonische
Wahl, den RKHS.

\begin{defn}
\label{defn:5.2.1}
\begin{defnenum}
\item Ein Hilbertraum $H$, der aus Funktionen $f: X\to\R$ besteht, heißt
Hilbertfunktionenraum über $X$. (HFS)\index{Hilbertfunktionenraum}
\item Ein HFS $H$ über $X$ heißt RKHS, falls die Dirac-Funktionale
\begin{align*}
\delta_x: H\to\R,\quad f\mapsto f(x)
\end{align*}
für alle $x\in X$ stetig sind.
\item\label{defn:5.2.1:3} Eine Funktion $k: X\times X\to \R$ heißt
\emph{reproduzierender Kern}
\index{reproduzierender Kern}
eines HFS über $X$, falls für jedes $x\in X$ $k(x,\cdot)\in H$ und
die \emph{reproduzierende Eigenschaft}\index{reproduzierende Eigenschaft} erfüllt ist,
\begin{align*}
f(x) = \lin{f,k(x,\cdot)},\qquad \text{für alle } f\in H,\; x\in X.\fishhere
\end{align*}
\end{defnenum}
\end{defn}

\begin{bem*}[Bemkerungen.]
\begin{bemenum}
\item $L_2([0,1])$ ist kein RKHS (nicht mal ein HFS).
\item Norm-Konvergenz in einem RKHS $H$ impliziert punktweise Konvergenz.
\begin{proof}
Sei $x\in X$ und gelte $\norm{f_n-f}_H\to 0$, dann gilt
\begin{align*}
f(x) = \delta_x(f) =
\delta_x\left(\lim\limits_{n\to\infty} f_n\right) =
\lim\limits_{n\to\infty} \delta_x(f_n) 
= \lim\limits_{n\to\infty} f_n(x).\qedhere
\end{align*}
\end{proof}
\item \ref{defn:5.2.1:3} wurde für den im Beweis von Satz \ref{prop:5.1.11}
konstruierten Hilbertraum schon "`fast"'
gezeigt.\maphere
\end{bemenum}
\end{bem*}

\begin{lem}
\label{prop:5.2.2}
Sei $H$ ein HFS über $X$ und $k: X\times X\to \R$ ein reproduzierender Kern von
$H$. Dann gelten
\begin{propenum}
\item $H$ ist ein RKHS.
\item $k$ ist ein Kern, $H$ ist ein FS von $k$ und
\begin{align*}
\Phi: X\to H,\qquad x\mapsto k(x,\cdot)
\end{align*}
die \emph{kanonische Featuremap} von $k$.\fishhere
\end{propenum}
\end{lem}
\begin{proof}
\begin{proofenum}
\item Eine direkte Rechung zeigt
\begin{align*}
\abs{\delta_x(f)} = \abs{f(x)} = \abs{\lin{f,k(x,\cdot)}_H} \le
\norm{f}_H\norm{k(x,\cdot)}_H.
\end{align*}
Somit ist $\delta_x$ stetig und $\norm{\delta_x} \le \norm{k(x,\cdot)}$.
\item Für $x'\in X$ fest schreibe $f\defl k(x',\cdot)$. Dann ist $f\in H$ per
definitionem und
\begin{align*}
\lin{\Phi(x),\Phi(x')}_H &= \lin{k(x,\cdot),f}_H = \lin{f,k(x,\cdot)}_H = f(x) =
k(x',x)\\ &= \lin{\Phi(x'),\Phi(x)}.\qedhere
\end{align*}
\end{proofenum}
\end{proof}

\begin{prop}[Existenz und Eindeutigkeit des reproduzierenden Kerns]
\label{prop:5.2.3}
Sei $H$ ein RKHS über $X$. Dann gilt
\begin{align*}
k: X\times X\to \R,\qquad
k(x,x')=\lin{\delta_x,\delta_{x'}},
\end{align*}
ist der einzige reproduzierende Kern von $H$. Ist ferner $(e_i)_{i\in\II}$ eine
Orthonormalbasis (ONB), so gilt
\begin{align*}
k(x,x') = \sum_{i\in\II} e_i(x) e_i(x'),\qquad x,x'\in X.\fishhere
\end{align*}
\end{prop}
\begin{proof}
\textit{Existenz}. Der Darstellungssatz von Fréchet-Riesz besagt, dass
\begin{align*}
I : H\to H',\qquad f\mapsto \lin{f,\cdot}_H
\end{align*}
ein isometrischer Isomorphismus ist, d.h. $H=H'$. Sei $J=I^{-1}: H'\to H$, so
ist auch $J$ ein isometrischer Isomorphismus und  für $f\in H$ und $\omega\in
H'$ gilt
\begin{align*}
\omega(f) = \lin{J\omega,f}_H.\tag{*}
\end{align*}
Für $x,x'$ folgt
\begin{align*}
k(x,x') = \lin{\delta_x,\delta_{x'}}_{H'} = \lin{J\delta_x,J\delta_{x'}}_H
\overset{\text{(*)}}{=} \delta_x(J\delta_{x'}) = J\delta_{x'}(x).
\end{align*}
Folglich ist $J\delta_{x'} = k(\cdot,x')$ und damit
\begin{align*}
f(x') - \delta_{x'}(f) = \lin{f,J\delta_{x'}}_H = 
\lin{f,k(\cdot,x')}_H = \lin{f,k(x',\cdot)}_H,
\end{align*}
d.h. $k$ ist reproduzierender Kern von $H$.

\textit{Eindeutigkeit}. Sei $\tilde{k}$ ein beliebiger reproduzierender Kern von
$H$.

Für $x'\in X$ gilt $k(x',\cdot)\in H$. Für die ONB $(e_i)_{i\in\II}$ folgt
\begin{align*}
k(x',\cdot) = \sum_{i\in\II} \lin{k(x',\cdot), e_i}_He_i =
\sum_{i\in\II} e_i(x')e_i,
\end{align*}
wobei
\begin{align*}
k(x',x) = \sum_{i\in\II } e_i(x')e_i(x)
\end{align*}
punktweise existiert. Insbesondere ist $\tilde{k}$ der einzige reproduzierende
Kern von $H$ und die Darstellung mithilfe der ONB $(e_i)_{i\in\II}$ hängt nicht
von der Wahl der ONB ab.\qedhere
\end{proof}

\begin{prop}
\label{prop:5.2.4}
Sei $k$ ein Kern über $X$ mit Featurespace $H_0$ und Featuremap $\Phi_0:X\to
H_0$. Dann gelten
\begin{propenum}
\item\label{prop:5.2.4:1} $H\defl\setdef{f: X\to\R}{\exists \omega\in H_0 : f(x) =
\lin{\omega,\Phi_0(x)}_{H_0} \forall x\in X}$ mit
\begin{align*}
\norm{f}_H \defl \inf\setdef{\norm{\omega}_{H_0}}{\lin{\omega,\Phi_0(\cdot)}_{H_0}
= f}
\end{align*}
ist der einzige RKHS von dem $k$ ein reproduzierender Kern ist.
\item\label{prop:5.2.4:2} Der Operator $V: H_0\to H$, $\omega\mapsto
\lin{\omega,\Phi_0(\cdot)}_{H_0}$ ist eine metrische Surjektion, d.h. linear,
beschränkt und $V\ocirc{B}_{H_0} = \ocirc{B}_H$.
\item\label{prop:5.2.4:3} $H_\pre = \setdef{\sum_{i=1}^n \alpha_i
k(x_i,\cdot)}{n\ge 1,\; \alpha_i\in \R,\; x_i\in X}$ ist dicht in $H$.
\item\label{prop:5.2.4:4} Für $f\in H_\pre$ mit $f=\sum_{i=1}^n \alpha_i
k(x_i,\cdot)$ gilt
\begin{align*}
\norm{f}_H^2 = \sum_{i,j=1}^n \alpha_i\alpha_j k(x_i,x_j).\fishhere 
\end{align*}
\end{propenum}
\end{prop}

\begin{bem*}[Verleich.]
\ref{prop:5.1.11}: Der FS ist eine Vervollständigung von $H_\pre$.

\ref{prop:5.2.4}: Es existiert eine Vervollständigung, die aus Funktionen $X\to
\R$ besteht und die RKHS ist.\maphere
\end{bem*}

\begin{proof}
\begin{proofenum}
\item
Zeige $H$ ist HFS über $X$. Offensichtlich besteht $H$ aus Funktionen $X\to \R$
und $V: H_0\to H$ ist linear. Für $f\in H$ gilt ferner
\begin{align*}
\norm{f}_H = \inf\setdef{\norm{\omega}_{H_0}}{\omega\in V^{-1}(\setd{f})}.
\end{align*}
Zeige $\norm{\cdot}_H$ ist Hilbertraumnorm auf $H$.
\begin{proofenuma}
\item $\ker V = \setdef{\omega\in H_0}{V\omega = 0}=V^{-1}(\setd{0})$ ist
abgeschlossen, da $V$ stetig.
% Für eine Folge $(\omega_n)$ in $\ker V$ mit $\omega_n\to \omega\in H_0$ gilt
% \begin{align*}
% \lin{\omega,\Phi_0(x)}_{H_0} = \lim\limits_{n\to\infty}
% \underbrace{\lin{\omega_n,\Phi_0(x)}_{H_0}}_{V\omega_n(x) = 0} = 0,
% \end{align*}
% d.h. $V\omega(x) = 0$ für alle $x\in X$ und daher ist $\omega\in \ker V$.
\item Sei $\tilde{H}$ das orthogonale Komplement von $\ker V$ in $H_0$. 
%, also $\tilde{H}\bot \ker V$ und $H_0= \tilde{H}\oplus \ker V$.
Dann ist
\begin{align*}
V\big|_{\tilde{H}} : \tilde{H}\to H
\end{align*}
injektiv nach Konstruktion und außerdem surjektiv, denn $V$ ist surjektiv, d.h.
zu $f\in H$ existiert ein $\omega\in H_0$ mit $V\omega = f$. Dann ist
\begin{align*}
&\omega = \tilde{\omega} + \omega_0,\quad
\tilde{\omega} \in \tilde{H},\; \omega_0\in H_0,\\
&f = V(\tilde{\omega}+\omega_0) = V(\tilde{\omega}) +
\underbrace{V(\omega_0)}_{=0} = V\big|_{\tilde{H}}(\tilde{\omega}).
\end{align*}
\item Zeige $\norm{f}_H^2 = \norm{(V\big|_{\tilde{H}})f}_H$.
\begin{align*}
\norm{f}_H^2 &= \inf \setdef{\norm{\omega_0 +
\tilde{\omega}}_{H_0}^2}
{\omega_0\in \ker V, \tilde{\omega}\in H} \\ &
= \inf \setdef{\norm{\omega_0}_{H_0}^2 + \norm{\tilde{\omega}}_{H_0}^2}
{\omega_0\in \ker V, \tilde{\omega}\in H}\\
&= \norm{V\big|_{\tilde{H}}^{-1}(f)}_{\tilde{H}}^2.
\end{align*}
Damit ist $H$ ein Hilbertraum, da $\tilde{H}$ ein Hilbertraum ist,
$V\big|_{\tilde{H}}: \tilde{H}\to H$ bijektiv linear und isometrisch.

Ferner folgt, dass $V$ eine metrische Surjektion ist.
\end{proofenuma}
\item
Zeige $H$ ist ein RKHS und $k$ ein reproduzierender Kern von $H$.
\begin{proofenuma}
\item Für $x\in X$ gilt
\begin{align*}
k(x,\cdot) = \lin{\Phi_0(x),\Phi_0(\cdot)}_{H_0} = V\Phi_0(x)
\end{align*}
und folglich ist $k(x,\cdot)\in H$.
\item $\Phi_0(x)\in \tilde{H} = (\ker V)^\bot$, denn es gilt für $\omega\in \ker
V$,
\begin{align*}
\lin{\omega, \Phi_0(x)}_{H_0} = V_\omega(x) = 0,\qquad x\in X.
\end{align*}
\item Zeige, dass $f(x)= \lin{f,k(x,\cdot)}_H$. Da $f =
V(V\big|_{\tilde{H}})^{-1}f$ und $V\big|_{\tilde{H}}$ Isometrie, folgt
\begin{align*}
f(x) = \lin{(V\big|_{\tilde{H}})^{-1}f,\Phi_0(x)}_{H_0} = 
\lin{f,V\big|_{\tilde{H}},\Phi_0(x)}_H = \lin{f, k(x,\cdot)}_H.
\end{align*}
Somit ist $k$ ein reproduzierender Kern und nach Lemma \ref{prop:5.2.2} $H$ ein
RKHS.
\end{proofenuma}
\item Zeige \ref{prop:5.2.4:3} und \ref{prop:5.2.4:4} gilt für jeden RKHS
$\bar{H}$ für den $k$ reproduzierender Kern ist. Wir zeigen dazu, dass

\begin{align*}
H_\pre = \setdef{\sum_{i=1}^n \alpha_i k(x_i)}{n\ge 1,\; \alpha_i\in\R,\;
x_i\in X}
\end{align*}
dicht ist in $\bar{H}$ und für $f = \sum_{i=1}^n \alpha_i k(x_i,\cdot)$ gilt
\begin{align*}
\norm{f}_{\bar{H}}^2 = \sum_{i,j=1}^n \alpha_i \alpha_j k(x_i,x_j).
\end{align*}
\begin{proofenuma}
\item $k(x,\cdot) \in H_\pre$ und folglich ist $H_\pre \subset \bar{H}$ und
$k(x,\cdot)\in \bar{H}$.
\item Zeige, dass $\bar{H_\pre}^{\bar{H}} = \bar{H}$. Angenommen
$\bar{H_\pre}^{\bar{H}} \subsetneq \bar{H}$, dann ist ${H_\pre}^\bot \neq (0)$,
d.h. es existiert ein $f\in {H_\pre}^\bot$ und ein $x\in X$ mit $f(x)\neq 0$
und $f\ \bot\  k(x,\cdot)$.
\begin{align*}
\Rightarrow 0 \neq f(x) = \lin{f,k(x,\cdot)}_{\bar{H}} = 0.\dipper 
\end{align*}
D.h. $\bar{H_\pre}^\bar{H}=\bar{H}$, also ist $H_\pre$ dicht in $\bar{H}$.

Sei nun $f=\sum_{i=1}^n \alpha_i k(x_i,\cdot)$, dann gilt
\begin{align*}
\norm{f}_{\bar{H}}^2 &= \norm{\sum_{i=1}^n \alpha_i k(x_i,\cdot)}_{\bar{H}}^2
= \lin{\sum_{i=1}^n \alpha_i k(x_i,\cdot),\sum_{j=1}^n \alpha_j k(x_j,\cdot)}\\
&= \sum_{i,j=1}^n \alpha_i\alpha_j \lin{k(x_i,\cdot),k(x_j,\cdot)}_{\bar{H}}
= \sum_{i,j=1}^n \alpha_i\alpha_j k(x_i,x_j)
\end{align*}
aufgrund der repräsentierenden Eigenschaft angewandt auf $g=k(x_i,x\cdot)$.
\end{proofenuma}
\item Zeige schließlich $H$ ist der einzige RKHS für den $k$ ein
reproduzierender Kern ist.

Seien dazu $H_1$ und $H_2$ RKHS für die $k$ ein reproduzierender Kern ist. Nach
dem bisher Gezeigten ist $H_\pre$ dicht in $H_1$ und $H_2$ und die Normen von
$H_1$ und $H_2$ sind auf $H_\pre$ identisch. Wir zeigen nun, dass $H_1\subset
H_2$ und $\norm{f}_{H_1}= \norm{f}_{H_2}$ für alle $f\in H_1$. Dann folgt die
Gleichheit aus der Symmetrie des Problems und $H$ ist eindeutig.

Sei also $f\in H_1$, dann existiert eine Folge $(f_n)$ in $H_\pre$ mit
\begin{align*}
\norm{f_n-f}_{H_1}\to 0,\qquad n\to\infty.
\end{align*}
Folglich ist $(f_n)$ Cauchyfolge in $H_\pre$
bezüglich $\norm{\cdot}_{H_1}$ und damit auch bezüglich $\norm{\cdot}_2$, also
ist $(f_n)$ Cauchyfolge in $H_2$, d.h. es existiert ein $g\in H_2$, so dass
\begin{align*}
\norm{f_n-g}_{H_2} \to 0,\qquad n\to\infty.
\end{align*}
Da die Konvergenz in $H_{1,2}$ der punktweisen Konvergenz entspricht gilt daher
auch
\begin{align*}
f(x) = \lim\limits_{n\to\infty} f_n(x) = g(x),\qquad x\in X.
\end{align*}
Folglich ist $f=g\in H_2$ und daher $H_1\subset H_2$. Da außerdem
\begin{align*}
\norm{f_n-f}_{H_2} \to 0,\qquad n\to\infty
\end{align*}
folgt $\norm{f_n}_{H_2}\to \norm{f}_{H_2}$, d.h.
\begin{align*}
\norm{f}_{H_1} =\lim\limits_{n\to\infty} \norm{f_n}_{H_1}
=
\lim\limits_{n\to\infty} \norm{f_n}_{H_2}
=
\norm{f}_{H_2}.\qedhere
\end{align*}
\end{proofenum}
\end{proof}

\begin{bem*}[Interpretation.]
\begin{itemize}
  \item[-] Zu jedem Kern $k$ existiert genau ein RKHS, so dass $k$ ein
  reproduzierender Kern von $H$ ist.
  \item[-] Zu jedem RKHS $H$ existiert genau ein reproduzierender Kern von $H$
  und dieser ist auch ein Kern.
\end{itemize}
\noindent
Zwischen Kern und RKHS besteht also eine 1:1 Relation.

Der RKHS ist der "`kleinste Featurespace"' im Sinne von Surjektionen. $H$
besteht aus den Funktionen, die entstehen, wenn Daten in einen Featurespace $H_0$
abgebildet werden und dann ein linearer Ansatz gemacht wird.\maphere
\end{bem*}

\section{Eigenschaften von Kernen und RKHS}

Ziel dieses Abschnittes ist es, eine Beziehung zwischen den Eigenschaften von
Kernen und den Eigenschaften der Funktionen eines RKHS herzustellen.

\begin{lem}
\label{prop:5.3.1}
Sei $k$ ein Kern auf $X$ und $H$ der RKHS von $k$, dann sind äquivalent
\begin{equivenum}
\item\label{prop:5.3.1:1} $k$ ist beschränkt.
\item\label{prop:5.3.1:2} Alle $f\in H$ sind beschränkt.
\end{equivenum}
In diesem Fall gilt zudem
\begin{align*}
\norm{\id: H\to \LL_\infty(X)} = \norm{k}_\infty ,\qquad
\norm{k}^2_\infty = \sup_{x\in X} k(x,x) < \infty.\fishhere
\end{align*}
% mit
% \begin{align*}
% 
% \end{align*}
\end{lem}
\begin{proof}
"`\ref{prop:5.3.1:1}$\Rightarrow$\ref{prop:5.3.1:2}"': Für $f\in H$ und $x\in X$
gilt da $k(x,\cdot)\in H_\pre$ nach Satz \ref{prop:5.2.4},
\begin{align*}
\abs{f(x)} &= \abs{\lin{f,k(x,\cdot)}_H} \le \norm{f}_H\norm{k(x,\cdot)}_H
= \norm{f}_H \sqrt{k(x,x)}\\
&\le \norm{f}_H \underbrace{\norm{k}_\infty}_{< \infty}.
\end{align*}
Somit ist $f\in \LL_\infty(X)$ und $\norm{f}_\infty \le \norm{k}_\infty
\norm{f}_H$.
Also gilt insbesondere
\begin{align*}
\norm{\id : H\to \LL_\infty(X)} \le \norm{k}_\infty.
\end{align*}
"`\ref{prop:5.3.1:2}$\Rightarrow$\ref{prop:5.3.1:1}"': $\id: H\to \LL_\infty(X)$
ist wohldefiniert nach Voraussetzung. Wir zeigen, dass $\id: H\to \LL_\infty(X)$
stetig ist. Dazu benutzen wir den Satz vom abgeschlossenen Graphen. Sei dazu
$(f_n)$ eine Folge in $H$ und $g\in \LL_\infty(X)$ mit
\begin{align*}
\norm{f_n-f}_H \to 0,\qquad \norm{\id(f_n) - g}_{\LL_\infty(X)} \to 0,\tag{*}
\end{align*}
so ist zu zeigen, dass $f=g$.

Nach (*) gelten
\begin{align*}
f_n(x)\to f(x),\qquad f_n(x)\to g(x),\qquad \forall x\in X.
\end{align*}
Somit ist $f(x) = g(x)$ für alle $x\in X$ und daher ist
\begin{align*}
\id : H\to \LL_\infty(X)
\end{align*}
%abgeschlossen also stetig?
stetig. Weiterhin ist
\begin{align*}
k(x,x) &\le \norm{k(x,\cdot)}_\infty
\le \norm{\id: H \to \LL_\infty(x)}\norm{k(x,\cdot)}_H\\
&= \norm{\id: H\to \LL_\infty(x)}\sqrt{k(x,x)}
\end{align*}
und daher
\begin{align*}
&\sqrt{k(x,x)} \le \norm{\id: H\to \LL_\infty(x)},\\
\Rightarrow &
\norm{k}_\infty \le  \norm{\id: H\to \LL_\infty(x)}.
\end{align*}
Insbesondere ist $k$ auf der Diagonalen beschränkt. Schließlich ist
\begin{align*}
\abs{k(x,x')} &= \abs{\lin{k(x,\cdot),k(x',\cdot)}_H}
\le \norm{k(x,\cdot)}_H\norm{k(x',\cdot)}_H\\
&= \sqrt{k(x,x)}\sqrt{k(x',x')}
\le \norm{k}_\infty^2.\qedhere
\end{align*}
\end{proof}

\begin{lem}
\label{prop:5.3.2}
Sei $k$ ein Kern auf $X$ und $H$ der RKHS von $k$, dann sind äquivalent
\begin{equivenum}
\item\label{prop:5.3.2:1} $k(x,\cdot)$ ist messbar für alle $x\in X$.
\item\label{prop:5.3.2:2} Alle $f\in H$ sind messbar.\fishhere
\end{equivenum}
\end{lem}
\begin{proof}
"`\ref{prop:5.3.2:2}$\Rightarrow$\ref{prop:5.3.2:1}"': $k(x,\cdot)\in H$.

"`\ref{prop:5.3.2:1}$\Rightarrow$\ref{prop:5.3.2:2}"': $k(x,\cdot)$ ist messbar
und folglich besteht
\begin{align*}
H_\pre = \span \setdef{h(x,\cdot)}{x\in X}
\end{align*}
aus messbaren Funktionen. Sei nun $f\in H$, dann existiert eine Folge $(f_n)$ in
$H_\pre$ mit $\norm{f_n-f}_H\to 0$ und folglich ist $f(x) =
\lim\limits_{n\to\infty} f_n(x)$ für alle $x\in X$.\qedhere
\end{proof}

\begin{lem}
\label{prop:5.3.3}
Sei $k$ Kern auf $X$ mit RKHS $H$ und $k(x,\cdot)$ messbar für alle $x\in X$ und
$H$ separabel. Dann ist $k$ messbar und die kanonische Featuremap $\Phi: X\to H$
ist messbar.\fishhere
\end{lem}
\begin{proof}
Wir überspringen den Beweis.\qedhere
\end{proof}

\begin{lem}
\label{prop:5.3.4}
Sei $X$ ein Hausdorff-Raum, $k$ ein Kern auf $X$ mit RKHS $H$. Dann sind
äquivalent
\begin{equivenum}
\item\label{prop:5.3.4:1} $k$ ist beschränkt und separat stetig, d.h.
$k(x,\cdot)$ ist stetig für alle $x\in X$.
\item\label{prop:5.3.4:2} Alle $f\in H$ sind stetig und beschränkt. In diesem
Fall gilt
\begin{align*}
\norm{\id: H\to C_b(X)} = \norm{k}_\infty < \infty.\fishhere
\end{align*}
\end{equivenum}
\end{lem}
\begin{proof}
"`\ref{prop:5.3.4:1}$\Rightarrow$\ref{prop:5.3.4:2}"': $H_\pre \subset C_b(X)$
nach Voraussetzung. Sei $f\in H$, dann existiert eine Folge $(f_n)$ in $H_\pre$
mit $\norm{f_n-f}_H \to 0$. Da $k$ beschränkt folgt mit Lemma
\ref{prop:5.3.1}, dass
\begin{align*}
\norm{f_n-f}_\infty \to 0
\end{align*}
und folglich ist $f\in C_b(X)$.

"`\ref{prop:5.3.4:2}$\Rightarrow$\ref{prop:5.3.4:1}"': $k(x,\cdot)\in H$ ist
stetig nach Voraussetzung und beschränkt nach Lemma \ref{prop:5.3.1}. Somit ist
$k$ beschränkt und $\norm{k}_\infty = \norm{\id: H\to C_b(X)}$.\qedhere

\end{proof}

\begin{defn}
\label{defn:5.3.5}
Sei $k$ ein Kern auf $X$ mit RKHS $H$ und kanonischer Featuremap $\Phi: X\to H$.
Dann heißt $d_k: X\times X\to [0,\infty)$
\begin{align*}
d_k(x,x') \defl \norm{\Phi(x)-\Phi(x')}_H
= \sqrt{k(x,x)+k(x',x')-2k(x,x')}
\end{align*}
die \emph{Kernmetrik}\index{Kernmetrik} von $k$.\fishhere
\end{defn}

Man sieht leicht ein, dass $d_k$ genau dann eine Metrik ist, wenn $\Phi$
injektiv ist.
Im Allgemeinen ist $d_k$ lediglich eine Semi-Metrik.

\begin{lem}
\label{prop:5.3.6}
Sei $(X,\tau)$ ein Hausdorff-Raum, $k$ ein Kern auf $X$ und $H$ ein RKHS von
$k$, $\Phi: X\to H$ die kanonische Featuremap. Dann sind äquivalent
\begin{equivenum}
\item\label{prop:5.3.6:1} $k$ ist stetig.
\item\label{prop:5.3.6:2} $k$ ist separat stetig und $x\mapsto k(x,x)$ ist
stetig.
\item\label{prop:5.3.6:3} $\Phi: X\to H$ ist stetig.
\item\label{prop:5.3.6:4} $\id : (X,\tau) \to (X,d_k)$ ist stetig, d.h. jede
$d_k$-offene Kugel ist $\tau$-offen.\fishhere
\end{equivenum}
\end{lem}

\begin{proof}
"`\ref{prop:5.3.6:1}$\Rightarrow$\ref{prop:5.3.6:2}"': Trivial.

"`\ref{prop:5.3.6:2}$\Rightarrow$\ref{prop:5.3.6:4}"': $d_k(\cdot,x):
(X,\tau)\to \R$ ist stetig und daher ist die Kugel
\begin{align*}
B_\ep(x',d_k) = \setdef{x'}{d_k(x,x') < \ep}
\end{align*}
$\tau$-offen. Diese Kugeln erzeugen aber gerade die Topologie von $d_k$, d.h.
die Toplogie von $\tau$ ist größer oder gleich der Topologie von $d_k$.

"`\ref{prop:5.3.6:4}$\Rightarrow$\ref{prop:5.3.6:3}"': $\Phi: (X,d_k)\to H$ ist
stetig nach Konstruktion von $d_k$ und somit ist $\Phi: (X,\tau)\to H$ stetig
nach Voraussetzung.

"`\ref{prop:5.3.6:3}$\Rightarrow$\ref{prop:5.3.6:1}"': Seien $(x_n)$ und
$(y_n)$ Folgen in $X$ mit $x,y\in X$, so dass $x_n\to x$ und $y_n\to y$, so gilt
\begin{align*}
\abs{k(x_n,y_n)-k(x,y)} &= \abs{\lin{\Phi(y_n),\Phi(x_n)-\Phi(x)}+
\lin{\Phi(x),\Phi(y_n)-\Phi(y)}}\\
&\le \norm{\Phi(y_n)}\norm{\Phi(x_n)-\Phi(x)} +
\norm{\Phi(x)}\norm{\Phi(y_n)-\Phi(y)}\\
&\le \norm{\Phi(y_n)}d_k(x_n,x) + \norm{\Phi(x)}d_k(y_n,y). 
\end{align*}
Nach Voraussetzung ist $\Phi$ stetig, d.h. $\norm{\Phi(y_n)}$ ist beschränkt.\qedhere 
\end{proof}

\begin{prop}
\label{prop:5.3.7}
Sei $X$ ein kompakter Hausdorffraum und $k$ ein stetiger Kern mit RKHS $H$. Dann
ist die Abbildung
\begin{align*}
\id : H\to C(X)
\end{align*}
wohldefiniert und kompakt.\fishhere
\end{prop}
\begin{proof}
Da $X$ kompakt und $k$ stetig, ist $k$ auch beschränkt. Foglich ist $\id :H\to
C(X)$ wohldefiniert und stetig (Lemma \ref{prop:5.3.4}).

Sei $C(X,d_k)$ der Raum der $d_k$-stetigen Funktionen $f: X\to \R$.
Offensichtlich ist $C_b(X,d_k)\subset l_\infty(X)$. Außerdem ist
$X$  kompakt und $\Phi$ stetig, d.h. $\Phi(X)\subset H$ ist kompakt. Da
$\Phi$ Isometrie bezüglich $d_k$, ist $(X,d_k)$ kompakt und folglich
\begin{align*}
C(X,d_k) = C_b(X,d_k).
\end{align*} 
Für $f\in B_H$ und $x,y\in H$ gilt
\begin{align*}
\abs{f(x)-f(y)} = \abs{\lin{f,\Phi(x)-\Phi(y)}_H}
\le \norm{f}\norm{\Phi(x)-\Phi(y)}
\le d_k(x,y).
\end{align*}
Also sind alle Funktionen in $B_H$ lipschitz bezüglich $d_k$ mit
lipschitz-Konstante $\le 1$. Der Satz von Arzelà-Ascoli besagt nun, dass
\begin{align*}
\bar{B_H} \text{ kompakt ist in } C(X,d_k).
\end{align*}
Somit ist $\id_{H\to C(X,d_k)}(B_H)$ relativkompakt und daher $\id: H\to
C(X,d_k)$ kompakt. Da $C(X,d_k)\subset C(X)$ mit Normgleichheit und
\begin{align*}
H \overset{\id}{\longrightarrow} C(X,d_k)
\overset{\text{stetig}}{\opento} C(X)
\end{align*}
ist $\id: H\to C(X)$ kompakt.\qedhere
\end{proof}

\begin{bem*}
Ist $X$ \textit{nicht} kompakt aber $k$ stetig und beschränkt, so ist
\begin{align*}
\id: H\to C_b(X)
\end{align*}
wohldefiniert und stetig aber im Allgemeinen \textit{nicht} kompakt.

Betrachte z.B. $X=\R$ und den Gauß-Kern $k_\sigma(x,y) =
\exp(-\sigma^2\norm{x-y}_2^2)$. Seien $m,n\in\N$ mit $n\neq m$, so gilt
\begin{align*}
\norm{k_\sigma(m,\cdot)-k_\sigma(n,\cdot)}_\infty
&\ge
\abs{k_\sigma(m,m)-k_\sigma(n,m)}\\
&= 1- \exp(-\sigma^2\abs{m-n}^2) \ge 1-e^{-\sigma^2},
\end{align*}
da $\abs{m-n} \ge 1$. Es ist aber
\begin{align*}
\norm{k_\sigma(n,\cdot)}_H = \sqrt{k_\sigma(n,n)} = 1
\end{align*}
und folglich $\id: H\to C_b(X)$ nicht kompakt.\maphere
\end{bem*}

\begin{lem}
\label{prop:5.3.8}
Sei $X$ ein separabler, metrischer Raum und $k$ ein stetiger Kern mit RKHS $H$.
Dann ist $H$ separabel.\fishhere
\end{lem}
\begin{proof}
$\Phi: X\to H$ ist stetig, da $k$ stetig und somit ist $\Phi(X)$ separabel.
Folglich ist auch $H_\pre = \mathrm{span}\ \Phi(X)$ separabel, denn man kann
$\mathrm{span}$ durch Linearkombinationen mit rationalen Koeffizienten
approximieren, und damit ist auch $H=\bar{H_\pre}$ separabel.~\qedhere
\end{proof}

Wir wollen nun untersuchen, wie sich die Differenzierbarkeit des Kerns auf die
Featuremap und den RKHS übertragen. Dazu betrachten wir einen Kern $k:
\R^d\times \R^d\to \R$ und die Abbildung
\begin{align*}
\tilde{k}: \R^{2d}\to \R,\qquad (x,y)\mapsto \tilde{k}((x,y)) = k(x,y).
\end{align*}
Falls die partiellen Ableitungen von $i$ und $i+d$ von $\tilde{k}$ existieren,
setzen wir
\begin{align*}
\partial_i \partial_{i+d} k\defl \partial_i \partial_{i+d}\tilde{k}.
\end{align*}
Analog lässt sich dies definieren, wenn $X\subset\R^d$ offen.

\begin{lem}
\label{prop:5.3.9}
Sei $X\subset\R^d$ offen, $k$ ein Kern mit Featurespace $H$ und Featuremap
$\Phi:~X\to H$, so dass $\partial_i\partial_{i+d} k$ für
$1\le i\le d$ existiert und stetig ist. Dann existiert auch
\begin{align*}
\partial_i \Phi: X\to H
\end{align*}
und ist stetig und es gilt
\begin{align*}
\lin{\partial_i \Phi(x),\partial_i\Phi(y)}_H = 
\partial_i\partial_{i+d}k(x,y) =
\partial_{i+d}\partial_{i}k(x,y).\fishhere 
\end{align*}
\end{lem}
\begin{proof}
Wir beweisen nur den Fall $X=\R^d$. Sei $e_i$ der $i$-te Einheitsvektor, dann
definiere für festes $h>0$,
\begin{align*}
\Delta_h \Phi(x) \defl \Phi(x+he_i) - \Phi(x).
\end{align*}

Wir zeigen nun, dass $(h_n^{-1}\Delta_{h_n}\Phi(x))_n$ für alle Folgen $h_n\to 0$
mit $h_n\neq 0$ konvergiert. Dann ist $\lim\limits_{n\to\infty} h_n^{-1}
\Delta_{h_n} \Phi(x) \defr \partial_i \Phi(x)$ unabhängig von der Folge $(h_n)$. Da
$H$ vollständig ist, genügt es zu zeigen, dass $(h_n^{-1}\Delta_{h_n}\Phi(x))$
eine Cauchyfolge ist. Es ist
\begin{align*}
&\norm{h_n^{-1}\Delta_{h_n}\Phi(x)-h_m^{-1}\Delta_{h_m} \Phi(x)}^2_H\\
&\quad= \norm{h_n^{-1}\Delta_{h_n}\Phi(x)}_H^2
+ \norm{h_m^{-1}\Delta_{h_m}\Phi(x)}_H^2\\
&\quad- 2 \lin{h_n^{-1}\Delta_{h_n} \Phi(x),h_m^{-1}\Delta h_m \Phi(x)}\\
&\quad\le
2\left(\ep + \partial_i\partial_{i+d}k(x,x)
-\lin{h_n^{-1}\Delta_{h_n} \Phi(x),h_m^{-1}\Delta_{h_m} \Phi(x)}\right)\tag{*}
\end{align*}
Wir definieren für festes $n$ und $x$,
\begin{align*}
K_{x,n}(y) = k(x+h_ne_i,y) - k(x,y),
\end{align*}
dann ist
\begin{align*}
\lin{\Delta_{h_n} \Phi(x),\Delta_{h_m} \Phi(y)}_H
&=
\lin{\Phi(x+h_ne_i)-\Phi(x),\Phi(y+h_me_i)-\Phi(y)}_H\\
&= k(x+h_ne_i,y+h_me_i) - k(x,y+h_me_i) \\ &
- k(x+h_ne_i,y) + k(x,y)\\ 
&= K_{x,n}(y+h_m e_i)-K_{x,n}(y).
\end{align*}
Wenden wir den Mittelwertsatz auf $K_{x,n}$ an, erhalten wir ein $\xi_{m,n}\in
[-\abs{h_m},\abs{h_m}]$ mit
\begin{align*}
h_m^{-1}\lin{\Delta_{h_n}\Phi(x),\Delta_{h_m} \Phi(y)}
&= h_m^{-1}\left(K_{x,n}(y+h_m e_i)-K_{x,n}(y) \right)\\
&= \partial_i K_{x,n}(y+\xi_{m,n}e_i) \\
&= \partial_{i+d}\left(k(x+h_n
e_i,y+\xi_{m,n}e_i) - k(x,y+\xi_{m,n}e_i)\right)
\end{align*} 
Wir wenden nun den Mittelwertsatz auf die erste Variable an und erhalten ein
$\eta_{n,m}\in [-\abs{h_n},\abs{h_n}]$, so dass
\begin{align*}
\lin{h_n^{-1}\Delta_{h_n}\Phi(x),h_m^{-1}\Delta_{h_m} \Phi(y)}_H
= \partial_i \partial_{i+d} k(x+\eta_{n,m}e_i,y+\xi_{m,n}e_i)
\end{align*}
Somit ist nach (*),
\begin{align*}
&\norm{h_n^{-1}\Delta_{h_n}\Phi(x)-h_m^{-1}\Delta_{h_m} \Phi(x)}^2_H\\
&\quad\le
2\left(\ep + \partial_i\partial_{i+d}k(x,x)
-\partial_i \partial_{i+d} k(x+\eta_{n,m}e_i,x+\xi_{m,n}e_i)\right)\\
&\quad < 4\ep,\qquad n,m\ge n_0,
\end{align*}
denn $\partial_i \partial_{i+d}k$ ist stetig. Also ist
$(h_n^{-1}\Delta_{h_n}\Phi(x))$ Cauchyfolge und daher existiert $\partial_i
\Phi(x)$ und ist stetig.

Weiterhin gilt
\begin{align*}
\lin{h_n^{-1}\Delta_{h_n}\Phi(x),h_m^{-1}\Delta_{h_m} \Phi(y)}_H\to \partial_i
\partial_{i+d} k(x,y)
\end{align*} 
und da $h_n^{-1}\Delta_{h_n}\Phi(x) \to \partial_i \Phi(x)$, folgt die
Formel.\qedhere
\end{proof}

\begin{defn}
\label{defn:5.3.10}
\index{Kern!stetig differenzierbar}
Ein Kern $k$ auf $X\subset\R^d$ offen heißt \emph{$m$-fach stetig
differenzierbar}, wenn
\begin{align*}
\partial^{\alpha,\alpha} k : X\times X\to \R
\end{align*}
existiert und stetig ist für alle $\alpha\in\N_0^d$ mit $\abs{\alpha}\le
m$.\fishhere
\end{defn}

\begin{cor}
\label{prop:5.3.11}
Sei $k$ ein $m$-fach stetig differenzierbarer Kern mit RKHS $H$. Dann sind alle
$f\in H$ $m$-fach stetig differenzierbar und es gilt
\begin{align*}
\abs{\partial^\alpha f(x)} \le \norm{f}\sqrt{\partial^{\alpha,\alpha}k(x,x)},
\end{align*}
für alle $x\in X$ und $\alpha\in \N_0^d$ mit $\abs{\alpha}\le m$.\fishhere
\end{cor}
\begin{proof}
Iteration von Lemma \ref{prop:5.3.9} impliziert, dass $\partial^\alpha \Phi: X
\to H$ Featuremap von $\partial^{\alpha,\alpha} k$. Da $\lin{f,\cdot}$ stetig,
folgt
\begin{align*}
\lin{f,\partial^\alpha \Phi(x)}_H = \partial^\alpha \lin{f,\Phi(x)} =
\partial^\alpha f(x)
\end{align*}
und nach Cauchy-Schwarz
\begin{align*}
\abs{\partial^\alpha f(x)} = \abs{\lin{f,\partial^\alpha \Phi(x)}} \le
\norm{f}_H \norm{\partial^\alpha \Phi(x)}_H
= \norm{f}_H \sqrt{\partial^{\alpha,\alpha}k(x,x)}.\qedhere
\end{align*}
\end{proof}

\section{Große RKHS}

Wann kann ein RKHS "`viele"' Funktionen approximieren?

\begin{defn}
\label{defn:5.4.1}
Sei $(X,d)$ ein kompakter metrischer Raum, dann heißt ein stetiger Kern $k$ auf
$X$ \emph{universell}\index{Kern!universell}, wenn der RKHS $H$ von $k$ dicht in
$C(X)$ ist, d.h.
\begin{align*}
\forall g\in C(X), \ep > 0 \exists f\in H : \norm{f-g}_\infty \le \ep.\fishhere
\end{align*}
\end{defn}

\begin{lem}
\label{prop:5.4.2}
Sei $(X,d)$ ein kompakter metrischer Raum und $k$ ein universeller Kern auf $X$.
Dann gelten
\begin{propenum}
\item Jede Featuremap von $k$ ist injektiv.
\item $k(x,x) > 0$ für alle $x\in X$.
\item Der Kern $k^* : X\times X \to \R$ gegeben durch
\begin{align*}
k^*(x,y) = \frac{k(x,y)}{\sqrt{k(x,x)k(y,y)}}
\end{align*}
ist universell.
\item Ist $M\subset X$ abgeschlossen, dann ist $k\big|_{M\times M}$
universell.\fishhere
\end{propenum}
\end{lem}
\begin{proof}
\begin{proofenum}
\item Seien $x_1\neq x_2\in X$ und $g: X\to \R$ mit $x\in X$,
\begin{align*}
g(x) \defl \frac{d(x_1,x)}{d(x_1,x)+d(x_2,x)} - \frac{d(x_2,x)}{d(x_1,x)+d(x_2,x)},
\end{align*}
$g(x_1)=-1$ und $g(x_2)=1$ und $g$ stetig auf $X$, also $g\in C(X)$. Sei nun
$H_0$ ein Featurespace von $k$ und $\Phi_0 : X\to H_0$ eine Featuremap und
ferner $H$ der RKHS von $k$. Da $k$ universell, existiert ein $f\in H$ mit
\begin{align*}
\norm{f-g}_\infty \le \frac{1}{2} \Rightarrow f(x_1) \le -\frac{1}{2}\text{ und
} f(x_2)\ge \frac{1}{2}.
\end{align*}
Nach Satz \ref{prop:5.2.4} existiert ein $\omega\in H_0$ mit
$f=\lin{\omega,\Phi_0(\cdot)}$ und daher
\begin{align*}
\begin{rcases}
\lin{\omega,\Phi_0(x_1)}_H \le -\frac{1}{2},\\
\lin{\omega,\Phi_0(x_2)}_H \ge \frac{1}{2},
\end{rcases}
\Rightarrow
\Phi_0(x_1)\neq \Phi_0(x_2).
\end{align*}
\item Zeige $k(x,x)>0$ für alle $x\in X$. In 1) haben wir gesehen, dass es für
$x\in X$ ein $g\in C(X)$ mit $g(x)=1$ gibt. Dazu gibt es ein $f\in H$ mit
$\norm{f-g}_\infty \le \frac{1}{2}$ und folglich
\begin{align*}
\frac{1}{2} \le f(x) = \lin{f,\Phi(x)}_H \Rightarrow \Phi(x)\neq 0.
\end{align*}
Somit ist $k(x,x) = \lin{\Phi(x),\Phi(x)}_H > 0$.
\item Zeige, dass der normalisierte Kern universell ist. Schreibe dazu
\begin{align*}
\alpha(x) \defl \left(k(x,x) \right)^{-1/2},
\end{align*}
dann ist $\alpha\Phi : X\to H$ eine Featuremap von $k^*$. Sei nun $g\in C(X)$
und $\ep > 0$. Sei weiterhin
\begin{align*}
c \defl \norm{\alpha}_\infty < \infty.
\end{align*}
Dann existiert ein $f\in H$ mit $\norm{f-\frac{g}{\alpha}}_\infty \le
\frac{\ep}{c}$ und damit
\begin{align*}
\norm{\lin{f,\alpha \Phi(\cdot)}_H-g}_\infty \le
\underbrace{\norm{\alpha}_\infty}_{\le c} \norm{f-\frac{g}{\alpha}}_\infty \le
\ep.
\end{align*}
\item Sei $M\subset X$ abgeschlossen, also kompakt, so ist
$k\big|_{M\times M}$ universell. Tietzes Fortsetzungssatz besagt
\begin{align*}
\forall g\in C(M) \exists \hat{g}\in C(X) : \hat{g}\big|_M = g.
\end{align*}
Zu jedem $g\in C(X)$ und $\ep >0$ existiert eine Fortsetzung $\hat{g}\in C(X)$
und ein $f\in H$ mit $\norm{\hat{g}-f}_\infty \le\ep$. Dann ist auch
\begin{align*}
\norm{\hat{g}\big|_M - f\big|_M}_\infty \le \ep,
\end{align*}
wobei $f\big|_M$ im RKHS von $k\big|_{M\times M}$.\qedhere
\end{proofenum} 
\end{proof}

\begin{prop}
\label{prop:5.4.3}
Sei $(X,d)$ ein kompakter metrischer Raum, $k$ stetiger Kern mit $k(x,x) >
0$ für alle $x\in X$. Ferner sei $\Phi: X\to l_2$ eine injektive Featuremap von
$k$. Schreibe $\Phi_n : X\to \R$ für die $n$-te Komponente von $\Phi$, d.h.
$\Phi(x) = (\Phi_n(x))_{n\ge 1}$ für alle $x\in X$.

Schreibe $\AA\defl\mathrm{span}\setdef{\Phi_n}{n\ge 1}$. Ist $\AA$ eine Algebra,
dann ist $k$ universell.~\fishhere
\end{prop}

Zum Beweis des Satzes benötigen wir den
\begin{prop*}[Satz von Stone-Weierstraß]
\index{Satz!von Stone-Weierstraß}
Sei $(X,d)$ ein kompakter metrischer Raum, $\AA\subset C(X)$ Algebra mit
\begin{propenum}
\item Zu jedem $x\in X$ existiert ein $f\in \AA$ mit $f(x)\neq 0$. "`$\AA$
verschwindet nicht"'
\item Zu $x\neq y\in X$ existiert ein $f\in \AA$ mit $f(x)\neq f(y)$. "`$\AA$
separiert"'.
\end{propenum}
Dann gilt $\AA\subset C(X)$ ist dicht.\fishhere
\end{prop*}

\begin{proof}
\begin{proofenum}
\item $\norm{(\Phi_n(x))_{n\ge 1}}_{l^2}^2 = \lin{\Phi(x),\Phi(x)}_{l^2} =
k(x,x) > 0$ und daher existiert ein $n\in\N$, so dass $\Phi_n(x) > 0$.
\item Sei nun $x\neq y$, dann ist $\Phi(x)\neq \Phi(y)$ nach Voraussetzung, d.h. es
gibt ein $n\in\N$, so dass $\Phi_n(x)\neq \Phi_n(y)$.
\item Zu zeigen ist noch, dass $\AA\subset C(X)$. Da $k$ stetig, ist $\Phi: X\to
l^2$ stetig und daher ist auch $\Phi_n$ stetig für alle $n\ge 1$, d.h.
$\AA\subset C(X)$. Damit zeigt Stone-Weierstraß, dass
$\bar{A}^{\norm{\cdot}_\infty} = C(X)$. Für $g\in C(X)$ und $\ep > 0$ existiert
somit ein $f\in\AA$ mit $\norm{f-g}_\infty \le \ep$ und
\begin{align*}
f = \sum_{j=1}^m \alpha_j \Phi_{n_j}.
\end{align*}
Setze
\begin{align*}
\omega_n \defl 
\begin{cases}
\alpha_j, & \text{falls } n_j = n,\\
0, & \text{sonst},
\end{cases}
\end{align*}
und $\omega\defl (\omega_n)_{n\ge 1}$. Somit ist $\omega\in l^2$ und
$f=\lin{\omega,\Phi(\cdot)}_{l^2}$ nach Konstruktion.\qedhere
\end{proofenum}
\end{proof}

\begin{cor}
\label{prop:5.4.4}
Für $r\in (0,\infty]$ sei $h: (-r,r)\to \R$ mit
\begin{align*}
h(t) = \sum_{n\ge 0} a_n t^n,\qquad t\in (-r,r).
\end{align*}
Falls $a_n >0$ für alle $n\ge 0$, ist der Taylorkern universell auf allen
kompakten Teilmengen $X\subset \sqrt{r}\ocirc{B}_{\R^d}$.\fishhere
\end{cor}
\begin{proof}
Wir haben schon gesehen, dass $\Phi: X\to l^2(\N_0^d)$,
\begin{align*}
\Phi(x) = \left(\underbrace{\sqrt{a_{j_1}\ldots a_{j_d}
c_{j_1,\ldots,j_d}}}_{>0}
\prod\limits_{i=1}^d x_i^{j_i}  \right)_{j_1,\ldots,j_d\ge 0}
\end{align*}
eine Featuremap von $k$ ist.
\begin{proofenum}
\item $k$ ist stetig, da $h$ und $\lin{\cdot,\cdot}_{\R^d}$ stetig.
\item $a_0 > 0$ und daher ist $k(x,x) = \sum_{n\ge 0} a_n(\lin{x,x})^n \ge a_0
> 0$.
\item $\Phi$ ist injektiv. Sei $x\neq y$ dann existiert ein
$i\in\setd{1,\ldots,d}$ mit $x_i\neq y_i$. Nimm $j_i = 1$, alle anderen $j_l =
0$. Folglich ist $\Phi_{j_1,\ldots,j_d}(x) \neq \Phi_{j_1,\ldots,j_d}(y)$ und
daher $\Phi(x)\neq\Phi(y)$.

Zu zeigen ist, dass
$\AA=\mathrm{span}\setdef{\Phi_{j_1,\ldots,j_d}}{j_1,\ldots,j_d \ge 0}$ ist
Algebra. Da $a_n \ge 0$ ist $\AA$ Algebra, denn $\AA$ enhält alle Monome und
Linearkombinationen davon. Also ist Satz \ref{prop:5.4.3} anwendbar.\qedhere
\end{proofenum}
\end{proof}

\begin{bsp*}
Der \textit{Exponentialkern}
\begin{align*}
h(x,x') \defl \exp\left( \lin{x,x'}_{\R^d}\right)
\end{align*}
ist universell auf allen Kompakta, da $a_n = \frac{1}{n!} >0$.\bsphere
\end{bsp*}

\begin{bsp*}
Für $\sigma > 0$ ist der \textit{Gaußkern}
\begin{align*}
k_\sigma(x,x') = \exp(-\sigma^2\abs{x-x'}^2)
\end{align*}
universell auf allen Kompakta. Da
\begin{align*}
h_\sigma(x,x') =
\frac{\exp(2\sigma^2\lin{x,x})}{\exp(\sigma^2\abs{x}^2)\exp(\sigma^2\abs{x'}^2)},
\end{align*}
$\exp(2\sigma^2\lin{x,x})$ universell und $h_\sigma$ eine normalisierte Fassung
davon.\bsphere
\end{bsp*}

\begin{bem*}[Bemerkungen.]
\begin{bemenum}
\item
Dahmen und Michelli zeigten 1987, dass ein Kern $k$ genau dann universell ist,
wenn
\begin{align*}
a_0 > 0 \text{ und } \sum_{a_{2n}>0} \frac{1}{2n} = \sum_{a_{2n+1}>0}
\frac{1}{2n+1} = \infty.
\end{align*}
\item Eine leichte Übung zeigt, dass jeder universelle Kern strikt positiv
ist.
\item Pirkus zeigte 2004, dass Taylorkerne genau dann strikt positiv sind, wenn
\begin{align*}
a_0 > 0\text{ und } \abs{\setdef{n}{a_{2n} >0}} = \abs{\setdef{n}{a_{2n+1}>0}} =
\infty.
\end{align*}
\item
Man kann auch untersuchen, wann $H\subset L_p(\mu)$ dicht ist.

Für Gaußkerne ist $H_\sigma\subset L_p(\mu)$ dicht für alle
$p\in[1,\infty)$ und alle W-Maße $\mu$ auf $\R^d$ und $\mu=\lambda^d$ auf
$\R^d$.
\item Sei $(X,d)$ kompakter metrischer Raum, dann existiert ein universeller
Kern auf $X$.
\item Sei $(X,\tau)$ kompakter Hausdorffraum, dann existiert genau dann ein
universeller Kern auf $X$, wenn $(X,\tau)$ metrisierbar ist.
\item In der Regel gilt für universelle RKHS $H$, dass $\dim H = \infty$,
insbesondere gilt dies für $X\subset \R^d$ mit $\ocirc{X}\neq
\varnothing$.\maphere
\end{bemenum}
\end{bem*}

\chapter{Support Vector Machines (SVMs)}

\section{Definition, einfache Eigenschaften und Beispiele}
\label{chap:6.1}

\begin{defn}
\label{defn:6.1.1}
Sei $H$ ein RKHS und $L: X\times Y \times \R\to [0,\infty)$ eine konvexe
Verlustfunktion. Ferner sei $\lambda > 0$ "`Regularisierungsparameter"'. Dann
heißt eine Lernmethode \emph{Support Vector Machine (SVM)}\index{SVM}, wenn
\begin{defnenum}
\item Für jedes $D\in (X\times Y)^m$ die Entscheidungsfunktion $f_{D,\lambda}\in
H$.
\item $\lambda \norm{f_{D,\lambda}}_H^2 + \RR_{L,D}(f_{D,\lambda}) = \inf_{f\in
H} \lambda \norm{f}_H^2 + \RR_{L,D}(f)$.
\end{defnenum}
D.h. $f_{D,\lambda}$ minimiert das regularisierte empirische Risiko,
\begin{align*}
\lambda \norm{\cdot}_{H^2} + \RR_{L,D}(\cdot),\qquad \text{über }H.\fishhere
\end{align*}
\end{defn}

\begin{lem}[Eindeutigkeit]
\index{SVM!Eindeutigkeit}
\label{lem:6.1.2}
Für alle $D\in (X\times Y)^n$, $n\ge 1$, $\lambda > 0$ existiert höchstens ein
$f_{D,\lambda}\in H$.\fishhere
\end{lem}
\begin{proof}
Angenommen es gibt $f_1\neq f_2\in H$ und beide minimieren 
$\lambda \norm{\cdot}_{H^2} + \RR_{L,D}(\cdot)$ über $H$. Sei
$f=\frac{1}{2}(f_1+f_2)$. $L$ konvex, dann ist $\RR_{L,D}(\cdot)$ konvex
\begin{align*}
\RR_{L,D}(f) \le \frac{1}{2}\RR_{L,D}(f_1) + \frac{1}{2}\RR_{L,D}(f_2)
\end{align*}
Parallelogrammgleichung in $H$:
\begin{align*}
\norm{f_1+f_2}_H^2 + \norm{f_1-f_2}_H^2 = 2\norm{f_1}_H^2 + 2\norm{f_2}_H^2.
\end{align*}
Da $f_1\neq f_2$ ist
\begin{align*}
\norm{\frac{1}{2}(f_1+f_2)}_H^2  < \frac{1}{2}\norm{f_1}_H^2 +
\frac{1}{2}\norm{f_2}_H^2
\end{align*}
und damit
\begin{align*}
\lambda\norm{f}_H^2 + \RR_{L,D}(f) &< 
\frac{1}{2}\left( \norm{f_1}_H^2 + \RR_{L,D}(f_1) \right)
+
\frac{1}{2}\left( \norm{f_2}_H^2 + \RR_{L,D}(f_2) \right)\\
&= \inf_{f\in H} \lambda \norm{f}_H^2 + \RR_{L,D}(f).\dipper\qedhere
\end{align*}
\end{proof}

\begin{prop}
\label{prop:6.1.3}
Ist $\dim H < \infty$, dann existiert ein $f_{D,\lambda}\in H$.\fishhere
\end{prop}
\begin{proof}
$L$ ist konvex, d.h. $L(x,y,\cdot): \R\to [0,\infty)$ konvex und lokal lipschitz
für alle $x,y$ also ist $L$ insbesondere stetig. Betrachte für $D\in (X\times
Y)^n$ das Risikofunktional
\begin{align*}
\RR_{L,D} : H\to [0,\infty),\qquad f\mapsto \RR_{L,D}(f).
\end{align*}
Dieses ist konvex und stetig, da
\begin{align*}
f\mapsto \RR_{L,D}(f) = \frac{1}{n}\sum_{i=1}^n L(x_i,y_i,f(x_i))
\end{align*}
und die Konvergenz in $H$ die punktweise Konvergenz impliziert. Ferner ist die
Abbildung
\begin{align*}
H\to [0,\infty),\qquad f\mapsto \lambda \norm{f}_H^2 
\end{align*}
stetig und konvex (aufgrund der Parallelogrammungleichung sogar strikt konvex)
und damit ist auch die Abbildung
\begin{align*}
H\to [0,\infty),\qquad f\mapsto \lambda \norm{f}_H^2 + \RR_{L,D}(f)\tag{*}
\end{align*}
stetig und konvex. Setze $m\defl\RR_{L,D}(0) < \infty$ und
\begin{align*}
A =\setdef{f\in H}{\lambda\norm{f}_H^2 + \RR_{L,D}(f)\le m}.
\end{align*}
Dann gelten
\begin{itemize}
  \item $0\in A$.
  \item $A$ ist abgeschlossen, da (*) stetig.
  \item $A$ ist beschränkt, da für $f\in A$ gilt
\begin{align*}
\lambda\norm{f}_H^2 \le \lambda \norm{f}_H^2 + \RR_{L,D}(f) \le m,
\end{align*} 
ist $\norm{f}_H \le \sqrt{m/\lambda}$.
\end{itemize}
Somit ist $A$ nichtleer und kompakt, d.h. (*) besitzt ein Minimum $f^*\in A$ mit
\begin{align*}
\lambda\norm{f^*}_H^2 + \RR_{L,D}(f^*)\le m
\end{align*}
und für $f\notin A$ gilt
\begin{align*}
\lambda\norm{f}_H^2 + \RR_{L,D}(f) > m \ge
\lambda\norm{f^*}_H^2 + \RR_{L,D}(f^*),
\end{align*}
also ist $f^*$ sogar ein globales Minimum.\qedhere
\end{proof}

\begin{lem}
\label{prop:6.1.4}
Falls $f_{D,\lambda}\in H$ existiert gelten
\begin{propenum}
\item $\norm{f_{D,\lambda}}_H \le \left(\lambda^{-1} \RR_{L,D}(0)\right)^{1/2}$.

Inbesondere ist $L(x,y,0) \le 1$ für alle $(x,y)\in X\times Y$ und folglich
\begin{align*}
\norm{f_{D,\lambda}}_H \le \lambda^{-1/2}.
\end{align*}
\item Falls ein $f^*\in H$ existiert mit $\RR_{L,D}(f^*) \le \RR_{L,D}(0)$, ist
$f_{D,\lambda}\neq 0$.\fishhere
\end{propenum}
\end{lem}
\begin{proof}
\begin{proofenum}
\item Man rechnet direkt nach, dass
\begin{align*}
\lambda\norm{f_{D,\lambda}}^2 &\le 
\lambda\norm{f_{D,\lambda}}^2 + 
\RR_{L,D}(f_{D,\lambda})
= \min_{f\in H} \lambda \norm{f}_H^2 + \RR_{L,D}(f)\\
&\le \lambda \norm{0}_H^2 + \RR_{L,D}(0)
= \RR_{L,D}(0).
\end{align*}
\item Für jedes $\alpha\in [0,1]$ sei
\begin{align*}
h(\alpha) \defl 2\lambda \alpha \norm{f^*}_H^2 + \RR_{L,D}(f^*)-\RR_{L,D}(0)
\end{align*}
%TODO: nachreichen\ldots

Da $\RR_{L,D}(f^*)\le \RR_{L,D}(0)$ und $\alpha \mapsto h(\alpha)$ quadratisch,
besitzt
\begin{align*}
h: [0,1]\to [0,\infty)
\end{align*}
ein globales Minimum $\alpha^*\in (0,1]$, denn
\begin{align*}
&2\lambda \alpha \norm{f^*}_H^2 + \RR_{L,D}(f^*)-\RR_{L,D}(0) = 0\\
\Leftrightarrow
\; &\alpha^* = \frac{\RR_{L,D}(0)-\RR_{L,D}(f^*)}{2\lambda \norm{f^*}_H^2}.
\end{align*}
Somit ist
\begin{align*}
\lambda \norm{\alpha^* f^*}_H^2 + \RR_{L,D}(\alpha^*f^*)
&\le h(\alpha^*) < h(0) = \RR_{L,D}(0) \\ &
= \lambda \norm{0}_H^2 +
\RR_{L,D}(0).
\end{align*}
Also ist $f_{D,\lambda}\neq 0$.\qedhere
\end{proofenum}
\end{proof}

\begin{prop}[Representer Theorem]
\index{Theorem!Representer}
\label{prop:6.1.5}
Sei $H$ ein RKHS mit Kern $k$, $L$ eine konvexe Verlustfunktion und $\lambda >
0$, $D\in (X\times Y)^n$. Schreibe $D=((x_1,y_1),\ldots,(x_n,y_n))$, dann
existieren ein $f_{D,\lambda}$ und $\alpha_1,\ldots,\alpha_n\in\R$ mit
\begin{align*}
f_{D,\lambda} = \sum_{i=1}^n \alpha_i k(x_i,\cdot).\fishhere
\end{align*}
\end{prop}

\begin{proof}
Sei $X'\defl\setd{x_1,\ldots,x_n}$ und
\begin{align*}
H\big|_{X'} \defl \setdef{\sum_{i=1}^n \alpha_i k(x_i,\cdot)}{\alpha_i\in\R} =
\mathrm{span} \setdef{k(x_i,\cdot)}{i=1,\ldots,n}.
\end{align*}
Wir zeigen, dass $H\big|_{X'}$ der RKHS von $k\big|_{X'\times X'}$ ist.

Satz \ref{prop:5.2.4} sagte nämlich $\mathrm{span}
\setdef{k(x_i,\cdot)}{i=1,\ldots,n}$ ist dicht im RKHS von $k\big|_{X'\times
X'}$. Da $\dim \mathrm{span}
\setdef{k(x_i,\cdot)}{i=1,\ldots,n}<\infty$, ist $\mathrm{span}
\setdef{k(x_i,\cdot)}{i=1,\ldots,n}$ abgeschlossen im RKHS von
$k\big|_{X'\times X'}$, d.h.
\begin{align*}
\mathrm{span}
\setd{k(x_i,\cdot)} =
\bar{\mathrm{span}
\setd{k(x_i,\cdot)}}
=\text{ RKHS von }k\big|_{X'\times X'}.
\end{align*}
Betrachte das SVM Problem bezüglich $k\big|_{X'\times X'}$,
\begin{align*}
\inf_{f\in H\big|_{X'}} \lambda \norm{f}_H^2 + \RR_{L,D}(f).\tag{*}
\end{align*}
Nach Lemma \ref{prop:6.1.3} existiert ein $f_{D,\lambda,H\big|_{X'}}\in
H\big|_{X'}$. Dieses ist Lösung von (*), denn $\dim H\big|_{X'}< \infty$. Da $H\big|_{X'}\subset H$
abgeschlossen, existiert das orthogonale Komplement $H\big|_{X'}^\bot$. Für
$f\in H\big|_{X'}^\bot$ gilt dann
\begin{align*}
f(x,\cdot)= \lin{f,k(x_i,\cdot)}_H = 0.
\end{align*}
Schreibe $P_{X'}: H\to H$ und $P_{X'}^\bot :H\to H$ für die orthogonale
Projektion auf $H\big|_{X'}$ bzw. $H\big|_{X'}^\bot$.
Für $f\in H$ gilt somit $f=P_{X'}f + P_{X'}^\bot f$ und nach (*) ist
\begin{align*}
P_{X'}^\bot f(x_i) = 0,\qquad i=1,\ldots,n.
\end{align*}
Somit ist auch
\begin{align*}
\RR_{L,D}(f) = \RR_{L,D}(P_{X'}f + P_{X'}^\bot f) = \RR_{L,D}(
P_{X'} f)
\end{align*}
und da ferner $\norm{P_{X'} f}_{H}^2\le \norm{f}_H^2$, gilt
\begin{align*}
\inf_{f\in H} \lambda\norm{f}_H^2  + \RR_{L,D}(f)
&\le 
\inf_{f\in H\big|_{X'}} \lambda\norm{f}_H^2  + \RR_{L,D}(f)\\
&=
\inf_{f\in H} \lambda\norm{P_{X'}f}_H^2  + \RR_{L,D}(P_{X'}f)\\
&\le
\inf_{f\in H} \lambda\norm{f}_H^2  + \RR_{L,D}(f).  
\end{align*}
Insbesondere ist daher
\begin{align*}
\inf_{f\in H} \lambda\norm{f}_H^2  + \RR_{L,D}(f)
&=
\inf_{f\in H\big|_{X'}} \lambda\norm{f}_{H\big|_{X'}}^2  + \RR_{L,D}(f)\\
&=
\lambda\norm{f_{D,\lambda,H\big|_{X'}}}_{H\big|_{X'}}^2  +
\RR_{L,D}(f_{D,\lambda,H\big|_{X'}}).
\end{align*}
Da $f_{D,\lambda,H\big|_{X'}} \in H$ löst $f_{D,\lambda,H\big|_{X'}}$ auch das
original SVM-Problem und da $f_{D,\lambda}$ eindeutig und $f_{D,\lambda}\in
H\big|_{X'}$ folgt die Darstellung.\qedhere
\end{proof}

Als Konsequenz ist für $f=\sum_{i=1}^n\alpha_i k(x_i,\cdot)$,
\begin{align*}
&\min_{f\in H} \lambda\norm{f}_H^2 + \RR_{L,D}(f) =
\min_{\alpha_1,\ldots,\alpha_n\in\R} \lambda\norm{f}^2_H + \RR_{L,D}(f)\\
&\qquad =
\min_{\alpha_1,\ldots,\alpha_n\in\R} \lambda\sum_{i,j=1}^n \alpha_i\alpha_j
k(x_i,x_j) + \RR_{L,D}\left(\sum_{i=1}^n \alpha_i k(x_i,\cdot) \right)
\end{align*}

Dies ist ein endlichdimensionales und \textit{konvexes} Optimierungsproblem.

\begin{bsp*}
Verlustfunktion der kleinsten Quadrate $L(x,y,t) = (y-t)^2$. Es ist
\begin{align*}
&\frac{\delta}{\delta \alpha_{i_0}} \left( 
\lambda \sum_{i,j=1}^n \alpha_i\alpha_j k(x_i,x_j) + \frac{1}{n}\sum_{i=1}^n 
\left(y_i -\sum_{j=1}^n \alpha_j k(x_i,x_j) \right)^2\right)\\
&=
2\lambda \sum_{i=1}^n \alpha_i k(x_i,x_{i_0}) + 
\frac{\delta}{\delta \alpha_{i_0}} \left( 
\frac{1}{n}\sum_{i=1}^n 
\left(y_i -\sum_{j=1}^n \alpha_j k(x_i,x_j) \right)^2\right),\tag{*}
\end{align*}
wobei
\begin{align*}
&\frac{\delta}{\delta \alpha_{i_0}} \left(  
\left(y_i -\sum_{j=1}^n \alpha_j k(x_i,x_j) \right)^2\right)\\
&= -2\left(y_i - \sum_{j=1}^n \alpha_j k(x_i,x_j) \right)
\underbrace{\frac{\delta}{\delta_{a_{i_0}}} \left(\sum_{j=1}^n \alpha_j
k(x_i,x_j)\right)}_{k(x_i,x_{i_0})}.
\end{align*}
Somit gilt
\begin{align*}
\text{(*)} &=
2\lambda \sum_{i=1}^n \alpha_i k(x_i,x_{i_0}) 
 -\frac{2}{n}\sum_{i=1}^n \left(y_i - \sum_{j=1}^n \alpha_j k(x_i,x_j) \right)
k(x_i,x_{i_0})\\
&\overset{!}{=} 0.
\end{align*}
Schreibe $k$ als $n\times n$-Matrix $k=\left(k(x_i,x_j)\right)_{i,j=1}^n$. Dann
ist $k=k^\top$ und die Optimierung ist gleichbedeutend mit
\begin{align*}
&2\lambda k\alpha - \frac{2}{n}ky + \frac{2}{n}kk\alpha \overset{!}{=}0
\Leftrightarrow
\lambda n k \alpha + kk\alpha = ky\\
\Leftrightarrow
&k(\lambda n E_n\alpha + k\alpha) = ky\tag{**}
\end{align*} 
wobei $E_n$ die $n$-te Einheitsmatrix bezeichne. Falls $\lambda_n E_n \alpha +
k\alpha = y$, dann ist (**) erfüllt.

Da $k$ ein Kern ist die Matrix $k$ positiv semi-definit, d.h. $k$ hat $n$
nichtnegative Eigenwerte,
\begin{align*}
\lambda n 
\begin{pmatrix}
1 \\
 &\ddots \\
 && 1
\end{pmatrix}
+
\begin{pmatrix}
\lambda_1 \\
 & \ddots \\
  & & \lambda_n
\end{pmatrix}.
\end{align*}
$\lambda n E_n + k$ hat folglich strikt positive Eigenwerte  und ist daher
invertierbar.
\begin{align*}
\alpha = (\lambda n E_n + k)^{-1}y
\end{align*}
ist die eindeutige Lösung und
\begin{align*}
f_{D,\lambda} = \sum_{i=1}^n \alpha_i k(x_i,\cdot).\bsphere
\end{align*}
\end{bsp*}

\subsection{Ausflug in das Reich der konvexen Analysis}

Sei $E$ ein endlichdimensionaler Vektorraum, $P: E\to\R$ konvex und stetig
differenzierbar sowie $h_i : E\to \R$ affin linear für $i=1,\ldots,m$.

\begin{defn*}[Primales Optimierungsproblem]
\index{Optimierungsproblem!Primales}
Wir suchen
\begin{align*}
P^* = \inf_{\omega\in E} P(\omega)
\end{align*}
unter den Nebenbedingungen $h_i(\omega) \le 0$ für alle $i=1,\ldots,m$.\fishhere
\end{defn*}

Wir nehmen für alles Weitere an, dass ein $\omega^*$ existiert mit
$h_i(\omega^*) \le 0$ für $i=1,\ldots,m$ und $P(\omega^*) = P^*$. Unser Ziel ist
es das primale Optimierungsproblem auf ein duales Optimierungsproblem
zurückzuführen.

\begin{defn*}
\index{Lagrangefunktion}
Die Funktion
\begin{align*}
L(\omega,\beta) \defl P(\omega) + \sum_{i=1}^m \beta_i h_i(\omega),\qquad \omega\in
E,\; \beta\in\R^n
\end{align*}
heißt \emph{Lagrangefunktion}.
\begin{align*}
D(\beta) \defl \inf_{\omega\in E} L(\omega, \beta)
\end{align*}
heißt die zur Lagrangefunktion gehörige \emph{duale Funktion}.\fishhere
\end{defn*}

Da $P$ konvex und die $h_i$ affin linear, ist auch $L$ konvex.

\begin{defn*}[Duales Optimierungsproblem]
\index{Optimierungsproblem!Duales}
\begin{align*}
D^* = \sup_{\beta\ge 0} D(\beta).\fishhere
\end{align*}
\end{defn*} 

Wie hängen das duale Problem mit seinen Lösungen mit dem primalen Problem
zusammen?

\begin{lem*}
Für $\beta \ge 0$ und $\omega\in E$ mit $h_i(\omega)\le 0$ für $i=1,\ldots,m$
gilt
\begin{align*}
D(\beta) \le P(\omega)
\end{align*}
und damit insbesondere $D^*\le P^*$.\fishhere
\end{lem*}
\begin{proof}
Man rechnet direkt nach, dass
\begin{align*}
D(\beta) = \inf_{\omega'\in E} L(\omega',\beta) \le
L(\omega,\beta) = P(\omega) + \sum_{i=1}^m \beta_i h_i(\omega) \le
P(\omega).\qedhere
\end{align*}
\end{proof}

\begin{prop*}
$D^*=P^*$.\fishhere
\end{prop*}
\begin{proof}
Wir überspringen den Beweis, da dieser zu weit in die konvexe Analysis
hineinführt.\qedhere
\end{proof}

\begin{cor*}
Sei $\beta^*\ge 0$ mit $D(\beta^*) = D^*$, dann gilt
\begin{align*}
D^* = \max_{\beta \ge 0} \inf_{\omega} L(\omega,\beta) = \min_{\omega}
\sup_{\beta \ge 0} L(\omega,\beta) = P^*.\fishhere
\end{align*}
\end{cor*}
\begin{proof}
"`$\le$"': gilt immer.\\
"`$\ge$"': Sei $\omega^*\in E$ eine primale Lösung, d.h.
\begin{align*}
h_i(\omega^*)\le 0,\quad i=1,\ldots,m,\qquad P(\omega^*) = P^*.
\end{align*}
Dann ist
\begin{align*}
&D^* = D(\beta^*) = \max_{\beta\ge 0}\inf_{\omega\in E}L(\omega,\beta),\\
&P^* = P(\omega^*) = \min_{\omega, h_i(\omega) = 0}
P(\omega) = \min_{\omega\in E} \sup_{\beta \ge 0} P(\omega) + \sum_{i=1}^m
\beta_i h_i(\omega)\tag{*}
\end{align*}
denn
\begin{align*}
\sup_{\beta \ge 0} P(\omega) + \sum_{i=1}^m \beta_i h_i(\omega) =
\begin{cases}
P(\omega), & h_i(\omega) \le 0,\; i=1,\ldots,m,\\
\infty, & \text{sonst}.
\end{cases}
\end{align*}
und folglich ist
\begin{align*}
\text{(*)} = \min_{\omega\in E}\sup_{\beta\ge 0} L(\omega,\beta).
\end{align*}
Die Behauptung folgt, dass $P^*=D^*$.\qedhere
\end{proof}

\begin{prop*}
Sei $\beta^*$ die duale Lösung, d.h. $\beta^*\ge 0$ und $D(\beta^*)= D^*$ und
$\omega^*$ die primale Lösung, d.h. $h_i(\omega^*) \le 0$ für $i=1,\ldots,m$
mit $P(\omega^*) = P^*$.
Dann folgt
\begin{align*}
\max_{\beta \ge 0} L(\omega^*,\beta) = L(\omega^*,\beta^*) = \min_{\omega\in E}
L(\omega, \beta^*)
\end{align*}
und $L(\omega^*,\beta^*) = D^*= P^*$.\fishhere
\end{prop*}

$(\omega^*,\beta^*)$ ist Sattelpunkt von $L$.

\begin{proof}
Es gilt
\begin{align*}
D^* = \max_{\beta\ge 0}\inf_{\omega\in E} L(\omega, \beta) = 
\min_{\omega\in E}\sup_{\beta\ge 0} L(\omega,\beta)
= P(\omega^*) = \sup_{\beta\ge 0} L(\omega^*,\beta).
\end{align*}
und
\begin{align*}
P^* = \min_{\omega\in E}\sup_{\beta\ge 0} L(\omega,\beta) = \max_{\beta \ge 0}
\inf_{\omega\in E} L(\omega, \beta) = D^* =D(\beta^*) = \inf_{\omega\in E}
L(\omega, \beta^*)
\end{align*}
sowie
\begin{align*}
L(\omega^*,\beta^*) &\ge \inf_{\omega\in E} L(\omega, \beta^*) = P^*,\\
L(\omega^*,\beta^*) &\le \sup_{\beta\ge 0} L(\omega^*, \beta) = D^*,
\end{align*}
und folglich ist
\begin{align*}
L(\omega^*,\beta^*) = \sup_{\beta \ge 0} L(\omega^*,\beta) = \inf_{\omega\in E}
L(\omega,\beta^*) = P^* = D^*
\end{align*}
und $\sup$ ist $\max$ und $\inf$ ist $\min$.\qedhere
\end{proof}

\begin{cor*}
Seien $\beta^*$ und $\omega^*$ wie oben. Dann ist
\begin{align*}
\beta_i^* h_i(\omega^*) = 0\text{ für alle } i=1,\ldots,m.\fishhere
\end{align*}
\end{cor*}
\begin{proof}
$P(\omega^*) = P^* = L(\omega^*,\beta^*) = P(\omega^*) + \sum_{i=1}^m \beta_i
h_i(\omega^*)$.\qedhere
\end{proof}

\begin{cor*}
Sei $E=E_1\times E_2$, $\omega=(\omega_1,\omega_2)\in E_1\times E_2$ und $L$
derart, dass:

Für alle $\beta^*\ge 0$ mit $D(\beta^*) = D^*$ existiert genau ein $\omega_1\in
E_1$ und ein (oder mehrere) $\omega_2\in E_2$ mit $h_i(\omega_1,\omega_2)\le 0$
für $i=1,\ldots,m$ und
\begin{align*}
L((\omega_1,\omega_2),\beta^*) = D(\beta^*) = \inf_{\omega\in E}
L(\omega,\beta^*).
\end{align*}
Dann gilt für jede primale Lösung $\omega^*=(\omega_1^*,\omega_2^*)$,
\begin{align*}
\omega_1 = \omega_1^*.
\end{align*}
Insbesondere: Falls wir wissen, dass es eine primale Lösung $\omega^*$ gibt, so
finden wir die erste Komponente $\omega_1^*$ indem wir
\begin{align*}
\inf_{\omega\in E} L(\omega,\beta^*)
\end{align*}
lösen und für die Lösung $\omega=(\omega_1,\omega_2)$ gilt $\omega_1^* =
\omega_1$.\fishhere
\end{cor*}
\begin{proof}
Sei $\omega^*$ eine Lösung, dann ist
\begin{align*}
h_i(\omega^*) = h_i(\omega_1^*,\omega_2^*) \le 0,\qquad i=1,\ldots,m
\end{align*}
und der vorige Satz zeigte $L(\omega^*,\beta^*) = D^*= D(\beta^*)$. Mit der
vorausgesetzten Eindeutigkeit in der ersten Komponente folgt $\omega_1^* =
\omega_1$.\qedhere
\end{proof}

Als Anwendung betrachten wir SVMs mit Hinge-loss.

\begin{bsp*}
SVMs mit Hinge-loss. Das Optimierungsproblem
\begin{align*}
\min_{f\in H} \lambda\norm{f}_H^2 + \frac{1}{n}\sum_{i=1}^n \max\setd{0,1-y_i
f(x)}
\end{align*}
ist ein SVM Optimierungsproblem, da $L_\text{hinge}(y,t) = \max\setd{0,1-yt}$.
Dieses ist äquivalent zu dem Optimierungsproblem
\begin{align*}
\min_{\atop{f\in H}{\xi\in\R^n}} \frac{1}{2}\lin{f,f} + c\sum_{i=1}^{n}
\xi_i,\qquad c= \frac{1}{2\lambda n}
\end{align*}
mit den Nebenbedingungen
\begin{align*}
\xi_i \ge 0,\quad \xi_i \ge 1-y_if(x_i)
\Leftrightarrow
\xi_i \ge \max{0,1-y_if(x_i)}.
\end{align*}
Die Lagrangefunktion des Problems ist
\begin{align*}
L(f,\xi,\beta,\gamma) = \frac{1}{2}\lin{f,f} + c\sum_{i=1}^n \xi_i -
\sum_{i=1}^n \beta_i \xi_i  + \sum_{i=1}^n \gamma_i(1-y_if(x_i)-\xi_i).
\end{align*}
$L$ ist differenzierbar und
\begin{align*}
\frac{\delta}{\delta f} L(f,\xi,\beta, \gamma)
= f - \sum_{i=1}^n \gamma_i y_i k(x_i,\cdot)
\overset{!}{=} 0
\end{align*}
führt auf die eindeutige Lösung
\begin{align*}
f = \sum_{i=1}^n \gamma_iy_i k(x_i,\cdot).
\end{align*}
Weiterhin führt
\begin{align*}
\frac{\delta}{\delta \xi_i} L(f,\xi,\beta,\gamma) = c-\beta_{i_0} - \gamma_{i_0}
\overset{!}{=} 0
\end{align*}
auf $\beta_{i_0} + \gamma_{i_0} = c$.

Ist $\beta + \gamma = c$, so minimiert jedes $\xi$. Ist andererseits
$\beta+\gamma\neq c$, so gibt es keine Lösung.

Setzen wir dies in unser Optimierungsproblem ein, so erhalten wir
\begin{align*}
D(\beta,\gamma) &= \frac{1}{2}\sum_{i=1}^n y_i y_j \gamma_i \gamma_j k(x_i,x_j)
+ \sum_{i=1}^n \gamma_i
- \sum_{i=1}^n y_i y_j \gamma_i \gamma_j k(x_i,x_j)\\
&= \sum_{i=1}^n \gamma_i
- \frac{1}{2}\sum_{i=1}^n y_i y_j \gamma_i \gamma_j k(x_i,x_j)
\end{align*}
mit $\beta\ge 0$, $\gamma\ge 0$ und $\beta + \gamma = c$, d.h.
$\gamma\in[0,c]^n$ und $\beta \defl c-\gamma$.

Damit ist das duale Problem
\begin{align*}
\max_{\gamma\in[0,c]^n} \sum_{i=1}^n \gamma_i - \frac{1}{2}
\sum_{i,j=1}^n y_iy_j \gamma_i\gamma_j k(x_i,x_j).\tag{*}
\end{align*}
- mit der Definition $\mathbf{k} = (y_iy_j k(x_i,x_j))_{i,j=1}^n$ -
\begin{align*}
\max_{\gamma\in[0,c]^n} \lin{\gamma,1} -
\frac{1}{2}\lin{\gamma,\mathbf{k}\gamma}.
\end{align*}
Ist $\gamma^*$ Lösung des dualen Problems, so ist
\begin{align*}
f_{D,x} = \sum_{i=1}^n y_i \gamma_i^* k(x_i,\cdot)
\end{align*}
die Lösung der ersten Komponente des primalen Problems.
\begin{align*}
\xi_i^* = \max\setd{0,1-y_i f_{D,x}(x_i)}
\end{align*} 
ist die zweite Komponente.

\begin{bem*}
Sei $\gamma\in[0,c]^n$ und $f_\gamma = \sum_{i=1}^n y_i \gamma_i k(x_i,\cdot)$.
Gilt nun 
\begin{align*}
P(f_\gamma,\xi_\gamma) - D(\gamma) \le \ep,
\end{align*}
dann ist auch
\begin{align*}
P(f_\gamma) - P^* \le \ep,
\end{align*}
da $D(\gamma)\le P^*$.
\begin{align*}
P(f_\gamma,\xi_\gamma) = \sum_{i,j=1}^n y_iy_j \gamma_i\gamma_j k(x_i,x_j) +
c\sum_{i=1}^n \max\setd{0,1-y_if_\gamma(x_i)} - \sum_{i=1}^n \gamma_i.\maphere
\end{align*}
\end{bem*}
Dann ist
\begin{align*}
\lambda\norm{f_\gamma}^2_H + \RR_{L,D}(f_\gamma) - \min_{f\in
H}\left(\lambda\norm{f}_H^2 + \RR_{L,D}(f) \right)
= 2\lambda\left(P(f_\gamma,\xi_\gamma) - P^* \right).
\end{align*}
Gilt nun
\begin{align*}
P(f_\gamma,\xi_\gamma) - P^* \le \frac{\ep}{2\lambda}\tag{+}
\end{align*}
so ist
\begin{align*}
\lambda\norm{f_\gamma}^2_H + \RR_{L,D}(f_\gamma) 
\le \inf_{f\in
H}\left(\lambda\norm{f}_H^2 + \RR_{L,D}(f) \right)
+\ep.\bsphere
\end{align*}
\end{bsp*}

\textit{Wie löst man das duale Problem?}
\begin{itemize}
  \item Standardsoftware.
  \item \emph{Gradient assent}. Zu $\gamma$ berechne den Gradienten von (*) und
  mache einen Step in Richtung des steilsten Anstiegs. Wiederhole den Vorgang,
  bis (+) erreicht ist.
  
  Ein Problem bei diesem Verfahren ist, dass die Kernmatrix ein enormer
  Platzfresser ist. Es gibt jedoch zahlreiche Verfahren, die nicht die gesamte
  Kernmatrix in den RAM laden muss.
  \item Man betrachtet die Abbildung
\begin{align*}
\gamma_i \mapsto D(\gamma + \gamma_ie_i).
\end{align*}
Diese ist eine eindimensionale konkave und quadratische Funktion und ihr
Optimierungsproblem ist explizit lösbar.
\begin{enumerate}
  \item Suche Richtung $i^*$ mit maximalem Gewinn in $D$ bei Optimierung in
  Richtung $i^*$.
  \item Optimiere in Richtung $i^*$.
  \item Gehe zu (a), falls (*) nicht erfüllt ist.
\end{enumerate}
Die meisten verfügbaren Programme machen dies so.
\end{itemize}

\section{Orakelungleichungen für SVMs}
\index{Orakelungleichung!SVM}

\begin{prop}
\label{prop:6.2.1}
Sei $(X,d)$ ein kompakter, metrischer Raum, $L$ eine konvexe und lokal
lipschitz-stetige Verlustfunktion mit $L(x,y,0) \le 1$ für alle $(x,y)\in
X\times Y$. Ferner sei $H$ ein RKHS über $X$ mit stetigem Kern $k$ und
$\norm{k}_\infty \le 1$ und $P$ sei ein Wahrscheinlichkeitsmaß auf $X\times Y$.

Für alle $\lambda > 0$, $n\ge 1$, $\ep > 0$ und $\tau > 0$ gilt mit einer
Wahrscheinlichkeit $P^n$ nicht kleiner als $1-\ep^{-\tau}$:
\begin{align*}
&\lambda\norm{f_{D,\lambda}}_H^2 + \RR_{L,P}(f_{D,\lambda})-\RR_{L,P}^*
< \inf_{f\in H}\left( \lambda\norm{f}_H^2 + \RR_{L,P}(f) - \RR_{L,P}^* \right)\\
&+ 4\ep \abs{L}_{2\lambda^{-1/2},1}
+ \left(2\abs{L}_{2\lambda^{-1/2},1}\lambda^{-1/2} + 1 \right)\\
&\cdot \sqrt{\frac{2\tau +2\log 2\NN(B_H, \norm{\cdot}_\infty,
\frac{1}{2}\lambda^{1/2}\ep)}{n}}.\fishhere
\end{align*}
\end{prop}

\begin{proof}
Satz \ref{prop:5.3.7} zeigte, dass die Abbildung
\begin{align*}
\id : H \to C(X)
\end{align*}
kompakt ist. Daraus folgt
\begin{align*}
\log\NN(B_H,\norm{\cdot}_\infty,\ep) < \ep,\qquad \ep > 0.
\end{align*}
Lemma \ref{prop:6.1.4} und \ref{prop:5.3.1} zeigten
\begin{align*}
&\norm{f_{D,\lambda}}_H \le \lambda^{-1/2},\qquad \lambda > 0,\\
&\norm{f_{D,\lambda}}_\infty \le \lambda^{-1/2},\qquad \text{da }\norm{\id: H\to
C(X)} = \norm{k}_\infty \le 1.
\end{align*} 
Sei nun $\delta > 0$, dann existiert ein $f_\delta\in H$ mit
\begin{align*}
\lambda\norm{f_\delta}_H^2 + \RR_{L,P}(f_\delta) \le
\inf_{f\in H} \left(\lambda \norm{f}_H^2  + \RR_{L,P}(f) \right) + \delta,
\end{align*}
und damit
\begin{align*}
&\lambda\norm{f_{D,\lambda}}^2_H  + \RR_{L,P}(f_{D,\lambda}) -
\inf_{f\in H}\left(\lambda\norm{f}_H^2 + \RR_{L,P}(f) \right)\\
&\le
\lambda\norm{f_{D,\lambda}}^2_H  + \RR_{L,P}(f_{D,\lambda}) -
\lambda\norm{f_\delta}_H^2 - \RR_{L,P}(f_\delta) + \delta\\
&\quad -\RR_{L,D}(f_{D,\lambda}) + \RR_{L,D}(f_{D,\lambda})
\tag{*}
\end{align*}
Weiterhin ist
\begin{align*}
\lambda\norm{f_{D,\lambda}}^2_H  + \RR_{L,D}(f_D) \le
\lambda\norm{f_\delta}_H^2 + \RR_{L,D}(f_\delta),
\end{align*}
da $f_{D,\lambda}$ diese Funktion minimiert. Es folgt
\begin{align*}
\text{(*)} &\le
\lambda\norm{f_\delta}_H^2 + \RR_{L,P}(f_{D,\lambda}) 
-\lambda\norm{f_\delta}_H^2 - \RR_{L,P}(f_\delta)
+ \delta\\
&\quad -\RR_{L,D}(f_{D,\lambda}) + \RR_{L,D}(f_\delta)\\
&=
\RR_{L,P}(f_{D,\lambda}) - \RR_{L,D}(f_{D,\lambda}) - \left(\RR_{L,P}(f_\delta)
- \RR_{L,D}(f_\delta)\right) + \delta\\
&\le 
\abs{\RR_{L,P}(f_{D,\lambda}) - \RR_{L,D}(f_{D,\lambda})} +
\abs{\left(\RR_{L,P}(f_\delta) - \RR_{L,D}(f_\delta)\right)} + \delta.\tag{**}
\end{align*}
Dann gilt ferner
\begin{align*}
\lambda\norm{f_\delta}_H^2 &\le \lambda\norm{f_\delta}_H^2 + \RR_{L,P}(f_\delta)
\le \inf_{f\in H} \left(\lambda\norm{f}_H^2 + \RR_{L,P}(f) \right) + \ep\\
&\le \RR_{L,P}(0) + \delta
\le 1 + \delta \le 4,
\end{align*}
falls $\delta \le 3$. Damit ist
\begin{align*}
&f_\delta \in 2\lambda^{-1/2}B_H,\\
&f_{D,\lambda}\in \lambda^{-1/2}B_H\subset 2\lambda^{-1/2}B_H.
\end{align*}
Folglich ist
\begin{align*}
\text{(**)} \le 2\sup_{f\in 2\lambda^{-1/2}B_H} \abs{\RR_{L,P}(f)-\RR_{L,D}(f)}
+ \delta
\end{align*}
für jedes $\delta > 0$ also gilt auch
\begin{align*}
\text{(**)} \le 2\sup_{f\in 2\lambda^{-1/2}B_H} \abs{\RR_{L,P}(f)-\RR_{L,D}(f)}.
\end{align*}
Definiere nun
\begin{align*}
B\defl 2\lambda^{-1/2}\abs{L}_{2\lambda^{-1/2}} + 1.
\end{align*}
Für $f\in 2\lambda^{-1/2}B_H$ gilt dann
\begin{align*}
L(x,y,f(x)) &\le \abs{L(x,y,f(x))-L(x,y,0)} + \underbrace{L(x,y,0)}_{\le 1}\\
&\le \abs{L}_{2\lambda^{-1/2},1}\abs{f(x)-0} + 1
\le \abs{L}_{2\lambda^{-1/2},1}2\lambda^{-1/2} + 1
\le B.
\end{align*}
Sein nun $\FF_\ep$ ein $\ep$-Netz von $2\lambda^{-1/2}B_H$ und
\begin{align*}
\abs{\FF_\ep} = \NN(2\lambda^{-1/2}B_H,\norm{\cdot}_\infty,\ep) =
\NN(B_H,\norm{\cdot}_\infty,\frac{1}{2}\lambda^{1/2}\ep).
\end{align*}
Für $f\in 2\lambda^{-1/2}B_H$ gibt es $g\in\FF_\ep$ mit $\norm{f-g}_\infty <\ep$
und
\begin{align*}
\abs{\RR_{L,D}(f)-\RR_{L,P}(f)} &\le
\abs{\RR_{L,D}(f)-\RR_{L,D}(g)} + 
\abs{\RR_{L,D}(g)-\RR_{L,P}(g)}\\
&\quad+ \abs{\RR_{L,P}(g)-\RR_{L,P}(f)}\\
&\le 2\ep \abs{L}_{2\lambda^{-1/2},1} + 
\abs{\RR_{L,D}(g)-\RR_{L,P}(g)}.
\end{align*}
Somit ist
\begin{align*}
&\lambda \norm{f_{D,\lambda}}_H^2
+ \RR_{L,P}(f_{D,\lambda}) - \inf_{f\in H}
\left(\lambda\norm{f}_H^2 + \RR_{L,P}(f) \right)\\
&\quad \le 4\ep \abs{L}_{2\lambda^{-1/2},1}
+ 2\sup_{g\in \FF_\ep} \abs{\RR_{L,P}(g)-\RR_{L,D}(g)}.
\end{align*}
Wir schätzen weiterhin ab,
\begin{align*}
&P^n\bigg(\bigg\{D\; :\; \lambda\norm{f_{D,\lambda}}^2_H +
\RR_{L,P}(f_{D,\lambda}) - \inf_{f\in H} \left( \lambda\norm{f}_H^2 +
\RR_{L,P}(f) \right) \\
&\qquad \qquad \quad \ge B\sqrt{\frac{2\tau}{m}} + 4\ep
\abs{L}_{2\lambda^{-1/2}} \bigg\} \bigg)\\
&\le
P^n\left(\setdef{D}{2\sup_{g\in \FF_\ep}\abs{\RR_{L,P}(g) - \RR_{L,D}(g)}\ge
B\sqrt{\frac{2\tau}{m}} } \right)\\
&\overset{!}{\le}
2\abs{\FF_\ep}e^{-\tau}
= \exp(-\tau + \log 2\NN(B_H,\norm{\cdot}_\infty, \frac{1}{2}\lambda^{1/2}\ep)
),
\end{align*}
wobei der union bound und die Hoeffdings Ungleichung verwendet wurden. Mit einer
Variablentransformation für $\tau$ folgt schließlich die Behauptung.\qedhere
\end{proof}

\begin{bem*}
Mit etwas mehr Theorie über Hilberträume kann man zeigen, dass das inf
tatsächlich angenommen wird.\maphere
\end{bem*}

\begin{cor}
\label{prop:6.2.2}
Es gelten die Voraussetzungen des Satzes \ref{prop:6.2.1} und außerdem sei $L$
lipschitz stetig mit $\abs{L}_1 \le 1$. Ferner existieren Konstanten $a\ge 1$,
$p > 0$ mit
\begin{align*}
\log \NN(B_H,\norm{\cdot}_\infty,\ep) \le a \ep^{-2p},\qquad \ep > 0,
\end{align*}
dann folgt, dass für alle $\lambda\in (0,1]$, $\tau > 0$ und $n\ge 1$ mit einer
Wahrscheinlichkeit $P^n$ nicht kleiner als $1-e^{-\tau}$ gilt,
\begin{align*}
\lambda\norm{f_{D,\lambda}}_H^2 + \RR_{L,P}(f_{D,\lambda})
- \RR_{L,P}^* &\le
\inf_{f\in H} \lambda\norm{f}_H^2 + \RR_{L,P}(f) - \RR_{L,P}^*\\
&\quad + 2\lambda^{-1/2}\left(\frac{a}{n}\right)^{(2+2p)^{-1}}
+ 4\lambda^{-1/2}\sqrt{\frac{2\tau+2}{n}}.\fishhere
\end{align*}
\end{cor}
\begin{proof}
Eine einfache Rechnung zeigt \ldots
\begin{align*}
\sqrt{2\tau + 2\log(2\NN(B_H,\norm{\cdot}_\infty, \frac{1}{2}\lambda^{1/2}\ep))}
\le
\sqrt{2\tau + 2\log 2 + a\left(\frac{1}{2}\lambda^{1/2}\ep \right)^{-2p}}
\end{align*}
und da $2\log 2\le 2$, $\sqrt{s+r}\le \sqrt{s}+\sqrt{r}$ für $s,r>0$,
\begin{align*}
\le \sqrt{2(\tau+1)} + a^{1/2}2^p\lambda^{-p/2}\ep^{-p}.
\end{align*}
Da $4\ep \abs{L}_{2\lambda^{-1/2},1} \le 4\ep$ und analog
\begin{align*}
2\abs{L}_{2\lambda^{-1/2},1}\lambda^{-1/2} + 1
\le 2\lambda^{-1/2}+1
\le 4\lambda^{-1/2},
\end{align*}
erhalten wir
\begin{align*}
&4\ep \abs{L}_{2\lambda^{-1/2},1} +
\left(2\abs{L}_{2\lambda^{-1/2},1}\lambda^{-1/2} + 1 \right)
\cdot \sqrt{\frac{2\tau +2\log 2\NN(B_H, \norm{\cdot}_\infty,
\frac{1}{2}\lambda^{1/2}\ep)}{n}}\\
&\le 4\ep + 4\lambda^{-1/2} \left( \sqrt{\frac{2(\tau+1)}{n}}
+ \frac{a^{1/2}2^p\lambda^{-p/2}\ep^{-p}}{\sqrt{n}}\right).
\end{align*}
Betrachten wir nun
\begin{align*}
&h(\ep) \defl \ep + \lambda^{-1/2} 2^p \lambda^{-p/2}
\ep^{-p}\left(\frac{a}{n}\right)^{1/2},
\end{align*}
so führt das Lösen von $h'(\ep) = 0$ auf die Minimalstelle
\begin{align*}
\ep^* = p^{(1+p)^{-1}} 2^{(1+p)^{-1}} \lambda^{-1/2}
\left(\frac{a}{n}\right)^{(2+2p)^{-1}}.
\end{align*}
Setzen wir diese ein, ergibt sich,
\begin{align*}
h(\ep^*) = \underbrace{2^{\frac{p}{1+p}}}_{\le 2}p^{\frac{1-p}{1+p}}
\lambda^{-1/2} \left(\frac{a}{n}\right)^{\frac{1}{2+2p}}
\end{align*}
Weiterhin besitzt
\begin{align*}
&q(p) = p^{\frac{1-p}{1+p}},\\
&q'(p) =
-\frac{p^{-\frac{2p}{1+p}}\left(2\ln p^p - 1 + p^2\right)}{(1+p)^2}
\end{align*}
ein Maximum in $p=1$. Folglich ist
\begin{align*}
h(\ep^*) \le 2 \lambda^{-1/2} \left(\frac{a}{n}\right)^{\frac{1}{2+2p}}.
\end{align*}
Setzen wir dies in die Orakelungleichung \ref{prop:6.2.1} ein, folgt die
Behauptung.\qedhere
\end{proof}

\begin{bsp*}
Sei $X\subset \R^n$ kompakt, $O = X^\circ$ nichtleer.
%  $k$ sei stetiger Kern auf
% $X$, so dass $k\big|_O$ $m$-fach stetig differenzierbar und alle Ableitungen
% stetig auf $X$ fortsetzbar sind.
\begin{align*}
C_b^m(\bar{O}) \defl
\setdef{f\in C(X)\cap C^m(O)}{f^{(k)} \text{
stetig fortsetzbar auf $X$},\; 1\le k\le m}.
\end{align*}
Weiterhin sei $k\in C_b^m(\bar{O})$ ein Kern mit RKHS $H$. Mit Korollar
\ref{prop:5.3.11} folgt, dass
\begin{align*}
\id : H\to C_b^m(\bar{O})
\end{align*}
eine stetige Abbildung ist. Aus den Bemerkungen in Kapitel \ref{sec:4.2} folgt
für die Entropiezahl $e_n$,
\begin{align*}
e_n\left(\id : C_b^m(\bar{O})\to l_\infty(\bar{O}) \right) \le
cn^{-\frac{m}{d}},\qquad n\ge 1
\end{align*}
mit einer Konstanten $c$.
%TODO: Diagramm

Aus der Multiplikativität von Entropiezahlen und $e_i(\cdot) = \norm{\cdot}$
folgt, weiter
\begin{align*}
e_n(\id : H\to l_\infty(\bar{O})) \le cn^{-\frac{m}{d}} \norm{\id: H\to
C_b^m(\bar{O})}.
\end{align*}
Wir können nun
Lemma \ref{prop:4.2.3} anwenden und erhalten
\begin{align*}
\log \NN(B_H,\norm{\cdot}_\infty,\ep) \le \log 4
\left(\frac{c}{\ep}\right)^{d/m}.
\end{align*}
Mit $a=\ln 4 c^{d/m}$ und $2p = \frac{d}{m}$ ist nun Korollar \ref{prop:6.2.2}
anwendbar.

Betrachten wir nun den Gauß-Kern
\begin{align*}
\Omega_\sigma(x,x') = \exp(-\sigma^2\norm{x-x'}^2)
\end{align*}
mit RKHS $H_\Omega$, dann können wir $m$ beliebig groß wählen. Es gilt dann
\begin{align*}
e_n(\id : H_\sigma \to l_\infty(X)) \le c \sigma^m n^{-m/d},\qquad \sigma\ge
1,\quad m\ge 1,\quad n\ge 1
\end{align*}
und $c$ einer von $n,m,\sigma$ unabhängigen Konstanten. Folglich ist
\begin{align*}
\log \NN(B_\sigma,\norm{\cdot}_\infty,\ep)
\le \log 4
\left(\frac{c\sigma^m}{\ep}\right)^{\frac{d}{m}}
=
\underbrace{c^{d/m}\log( 4) \sigma^d}_{\defr a} \ep^{-\frac{d}{m}}.
\end{align*}
Die rechte Seite der Orakelungleichung ist somit bis auf Konstanten, die von
$n$, $\tau$, $m$, $\lambda$ unabhängig sind
\begin{align*}
\lambda^{-1/2} \sigma^{d\cdot \frac{1}{2+2\frac{d}{2m}}}
n^{-\frac{1}{2+2\frac{d}{2m}}} + 4\lambda^{-1/2}\sqrt{\frac{2\tau + 2}{n}}
= \lambda^{-1/2}\sigma^{\frac{dm}{2m+d}}n^{-\frac{m}{2m+d}}+
4\lambda^{-1/2}\sqrt{\frac{2\tau + 2}{n}}
\end{align*}
Für $m>>1$ entspricht dies ungefähr
\begin{align*}
\lambda^{-1/2}\sigma^{d/2}n^{-1/2} +
4\lambda^{-1/2}\sqrt{\frac{2\tau+2}{n}}.\bsphere
\end{align*}
\end{bsp*}

\section{Die Funktion $A(\lambda)$}

Wir definieren die Funktion $A$ durch
\begin{align*}
A(\lambda) \defl \inf_{f\in H} \left(\lambda\norm{f}_H^2 + \RR_{L,P}(f) \right) -
\RR_{L,P}^*,\qquad \lambda > 0.
\end{align*}
Im Folgenden untersuchen wir, wann $A(\lambda)\approx 0$.

\begin{prop}
\label{prop:6.3.1}
Sei $P$ ein W-Maß auf $X\times Y$ und $L$ eine $P$-integrierbare NVF. Dann gilt
für jedes $p\in (0,\infty]$
\begin{align*}
\RR_{L,P}^* =
\inf_{f\in\LL_p(P_X)}\RR_{L,P}(f) = \RR_{L,P,\LL_p(P_X)}^*.\fishhere
\end{align*}
\end{prop}

Es genügt also beschränkte, messbare Funktionen zur Berechnung des Bayes-Risikos
zu betrachten.

\begin{proof}
"`$p=\infty$"': Sei $f: Y\to \R$ messbar mit $\RR_{L,P}(f)  < \infty$. (Falls es
keine solche Funktion gibt, sind wird fertig.) Setze nun
\begin{align*}
f_n \defl f\cdot \chi_{\setd{\abs{f}\le n}},
\end{align*}
dann ist $f_n\in \LL_\infty$ für alle $n\ge 1$. Weiterhin gilt
\begin{align*}
\abs{\RR_{L,P}(f_n)-\RR_{L,P}(f)} &\le 
\int_{X\times Y} \abs{L(x,y,f_n(x))-L(x,y,f(x))}\dP(x,y)\\
&= 
\int_{\setd{\abs{f} > n}} \abs{L(x,y,0)-L(x,y,f(x))} \dP(x,y)\\
&\le
\int_{\setd{\abs{f} > n}} L(x,y,0)+L(x,y,f(x)) \dP(x,y)
\end{align*}
und da $L(x,y,t)\le b(x,y) + h(\abs{t})$ folgt
\begin{align*}
\le \int_{\setd{\abs{f} > n}} \underbrace{b(x,y)+ h(0)}_{\text{(*)}} +
\underbrace{L(x,y,f(x))}_{\text{(**)}} \dP(x,y).
\end{align*}
(*) $\in L^1(P)$, denn $L$ eine $P$-integrierbare NVF, (**) $\in L^1(P)$, da
$\RR_{L,P}(f) < \infty$ und folglich
\begin{align*}
\le \int_{X\times Y} \chi_{\setd{\abs{t}> n}} g \dP \to 0,\qquad n\to\infty,
\end{align*}
denn $g_n\le g$ und $g_n\to 0$. Für $\ep\downarrow 0$ gibt es daher eine
beschränkte Funktion $f_\ep$ mit
\begin{align*}
\abs{\RR_{L,P}(f_\ep)-\RR_{L,P}(f)} \le \ep.
\end{align*}
Dann ist
\begin{align*}
\RR_{L,P}(f) - \ep \le \RR_{L,P}(f_\ep) \le \RR_{L,P}(f) + \ep
\end{align*}
und die Behauptung folgt.

"`$p<\infty$"': $\LL_\infty(P_Y) \subset \LL_p(P_X)$. Damit sind wir
fertig.\qedhere
\end{proof}

\begin{prop}
\label{prop:6.3.2}
Sei $P$ ein W-Maß auf $X\times Y$ und $L$ eine $P$-integrierbare NVF der Ordnung
$p\in [1,\infty)$ und ist $H$ ein universeller RKHS (also insbesondere
ist $X$ kompakt). Dann gilt
\begin{align*}
\RR_{L,P}^* = \RR_{L,P,H}^* \defl \inf_{f\in H} \RR_{L,P}(f).\fishhere
\end{align*}
\end{prop}
\begin{proof}
Satz \ref{prop:2.1.6} besagt, dass die Abbildung
\begin{align*}
\RR_{L,P} : \LL_p(P_X) \to \R
\end{align*} 
stetig ist; weiterhin sind die Abbildungen
\begin{align*}
\id : H\to C(X),\qquad \id : C(X)\to \LL_p(P_X)
\end{align*}
stetig und haben dichtes Bild. Folglich existiert zu $g\in\LL_p(P_X)$ eine Folge
$(f_n)$ in $H$ mit $\norm{f_n-g}_{\LL_p(P_X)} \to 0$.  Aufgrund der Stetigkeit
von $\RR_{L,P}$ ist
\begin{align*}
\lim\limits_{n\to\infty} \RR_{L,P}(f_n) = \RR_{L,P}(g)
\end{align*}
und mit Satz \ref{prop:6.3.1} folgt die Behauptung
\begin{align*}
\RR_{L,P,H}^* = \RR_{L,P,\LL_p(X)}^* = \RR_{L,P}^*.\qedhere
\end{align*}
\end{proof}

\begin{cor*}
Falls die Voraussetzungen des Satzes \ref{prop:6.3.2} erfüllt sind, ist
\begin{align*}
A(\lambda) = \inf_{f\in H} \left(\lambda \norm{f}_H^2 + \RR_{L,P}(f)\right) -
\RR_{L,P,H}^*,\qquad \lambda \ge 0.\fishhere
\end{align*}
\end{cor*}

\begin{bem*}
Satz \ref{prop:6.3.2} gilt auch für nicht universelle Kerne, falls der RKHS $H$
dicht in $\LL_p(P_X)$ liegt.\maphere
\end{bem*}

\begin{lem}
\label{prop:6.3.3}
Sei $L$ eine Verlustfunktion, $H$ ein RKHS über $X$ und $P$ ein W-Maß auf
$X\times Y$, so dass $\RR_{L,P,H}^* = \RR_{L,P}^*$. Dann erfüllt die Funktion
$A: [0,\infty)\to [0,\infty]$ die folgenden Eigenschaften:
\begin{equivenum}
\item\label{prop:6.3.3:1} $A$ ist monoton steigend, stetig und konkav.
\item\label{prop:6.3.3:2} $A(0) = 0$.
\item\label{prop:6.3.3:3} $\mu^{-1}A(\mu)\le \lambda^{-1}A(\lambda)$ für $0 <
\lambda < \mu$.
\item\label{prop:6.3.3:4} $A(\lambda)\le \RR_{L,P}(0)-\RR_{L,P,H}^*$ für alle
$\lambda \ge 0$.
\item\label{prop:6.3.3:5} $A$ ist subadditiv, d.h. $A(\lambda + \mu) \le
A(\lambda) + A(\mu)$ für $\lambda,\mu\ge 0$.
\item\label{prop:6.3.3:6} Falls eine Funktion $h:[0,1]\to [0,\infty)$ mit
$\lim_{\lambda\to 0} h(\lambda) = 0$ und $A(\lambda)\le \lambda h(\lambda)$ für $\lambda\in[0,1]$
existiert, dann ist
\begin{align*}
A(\lambda) = 0,\qquad \lambda \ge 0.\fishhere
\end{align*}
\end{equivenum}
\end{lem}

\begin{proof}
"`\ref{prop:6.3.3:2}"': Man rechnet direkt nach, dass
\begin{align*}
A(0) =  \inf_{f\in H} \left(0\cdot \norm{f}_H^2 + \RR_{L,P}(f) \right) -
\RR_{L,P,H}^*
= \inf_{f\in H} \RR_{L,P}(f) - \RR_{L,P,H}^* = 0.
\end{align*}

"`\ref{prop:6.3.3:1}"': \textit{$A$ ist konkav}. Zu $\lambda\in[0,\infty)$ und
$f\in H$ definiere
\begin{align*}
h_f(\lambda) \defl \lambda\norm{f}_H^2 + \RR_{L,P}(f) -\RR_{L,P,H}^*. 
\end{align*}
Dann ist $h_f$ affin linear und
\begin{align*}
A(\lambda) = \inf_{f\in H} h_f(\lambda).
\end{align*}
Seien $\ep > 0$, $\lambda,\mu \ge 0$ und $\alpha\in [0,1]$. Wir zeigen nun, dass
\begin{align*}
\alpha (A\lambda) + (1-\alpha)A(\mu) \le A(\alpha \lambda + (1-\alpha)\mu).
\end{align*}
Da $A$ das Infimum über $h_f$ ist, existieren $f_1,f_2,f_3\in H$, so dass
\begin{align*}
&A(\lambda) \le h_{f_1}(\lambda) \le A(\lambda)+ \ep,\\
&A(\mu) \le h_{f_2}(\mu) \le A(\mu) + \ep,\\
&h_{f_3}(\alpha\lambda + (1-\alpha)\mu)
\le A(\alpha\lambda + (1-\alpha)\mu) + \ep
\end{align*}
Somit gilt
\begin{align*}
\alpha (A\lambda) + (1-\alpha)A(\mu) &\le \alpha h_{f_1}(\lambda) + (1-\alpha)
h_{f_2}(\mu)\\
&\le \alpha A(\lambda) + (1-\alpha)A(\mu) + \ep\\
&\le \alpha h_{f_3}(\lambda) + (1-\alpha)h_{f_3}(\mu) + \ep
\end{align*}
und da $h_f$ affin linear also konkav folgt
\begin{align*}
\ldots \le
h_{f_3}(\alpha\lambda + (1-\alpha)\mu)+ \ep
\le A(\alpha \lambda + (1-\alpha)\mu) + 2\ep.
\end{align*}
Grenzübergang $\ep\to 0$ liefert, dass $A$ konkav ist.

\textit{$A$ ist stetig in $0$}. Sei $\ep > 0$, dann existiert ein $f_\ep\in H$
mit
\begin{align*}
\RR_{L,P}(f_\ep) - \RR_{L,P,H}^* \le \ep.
\end{align*}
Ohne Einschränkung ist $f_\ep\neq 0$, denn sonst ist $A\equiv 0$. Für
$\lambda\le \norm{f_\ep}_H^{-2}\ep$ folgt daher,
\begin{align*}
0 \le A(\lambda) = \inf_{f\in H} \left(\lambda \norm{f}_H^2 + \RR_{L,P}(f)
\right) - \RR_{L,P,H}^*
\le
\underbrace{\lambda\norm{f_\ep}^2}_{\le \ep} + \underbrace{\RR_{L,P}(f_\ep) -
\RR_{L,P,H}^*}_{\le \ep} \le 2\ep.
\end{align*}
Somit folgt $A(\lambda)\to 0 = A(0)$ für $\lambda \to 0$.

\textit{$A$ ist monoton steigend}. Seien $0\le \lambda\le \mu$, dann ist
\begin{align*}
h_f(\lambda) \le h_f(\mu),\qquad f\in H
\end{align*}
und folglich
\begin{align*}
A(\lambda) = \inf_{f'\in H} h_{f'}(\lambda) \le h_f(\mu)
\end{align*}
Dies gilt für jedes $f$, also auch für das Minimum
\begin{align*}
A(\lambda)\le A(\mu).
\end{align*}

"`\ref{prop:6.3.3:3}"': Sei $\lambda\le \mu$, dann ist
\begin{align*}
&A(\lambda) = A\left(\frac{\lambda}{\mu}\mu +
\left(1-\frac{\lambda}{\mu}\right)\cdot 0 \right)
\ge
\frac{\lambda}{\mu}A(\mu) + \left(1-\frac{\lambda}{\mu}\right)A(0)
= \frac{\lambda}{\mu}A(\mu)\\
\Rightarrow\;
&\frac{1}{\mu}A(\mu) \le \frac{1}{\lambda}A(\lambda).
\end{align*}

"`\ref{prop:6.3.3:5}"': Ohne Einschränkung sei $\lambda\le \mu$ und folglich,
\begin{align*}
A(\lambda + \mu) \le \frac{\lambda+\mu}{\mu}A(\mu)
=\frac{\lambda}{\mu}A(\mu) + A(\mu)
\le A(\lambda) + A(\mu)
\end{align*}

\textit{$A$ ist stetig}. Sei $\lambda > 0$, dann gilt für $\mu\ge 0$,
\begin{align*}
A(\mu) \le A(\lambda + \mu) \le A(\lambda) + A(\mu) \to A(\mu),\qquad
\lambda \to 0.
\end{align*}

"`\ref{prop:6.3.3:4}"': 
\begin{align*}
A(\lambda) &
= \inf_{f\in H} h_f(\lambda)
= \inf_{f\in H} \left(\lambda\norm{f}_H^2 + \RR_{L,P}(f) \right) +
\RR_{L,P,H}^*\\
& \le \RR_{L,P}(0) - \RR_{L,P,H}^*.
\end{align*}

"`\ref{prop:6.3.3:6}"': 
Für $\lambda\in (0,1]$ gilt nach \ref{prop:6.3.3:3}
\begin{align*}
A(1) \le \lambda^{-1}A(\lambda) \le h(\lambda).
\end{align*}
Konvergiert $h(\lambda)\to 0$ für $\lambda\to 0$, so ist
$A(1) = 0$. Da $A$ konkav und nichtnegativ folgt $A\equiv 0$.\qedhere
\end{proof}

\begin{prop}
\label{prop:6.3.4}
Existiert ein $f^*\in H$ mit
\begin{align*}
\RR_{L,P}(f^*) = \RR_{L,P}^* = \RR_{L,P,H}^*.
\end{align*}
Dann ist
\begin{align*}
A(\lambda) \le \lambda\norm{f^*}_H^2,\qquad \lambda \ge 0.\fishhere
\end{align*}
\end{prop}
\begin{proof}
Sei $\lambda \ge 0$, dann ist
\begin{align*}
A(\lambda) = \inf_{f\in H} \left( \lambda\norm{f}_H^2 + \RR_{L,P}(f) \right) -
\RR_{L,P}^* \le \lambda\norm{f^*}_H^2.\qedhere
\end{align*}
\end{proof}

\begin{bem*}[Bemerkungen.]
\begin{bemenum}
\item Sei $L$ konvex und $k$ beschränkt, dann gilt
\begin{align*}
\exists c\ge 0 \forall \lambda \ge 0 : A(\lambda) \le c\lambda
\Leftrightarrow
\exists f^*\in H : \RR_{L,P}(f^*) = \RR_{L,P}^*.\tag{*}
\end{align*}
\item Ein schnelleres Konvergenzverhalten als in (*) ist nur in trivialen
Situationen möglich. Lemma \ref{prop:6.3.3} zeigte, dass dann $A(\lambda)=0$ für
$\lambda \ge 0$, d.h.
\begin{align*}
\RR_{L,P}(0) = \RR_{L,P}^*.
\end{align*}
\item In vielen Situationen ist $A(\lambda)\le c\lambda^\beta$ für $\lambda \ge
0$, mit $c\ge 0$ und $\beta\in(0,1]$ (z.B. gilt dies für die Verlustfunktion
der kleinsten Quadrate), wobei
\begin{align*}
\exists c\ge 0 \forall \lambda \ge 0 : A(\lambda)\le c\lambda^\beta,\quad
\beta\in(0,1] \Leftrightarrow
\exists f_{L,P}^* \in \left[L_2(P_X),H \right]_{\beta,\infty}.\maphere
\end{align*}
\end{bemenum}
\end{bem*}

\begin{lem}
\label{prop:6.3.5}
Sei $\RR_{L,P}^*= \RR_{L,P,H}^*$ und $I\subset (0,\infty)$ ein beschränktes
Intervall und $\Lambda$ sei ein endliches $\ep$-Netz von $I$.

Für Konstanten $\alpha,c\in (0,\infty)$ gilt dann
\begin{align*}
\min_{\lambda\in\Lambda} \left(A(\lambda) + c\lambda^{-\alpha} \right)
\le A(2\ep) + \inf_{\lambda\in I} \left(A(\lambda) + c\lambda^{-\alpha}
\right).\fishhere
\end{align*}
\end{lem}

\begin{proof}
Ohne Einschränkgung ist $\Lambda\defl\setd{\lambda_1,\ldots,\lambda_m}$ mit
$\lambda_{i-1}\le \lambda_i$ und $\lambda_0 \defl \inf I$.

Es gilt $0 < \lambda_i-\lambda_{i-1} \le 2 \ep$ für alle $i=1,\ldots,m$,
denn $B_\ep(\lambda_i)\cap B_\ep(\lambda_{i+1})\neq \varnothing$, sonst wäre
$\Lambda$ kein $\ep$-Netz. 

Sei nun $\delta > 0$, dann existiert ein $\lambda^*\in I$ mit
\begin{align*}
A(\lambda^*) + c(\lambda^*)^{-\alpha} \le 
\inf_{\lambda\in I} \left(A(\lambda) + c\lambda^{-\alpha} \right) + \delta
\end{align*}
und folglich existiert ein $i$ mit $\lambda_{i-1}\le \lambda^*\le\lambda_i$,
also $\lambda_i \le \lambda^*+ 2\ep$. Dann gilt
\begin{align*}
\min_{\lambda\in\Lambda} \left(A(\lambda) + c\lambda^{-\alpha}\right)
&\le A(\lambda_i) + c(\lambda_i)^{-\alpha}
\le A(\lambda^*+2\ep) + c(\lambda^*)^{-\alpha}\\
&\le A(2\ep) + \inf_{\lambda\in I} \left(A(\lambda) + c\lambda^{-\alpha} \right)
+ \delta.
\end{align*}
Im Limes für $\delta \to 0$ folgt die Behauptung.\qedhere 
\end{proof}

\section{Konsistenz und Lernraten für SVMs}

\begin{prop}
\label{prop:6.4.1}
Sei $(X,d)$ ein kompakter, metrischer Raum, $L$ eine konvexe, lipschitz-stetige
Verlustfunktion mit $\abs{L}_1 \le 1$ und $L(x,y,0)\le 1$ für alle $(x,y)\in
X\times Y$. Ferner sei $H$ ein universeller RKHS mit Kern $k$,
$\norm{k}_\infty\le 1$ und
\begin{align*}
\log \NN(B_H,\norm{\cdot}_\infty,\ep) \le a \ep^{-2p},\qquad \ep > 0,
\end{align*}
wobei $a\ge 1$ und $p>0$ Konstanten. Dann ist für alle Folgen $(\lambda_n)$ in
$[0,1]$ mit 
\begin{align*}
\lambda_n\to 0\quad  \text{und} \quad
n\lambda_n^{1+p} \to \infty
\end{align*}
die Lernmethode
\begin{align*}
(X,Y)^n \ni D \mapsto f_{D,\lambda_n}
\end{align*}
universell konsistent.\fishhere
\end{prop}

SVMs mit geeigneter Regularisierung sind also universell konsistent.

\begin{proof}
$L$ ist konvex, lipschitz-stetig und erfüllt $L(x,y,0)\le 1$. Somit ist $L$ eine
$P$-integrierbare NVF der Ordnung $1$ und es folgt
\begin{align*}
\RR_{L,P,H}^* = \RR_{L,P}^*.
\end{align*}
Korollar \ref{prop:6.2.2} sichert weiterhin, dass
\begin{align*}
\RR_{L,P}(f_{D,\lambda_n}) - \RR_{L,P}^*
\le A(\lambda_n) + 2 \lambda_n^{-1/2} \left(\frac{a}{n}\right)^{\frac{1}{2+2p}}
+ 4 \lambda_n^{-1/2} \sqrt{\frac{2\tau+2}{n}}
\end{align*}
mit einer Wahrscheinlichkeit nicht kleiner als $1-e^{-\tau}$.

Da $A(\lambda_n)\to 0$ und $n\lambda^{1+p}\to \infty$ folgt, dass
\begin{align*}
\lambda_n^{-1/2} \left(\frac{1}{n}\right)^{\frac{1}{2+2p}}
= \left(\frac{1}{\lambda_n^{1+p}n}\right)^{\frac{1}{2+2p}} \to 0,
\end{align*}
sowie
\begin{align*}
4\lambda_n^{-1/2}\sqrt{\frac{2\tau+2}{n}}
= 4\sqrt{2\tau+2}\frac{1}{\sqrt{\lambda_n n}}
\le
4\sqrt{2\tau+2}\frac{1}{\sqrt{\lambda_n^{1+p} n}}
\to 0.\qedhere
\end{align*}
\end{proof}

\begin{bsp*}
Betrachte die Verlustfunktion $L=L_\mathrm{hinge}$, dann besagt Zhang's
Ungleichung \ref{prop:2.2.8}
\begin{align*}
\RR_{L_\mathrm{class},P}(f)-\RR_{L_\mathrm{class},P}^*
\le \RR_{L,P}(f)-\RR_{L,P}^*.
\end{align*}
Damit ist die SVM, die $L$ benutzt, unter den Voraussetzungen des Satzes
\ref{prop:6.4.1} nicht nur universell konsistent bezüglich $L$, sondern auch
universell klassifikationskonsistent.\maphere
\end{bsp*}

\begin{prop}
\label{prop:6.4.2}
Es gelten die Voraussetzungen von Satz \ref{prop:6.4.1}. Zudem seien $c > 0$ und
$\beta\in (0,1]$ Konstanten mit
\begin{align*}
A(\lambda) \le c\lambda^\beta.
\end{align*}
Für $n\ge 1$ definiere dann
\begin{align*}
\lambda_n \defl n^{-\frac{1}{(1+p)(1+2\beta)}}.
\end{align*}
Dann lernt die Lernmethode
\begin{align*}
(X\times Y)^n \ni D \mapsto f_{D,\lambda_n}
\end{align*}
in Verteilung mit der Rate $n^{-\frac{1}{(1+p)(1+2\beta)}}$.\fishhere
\end{prop}

\begin{proof}
Mit Korollar \ref{prop:6.2.2} folgt,
\begin{align*}
\lambda\norm{f_{D,\lambda_n}}_H^2 + \RR_{L,P}(f_{D,\lambda_n})-\RR_{L,P}^*
\le c\lambda_n^\beta  + 2\lambda_n^{-1/2}\left(\frac{a}{n}\right)^{(2+2p)^{-1}}
+ 4\lambda_n^{-1/2}\sqrt{\frac{2\tau+2}{n}}
\end{align*}
mit einer Wahrscheinlichkeit nicht kleiner als $1-\ep^{-\tau}$. Einsetzen ergibt
die Behauptung.\qedhere
\end{proof}

\begin{bem*}[Bemerkungen.]
\begin{bemenum}
\item Die Definition von $\lambda_n$ ist asymptotisch (bezgl. $n$) für die
Orakelungleichung aus Korollar \ref{prop:6.2.2} die beste Wahl.
\item Um $\lambda_n$ so definieren zu können, benötigt man Wissen über $\beta$.
In der Regel verfügt man darüber jedoch nicht!\maphere
\end{bemenum}
\end{bem*}

\begin{defn}
\label{defn:6.4.3}
Sei $L$ eine Verlustfunktion, die bei 1 abgeschnitten werden kann, $H$ ein RKHS
über $X$ und $\Lambda = (\Lambda_n)$ eine Familie von endlichen Teilmengen von
$(0,1]$. Für $n\ge 3$ und $D=((x_1,y_1),\ldots,(x_n,y_n))\in (X\times Y)^n$
definiere
\begin{align*}
&m \defl \left\lfloor \frac{n}{2} \right\rfloor,\\
&D_1 = ((x_1,y_1),\ldots,(x_m,y_m)),\\
&D_2 = ((x_{m+1},y_{m+1}),\ldots,(x_n,y_n)),
\end{align*}
und betrachte
\begin{align*}
f_{D_1,\lambda} &= \argmin\limits_{f\in H} \left(\lambda\norm{f}_H^2 +
\RR_{L,D_1}(f) \right),\\
\lambda_{D_2} &\in \argmin_{\lambda\in \Lambda_n}
\RR_{L,D_2}(\cut{f}_{D_1,\lambda}).
\end{align*}
Dann heißt $D\mapsto f_{D,\lambda_{D_2}}$ \emph{TV-SVM
(Training/Validation-Support vector machine)}.~\fishhere
\index{SVM!TV}
\end{defn}

\begin{prop}
\label{prop:6.4.4}
Es gelten die Voraussetzungen von Satz \ref{prop:6.4.1} und $L$ sei bei $1$
abschneidbar. Ferner sei $\Lambda_n$ ein $\ep_n$-Netz von $(0,1]$, wobei
$\ep_n>0$. Für $\tau \ge 1$ sei ferner
\begin{align*}
\tau_n=2\tau+4\log \abs{\Lambda_n} + 1.
\end{align*}
Dann gilt mit einer Wahrscheinlichkeit $P^n$ nicht kleiner als $1-e^{-\tau}$,
\begin{align*}
\RR_{L,P}(\cut{f}_{D_1,\lambda_2}) - \RR_{L,P}^* &\le
\inf_{\lambda\in (0,1]} \bigg( A(\lambda) +
3\lambda^{-1/2}\left(\frac{a}{n}\right)^{(2+2p)^{-1}} 
\\ &
+12\lambda^{-1/2}\sqrt{\frac{\tau_n}{n}} \bigg) + A(2\ep_n).
\end{align*}
Insbesondere ist die TV-SVM universell konsistent, falls $\ep_n\to 0$ und
$n^{-1}\log \abs{\Lambda_n}\to 0$. Ist außerdem $A(\lambda)\le
c\lambda^\beta$ für $\lambda \ge 0$ und geeignetes $c\ge 0$ und $\beta\in(0,1]$,
so lernt die TV-SVM in Verteilung mit Rate $n^{-\frac{\beta}{(1+p)(1+2\beta)}}$,
sofern $\ep \le \min\setd{n^{-1},\abs{\Lambda_n}}$, verhält sich die Lernrate
polynomial in $n$.\fishhere
\end{prop}

\begin{proof}
Da $m=\lfloor\frac{n}{2}\rfloor$ ist $m> \frac{n}{2}$. Mit Korollar
\ref{prop:6.2.2} folgt somit
\begin{align*}
\RR_{L,P}(f_{D_1,\lambda} ) - \RR_{L,P}^*  &\le A(\lambda) +
2\lambda^{-1/2}\left(\frac{a}{n}\right)^{(2+2p)^{-1}} +
4\lambda^{-1/2}\sqrt{\frac{2\tau+2}{m}}\\
&\le A(\lambda) + 3\lambda^{-1/2}\left(\frac{a}{n}\right)^{(2+2p)^{-1}}
+ 8\lambda^{-1/2}\sqrt{\frac{\tau+1}{n}}
\end{align*}
mit einer Wahrscheinlichkeit nicht kleiner als $1-e^{-\tau}$. Mit dem union
bound folgt, dass diese Abschätzung für alle $\lambda\in \Lambda_n$ simultan mit
einer Wahrscheinlichkeit $P^n$ nicht kleiner als $1-\abs{\Lambda_n}e^{-\tau}$
gilt.
Ferner gilt
\begin{align*}
L(x,y,\cut{t}) \le \underbrace{\abs{L}_1}_{\le 1} + \underbrace{L(x,y,0)}_{\le
1} \le 2 \defr B.
\end{align*}
Satz \ref{prop:4.1.2} impliziert mit $n-m \ge n/2-1\ge n/4$, dass
\begin{align*}
\RR_{L,P}(f_{D_1},\lambda_{D_2}) \le
\inf_{\lambda\in \Lambda_n}  \RR_{L,P}(\cut{f}_{D,\lambda}) +
4\sqrt{\frac{2\tau + 2\log 2\abs{\Lambda_n}}{n}} 
\end{align*}
mit Wahrscheinlichkeit $P^{n-m}$ nicht kleiner als $1-e^{-\tau}$. Man sieht mit
Lemma \ref{prop:6.3.5} leicht ein, dass somit auch mit einer Wahrscheinlichkeit
$P^n$ nicht kleiner als $1-(1+\abs{\Lambda_n})e^{-\tau}$ gilt
\begin{align*}
\RR_{L,P}(\cut{f}_{D_1,\lambda_{D_2}}) - \RR_{L,P}^* 
&\le \inf_{\lambda\in \Lambda_n}  \left(\RR_{L,P}(\cut{f}_{D,\lambda}) -
\RR_{L,P}^*\right) + 4\sqrt{\frac{2\tau + 2\log 2\abs{\Lambda_n}}{n}}\\
&\le
\inf_{\lambda\in \Lambda_n}
\left( A(\lambda)+ 3\lambda^{-1/2}\left(\frac{a}{n} \right)^{(2+2p)^{-1}}
+ 8\lambda^{-1/2}\sqrt{\frac{\tau+1}{n}}
\right)\\
&+ 4\sqrt{\frac{2\tau+2\log 2\abs{\Lambda_n}}{n}}\\
&\le
\inf_{\lambda\in (0,1]}
\left(A(\lambda) + 3\lambda^{-1/2}\left(\frac{a}{n} \right)^{(2+2p)^{-1}}+
8\lambda^{-1/2}\sqrt{\frac{\tau+1}{n}} \right)\\
& + A(2\ep)
+ 4\sqrt{\frac{2\tau+2\log 2\abs{\Lambda_n}}n}.
\end{align*}
Weiterhin ist
\begin{align*}
2\tau + 2\log 2 \abs{\Lambda_n}
\le 2\tau + 2 + 2\log \abs{\Lambda_n}
\le \tau_n 
\end{align*}
sowie $\tau+1\le \tau_n$ und $\lambda^{-1/2}\ge 1$ für $\lambda\in (0,1]$. Somit
folgt die Behauptung.\qedhere
\end{proof}

\begin{bsp*}
Wähle $\Lambda_n$ als $\frac{1}{n}$-Netz mit Kardinalität $\le 2-n$. Dann ist
die Konstruktion von $\beta$ unabhängig von $D$ und die TV-SVM lernt mit Rate
$n^{-\frac{\beta}{(1+p)(1+2\beta)}}$.\bsphere
\end{bsp*}

\begin{bem*}[Bemkerungen.]
\begin{bemenum}
\item Die TV-SVM kann bei geeigneter Modifikation auch weitere Parameter adaptiv
bestimmen, wie z.B. den Kernparameter $\sigma$ des Gaußkerns.
\item In der Praxis werden kleinere Netze $\Lambda_n$ benutzt, z.B.
$10\le \abs{\Lambda_n}\le 20$ mit geometrischer Verteilung.
\item Die benutzte Technik mit $\norm{\cdot}_\infty$-Überdeckungszahlen war
Stand der Forschung bis ca. 2002. Dies kann noch deutlich verbessert werden,
denn in Satz \ref{prop:6.2.1} haben wir ausgenutzt, dass $f_{D,\lambda}\in
\lambda^{-1/2}B_H$. Dies führte zu $b\approx \lambda^{-1/2}$ in Hoeffdings
Ungleichung. Man kann dies aus mehreren Gründen noch deutlich verbesern
\begin{enumerate}[label=\arabic{*}.),leftmargin=2pt]
  \item Sei $f_{D,\lambda}\in \argmin_{f\in H}\left( \lambda\norm{f}_H^2 +
  \RR_{L,P}(f)\right)$, dann gilt
\begin{align*}
\lambda\norm{f_{D,\lambda}}_H^2
\le
\lambda\norm{f_{D,\lambda}}_H^2 + \RR_{L,P}(f_{D,\lambda}) - \RR_{L,P}^*
= A(\lambda). 
\end{align*}
Für $A(\lambda)\le c\lambda^\beta$ folgt somit
\begin{align*}
\norm{f_{D,\lambda}}_H \le
\sqrt{c}\lambda^{\frac{\beta-1}{2}},\qquad \lambda \ge 0.
\end{align*}
Wir sind bisher also stets von $\beta = 0$ und $c=1$ ausgegangen\ldots
\item Sei $\lambda_n$ eine Nullfolge mit $\lambda_n^{-1/2} n^{-(2+2p)^{-1}} \to
0$ polynomial. Dann besagt Korollar \ref{prop:6.2.2}, dass mit hoher
Wahrscheinlichkeit gilt
\begin{align*}
\lambda_n \norm{f_{D,\lambda_n}}^2 &\le \lambda_n \norm{f_{D,\lambda_n}}_H^2 +
\RR_{L,P}(f_{D,\lambda_n}) - \RR_{L,P}^*\\ & \le A(\lambda_n)
+ \lambda_n^{-1/2}n^{-(2+2p)},
\end{align*}
was polynomial gegen Null konvergiert, falls $A(\lambda)\le c\lambda^\beta$. Sei
$\alpha$ der entsprechende Exponent, dann folgt
\begin{align*}
\norm{f_{D,\lambda_n}}_H^2 \le \kappa \lambda^{-1/2}n^{-\alpha},\tag{*}
\end{align*}
mit hoher Wahrscheinlichkeit. Betrachte nun im Beweis von Satz \ref{prop:6.2.1}
nur noch die Datensätze für die (*) gilt. Dies ergibt eine bessere
Orakelungleichung, welche wiederum auf eine Verbesserung von (*) führt.
Iteration dieses Arguments liefert eine deutlich verbesserte Orakelungleichung.
\item Kann $L$ abgeschnitten werden, z.B. bei $M=1$, so folgt
\begin{align*}
\norm{\cut{f}_{D,\lambda}}_\infty \le 1,
\end{align*}
wobei wir bisher verwendet haben, dass
\begin{align*}
\norm{f_{D,\lambda}}_\infty \le \lambda^{-1/2}.
\end{align*}
Man sollte daher eigentlich
\begin{align*}
\RR_{L,P}(\cut{f}_{D,\lambda}) - \RR_{L,P}^*
\end{align*}
abschätzen.
\item In vielen Fällen liegt ein sogenanntes \emph{variance bound} vor,
\begin{align*}
\E (L\circ f  - L\circ f_{L,P}^*)^2 \le
V(\E(L\circ f - L\circ f_{L,P}^*))^\th,
\end{align*}
wobei $L\circ f(x,y) = L(x,y,f(x))$ und $V>0$, $\th\in [0,1]$ Konstanten.

Vergleichen wir nun die Fehlerterme von Bernsteins-
\begin{align*}
\sqrt{\frac{\sigma^2 \tau}{n}} + \frac{B}{n}
\end{align*}
und Hoeffdings-Ungleichung
\begin{align*}
B\sqrt{\frac{\tau}{n}},
\end{align*}
so können wir $\sigma^2$ durch $\E (L\circ f  - L\circ f_{L,P}^*)^2$
"`ersetzen"'. Dies führt zu einer neuen Orakelungleichung, die zeigt, dass
\begin{align*}
\RR_{L,P}(f)-\RR_{L,P}^*\tag{**}
\end{align*}
mit hoher Wahrscheinlichkeit klein ist (z.B. $O(n^{-\alpha\th})$). Das variance
bound zeigt dann, dass $\sigma^2 \hat{=} \E (L\circ f  - L\circ f_{L,P}^*)^2$
ebenfalls mit hoher Wahrscheinlichkeit klein ist und folglich ist (**) noch
kleiner (z.B. $O(n^{-\alpha\th-1/2})$). Iteration dieses Arguments ergibt eine
deutliche Verbesserung.
\item Dies zusammengenommen kann Lernraten bis zu $O(n^{-1})$ ergeben.
Vergleiche  dies mit unserem Ergebnis
$O(n^{-\frac{\beta}{(1+p)(1+2\beta)}})$, welches nie kleiner ist als
$O(n^{-1/3})$.
\item Bisher haben wir nur Überdeckungszahlen bezüglich der Supremumsnorm
verwendet. Dies erfordert, dass der Eingaberaum kompakt ist.
Oftmals sind die Daten jedoch auf ganz $\R^n$ verteilt (z.B.
Gauß-verteilt).
Man kann anstatt $\NN(B_H,\norm{\cdot}_\infty,\ep)$ bessere Überdeckungszahlen
wie $\NN(B_H,\norm{\cdot}_{L^2(P_X)},\ep)$ betrachten. Dies geht jedoch nicht
mehr mit elementaren Methoden. Dafür erhält man z.B. für die Verlustfunktion der
kleinsten Quadrate optimale Lernraten. Resultate in dieser Richtung sind relativ
neu (2007-2009).\maphere
\end{enumerate}
\end{bemenum}
\end{bem*}


\chapter{Verlustfunktionen II}

Bisher haben wir die Verlustfunktionen studiert, die das Lernziel gut
beschrieben ($L_\mathrm{class}$, $L_\mathrm{hinge}$, $L_\mathrm{LS}$, \ldots).
Wir haben aber auch schon gesehen, dass $L_\mathrm{hinge}$ in einem günstigen
Verhältnis zu $L_\mathrm{class}$ steht (Zhang's Ungleichung)
\begin{align*}
\RR_{L_\mathrm{class},P}(f)-\RR_{\mathrm{class},P}^* 
\le \RR_{L_\mathrm{hinge},P}(f)-\RR_{L_\mathrm{hinge},P}^*.
\end{align*}
Während die Minimierung des Überschussrisikos von $L_\mathrm{class}$ ein
kombinatorisches Optimierungsproblem darstellt und somit äußerst diffiziel
werden kann, beschreibt die Minimierung des Überschussrisikos von
$L_\mathrm{hinge}$ ein konvexes Optimierungsproblem, zu dessen Lösung wir
bereits Methoden kennengelernt haben (siehe Kapitel \ref{chap:6.1}). Dies
motiviert den Einsatz von $L_\mathrm{hinge}$ z.B. in SMVs. Weiterhin haben wir
in den Übungen gesehen, dass
\begin{align*}
\RR_{L_\mathrm{class},P}(f)-\RR_{L_\mathrm{class},P}^* \le
\sqrt{\RR_{L_\mathrm{LS},P}(f)-\RR_{L_\mathrm{LS},P}^*}.
\end{align*}

Ziel dieses Abschnitts ist es nun zu untersuchen, wie man eine
Zielverlustfunktion $L$, welche das Lernziel beschreibt, durch eine
Ersatzverlustfunktion ersetzen kann, die einer "`solchen"' Ungleichung genügt. 

\section{Marginbasierte Verlustfunktionen}

\begin{prop}[Satz von Bartlett, Jordan, Mc Auliffe (2007)]
\label{prop:7.1}
Sei $L$ eine konvexe, marginbasierte Verlustfunktion, die durch
\begin{align*}
\phi: \R\to [0,\infty)  
\end{align*}
dargestellt wird ($L(y,t) = \phi(y-t)$). Dann sind äquivalent
\begin{equivenum}
\item $\phi$ ist in $0$ differenzierbar mit $\phi'\big|_0 = 0$.
\item Es existiert eine Funktion $Y: [0,1]\to [0,\infty)$ streng monoton und
\begin{align*}
Y(\RR_{L_\mathrm{class},P}(f)-\RR_{L_\mathrm{class},P}^*)\tag{*}
\le \RR_{L,P}(f)-\RR_{L,P}^*
\end{align*}
für alle W-Maße $P$ auf $X\times \setd{-1,1}$ und $f: X\to \R$.

In diesem Fall gilt
\begin{align*}
Y(t) &= \phi(0) - \inf_{s\in\R}
\left(\frac{t+1}{2}\phi(s) + \left(1-\frac{t+1}{2}\right)\phi(-s) \right)\\
&\overset{\eta = \frac{t+1}{2}}{=}
\phi(0) - \inf_{s\in\R} \left(\eta \phi(s) + (1-\eta)\phi(s) \right).\fishhere 
\end{align*} 
\end{equivenum}
\end{prop}

\begin{bsp*}
Die Äquivalenz gilt für $L_\mathrm{hinge}$ und $L_\mathrm{LS}$ mit 
\begin{align*}
&L_\mathrm{hinge}, && Y(t) = t, && \text{für alle } t\in [0,1],\\
&L_\mathrm{LS}, && Y(t) = t^2, && \text{für alle }t\in [0,1].\bsphere
\end{align*}
\end{bsp*}

\begin{prop*}
Existieren $q\in [0,\infty]$, $c_p > 0$ mit
\begin{align*}
P_X\left(\setdef{x\in X}{\abs{2\eta(x)-1} < t}\right)
\le (c_p t)^q,\qquad t\in [0,\infty),\tag{**}
\end{align*}
und gilt für das in (*) definiert $Y$, $Y(t) \ge ct^p$ für ein $p>1$. Dann gilt
\begin{align*}
\RR_{L_\mathrm{class},P}(f)-\RR_{L_\mathrm{class},P}^*
\le 2c^{-\frac{q+1}{q+p}} c_p^{\frac{q(p-1)}{q+p}}\left(\RR_{L,P}(f)-\RR_{L,P}^*
\right)^{\frac{q+1}{q+p}}.\fishhere
\end{align*}
\end{prop*}

(**) heißt \emph{Tsybakov's Noise Assumption}  (2004). Es ist
\begin{align*}
&\eta = \frac{1}{2} \Leftrightarrow
\abs{2\eta-1} = 0,
&&\eta \approx \frac{1}{2} \Leftrightarrow
\abs{2\eta-1} \approx 0.
\end{align*}  
Die linke Seite in (**) misst die Masse der $x$, bei denen $\eta(x)$ nahe bei
$\frac{1}{2}$ ist. Die rechte Seite beschränkt diese Masse. Für $t\to 0$ sagt
die Noise Assumption
\begin{align*}
P_X\left(\setdef{x\in X}{\abs{2\eta(x)-1} < t}\right)
 = O(t^q).
\end{align*}

\newcommand{\aclass}{{\alpha-\mathrm{class}}}
Wir definieren
\begin{align*}
L_{\aclass}(y,t) \defl
\begin{cases}
1-\alpha, & y = 1,\quad t < 0,\\
\alpha, & y= -1, \quad t\ge 0,\\
0 , & \text{sonst}.
\end{cases}
\end{align*}

\begin{prop}
\label{prop:7.2}
Sei $L$ eine konvexe Verlustfunktion, die durch $\phi$ dargestellt werden kann,
mit
\begin{align*}
L_\aclass(y,t) =
\begin{cases}
1-\alpha\phi(t), & y=1,\\
\alpha\phi(t), & y=-1.
\end{cases}
\end{align*}
Dann sind äquivalent
\begin{equivenum}
\item $\phi$ ist in $0$ differenzierbar mit $\phi'(0) = 0$.
\item Es gibt eine streng monoton steigende Funktion $Y: [0,1]\to [0,\infty)$
mit
\begin{align*}
Y\left(\RR_{\aclass,P}(f)-\RR_{\aclass,P}^* \right)
\le \RR_{L_\alpha,P}(f)-\RR_{L_\alpha,P}^*.\fishhere
\end{align*}
\end{equivenum}
\end{prop}

\begin{bsp*}
Für $0<\alpha\le \frac{1}{2}$ sind
\begin{align*}
&L_\mathrm{hinge}, && Y(t) = t,\\
&L_\mathrm{LS}, && Y(t) = \frac{t^2}{2\alpha(1-\alpha)+(1-2\alpha)}.\bsphere
\end{align*}
\end{bsp*}

\section{Distanzbasierte Verlustfunktionen}

\newcommand{\Med}{\mathrm{Median}}
Für die Verlustfunktion der kleinsten Quadrate $L_\mathrm{LS}$ ist
\begin{align*}
\norm{f-f_{L,P}^*}_{L^2(P_X)}^2
= \RR_{L,P}(f) - \RR_{L,P}^*,\qquad f_{L,P}^* = \E(Y\mid x).
\end{align*}
Wie wollen nun den bedingten Median schätzen. Dazu nehmen wir an, dass
\begin{align*}
\Med(Y\mid x) = \setd{f^*(x)},\qquad P_X-\fs
\end{align*}

\begin{prop}
\label{prop:7.2.1}
Sei $L(y,t) = \abs{y-t}$ für $y\in\R$ und $t\in \R$. Dann sind für $f: X\to \R$
äquivalent
\begin{equivenum}
\item $\RR_{L,P}(f) = \RR_{L,P}^*$,
\item $f= f^*$ $P_X$-$\fs$
\end{equivenum}
D.h. der Minimierer des Überschussrisikos entspricht dem Median.\fishhere
\end{prop}

Wenn wir bereits wissen, das $\RR_{L,P}(f)\approx \RR_{L,P}(f^*)$, wie nahe ist
dann $f$ an $f^*$?

\begin{defn*}
Eine Verteilung $Q$ auf $[-1,1]$ hat den \emph{Mediantyp} $q\in (1,\infty)$,
wenn $\alpha_Q\in [0,2]$ und ein $b_Q> 0$ existieren, so dass für $t^*=\Med(Q)$
und jedes $s\in [0,\alpha_Q]$ gelten
\begin{align*}
Q((t^*-s,t^*)) \ge b_Q s^{q-1},\qquad
Q((t^*,t^*+s)) \ge b_Q s^{q-1}.\fishhere
\end{align*}
\end{defn*}

\begin{defn*}
Sei $p\in (0,\infty]$, $q\in (1,\infty)$ und $P$ ein W-Maß auf $X\times [-1,1]$.
Dann hat $P$ einen \emph{$p$-mittelbaren Mediantyp $q$}, wenn folgende
Eigenschaften erfüllt sind
\begin{defnenum}
\item $P(Y\mid x)$ hat Mediantyp $q$ $P_X$-$\fs$.
\item Die Abbildung $\gamma: X\to \R,\quad x\mapsto b_{P(Y\mid
x)}\alpha_{P(Y\mid x)}^{q-1}$ erfüllt $\gamma^{-1}\in L_p(P_X)$.\fishhere
\end{defnenum}
\end{defn*}

\begin{bsp*}
Alle Dichten von $P(Y\mid x)$ erfüllen für $p=\infty$ und $q=2$,
\begin{align*}
\min\setd{1+f^*(x),1-f^*(x)} \ge \alpha^*.
\end{align*}
Unter diesen Voraussetzungen gilt
\begin{align*}
&r \defl \frac{p_q}{p+1},\\
&\norm{f-f^*}_{L_r(P_X)}
\le c\left(\RR_{L,P}(f)-\RR_{L,P}^* \right)^{1/q}
\end{align*}
für alle $f: X\to [-1,1]$ und $r=q=2$.\bsphere
\end{bsp*}

\printindex

\begin{thebibliography}{50}
% \bibitem[Reed, Simon]{SimonReed95} Michael Reed and Barry Simon: Methods of
% modern mathematical physics, Band 1. Academic Press 1995
% \bibitem[Reed, Simon]{SimonReed95} Michael Reed and Barry Simon: Methods of
% modern mathematical physics, Band 1. Academic Press 1995
% \bibitem[Reed, Simon]{SimonReed95} Michael Reed and Barry Simon: Methods of
% modern mathematical physics, Band 1. Academic Press 1995
% \bibitem[Reed, Simon]{SimonReed95} Michael Reed and Barry Simon: Methods of
% modern mathematical physics, Band 1. Academic Press 1995
% \bibitem[Funkana]{Funkana07} Griesemer, Marcel: Funktionalanalysis,
% Sommersemester 2007
% \bibitem[Werner]{Werner07} Werner, Dirk: Funktionalanalysis. 6. Auflage,
% Springer 2007
% \bibitem[Yosida]{Kosaku80} Kosaku Yosida: Functional analysis. Springer
% 1980
\bibitem{CukerZhou07}
F. Cucker and D.X. Zhou, Learning Theory: An Approximation Theory Viewpoint, Cambridge University Press (2007)
\bibitem{Devroye96}
L. Devroye, L. Györfi und G. Lugosi: A Probabilistic Theory of Pattern Recognition, Springer (1996)
\bibitem{Gyoerfi02}
L. Györfi, M. Kohler, A. Krzyzak und Harro Walk, A Distribution-Free Theory of Nonparametric Regression, Springer (2002)
\bibitem{Hastie09}
T. Hastie, R. Tibshirani und J. Friedman, The Elements of Statistical Learning: Data Mining, Inference, and Prediction, Second Edition, Springer (2009)
\bibitem{Hastie08}
I. Steinwart und A. Christmann, Support Vector Machines, Springer (2008)
\bibitem{bauerII} Bauer, H. Wahrscheinlichkeitstheorie; de Gruyter, Berlin
  (1990), 4.\ Aufl.
\bibitem{bauerI} Bauer, H. Maß- und Integrationstheorie; de Gruyter, Berlin
  (1998), 2.~Aufl.
\bibitem{Werner07} Werner, Dirk: Funktionalanalysis. 6. Auflage,
Springer 2007
\end{thebibliography}

% G. Teschl: Mathematical Methods in Quantum Mechanics
% Thirring: Lehrbuch der Mathematischen Physik, Band 3.
% Kato: Perturbation Theory of Linear Operators
% Hislop, Sigal: Introduction to Spectral Theory. With Applications to Schrödinger Operators.
% Cycon, Froese, Kirsch, Simon: Schödinger Operators


\end{document}