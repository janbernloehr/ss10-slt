\chapter{Reproduzierende Kernhilberträume (RKHS)}



Es gibt zwei schöne Klassen von Banachräumen
\begin{itemize}
  \item Funktionenräume.
  \item Hilberträume.
\end{itemize}

RKHS sind beides, der $L^2([0,1])$ übrigens \textit{kein} Funktionenraum!

\section{Kerne und Beispiele}

\begin{defn}
\label{defn:5.1.1}
Eine Funktion $k: X\times X\to\R$ für die ein Hilbertraum $H$ und ein $\Phi:
X\to H$ existieren, so dass
\begin{align*}
k(x,x') = \lin{\Phi(x),\Phi(x')}_H,\qquad \forall x,x'\in X,
\end{align*}
heißt \emph{Kern}.\fishhere\index{Kern}
\end{defn}

\begin{bem*}[Bemerkungen.]
\begin{bemenum}
\item Im Allgemeinen sind $H$ und $\Phi$ nicht eindeutig.\\
Betrachte z.B. $k(x,x') = x\cdot x'$ für $x,x'$, so ist $k$ ein Kern, denn
\begin{align*}
&H_1 = \R,\quad \Phi_1 = \id_\R,\quad \lin{\Phi_1(x),\Phi_1(x')} = x\cdot x'.\\
&H_2 = \R^2,\quad \Phi_2(x) = \frac{1}{\sqrt{2}}\left(x,x\right)
\end{align*}
\item $H$ heißt \emph{Featurespace} (Merkmalraum), $\Phi$ heißt
Featuremap.\maphere
\end{bemenum}
\end{bem*}

\begin{lem}
\label{prop:5.1.2}
Seien $f_n:X\to\R$, so dass $(f_n(x))_{n\ge 1}\in l_2$ für alle $x\in X$, so
definiert
\begin{align*}
k(x,x') = \sum_{n\ge 1} f_n(x)f_n(x'),\qquad x,x'\in X
\end{align*}
einen Kern.\fishhere
\end{lem}
\begin{proof}
$k$ ist wohl definiert, denn nach Cauchy-Schwarz ist
\begin{align*}
\sum_{n\ge 1} \abs{f_n(x)f_n(x')}
\le \norm{(f_n(x))}_{l_2}\norm{(f_n(x'))}_{l_2} < \infty.
\end{align*}
Sei $H=l^2$ und $\Phi: X\to H$ gegeben durch
\begin{align*}
\Phi(x) = (f_n(x))_{n\ge 1},\qquad x\in X.
\end{align*}
Dann ist $\lin{\Phi(x),\Phi(x')} = k(x,x')$ und daher $k$ ein Kern.\qedhere
\end{proof}

\begin{lem}
\label{prop:5.1.3}
Ist $k$ ein Kern auf $X$ und $A:X'\to X$ eine Abbildung, dann definiert
\begin{align*}
k'(x,x') = k(A(x),A(x')),\qquad x,x'\in X
\end{align*}
einen Kern auf $X'$.\fishhere
\end{lem}
\begin{proof}
Sei $H$ ein Featurespace von $k$ und $\Phi: X\to H$ die zugehörige Featuremap.
Setze
\begin{align*}
\Psi(x) = \Phi(A(x)), 
\end{align*}
so gilt
\begin{align*}
k'(x,x') = k(A(x),A(x')) = \lin{\Phi(A(x)),\Phi(A(x'))}_H
= \lin{\Psi(x),\Psi(x')}_H,
\end{align*}
d.h. $\Psi$ ist Featuremap von $k'$.\qedhere
\end{proof}

\begin{lem}
\label{prop:5.1.4}
Sind $k_1$ und $k_2$ Kerne auf $X$ und $\alpha\ge 0$. Dann sind auch $k_1+k_2$
und $\alpha \cdot k_1$ Kerne auf $X$.\fishhere
\end{lem}
\begin{proof}
Der Beweis ist eine leichte Übung.\qedhere
\end{proof}

\begin{lem}
\label{prop:5.1.5}
Sind $k_1$ und $k_2$ Kerne auf $X$, so ist auch $k_1\cdot k_2$ ein Kern auf
$X$.\fishhere
\end{lem}
\begin{proof}
Der Beweis erfolgt unter Verwendung des Tensorprodukts von Hilberträumen. Wir
überspringen diesen, da wir die Aussage nicht benötigen.\qedhere
\end{proof}

\begin{lem}[Lemma für Polynom-Kerne]
\label{prop:5.1.6}
\index{Lemma!für Polynom-Kerne}
Seien $m\ge 0$, $d\ge 1$ natürliche Zahlen und $0 \le c \in\R$. Dann definiert
\begin{align*}
k(x,x') = \left(\lin{x,x'}_{\R^d} + c\right)^m,\qquad x,x'\in \R^d
\end{align*}
einen Kern auf $\R^d$.\fishhere
\end{lem}
\begin{proof}
Nach Konstruktion existiert ein Polynom $p$ der Ordnung $m$ mit
\begin{align*}
k(x,x') = p(\lin{x,x'}),\qquad x,x'\in \R^d,
\end{align*}
wobei $p$ nur nichtnegative Koeffizienten besitzt. Es genügt daher zu zeigen,
dass die Monome
\begin{align*}
(x,x') \mapsto \lin{x,x'}^m
\end{align*}
einen Kern definieren. Nun ist
\begin{align*}
\lin{x,x'}_{\R^d}^m = \left(\sum_{i=1}^d x_ix_i' \right)^m
\end{align*}
also können wir Lemma \ref{prop:5.1.2} anwenden mit $f_i(x) = \pi_i(x)$ der
Projektion auf die $i$-te Koordinate und deren Potenzen.\qedhere
\end{proof}

\begin{lem}[Lemma für Taylor-Kerne]
\label{prop:5.1.7}
\index{Lemma!für Taylor-Kerne}
%Sei $\ocirc{B}_{\R^d}\defl \setdef{x\in\R^d}{\norm{x}_2 < 1}$ und 
Sei $f: (-r,r)\to \R$
mit $r\in (0,\infty]$,
\begin{align*}
f(x) = \sum_{n=0}^\infty a_n t^n,\qquad t\in (-r,r).
\end{align*}
Falls $a_n\ge 0$ für alle $n\ge 0$, definiert
\begin{align*}
k(x,x') = f(\lin{x,x'}_{\R^d}) = \sum_{n=0}^\infty a_n \lin{x,x'}_{\R^d}^n
\end{align*}
einen Kern auf $\sqrt{r}\ocirc{B}_{\R^d}=\setdef{x\in\R^d}{\norm{x}_2 <
\sqrt{r}}$.\fishhere
\end{lem}
\begin{proof}
Für $x,x'\in \sqrt{r}\ocirc{B}_{\R^d}$ ist
\begin{align*}
\abs{\lin{x,x'}_{\R^d}} \le \norm{x}_2\norm{x'}_2 < r
\end{align*}
und daher die Konstruktion sinnvoll. Ferner seien
\begin{align*}
c_{j_1,\ldots,j_d} =
\frac{n!}{\prod_{i=1}^d j_i!},\qquad\qquad 
\sum_{j=1}^d j_i = n,
\end{align*}
dann gilt
\begin{align*}
k(x,x') &= \sum_{n=0}^\infty a_n \left( \sum_{i=1}^d x_i x_i'\right)^n
=
\sum_{n=0}^\infty a_n \sum_{\atop{j_1,\ldots,j_d\ge 0}{j_1+\ldots+j_d =n}}
c_{j_1,\ldots,j_d} \prod_{i=1}^d x_i^{j_i}(x_i')^{j_i}\\
&=
\sum_{n=0}^\infty a_n \sum_{\atop{j_1,\ldots,j_d\ge 0}{j_1+\ldots+j_d =n}}
c_{j_1,\ldots,j_d} \left(\prod_{i=1}^d x_i^{j_i}\right)\left(\prod_{i=1}^d
(x_i')^{j_i}\right).
\end{align*}
Setze $\Phi: X\to l_2(\N_0^d)$, wobei $l_2(I)\defl
\setdef{(a_i)_{i\in\II}}{\norm{(a_i)}_2 < \infty}$, dann ist
\begin{align*}
\Phi(x) = \left(\sqrt{a_{j_1+\ldots+j_d}c_{j_1,\ldots,j_d}}\prod_{i=1}^d
x_i^{j_i} \right)_{(j_1,\ldots,j_d)\in\N_0^d}
\end{align*}
eine Featuremap, denn
\begin{align*}
&\sum_{n=0}^\infty a_n \sum_{\atop{j_1,\ldots,j_d\ge 0}{j_1+\ldots+j_d =n}}
c_{j_1,\ldots,j_d} \left(\prod_{i=1}^d x_i^{j_i}\right)\left(\prod_{i=1}^d
(x_i')^{j_i}\right)\\
&=
\sum_{n=0}^\infty \sum_{\atop{j_1,\ldots,j_d\ge 0}{j_1+\ldots+j_d =n}}
\sqrt{a_{j_1+\ldots+j_d}c_{j_1,\ldots,j_d}}\left(\prod_{i=1}^d x_i^{j_i}\right)
\sqrt{a_{j_1+\ldots+j_d}c_{j_1,\ldots,j_d}}\left(\prod_{i=1}^d
(x_i')^{j_i}\right)\\
&= \lin{\Phi(x),\Phi(x')}_{l^2(\N_0^d)}.\qedhere
\end{align*}
\end{proof}


\begin{bsp}
\label{bsp:5.1.8}
\index{Exponential Kern}
\textit{Exponential Kern}. Die Abbildung
\begin{align*}
k(x,x') = \exp\left(\lin{x,x'}_{\R^d} \right),\qquad x,x'\in\R^d
\end{align*}
ist ein Kern.\bsphere
\end{bsp}

\begin{bsp}
\label{bsp:5.1.9}
\index{Gauß Kern}
\textit{Gauß Kern} oder \textit{Gauß'scher RBF-Kern}. Die Abbildung
\begin{align*}
k_\sigma(x,x') = \exp\left(-\sigma^2\norm{x-x'}_{\R^d}^2\right),\qquad
x,x'\in\R^d
\end{align*}
ist ein Kern.
\begin{proof}
Eine einfache Rechnung zeigt
\begin{align*}
\norm{x-x'}^2 &= \lin{x-x',x-x'} = 
\lin{x,x} - 2\lin{x,x'} + \lin{x',x'}\\
&= \norm{x}^2 + \norm{x'}^2 - 2\lin{x,x'}.
\end{align*}
Setzen wir $h(x) = \exp\left(\sigma^2\norm{x}_{\R^d}^2\right)$, so können wir
den Gauß-Kern schreiben als
\begin{align*}
\exp\left(-\sigma^2\norm{x-x'}_{\R^d}^2\right)
=
\frac{\exp\left(2\sigma^2\lin{x,x'}_{\R^d}\right)}
{h(x)
h(x')}.
\end{align*}
Ferner ist $(x,x')\mapsto \exp(2\sigma^2\lin{x,x'})$ ein Kern nach BSP
\ref{bsp:5.1.8} und Lemma \ref{prop:5.1.3}. Betrachten wir einen zugehörigen
Featurespace $H$ und eine Featuremap $\Phi$ und setzen
\begin{align*}
\Phi_\sigma = \frac{1}{h}\Phi : X\to H,
\end{align*}
so gilt
\begin{align*}
\lin{\Phi_\sigma(x),\Phi_\sigma(x')} &= 
\lin{\frac{\Phi(x)}{h(x)},\frac{\Phi(x')}{h(x')}}
= \frac{1}{h(x)h(x')} \lin{\Phi(x),\Phi(x')}\\
&= \frac{\exp(2\sigma^2\lin{x,x'})}{h(x)h(x')}.\qedhere\bsphere
\end{align*}
\end{proof}
\end{bsp}

\begin{defn}
\label{defn:5.1.10}
Eine Funktion $k: X\times X\to\R$ heißt \emph{positiv definit}\index{positiv
definit}, falls
\begin{align*}
\sum_{i,j=1}^n \alpha_i\alpha_j k(x_i,x_j) \ge 0
\end{align*}
für alle $n\ge 1$ und $\alpha_1,\ldots,\alpha_n\in\R$, d.h. die Grammatritzen
$(k(x_i,x_j))_{i,j=1}^n$ sind positiv definit.

$k$ heißt \emph{strikt positiv
definit}\index{positiv definit!strikt}, falls für alle paarweise verschiedenen
$x_1,\ldots,x_n$ und $\alpha_1,\ldots,\alpha_n\in\R$ gilt
\begin{align*}
\sum_{i,j=1}^n \alpha_i\alpha_j k(x_i,x_j) = 0 \Rightarrow
\alpha_1=\ldots=\alpha_n = 0.
\end{align*}

$k$ heißt \emph{symmetrisch}\index{symmetrisch} falls $k(x,x')=k(x',x)$ für alle
$x,x'\in X$.\fishhere
\end{defn}

\begin{prop}
\label{prop:5.1.11}
$k: X\times X\to\R$ ist genau dann ein Kern, wenn $k$ positiv definit und
symmetrisch.\fishhere
\end{prop}
\begin{proof}
"`$\Rightarrow$"': Sei $\Phi: X\to H$ eine Featuremap von $k$, dann gilt
\begin{align*}
k(x,x') = \lin{\Phi(x),\Phi(x')}
\end{align*}
und jedes Skalarprodukt ist positiv definit und symmetrisch
\begin{align*}
0 &\le \lin{h,h} = \lin{\sum_{i=1}^n \alpha_i\Phi(x_i),\sum_{j=1}^n \alpha_j
\Phi(x_j) }\\
&=
\sum_{i,j=1}^n \alpha_i \alpha_j \lin{\Phi(x_i),\Phi(x_j)}
= 
\sum_{i,j=1}^n \alpha_i \alpha_j k(x_i,x_j).
\end{align*}
"`$\Leftarrow$"':
Wir müssen Featurespace und -map konstruieren. Dazu setzen wir
\begin{align*}
H_\mathrm{pre} = \setdef{\sum_{i=1}^n \alpha_i k(x_i,\cdot)}{n\ge 1,\;
\alpha_1,\ldots,\alpha_n\in\R,\; x_1,\ldots,x_n\in X}
\end{align*}
und für $f=\sum_{i=1}^n \alpha_i k(x_i,\cdot)$ und $g=\sum_{j=1}^n \beta_j
k(x_j,\cdot)$ setze in $H_\mathrm{pre}$,
\begin{align*}
\lin{f,g}_{H_\mathrm{pre}} = 
\sum_{i=1}^n\sum_{j=1}^n \alpha_i \beta_j k(x_i,x_j).
\end{align*}
Weiterhin ist
\begin{align*}
&\lin{f,g}_{H_\mathrm{pre}} = 
\sum_{j=1}^m \beta_j f(x_j'),\\
&\lin{f,g}_{H_\mathrm{pre}} = 
\sum_{i=1}^m \alpha_i g(x_i),
\end{align*}
also ist $\lin{\cdot,\cdot}_{H_\mathrm{pre}}$ von der speziellen Darstellung von
$f$ oder $g$ unabhängig und daher wohldefiniert.

Offensichtlich ist $\lin{\cdot,\cdot}_{H_\mathrm{pre}}$ bilinear und
symmetrisch, da $k$ symmetrisch.

Zu zeigen ist also noch, dass $\lin{f,f}=0\Rightarrow f=0$. Wir zeigen dazu,
dass $\lin{\cdot,\cdot}_{H_\mathrm{pre}}$ die Cauchy-Schwarz'sche Ungleichung
erfüllt.

\textit{1. Fall}. Ist $\lin{f,f}=0$ und $\lin{g,g} = 0$, dann ist
\begin{align*}
&0\le \lin{f+g,f+g} = \lin{f,f} + 2\lin{f,g} + \lin{g,g} = 2\lin{f,g},\\
&0\le \lin{f-g,f-g} = \lin{f,f} - 2\lin{f,g} + \lin{g,g} = -2\lin{f,g}.
\end{align*}
Folglich ist $\lin{f,g}=0$ und die Cauchy-Schwarz'sche Ungleichung folgt. 

\textit{2. Fall}. Ohne Einschränkung ist $\lin{g,g}>0$, setzen wir also $\alpha
= - \frac{\lin{f,g}}{\lin{g,g}}$. Dann folgt
\begin{align*}
0\le \lin{f+\alpha g,f+\alpha g} = \ldots = \lin{f,f} -
\frac{\lin{f,g}^2}{\lin{g,g}}.
\end{align*}

% Somit ist $\lin{\cdot,\cdot}_{H_\mathrm{pre}}$ ein Skalarprodukt auf
% $H_\mathrm{pre}$. 

Sei nun $f\in H_\mathrm{pre}$ mit $\lin{f,f} = 0$ und
$
f= \sum_{i=1}^n \alpha_i k(x_i,\cdot).
$
Dann ist
\begin{align*}
\abs{f(x)}^2 &= \abs{\sum_{i=1}^n \alpha_i k(x_i,\cdot)}^2
=
\abs{\lin{f,k(x,\cdot)}}^2\\
&\le \lin{f,f}\lin{k(x,\cdot),k(x,\cdot)} = 0.
\end{align*}
Folglich ist $\lin{\cdot,\cdot}_{H_\mathrm{pre}}$ ein Skalarprodukt und damit
$H_\mathrm{pre}$ ein Prähilbertraum.

Zu $H_\mathrm{pre}$ existiert eine Vervollständigung $H$ mit einer isometrischen
Einbettung $J: H_\mathrm{pre}\to H$.
Setze nun $\Phi: X \to H$, $x\mapsto JK(x,\cdot)$, dann gilt
\begin{align*}
\lin{\Phi(x),\Phi(x')}_H &= \lin{Jk(x,\cdot),Jk(x',\cdot)}_H =
\lin{k(x,\cdot),k(x',\cdot)}_{H_\mathrm{pre}} \\ &= k(x,x').\qedhere
\end{align*}
\end{proof}

\begin{cor}
\label{prop:5.1.12}
Sind $k_n$, $n\ge 1$ Kerne auf $X$ und gibt es eine Funktion
$k: X\times X \to \R$ mit $k_n(x,x')\to k(x,x')$ für alle $x,x'\in X$, so ist
$k$ ein Kern.\fishhere
\end{cor}
\begin{proof}
Alle $k_n$ sind symmetrisch und daher ist es die Grenzfunktion $k$ auch.
Weiterhin gilt
\begin{align*}
0 \le \sum_{i,j=1}^d \alpha_i\alpha_j k_n(x_i,x_j) \to
\sum_{i,j=1}^n \alpha_i\alpha_j k(x_i,x_j)
\end{align*}
und daher ist $k$ positiv definit.\qedhere
\end{proof}

\section{RKHS (Reproduzierende Kern-Hilbert-Räume)}

Bisher waren Featurespace und -map nicht eindeutig, es gibt aber eine kanonische
Wahl, den RKHS.

\begin{defn}
\label{defn:5.2.1}
\begin{defnenum}
\item Ein Hilbertraum $H$, der aus Funktionen $f: X\to\R$ besteht, heißt
Hilbertfunktionenraum über $X$. (HFS)\index{Hilbertfunktionenraum}
\item Ein HFS $H$ über $X$ heißt RKHS, falls die Dirac-Funktionale
\begin{align*}
\delta_x: H\to\R,\quad f\mapsto f(x)
\end{align*}
für alle $x\in X$ stetig sind.
\item\label{defn:5.2.1:3} Eine Funktion $k: X\times X\to \R$ heißt
\emph{reproduzierender Kern}
\index{reproduzierender Kern}
eines HFS über $X$, falls für jedes $x\in X$ $k(x,\cdot)\in H$ und
die \emph{reproduzierende Eigenschaft}\index{reproduzierende Eigenschaft} erfüllt ist,
\begin{align*}
f(x) = \lin{f,k(x,\cdot)},\qquad \text{für alle } f\in H,\; x\in X.\fishhere
\end{align*}
\end{defnenum}
\end{defn}

\begin{bem*}[Bemkerungen.]
\begin{bemenum}
\item $L_2([0,1])$ ist kein RKHS (nicht mal ein HFS).
\item Norm-Konvergenz in einem RKHS $H$ impliziert punktweise Konvergenz.
\begin{proof}
Sei $x\in X$ und gelte $\norm{f_n-f}_H\to 0$, dann gilt
\begin{align*}
f(x) = \delta_x(f) =
\delta_x\left(\lim\limits_{n\to\infty} f_n\right) =
\lim\limits_{n\to\infty} \delta_x(f_n) 
= \lim\limits_{n\to\infty} f_n(x).\qedhere
\end{align*}
\end{proof}
\item \ref{defn:5.2.1:3} wurde für den im Beweis von Satz \ref{prop:5.1.11}
konstruierten Hilbertraum schon "`fast"'
gezeigt.\maphere
\end{bemenum}
\end{bem*}

\begin{lem}
\label{prop:5.2.2}
Sei $H$ ein HFS über $X$ und $k: X\times X\to \R$ ein reproduzierender Kern von
$H$. Dann gelten
\begin{propenum}
\item $H$ ist ein RKHS.
\item $k$ ist ein Kern, $H$ ist ein FS von $k$ und
\begin{align*}
\Phi: X\to H,\qquad x\mapsto k(x,\cdot)
\end{align*}
die \emph{kanonische Featuremap} von $k$.\fishhere
\end{propenum}
\end{lem}
\begin{proof}
\begin{proofenum}
\item Eine direkte Rechung zeigt
\begin{align*}
\abs{\delta_x(f)} = \abs{f(x)} = \abs{\lin{f,k(x,\cdot)}_H} \le
\norm{f}_H\norm{k(x,\cdot)}_H.
\end{align*}
Somit ist $\delta_x$ stetig und $\norm{\delta_x} \le \norm{k(x,\cdot)}$.
\item Für $x'\in X$ fest schreibe $f\defl k(x',\cdot)$. Dann ist $f\in H$ per
definitionem und
\begin{align*}
\lin{\Phi(x),\Phi(x')}_H &= \lin{k(x,\cdot),f}_H = \lin{f,k(x,\cdot)}_H = f(x) =
k(x',x)\\ &= \lin{\Phi(x'),\Phi(x)}.\qedhere
\end{align*}
\end{proofenum}
\end{proof}

\begin{prop}[Existenz und Eindeutigkeit des reproduzierenden Kerns]
\label{prop:5.2.3}
Sei $H$ ein RKHS über $X$. Dann gilt
\begin{align*}
k: X\times X\to \R,\qquad
k(x,x')=\lin{\delta_x,\delta_{x'}},
\end{align*}
ist der einzige reproduzierende Kern von $H$. Ist ferner $(e_i)_{i\in\II}$ eine
Orthonormalbasis (ONB), so gilt
\begin{align*}
k(x,x') = \sum_{i\in\II} e_i(x) e_i(x'),\qquad x,x'\in X.\fishhere
\end{align*}
\end{prop}
\begin{proof}
\textit{Existenz}. Der Darstellungssatz von Fréchet-Riesz besagt, dass
\begin{align*}
I : H\to H',\qquad f\mapsto \lin{f,\cdot}_H
\end{align*}
ein isometrischer Isomorphismus ist, d.h. $H=H'$. Sei $J=I^{-1}: H'\to H$, so
ist auch $J$ ein isometrischer Isomorphismus und  für $f\in H$ und $\omega\in
H'$ gilt
\begin{align*}
\omega(f) = \lin{J\omega,f}_H.\tag{*}
\end{align*}
Für $x,x'$ folgt
\begin{align*}
k(x,x') = \lin{\delta_x,\delta_{x'}}_{H'} = \lin{J\delta_x,J\delta_{x'}}_H
\overset{\text{(*)}}{=} \delta_x(J\delta_{x'}) = J\delta_{x'}(x).
\end{align*}
Folglich ist $J\delta_{x'} = k(\cdot,x')$ und damit
\begin{align*}
f(x') - \delta_{x'}(f) = \lin{f,J\delta_{x'}}_H = 
\lin{f,k(\cdot,x')}_H = \lin{f,k(x',\cdot)}_H,
\end{align*}
d.h. $k$ ist reproduzierender Kern von $H$.

\textit{Eindeutigkeit}. Sei $\tilde{k}$ ein beliebiger reproduzierender Kern von
$H$.

Für $x'\in X$ gilt $k(x',\cdot)\in H$. Für die ONB $(e_i)_{i\in\II}$ folgt
\begin{align*}
k(x',\cdot) = \sum_{i\in\II} \lin{k(x',\cdot), e_i}_He_i =
\sum_{i\in\II} e_i(x')e_i,
\end{align*}
wobei
\begin{align*}
k(x',x) = \sum_{i\in\II } e_i(x')e_i(x)
\end{align*}
punktweise existiert. Insbesondere ist $\tilde{k}$ der einzige reproduzierende
Kern von $H$ und die Darstellung mithilfe der ONB $(e_i)_{i\in\II}$ hängt nicht
von der Wahl der ONB ab.\qedhere
\end{proof}

\begin{prop}
\label{prop:5.2.4}
Sei $k$ ein Kern über $X$ mit Featurespace $H_0$ und Featuremap $\Phi_0:X\to
H_0$. Dann gelten
\begin{propenum}
\item\label{prop:5.2.4:1} $H\defl\setdef{f: X\to\R}{\exists \omega\in H_0 : f(x) =
\lin{\omega,\Phi_0(x)}_{H_0} \forall x\in X}$ mit
\begin{align*}
\norm{f}_H \defl \inf\setdef{\norm{\omega}_{H_0}}{\lin{\omega,\Phi_0(\cdot)}_{H_0}
= f}
\end{align*}
ist der einzige RKHS von dem $k$ ein reproduzierender Kern ist.
\item\label{prop:5.2.4:2} Der Operator $V: H_0\to H$, $\omega\mapsto
\lin{\omega,\Phi_0(\cdot)}_{H_0}$ ist eine metrische Surjektion, d.h. linear,
beschränkt und $V\ocirc{B}_{H_0} = \ocirc{B}_H$.
\item\label{prop:5.2.4:3} $H_\pre = \setdef{\sum_{i=1}^n \alpha_i
k(x_i,\cdot)}{n\ge 1,\; \alpha_i\in \R,\; x_i\in X}$ ist dicht in $H$.
\item\label{prop:5.2.4:4} Für $f\in H_\pre$ mit $f=\sum_{i=1}^n \alpha_i
k(x_i,\cdot)$ gilt
\begin{align*}
\norm{f}_H^2 = \sum_{i,j=1}^n \alpha_i\alpha_j k(x_i,x_j).\fishhere 
\end{align*}
\end{propenum}
\end{prop}

\begin{bem*}[Verleich.]
\ref{prop:5.1.11}: Der FS ist eine Vervollständigung von $H_\pre$.

\ref{prop:5.2.4}: Es existiert eine Vervollständigung, die aus Funktionen $X\to
\R$ besteht und die RKHS ist.\maphere
\end{bem*}

\begin{proof}
\begin{proofenum}
\item
Zeige $H$ ist HFS über $X$. Offensichtlich besteht $H$ aus Funktionen $X\to \R$
und $V: H_0\to H$ ist linear. Für $f\in H$ gilt ferner
\begin{align*}
\norm{f}_H = \inf\setdef{\norm{\omega}_{H_0}}{\omega\in V^{-1}(\setd{f})}.
\end{align*}
Zeige $\norm{\cdot}_H$ ist Hilbertraumnorm auf $H$.
\begin{proofenuma}
\item $\ker V = \setdef{\omega\in H_0}{V\omega = 0}=V^{-1}(\setd{0})$ ist
abgeschlossen, da $V$ stetig.
% Für eine Folge $(\omega_n)$ in $\ker V$ mit $\omega_n\to \omega\in H_0$ gilt
% \begin{align*}
% \lin{\omega,\Phi_0(x)}_{H_0} = \lim\limits_{n\to\infty}
% \underbrace{\lin{\omega_n,\Phi_0(x)}_{H_0}}_{V\omega_n(x) = 0} = 0,
% \end{align*}
% d.h. $V\omega(x) = 0$ für alle $x\in X$ und daher ist $\omega\in \ker V$.
\item Sei $\tilde{H}$ das orthogonale Komplement von $\ker V$ in $H_0$. 
%, also $\tilde{H}\bot \ker V$ und $H_0= \tilde{H}\oplus \ker V$.
Dann ist
\begin{align*}
V\big|_{\tilde{H}} : \tilde{H}\to H
\end{align*}
injektiv nach Konstruktion und außerdem surjektiv, denn $V$ ist surjektiv, d.h.
zu $f\in H$ existiert ein $\omega\in H_0$ mit $V\omega = f$. Dann ist
\begin{align*}
&\omega = \tilde{\omega} + \omega_0,\quad
\tilde{\omega} \in \tilde{H},\; \omega_0\in H_0,\\
&f = V(\tilde{\omega}+\omega_0) = V(\tilde{\omega}) +
\underbrace{V(\omega_0)}_{=0} = V\big|_{\tilde{H}}(\tilde{\omega}).
\end{align*}
\item Zeige $\norm{f}_H^2 = \norm{(V\big|_{\tilde{H}})f}_H$.
\begin{align*}
\norm{f}_H^2 &= \inf \setdef{\norm{\omega_0 +
\tilde{\omega}}_{H_0}^2}
{\omega_0\in \ker V, \tilde{\omega}\in H} \\ &
= \inf \setdef{\norm{\omega_0}_{H_0}^2 + \norm{\tilde{\omega}}_{H_0}^2}
{\omega_0\in \ker V, \tilde{\omega}\in H}\\
&= \norm{V\big|_{\tilde{H}}^{-1}(f)}_{\tilde{H}}^2.
\end{align*}
Damit ist $H$ ein Hilbertraum, da $\tilde{H}$ ein Hilbertraum ist,
$V\big|_{\tilde{H}}: \tilde{H}\to H$ bijektiv linear und isometrisch.

Ferner folgt, dass $V$ eine metrische Surjektion ist.
\end{proofenuma}
\item
Zeige $H$ ist ein RKHS und $k$ ein reproduzierender Kern von $H$.
\begin{proofenuma}
\item Für $x\in X$ gilt
\begin{align*}
k(x,\cdot) = \lin{\Phi_0(x),\Phi_0(\cdot)}_{H_0} = V\Phi_0(x)
\end{align*}
und folglich ist $k(x,\cdot)\in H$.
\item $\Phi_0(x)\in \tilde{H} = (\ker V)^\bot$, denn es gilt für $\omega\in \ker
V$,
\begin{align*}
\lin{\omega, \Phi_0(x)}_{H_0} = V_\omega(x) = 0,\qquad x\in X.
\end{align*}
\item Zeige, dass $f(x)= \lin{f,k(x,\cdot)}_H$. Da $f =
V(V\big|_{\tilde{H}})^{-1}f$ und $V\big|_{\tilde{H}}$ Isometrie, folgt
\begin{align*}
f(x) = \lin{(V\big|_{\tilde{H}})^{-1}f,\Phi_0(x)}_{H_0} = 
\lin{f,V\big|_{\tilde{H}},\Phi_0(x)}_H = \lin{f, k(x,\cdot)}_H.
\end{align*}
Somit ist $k$ ein reproduzierender Kern und nach Lemma \ref{prop:5.2.2} $H$ ein
RKHS.
\end{proofenuma}
\item Zeige \ref{prop:5.2.4:3} und \ref{prop:5.2.4:4} gilt für jeden RKHS
$\bar{H}$ für den $k$ reproduzierender Kern ist. Wir zeigen dazu, dass

\begin{align*}
H_\pre = \setdef{\sum_{i=1}^n \alpha_i k(x_i)}{n\ge 1,\; \alpha_i\in\R,\;
x_i\in X}
\end{align*}
dicht ist in $\bar{H}$ und für $f = \sum_{i=1}^n \alpha_i k(x_i,\cdot)$ gilt
\begin{align*}
\norm{f}_{\bar{H}}^2 = \sum_{i,j=1}^n \alpha_i \alpha_j k(x_i,x_j).
\end{align*}
\begin{proofenuma}
\item $k(x,\cdot) \in H_\pre$ und folglich ist $H_\pre \subset \bar{H}$ und
$k(x,\cdot)\in \bar{H}$.
\item Zeige, dass $\bar{H_\pre}^{\bar{H}} = \bar{H}$. Angenommen
$\bar{H_\pre}^{\bar{H}} \subsetneq \bar{H}$, dann ist ${H_\pre}^\bot \neq (0)$,
d.h. es existiert ein $f\in {H_\pre}^\bot$ und ein $x\in X$ mit $f(x)\neq 0$
und $f\ \bot\  k(x,\cdot)$.
\begin{align*}
\Rightarrow 0 \neq f(x) = \lin{f,k(x,\cdot)}_{\bar{H}} = 0.\dipper 
\end{align*}
D.h. $\bar{H_\pre}^\bar{H}=\bar{H}$, also ist $H_\pre$ dicht in $\bar{H}$.

Sei nun $f=\sum_{i=1}^n \alpha_i k(x_i,\cdot)$, dann gilt
\begin{align*}
\norm{f}_{\bar{H}}^2 &= \norm{\sum_{i=1}^n \alpha_i k(x_i,\cdot)}_{\bar{H}}^2
= \lin{\sum_{i=1}^n \alpha_i k(x_i,\cdot),\sum_{j=1}^n \alpha_j k(x_j,\cdot)}\\
&= \sum_{i,j=1}^n \alpha_i\alpha_j \lin{k(x_i,\cdot),k(x_j,\cdot)}_{\bar{H}}
= \sum_{i,j=1}^n \alpha_i\alpha_j k(x_i,x_j)
\end{align*}
aufgrund der repräsentierenden Eigenschaft angewandt auf $g=k(x_i,x\cdot)$.
\end{proofenuma}
\item Zeige schließlich $H$ ist der einzige RKHS für den $k$ ein
reproduzierender Kern ist.

Seien dazu $H_1$ und $H_2$ RKHS für die $k$ ein reproduzierender Kern ist. Nach
dem bisher Gezeigten ist $H_\pre$ dicht in $H_1$ und $H_2$ und die Normen von
$H_1$ und $H_2$ sind auf $H_\pre$ identisch. Wir zeigen nun, dass $H_1\subset
H_2$ und $\norm{f}_{H_1}= \norm{f}_{H_2}$ für alle $f\in H_1$. Dann folgt die
Gleichheit aus der Symmetrie des Problems und $H$ ist eindeutig.

Sei also $f\in H_1$, dann existiert eine Folge $(f_n)$ in $H_\pre$ mit
\begin{align*}
\norm{f_n-f}_{H_1}\to 0,\qquad n\to\infty.
\end{align*}
Folglich ist $(f_n)$ Cauchyfolge in $H_\pre$
bezüglich $\norm{\cdot}_{H_1}$ und damit auch bezüglich $\norm{\cdot}_2$, also
ist $(f_n)$ Cauchyfolge in $H_2$, d.h. es existiert ein $g\in H_2$, so dass
\begin{align*}
\norm{f_n-g}_{H_2} \to 0,\qquad n\to\infty.
\end{align*}
Da die Konvergenz in $H_{1,2}$ der punktweisen Konvergenz entspricht gilt daher
auch
\begin{align*}
f(x) = \lim\limits_{n\to\infty} f_n(x) = g(x),\qquad x\in X.
\end{align*}
Folglich ist $f=g\in H_2$ und daher $H_1\subset H_2$. Da außerdem
\begin{align*}
\norm{f_n-f}_{H_2} \to 0,\qquad n\to\infty
\end{align*}
folgt $\norm{f_n}_{H_2}\to \norm{f}_{H_2}$, d.h.
\begin{align*}
\norm{f}_{H_1} =\lim\limits_{n\to\infty} \norm{f_n}_{H_1}
=
\lim\limits_{n\to\infty} \norm{f_n}_{H_2}
=
\norm{f}_{H_2}.\qedhere
\end{align*}
\end{proofenum}
\end{proof}

\begin{bem*}[Interpretation.]
\begin{itemize}
  \item[-] Zu jedem Kern $k$ existiert genau ein RKHS, so dass $k$ ein
  reproduzierender Kern von $H$ ist.
  \item[-] Zu jedem RKHS $H$ existiert genau ein reproduzierender Kern von $H$
  und dieser ist auch ein Kern.
\end{itemize}
\noindent
Zwischen Kern und RKHS besteht also eine 1:1 Relation.

Der RKHS ist der "`kleinste Featurespace"' im Sinne von Surjektionen. $H$
besteht aus den Funktionen, die entstehen, wenn Daten in einen Featurespace $H_0$
abgebildet werden und dann ein linearer Ansatz gemacht wird.\maphere
\end{bem*}

\section{Eigenschaften von Kernen und RKHS}

Ziel dieses Abschnittes ist es, eine Beziehung zwischen den Eigenschaften von
Kernen und den Eigenschaften der Funktionen eines RKHS herzustellen.

\begin{lem}
\label{prop:5.3.1}
Sei $k$ ein Kern auf $X$ und $H$ der RKHS von $k$, dann sind äquivalent
\begin{equivenum}
\item\label{prop:5.3.1:1} $k$ ist beschränkt.
\item\label{prop:5.3.1:2} Alle $f\in H$ sind beschränkt.
\end{equivenum}
In diesem Fall gilt zudem
\begin{align*}
\norm{\id: H\to \LL_\infty(X)} = \norm{k}_\infty ,\qquad
\norm{k}^2_\infty = \sup_{x\in X} k(x,x) < \infty.\fishhere
\end{align*}
% mit
% \begin{align*}
% 
% \end{align*}
\end{lem}
\begin{proof}
"`\ref{prop:5.3.1:1}$\Rightarrow$\ref{prop:5.3.1:2}"': Für $f\in H$ und $x\in X$
gilt da $k(x,\cdot)\in H_\pre$ nach Satz \ref{prop:5.2.4},
\begin{align*}
\abs{f(x)} &= \abs{\lin{f,k(x,\cdot)}_H} \le \norm{f}_H\norm{k(x,\cdot)}_H
= \norm{f}_H \sqrt{k(x,x)}\\
&\le \norm{f}_H \underbrace{\norm{k}_\infty}_{< \infty}.
\end{align*}
Somit ist $f\in \LL_\infty(X)$ und $\norm{f}_\infty \le \norm{k}_\infty
\norm{f}_H$.
Also gilt insbesondere
\begin{align*}
\norm{\id : H\to \LL_\infty(X)} \le \norm{k}_\infty.
\end{align*}
"`\ref{prop:5.3.1:2}$\Rightarrow$\ref{prop:5.3.1:1}"': $\id: H\to \LL_\infty(X)$
ist wohldefiniert nach Voraussetzung. Wir zeigen, dass $\id: H\to \LL_\infty(X)$
stetig ist. Dazu benutzen wir den Satz vom abgeschlossenen Graphen. Sei dazu
$(f_n)$ eine Folge in $H$ und $g\in \LL_\infty(X)$ mit
\begin{align*}
\norm{f_n-f}_H \to 0,\qquad \norm{\id(f_n) - g}_{\LL_\infty(X)} \to 0,\tag{*}
\end{align*}
so ist zu zeigen, dass $f=g$.

Nach (*) gelten
\begin{align*}
f_n(x)\to f(x),\qquad f_n(x)\to g(x),\qquad \forall x\in X.
\end{align*}
Somit ist $f(x) = g(x)$ für alle $x\in X$ und daher ist
\begin{align*}
\id : H\to \LL_\infty(X)
\end{align*}
%abgeschlossen also stetig?
stetig. Weiterhin ist
\begin{align*}
k(x,x) &\le \norm{k(x,\cdot)}_\infty
\le \norm{\id: H \to \LL_\infty(x)}\norm{k(x,\cdot)}_H\\
&= \norm{\id: H\to \LL_\infty(x)}\sqrt{k(x,x)}
\end{align*}
und daher
\begin{align*}
&\sqrt{k(x,x)} \le \norm{\id: H\to \LL_\infty(x)},\\
\Rightarrow &
\norm{k}_\infty \le  \norm{\id: H\to \LL_\infty(x)}.
\end{align*}
Insbesondere ist $k$ auf der Diagonalen beschränkt. Schließlich ist
\begin{align*}
\abs{k(x,x')} &= \abs{\lin{k(x,\cdot),k(x',\cdot)}_H}
\le \norm{k(x,\cdot)}_H\norm{k(x',\cdot)}_H\\
&= \sqrt{k(x,x)}\sqrt{k(x',x')}
\le \norm{k}_\infty^2.\qedhere
\end{align*}
\end{proof}

\begin{lem}
\label{prop:5.3.2}
Sei $k$ ein Kern auf $X$ und $H$ der RKHS von $k$, dann sind äquivalent
\begin{equivenum}
\item\label{prop:5.3.2:1} $k(x,\cdot)$ ist messbar für alle $x\in X$.
\item\label{prop:5.3.2:2} Alle $f\in H$ sind messbar.\fishhere
\end{equivenum}
\end{lem}
\begin{proof}
"`\ref{prop:5.3.2:2}$\Rightarrow$\ref{prop:5.3.2:1}"': $k(x,\cdot)\in H$.

"`\ref{prop:5.3.2:1}$\Rightarrow$\ref{prop:5.3.2:2}"': $k(x,\cdot)$ ist messbar
und folglich besteht
\begin{align*}
H_\pre = \span \setdef{h(x,\cdot)}{x\in X}
\end{align*}
aus messbaren Funktionen. Sei nun $f\in H$, dann existiert eine Folge $(f_n)$ in
$H_\pre$ mit $\norm{f_n-f}_H\to 0$ und folglich ist $f(x) =
\lim\limits_{n\to\infty} f_n(x)$ für alle $x\in X$.\qedhere
\end{proof}

\begin{lem}
\label{prop:5.3.3}
Sei $k$ Kern auf $X$ mit RKHS $H$ und $k(x,\cdot)$ messbar für alle $x\in X$ und
$H$ separabel. Dann ist $k$ messbar und die kanonische Featuremap $\Phi: X\to H$
ist messbar.\fishhere
\end{lem}
\begin{proof}
Wir überspringen den Beweis.\qedhere
\end{proof}

\begin{lem}
\label{prop:5.3.4}
Sei $X$ ein Hausdorff-Raum, $k$ ein Kern auf $X$ mit RKHS $H$. Dann sind
äquivalent
\begin{equivenum}
\item\label{prop:5.3.4:1} $k$ ist beschränkt und separat stetig, d.h.
$k(x,\cdot)$ ist stetig für alle $x\in X$.
\item\label{prop:5.3.4:2} Alle $f\in H$ sind stetig und beschränkt. In diesem
Fall gilt
\begin{align*}
\norm{\id: H\to C_b(X)} = \norm{k}_\infty < \infty.\fishhere
\end{align*}
\end{equivenum}
\end{lem}
\begin{proof}
"`\ref{prop:5.3.4:1}$\Rightarrow$\ref{prop:5.3.4:2}"': $H_\pre \subset C_b(X)$
nach Voraussetzung. Sei $f\in H$, dann existiert eine Folge $(f_n)$ in $H_\pre$
mit $\norm{f_n-f}_H \to 0$. Da $k$ beschränkt folgt mit Lemma
\ref{prop:5.3.1}, dass
\begin{align*}
\norm{f_n-f}_\infty \to 0
\end{align*}
und folglich ist $f\in C_b(X)$.

"`\ref{prop:5.3.4:2}$\Rightarrow$\ref{prop:5.3.4:1}"': $k(x,\cdot)\in H$ ist
stetig nach Voraussetzung und beschränkt nach Lemma \ref{prop:5.3.1}. Somit ist
$k$ beschränkt und $\norm{k}_\infty = \norm{\id: H\to C_b(X)}$.\qedhere

\end{proof}

\begin{defn}
\label{defn:5.3.5}
Sei $k$ ein Kern auf $X$ mit RKHS $H$ und kanonischer Featuremap $\Phi: X\to H$.
Dann heißt $d_k: X\times X\to [0,\infty)$
\begin{align*}
d_k(x,x') \defl \norm{\Phi(x)-\Phi(x')}_H
= \sqrt{k(x,x)+k(x',x')-2k(x,x')}
\end{align*}
die \emph{Kernmetrik}\index{Kernmetrik} von $k$.\fishhere
\end{defn}

Man sieht leicht ein, dass $d_k$ genau dann eine Metrik ist, wenn $\Phi$
injektiv ist.
Im Allgemeinen ist $d_k$ lediglich eine Semi-Metrik.

\begin{lem}
\label{prop:5.3.6}
Sei $(X,\tau)$ ein Hausdorff-Raum, $k$ ein Kern auf $X$ und $H$ ein RKHS von
$k$, $\Phi: X\to H$ die kanonische Featuremap. Dann sind äquivalent
\begin{equivenum}
\item\label{prop:5.3.6:1} $k$ ist stetig.
\item\label{prop:5.3.6:2} $k$ ist separat stetig und $x\mapsto k(x,x)$ ist
stetig.
\item\label{prop:5.3.6:3} $\Phi: X\to H$ ist stetig.
\item\label{prop:5.3.6:4} $\id : (X,\tau) \to (X,d_k)$ ist stetig, d.h. jede
$d_k$-offene Kugel ist $\tau$-offen.\fishhere
\end{equivenum}
\end{lem}

\begin{proof}
"`\ref{prop:5.3.6:1}$\Rightarrow$\ref{prop:5.3.6:2}"': Trivial.

"`\ref{prop:5.3.6:2}$\Rightarrow$\ref{prop:5.3.6:4}"': $d_k(\cdot,x):
(X,\tau)\to \R$ ist stetig und daher ist die Kugel
\begin{align*}
B_\ep(x',d_k) = \setdef{x'}{d_k(x,x') < \ep}
\end{align*}
$\tau$-offen. Diese Kugeln erzeugen aber gerade die Topologie von $d_k$, d.h.
die Toplogie von $\tau$ ist größer oder gleich der Topologie von $d_k$.

"`\ref{prop:5.3.6:4}$\Rightarrow$\ref{prop:5.3.6:3}"': $\Phi: (X,d_k)\to H$ ist
stetig nach Konstruktion von $d_k$ und somit ist $\Phi: (X,\tau)\to H$ stetig
nach Voraussetzung.

"`\ref{prop:5.3.6:3}$\Rightarrow$\ref{prop:5.3.6:1}"': Seien $(x_n)$ und
$(y_n)$ Folgen in $X$ mit $x,y\in X$, so dass $x_n\to x$ und $y_n\to y$, so gilt
\begin{align*}
\abs{k(x_n,y_n)-k(x,y)} &= \abs{\lin{\Phi(y_n),\Phi(x_n)-\Phi(x)}+
\lin{\Phi(x),\Phi(y_n)-\Phi(y)}}\\
&\le \norm{\Phi(y_n)}\norm{\Phi(x_n)-\Phi(x)} +
\norm{\Phi(x)}\norm{\Phi(y_n)-\Phi(y)}\\
&\le \norm{\Phi(y_n)}d_k(x_n,x) + \norm{\Phi(x)}d_k(y_n,y). 
\end{align*}
Nach Voraussetzung ist $\Phi$ stetig, d.h. $\norm{\Phi(y_n)}$ ist beschränkt.\qedhere 
\end{proof}

\begin{prop}
\label{prop:5.3.7}
Sei $X$ ein kompakter Hausdorffraum und $k$ ein stetiger Kern mit RKHS $H$. Dann
ist die Abbildung
\begin{align*}
\id : H\to C(X)
\end{align*}
wohldefiniert und kompakt.\fishhere
\end{prop}
\begin{proof}
Da $X$ kompakt und $k$ stetig, ist $k$ auch beschränkt. Foglich ist $\id :H\to
C(X)$ wohldefiniert und stetig (Lemma \ref{prop:5.3.4}).

Sei $C(X,d_k)$ der Raum der $d_k$-stetigen Funktionen $f: X\to \R$.
Offensichtlich ist $C_b(X,d_k)\subset l_\infty(X)$. Außerdem ist
$X$  kompakt und $\Phi$ stetig, d.h. $\Phi(X)\subset H$ ist kompakt. Da
$\Phi$ Isometrie bezüglich $d_k$, ist $(X,d_k)$ kompakt und folglich
\begin{align*}
C(X,d_k) = C_b(X,d_k).
\end{align*} 
Für $f\in B_H$ und $x,y\in H$ gilt
\begin{align*}
\abs{f(x)-f(y)} = \abs{\lin{f,\Phi(x)-\Phi(y)}_H}
\le \norm{f}\norm{\Phi(x)-\Phi(y)}
\le d_k(x,y).
\end{align*}
Also sind alle Funktionen in $B_H$ lipschitz bezüglich $d_k$ mit
lipschitz-Konstante $\le 1$. Der Satz von Arzelà-Ascoli besagt nun, dass
\begin{align*}
\bar{B_H} \text{ kompakt ist in } C(X,d_k).
\end{align*}
Somit ist $\id_{H\to C(X,d_k)}(B_H)$ relativkompakt und daher $\id: H\to
C(X,d_k)$ kompakt. Da $C(X,d_k)\subset C(X)$ mit Normgleichheit und
\begin{align*}
H \overset{\id}{\longrightarrow} C(X,d_k)
\overset{\text{stetig}}{\opento} C(X)
\end{align*}
ist $\id: H\to C(X)$ kompakt.\qedhere
\end{proof}

\begin{bem*}
Ist $X$ \textit{nicht} kompakt aber $k$ stetig und beschränkt, so ist
\begin{align*}
\id: H\to C_b(X)
\end{align*}
wohldefiniert und stetig aber im Allgemeinen \textit{nicht} kompakt.

Betrachte z.B. $X=\R$ und den Gauß-Kern $k_\sigma(x,y) =
\exp(-\sigma^2\norm{x-y}_2^2)$. Seien $m,n\in\N$ mit $n\neq m$, so gilt
\begin{align*}
\norm{k_\sigma(m,\cdot)-k_\sigma(n,\cdot)}_\infty
&\ge
\abs{k_\sigma(m,m)-k_\sigma(n,m)}\\
&= 1- \exp(-\sigma^2\abs{m-n}^2) \ge 1-e^{-\sigma^2},
\end{align*}
da $\abs{m-n} \ge 1$. Es ist aber
\begin{align*}
\norm{k_\sigma(n,\cdot)}_H = \sqrt{k_\sigma(n,n)} = 1
\end{align*}
und folglich $\id: H\to C_b(X)$ nicht kompakt.\maphere
\end{bem*}

\begin{lem}
\label{prop:5.3.8}
Sei $X$ ein separabler, metrischer Raum und $k$ ein stetiger Kern mit RKHS $H$.
Dann ist $H$ separabel.\fishhere
\end{lem}
\begin{proof}
$\Phi: X\to H$ ist stetig, da $k$ stetig und somit ist $\Phi(X)$ separabel.
Folglich ist auch $H_\pre = \mathrm{span}\ \Phi(X)$ separabel, denn man kann
$\mathrm{span}$ durch Linearkombinationen mit rationalen Koeffizienten
approximieren, und damit ist auch $H=\bar{H_\pre}$ separabel.~\qedhere
\end{proof}

Wir wollen nun untersuchen, wie sich die Differenzierbarkeit des Kerns auf die
Featuremap und den RKHS übertragen. Dazu betrachten wir einen Kern $k:
\R^d\times \R^d\to \R$ und die Abbildung
\begin{align*}
\tilde{k}: \R^{2d}\to \R,\qquad (x,y)\mapsto \tilde{k}((x,y)) = k(x,y).
\end{align*}
Falls die partiellen Ableitungen von $i$ und $i+d$ von $\tilde{k}$ existieren,
setzen wir
\begin{align*}
\partial_i \partial_{i+d} k\defl \partial_i \partial_{i+d}\tilde{k}.
\end{align*}
Analog lässt sich dies definieren, wenn $X\subset\R^d$ offen.

\begin{lem}
\label{prop:5.3.9}
Sei $X\subset\R^d$ offen, $k$ ein Kern mit Featurespace $H$ und Featuremap
$\Phi:~X\to H$, so dass $\partial_i\partial_{i+d} k$ für
$1\le i\le d$ existiert und stetig ist. Dann existiert auch
\begin{align*}
\partial_i \Phi: X\to H
\end{align*}
und ist stetig und es gilt
\begin{align*}
\lin{\partial_i \Phi(x),\partial_i\Phi(y)}_H = 
\partial_i\partial_{i+d}k(x,y) =
\partial_{i+d}\partial_{i}k(x,y).\fishhere 
\end{align*}
\end{lem}
\begin{proof}
Wir beweisen nur den Fall $X=\R^d$. Sei $e_i$ der $i$-te Einheitsvektor, dann
definiere für festes $h>0$,
\begin{align*}
\Delta_h \Phi(x) \defl \Phi(x+he_i) - \Phi(x).
\end{align*}

Wir zeigen nun, dass $(h_n^{-1}\Delta_{h_n}\Phi(x))_n$ für alle Folgen $h_n\to 0$
mit $h_n\neq 0$ konvergiert. Dann ist $\lim\limits_{n\to\infty} h_n^{-1}
\Delta_{h_n} \Phi(x) \defr \partial_i \Phi(x)$ unabhängig von der Folge $(h_n)$. Da
$H$ vollständig ist, genügt es zu zeigen, dass $(h_n^{-1}\Delta_{h_n}\Phi(x))$
eine Cauchyfolge ist. Es ist
\begin{align*}
&\norm{h_n^{-1}\Delta_{h_n}\Phi(x)-h_m^{-1}\Delta_{h_m} \Phi(x)}^2_H\\
&\quad= \norm{h_n^{-1}\Delta_{h_n}\Phi(x)}_H^2
+ \norm{h_m^{-1}\Delta_{h_m}\Phi(x)}_H^2\\
&\quad- 2 \lin{h_n^{-1}\Delta_{h_n} \Phi(x),h_m^{-1}\Delta h_m \Phi(x)}\\
&\quad\le
2\left(\ep + \partial_i\partial_{i+d}k(x,x)
-\lin{h_n^{-1}\Delta_{h_n} \Phi(x),h_m^{-1}\Delta_{h_m} \Phi(x)}\right)\tag{*}
\end{align*}
Wir definieren für festes $n$ und $x$,
\begin{align*}
K_{x,n}(y) = k(x+h_ne_i,y) - k(x,y),
\end{align*}
dann ist
\begin{align*}
\lin{\Delta_{h_n} \Phi(x),\Delta_{h_m} \Phi(y)}_H
&=
\lin{\Phi(x+h_ne_i)-\Phi(x),\Phi(y+h_me_i)-\Phi(y)}_H\\
&= k(x+h_ne_i,y+h_me_i) - k(x,y+h_me_i) \\ &
- k(x+h_ne_i,y) + k(x,y)\\ 
&= K_{x,n}(y+h_m e_i)-K_{x,n}(y).
\end{align*}
Wenden wir den Mittelwertsatz auf $K_{x,n}$ an, erhalten wir ein $\xi_{m,n}\in
[-\abs{h_m},\abs{h_m}]$ mit
\begin{align*}
h_m^{-1}\lin{\Delta_{h_n}\Phi(x),\Delta_{h_m} \Phi(y)}
&= h_m^{-1}\left(K_{x,n}(y+h_m e_i)-K_{x,n}(y) \right)\\
&= \partial_i K_{x,n}(y+\xi_{m,n}e_i) \\
&= \partial_{i+d}\left(k(x+h_n
e_i,y+\xi_{m,n}e_i) - k(x,y+\xi_{m,n}e_i)\right)
\end{align*} 
Wir wenden nun den Mittelwertsatz auf die erste Variable an und erhalten ein
$\eta_{n,m}\in [-\abs{h_n},\abs{h_n}]$, so dass
\begin{align*}
\lin{h_n^{-1}\Delta_{h_n}\Phi(x),h_m^{-1}\Delta_{h_m} \Phi(y)}_H
= \partial_i \partial_{i+d} k(x+\eta_{n,m}e_i,y+\xi_{m,n}e_i)
\end{align*}
Somit ist nach (*),
\begin{align*}
&\norm{h_n^{-1}\Delta_{h_n}\Phi(x)-h_m^{-1}\Delta_{h_m} \Phi(x)}^2_H\\
&\quad\le
2\left(\ep + \partial_i\partial_{i+d}k(x,x)
-\partial_i \partial_{i+d} k(x+\eta_{n,m}e_i,x+\xi_{m,n}e_i)\right)\\
&\quad < 4\ep,\qquad n,m\ge n_0,
\end{align*}
denn $\partial_i \partial_{i+d}k$ ist stetig. Also ist
$(h_n^{-1}\Delta_{h_n}\Phi(x))$ Cauchyfolge und daher existiert $\partial_i
\Phi(x)$ und ist stetig.

Weiterhin gilt
\begin{align*}
\lin{h_n^{-1}\Delta_{h_n}\Phi(x),h_m^{-1}\Delta_{h_m} \Phi(y)}_H\to \partial_i
\partial_{i+d} k(x,y)
\end{align*} 
und da $h_n^{-1}\Delta_{h_n}\Phi(x) \to \partial_i \Phi(x)$, folgt die
Formel.\qedhere
\end{proof}

\begin{defn}
\label{defn:5.3.10}
\index{Kern!stetig differenzierbar}
Ein Kern $k$ auf $X\subset\R^d$ offen heißt \emph{$m$-fach stetig
differenzierbar}, wenn
\begin{align*}
\partial^{\alpha,\alpha} k : X\times X\to \R
\end{align*}
existiert und stetig ist für alle $\alpha\in\N_0^d$ mit $\abs{\alpha}\le
m$.\fishhere
\end{defn}

\begin{cor}
\label{prop:5.3.11}
Sei $k$ ein $m$-fach stetig differenzierbarer Kern mit RKHS $H$. Dann sind alle
$f\in H$ $m$-fach stetig differenzierbar und es gilt
\begin{align*}
\abs{\partial^\alpha f(x)} \le \norm{f}\sqrt{\partial^{\alpha,\alpha}k(x,x)},
\end{align*}
für alle $x\in X$ und $\alpha\in \N_0^d$ mit $\abs{\alpha}\le m$.\fishhere
\end{cor}
\begin{proof}
Iteration von Lemma \ref{prop:5.3.9} impliziert, dass $\partial^\alpha \Phi: X
\to H$ Featuremap von $\partial^{\alpha,\alpha} k$. Da $\lin{f,\cdot}$ stetig,
folgt
\begin{align*}
\lin{f,\partial^\alpha \Phi(x)}_H = \partial^\alpha \lin{f,\Phi(x)} =
\partial^\alpha f(x)
\end{align*}
und nach Cauchy-Schwarz
\begin{align*}
\abs{\partial^\alpha f(x)} = \abs{\lin{f,\partial^\alpha \Phi(x)}} \le
\norm{f}_H \norm{\partial^\alpha \Phi(x)}_H
= \norm{f}_H \sqrt{\partial^{\alpha,\alpha}k(x,x)}.\qedhere
\end{align*}
\end{proof}

\section{Große RKHS}

Wann kann ein RKHS "`viele"' Funktionen approximieren?

\begin{defn}
\label{defn:5.4.1}
Sei $(X,d)$ ein kompakter metrischer Raum, dann heißt ein stetiger Kern $k$ auf
$X$ \emph{universell}\index{Kern!universell}, wenn der RKHS $H$ von $k$ dicht in
$C(X)$ ist, d.h.
\begin{align*}
\forall g\in C(X), \ep > 0 \exists f\in H : \norm{f-g}_\infty \le \ep.\fishhere
\end{align*}
\end{defn}

\begin{lem}
\label{prop:5.4.2}
Sei $(X,d)$ ein kompakter metrischer Raum und $k$ ein universeller Kern auf $X$.
Dann gelten
\begin{propenum}
\item Jede Featuremap von $k$ ist injektiv.
\item $k(x,x) > 0$ für alle $x\in X$.
\item Der Kern $k^* : X\times X \to \R$ gegeben durch
\begin{align*}
k^*(x,y) = \frac{k(x,y)}{\sqrt{k(x,x)k(y,y)}}
\end{align*}
ist universell.
\item Ist $M\subset X$ abgeschlossen, dann ist $k\big|_{M\times M}$
universell.\fishhere
\end{propenum}
\end{lem}
\begin{proof}
\begin{proofenum}
\item Seien $x_1\neq x_2\in X$ und $g: X\to \R$ mit $x\in X$,
\begin{align*}
g(x) \defl \frac{d(x_1,x)}{d(x_1,x)+d(x_2,x)} - \frac{d(x_2,x)}{d(x_1,x)+d(x_2,x)},
\end{align*}
$g(x_1)=-1$ und $g(x_2)=1$ und $g$ stetig auf $X$, also $g\in C(X)$. Sei nun
$H_0$ ein Featurespace von $k$ und $\Phi_0 : X\to H_0$ eine Featuremap und
ferner $H$ der RKHS von $k$. Da $k$ universell, existiert ein $f\in H$ mit
\begin{align*}
\norm{f-g}_\infty \le \frac{1}{2} \Rightarrow f(x_1) \le -\frac{1}{2}\text{ und
} f(x_2)\ge \frac{1}{2}.
\end{align*}
Nach Satz \ref{prop:5.2.4} existiert ein $\omega\in H_0$ mit
$f=\lin{\omega,\Phi_0(\cdot)}$ und daher
\begin{align*}
\begin{rcases}
\lin{\omega,\Phi_0(x_1)}_H \le -\frac{1}{2},\\
\lin{\omega,\Phi_0(x_2)}_H \ge \frac{1}{2},
\end{rcases}
\Rightarrow
\Phi_0(x_1)\neq \Phi_0(x_2).
\end{align*}
\item Zeige $k(x,x)>0$ für alle $x\in X$. In 1) haben wir gesehen, dass es für
$x\in X$ ein $g\in C(X)$ mit $g(x)=1$ gibt. Dazu gibt es ein $f\in H$ mit
$\norm{f-g}_\infty \le \frac{1}{2}$ und folglich
\begin{align*}
\frac{1}{2} \le f(x) = \lin{f,\Phi(x)}_H \Rightarrow \Phi(x)\neq 0.
\end{align*}
Somit ist $k(x,x) = \lin{\Phi(x),\Phi(x)}_H > 0$.
\item Zeige, dass der normalisierte Kern universell ist. Schreibe dazu
\begin{align*}
\alpha(x) \defl \left(k(x,x) \right)^{-1/2},
\end{align*}
dann ist $\alpha\Phi : X\to H$ eine Featuremap von $k^*$. Sei nun $g\in C(X)$
und $\ep > 0$. Sei weiterhin
\begin{align*}
c \defl \norm{\alpha}_\infty < \infty.
\end{align*}
Dann existiert ein $f\in H$ mit $\norm{f-\frac{g}{\alpha}}_\infty \le
\frac{\ep}{c}$ und damit
\begin{align*}
\norm{\lin{f,\alpha \Phi(\cdot)}_H-g}_\infty \le
\underbrace{\norm{\alpha}_\infty}_{\le c} \norm{f-\frac{g}{\alpha}}_\infty \le
\ep.
\end{align*}
\item Sei $M\subset X$ abgeschlossen, also kompakt, so ist
$k\big|_{M\times M}$ universell. Tietzes Fortsetzungssatz besagt
\begin{align*}
\forall g\in C(M) \exists \hat{g}\in C(X) : \hat{g}\big|_M = g.
\end{align*}
Zu jedem $g\in C(X)$ und $\ep >0$ existiert eine Fortsetzung $\hat{g}\in C(X)$
und ein $f\in H$ mit $\norm{\hat{g}-f}_\infty \le\ep$. Dann ist auch
\begin{align*}
\norm{\hat{g}\big|_M - f\big|_M}_\infty \le \ep,
\end{align*}
wobei $f\big|_M$ im RKHS von $k\big|_{M\times M}$.\qedhere
\end{proofenum} 
\end{proof}

\begin{prop}
\label{prop:5.4.3}
Sei $(X,d)$ ein kompakter metrischer Raum, $k$ stetiger Kern mit $k(x,x) >
0$ für alle $x\in X$. Ferner sei $\Phi: X\to l_2$ eine injektive Featuremap von
$k$. Schreibe $\Phi_n : X\to \R$ für die $n$-te Komponente von $\Phi$, d.h.
$\Phi(x) = (\Phi_n(x))_{n\ge 1}$ für alle $x\in X$.

Schreibe $\AA\defl\mathrm{span}\setdef{\Phi_n}{n\ge 1}$. Ist $\AA$ eine Algebra,
dann ist $k$ universell.~\fishhere
\end{prop}

Zum Beweis des Satzes benötigen wir den
\begin{prop*}[Satz von Stone-Weierstraß]
\index{Satz!von Stone-Weierstraß}
Sei $(X,d)$ ein kompakter metrischer Raum, $\AA\subset C(X)$ Algebra mit
\begin{propenum}
\item Zu jedem $x\in X$ existiert ein $f\in \AA$ mit $f(x)\neq 0$. "`$\AA$
verschwindet nicht"'
\item Zu $x\neq y\in X$ existiert ein $f\in \AA$ mit $f(x)\neq f(y)$. "`$\AA$
separiert"'.
\end{propenum}
Dann gilt $\AA\subset C(X)$ ist dicht.\fishhere
\end{prop*}

\begin{proof}
\begin{proofenum}
\item $\norm{(\Phi_n(x))_{n\ge 1}}_{l^2}^2 = \lin{\Phi(x),\Phi(x)}_{l^2} =
k(x,x) > 0$ und daher existiert ein $n\in\N$, so dass $\Phi_n(x) > 0$.
\item Sei nun $x\neq y$, dann ist $\Phi(x)\neq \Phi(y)$ nach Voraussetzung, d.h. es
gibt ein $n\in\N$, so dass $\Phi_n(x)\neq \Phi_n(y)$.
\item Zu zeigen ist noch, dass $\AA\subset C(X)$. Da $k$ stetig, ist $\Phi: X\to
l^2$ stetig und daher ist auch $\Phi_n$ stetig für alle $n\ge 1$, d.h.
$\AA\subset C(X)$. Damit zeigt Stone-Weierstraß, dass
$\bar{A}^{\norm{\cdot}_\infty} = C(X)$. Für $g\in C(X)$ und $\ep > 0$ existiert
somit ein $f\in\AA$ mit $\norm{f-g}_\infty \le \ep$ und
\begin{align*}
f = \sum_{j=1}^m \alpha_j \Phi_{n_j}.
\end{align*}
Setze
\begin{align*}
\omega_n \defl 
\begin{cases}
\alpha_j, & \text{falls } n_j = n,\\
0, & \text{sonst},
\end{cases}
\end{align*}
und $\omega\defl (\omega_n)_{n\ge 1}$. Somit ist $\omega\in l^2$ und
$f=\lin{\omega,\Phi(\cdot)}_{l^2}$ nach Konstruktion.\qedhere
\end{proofenum}
\end{proof}

\begin{cor}
\label{prop:5.4.4}
Für $r\in (0,\infty]$ sei $h: (-r,r)\to \R$ mit
\begin{align*}
h(t) = \sum_{n\ge 0} a_n t^n,\qquad t\in (-r,r).
\end{align*}
Falls $a_n >0$ für alle $n\ge 0$, ist der Taylorkern universell auf allen
kompakten Teilmengen $X\subset \sqrt{r}\ocirc{B}_{\R^d}$.\fishhere
\end{cor}
\begin{proof}
Wir haben schon gesehen, dass $\Phi: X\to l^2(\N_0^d)$,
\begin{align*}
\Phi(x) = \left(\underbrace{\sqrt{a_{j_1}\ldots a_{j_d}
c_{j_1,\ldots,j_d}}}_{>0}
\prod\limits_{i=1}^d x_i^{j_i}  \right)_{j_1,\ldots,j_d\ge 0}
\end{align*}
eine Featuremap von $k$ ist.
\begin{proofenum}
\item $k$ ist stetig, da $h$ und $\lin{\cdot,\cdot}_{\R^d}$ stetig.
\item $a_0 > 0$ und daher ist $k(x,x) = \sum_{n\ge 0} a_n(\lin{x,x})^n \ge a_0
> 0$.
\item $\Phi$ ist injektiv. Sei $x\neq y$ dann existiert ein
$i\in\setd{1,\ldots,d}$ mit $x_i\neq y_i$. Nimm $j_i = 1$, alle anderen $j_l =
0$. Folglich ist $\Phi_{j_1,\ldots,j_d}(x) \neq \Phi_{j_1,\ldots,j_d}(y)$ und
daher $\Phi(x)\neq\Phi(y)$.

Zu zeigen ist, dass
$\AA=\mathrm{span}\setdef{\Phi_{j_1,\ldots,j_d}}{j_1,\ldots,j_d \ge 0}$ ist
Algebra. Da $a_n \ge 0$ ist $\AA$ Algebra, denn $\AA$ enhält alle Monome und
Linearkombinationen davon. Also ist Satz \ref{prop:5.4.3} anwendbar.\qedhere
\end{proofenum}
\end{proof}

\begin{bsp*}
Der \textit{Exponentialkern}
\begin{align*}
h(x,x') \defl \exp\left( \lin{x,x'}_{\R^d}\right)
\end{align*}
ist universell auf allen Kompakta, da $a_n = \frac{1}{n!} >0$.\bsphere
\end{bsp*}

\begin{bsp*}
Für $\sigma > 0$ ist der \textit{Gaußkern}
\begin{align*}
k_\sigma(x,x') = \exp(-\sigma^2\abs{x-x'}^2)
\end{align*}
universell auf allen Kompakta. Da
\begin{align*}
h_\sigma(x,x') =
\frac{\exp(2\sigma^2\lin{x,x})}{\exp(\sigma^2\abs{x}^2)\exp(\sigma^2\abs{x'}^2)},
\end{align*}
$\exp(2\sigma^2\lin{x,x})$ universell und $h_\sigma$ eine normalisierte Fassung
davon.\bsphere
\end{bsp*}

\begin{bem*}[Bemerkungen.]
\begin{bemenum}
\item
Dahmen und Michelli zeigten 1987, dass ein Kern $k$ genau dann universell ist,
wenn
\begin{align*}
a_0 > 0 \text{ und } \sum_{a_{2n}>0} \frac{1}{2n} = \sum_{a_{2n+1}>0}
\frac{1}{2n+1} = \infty.
\end{align*}
\item Eine leichte Übung zeigt, dass jeder universelle Kern strikt positiv
ist.
\item Pirkus zeigte 2004, dass Taylorkerne genau dann strikt positiv sind, wenn
\begin{align*}
a_0 > 0\text{ und } \abs{\setdef{n}{a_{2n} >0}} = \abs{\setdef{n}{a_{2n+1}>0}} =
\infty.
\end{align*}
\item
Man kann auch untersuchen, wann $H\subset L_p(\mu)$ dicht ist.

Für Gaußkerne ist $H_\sigma\subset L_p(\mu)$ dicht für alle
$p\in[1,\infty)$ und alle W-Maße $\mu$ auf $\R^d$ und $\mu=\lambda^d$ auf
$\R^d$.
\item Sei $(X,d)$ kompakter metrischer Raum, dann existiert ein universeller
Kern auf $X$.
\item Sei $(X,\tau)$ kompakter Hausdorffraum, dann existiert genau dann ein
universeller Kern auf $X$, wenn $(X,\tau)$ metrisierbar ist.
\item In der Regel gilt für universelle RKHS $H$, dass $\dim H = \infty$,
insbesondere gilt dies für $X\subset \R^d$ mit $\ocirc{X}\neq
\varnothing$.\maphere
\end{bemenum}
\end{bem*}