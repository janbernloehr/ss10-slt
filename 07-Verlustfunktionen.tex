\chapter{Verlustfunktionen II}

Bisher haben wir die Verlustfunktionen studiert, die das Lernziel gut
beschrieben ($L_\mathrm{class}$, $L_\mathrm{hinge}$, $L_\mathrm{LS}$, \ldots).
Wir haben aber auch schon gesehen, dass $L_\mathrm{hinge}$ in einem günstigen
Verhältnis zu $L_\mathrm{class}$ steht (Zhang's Ungleichung)
\begin{align*}
\RR_{L_\mathrm{class},P}(f)-\RR_{\mathrm{class},P}^* 
\le \RR_{L_\mathrm{hinge},P}(f)-\RR_{L_\mathrm{hinge},P}^*.
\end{align*}
Während die Minimierung des Überschussrisikos von $L_\mathrm{class}$ ein
kombinatorisches Optimierungsproblem darstellt und somit äußerst diffiziel
werden kann, beschreibt die Minimierung des Überschussrisikos von
$L_\mathrm{hinge}$ ein konvexes Optimierungsproblem, zu dessen Lösung wir
bereits Methoden kennengelernt haben (siehe Kapitel \ref{chap:6.1}). Dies
motiviert den Einsatz von $L_\mathrm{hinge}$ z.B. in SMVs. Weiterhin haben wir
in den Übungen gesehen, dass
\begin{align*}
\RR_{L_\mathrm{class},P}(f)-\RR_{L_\mathrm{class},P}^* \le
\sqrt{\RR_{L_\mathrm{LS},P}(f)-\RR_{L_\mathrm{LS},P}^*}.
\end{align*}

Ziel dieses Abschnitts ist es nun zu untersuchen, wie man eine
Zielverlustfunktion $L$, welche das Lernziel beschreibt, durch eine
Ersatzverlustfunktion ersetzen kann, die einer "`solchen"' Ungleichung genügt. 

\section{Marginbasierte Verlustfunktionen}

\begin{prop}[Satz von Bartlett, Jordan, Mc Auliffe (2007)]
\label{prop:7.1}
Sei $L$ eine konvexe, marginbasierte Verlustfunktion, die durch
\begin{align*}
\phi: \R\to [0,\infty)  
\end{align*}
dargestellt wird ($L(y,t) = \phi(y-t)$). Dann sind äquivalent
\begin{equivenum}
\item $\phi$ ist in $0$ differenzierbar mit $\phi'\big|_0 = 0$.
\item Es existiert eine Funktion $Y: [0,1]\to [0,\infty)$ streng monoton und
\begin{align*}
Y(\RR_{L_\mathrm{class},P}(f)-\RR_{L_\mathrm{class},P}^*)\tag{*}
\le \RR_{L,P}(f)-\RR_{L,P}^*
\end{align*}
für alle W-Maße $P$ auf $X\times \setd{-1,1}$ und $f: X\to \R$.

In diesem Fall gilt
\begin{align*}
Y(t) &= \phi(0) - \inf_{s\in\R}
\left(\frac{t+1}{2}\phi(s) + \left(1-\frac{t+1}{2}\right)\phi(-s) \right)\\
&\overset{\eta = \frac{t+1}{2}}{=}
\phi(0) - \inf_{s\in\R} \left(\eta \phi(s) + (1-\eta)\phi(s) \right).\fishhere 
\end{align*} 
\end{equivenum}
\end{prop}

\begin{bsp*}
Die Äquivalenz gilt für $L_\mathrm{hinge}$ und $L_\mathrm{LS}$ mit 
\begin{align*}
&L_\mathrm{hinge}, && Y(t) = t, && \text{für alle } t\in [0,1],\\
&L_\mathrm{LS}, && Y(t) = t^2, && \text{für alle }t\in [0,1].\bsphere
\end{align*}
\end{bsp*}

\begin{prop*}
Existieren $q\in [0,\infty]$, $c_p > 0$ mit
\begin{align*}
P_X\left(\setdef{x\in X}{\abs{2\eta(x)-1} < t}\right)
\le (c_p t)^q,\qquad t\in [0,\infty),\tag{**}
\end{align*}
und gilt für das in (*) definiert $Y$, $Y(t) \ge ct^p$ für ein $p>1$. Dann gilt
\begin{align*}
\RR_{L_\mathrm{class},P}(f)-\RR_{L_\mathrm{class},P}^*
\le 2c^{-\frac{q+1}{q+p}} c_p^{\frac{q(p-1)}{q+p}}\left(\RR_{L,P}(f)-\RR_{L,P}^*
\right)^{\frac{q+1}{q+p}}.\fishhere
\end{align*}
\end{prop*}

(**) heißt \emph{Tsybakov's Noise Assumption}  (2004). Es ist
\begin{align*}
&\eta = \frac{1}{2} \Leftrightarrow
\abs{2\eta-1} = 0,
&&\eta \approx \frac{1}{2} \Leftrightarrow
\abs{2\eta-1} \approx 0.
\end{align*}  
Die linke Seite in (**) misst die Masse der $x$, bei denen $\eta(x)$ nahe bei
$\frac{1}{2}$ ist. Die rechte Seite beschränkt diese Masse. Für $t\to 0$ sagt
die Noise Assumption
\begin{align*}
P_X\left(\setdef{x\in X}{\abs{2\eta(x)-1} < t}\right)
 = O(t^q).
\end{align*}

\newcommand{\aclass}{{\alpha-\mathrm{class}}}
Wir definieren
\begin{align*}
L_{\aclass}(y,t) \defl
\begin{cases}
1-\alpha, & y = 1,\quad t < 0,\\
\alpha, & y= -1, \quad t\ge 0,\\
0 , & \text{sonst}.
\end{cases}
\end{align*}

\begin{prop}
\label{prop:7.2}
Sei $L$ eine konvexe Verlustfunktion, die durch $\phi$ dargestellt werden kann,
mit
\begin{align*}
L_\aclass(y,t) =
\begin{cases}
1-\alpha\phi(t), & y=1,\\
\alpha\phi(t), & y=-1.
\end{cases}
\end{align*}
Dann sind äquivalent
\begin{equivenum}
\item $\phi$ ist in $0$ differenzierbar mit $\phi'(0) = 0$.
\item Es gibt eine streng monoton steigende Funktion $Y: [0,1]\to [0,\infty)$
mit
\begin{align*}
Y\left(\RR_{\aclass,P}(f)-\RR_{\aclass,P}^* \right)
\le \RR_{L_\alpha,P}(f)-\RR_{L_\alpha,P}^*.\fishhere
\end{align*}
\end{equivenum}
\end{prop}

\begin{bsp*}
Für $0<\alpha\le \frac{1}{2}$ sind
\begin{align*}
&L_\mathrm{hinge}, && Y(t) = t,\\
&L_\mathrm{LS}, && Y(t) = \frac{t^2}{2\alpha(1-\alpha)+(1-2\alpha)}.\bsphere
\end{align*}
\end{bsp*}

\section{Distanzbasierte Verlustfunktionen}

\newcommand{\Med}{\mathrm{Median}}
Für die Verlustfunktion der kleinsten Quadrate $L_\mathrm{LS}$ ist
\begin{align*}
\norm{f-f_{L,P}^*}_{L^2(P_X)}^2
= \RR_{L,P}(f) - \RR_{L,P}^*,\qquad f_{L,P}^* = \E(Y\mid x).
\end{align*}
Wie wollen nun den bedingten Median schätzen. Dazu nehmen wir an, dass
\begin{align*}
\Med(Y\mid x) = \setd{f^*(x)},\qquad P_X-\fs
\end{align*}

\begin{prop}
\label{prop:7.2.1}
Sei $L(y,t) = \abs{y-t}$ für $y\in\R$ und $t\in \R$. Dann sind für $f: X\to \R$
äquivalent
\begin{equivenum}
\item $\RR_{L,P}(f) = \RR_{L,P}^*$,
\item $f= f^*$ $P_X$-$\fs$
\end{equivenum}
D.h. der Minimierer des Überschussrisikos entspricht dem Median.\fishhere
\end{prop}

Wenn wir bereits wissen, das $\RR_{L,P}(f)\approx \RR_{L,P}(f^*)$, wie nahe ist
dann $f$ an $f^*$?

\begin{defn*}
Eine Verteilung $Q$ auf $[-1,1]$ hat den \emph{Mediantyp} $q\in (1,\infty)$,
wenn $\alpha_Q\in [0,2]$ und ein $b_Q> 0$ existieren, so dass für $t^*=\Med(Q)$
und jedes $s\in [0,\alpha_Q]$ gelten
\begin{align*}
Q((t^*-s,t^*)) \ge b_Q s^{q-1},\qquad
Q((t^*,t^*+s)) \ge b_Q s^{q-1}.\fishhere
\end{align*}
\end{defn*}

\begin{defn*}
Sei $p\in (0,\infty]$, $q\in (1,\infty)$ und $P$ ein W-Maß auf $X\times [-1,1]$.
Dann hat $P$ einen \emph{$p$-mittelbaren Mediantyp $q$}, wenn folgende
Eigenschaften erfüllt sind
\begin{defnenum}
\item $P(Y\mid x)$ hat Mediantyp $q$ $P_X$-$\fs$.
\item Die Abbildung $\gamma: X\to \R,\quad x\mapsto b_{P(Y\mid
x)}\alpha_{P(Y\mid x)}^{q-1}$ erfüllt $\gamma^{-1}\in L_p(P_X)$.\fishhere
\end{defnenum}
\end{defn*}

\begin{bsp*}
Alle Dichten von $P(Y\mid x)$ erfüllen für $p=\infty$ und $q=2$,
\begin{align*}
\min\setd{1+f^*(x),1-f^*(x)} \ge \alpha^*.
\end{align*}
Unter diesen Voraussetzungen gilt
\begin{align*}
&r \defl \frac{p_q}{p+1},\\
&\norm{f-f^*}_{L_r(P_X)}
\le c\left(\RR_{L,P}(f)-\RR_{L,P}^* \right)^{1/q}
\end{align*}
für alle $f: X\to [-1,1]$ und $r=q=2$.\bsphere
\end{bsp*}