\chapter{Konzentrationsungleichungen}

Im Lemma \ref{lem:1.3.7} hatten wir gezeigt:
\begin{align*}
\mu^n\left(\setdef{\omega\in\Omega^n}{\frac{1}{n}\sum_{i=1}^n
h(\omega_i)-\E_\mu h \ge t} \right) \le \frac{\E_\mu h^2}{t^2n}
\end{align*}
für alle Maße $\mu$, $n\ge 1$, $t>0$ und $h: \Omega\to\R$.

\begin{prop}
\label{prop:3.1}
Sei $(\Omega,\AA,\mu)$ ein Wahrscheinlichkeitsraum, dann gilt für alle $f:
\Omega\to\R$ mit $\E_\mu \abs{f}< \infty$ und $t>0$,
\begin{align*}
\mu\left(\setdef{\omega\in\Omega}{f(\omega)\ge t} \right)
\le \frac{\E_\mu \abs{f}}{t}.\fishhere
\end{align*}
\end{prop}

\begin{lem}
\label{prop:3.2}
Für $x>-1$ gilt
\begin{align*}
(1+x)\ln(1+x) -x \ge \frac{3}{2}\frac{x^2}{x+3}.\fishhere
\end{align*}
\end{lem}
\begin{proof}
Seien $f(x) = (1+x)\ln(1+x)-x$ und $g(x) = \frac{3}{2}\frac{x^2}{x+3}$. Dann
folgt
\begin{align*}
&f'(x) = \ln(1+x), && g'(x) = \frac{3}{2}\frac{x^2+6x}{(x+3)^2},\\
&f''(x) = \frac{1}{1+x}, && g''(x) = \frac{27}{(x+3)^3}, 
\end{align*}
und es gelten
\begin{align*}
&f'(0) = 0,\quad f(0) = 0,\\
&f'(0) = 0, \quad g(0) = 0. 
\end{align*}
Wir zeigen nun $f''(x)\ge g''(x)$ für alle $x>-1$. Es ist
\begin{align*}
&0 \le x^2(x+9) = x^3+9x^2\\
\Rightarrow\;
&x^3+9x^2+27x+27 = (x+3)^3 \ge  27(1+x) = 27x + 27\\
\Rightarrow\;
&\frac{27}{(x+3)^3} \le \frac{1}{1+x}.
\end{align*}
Für $x\ge 0$ folgt mit dem Hauptsatz der Differential- und Integralrechnung,
\begin{align*}
f'(x) = f'(x)-f'(0) = \int_0^x f''(t)\dt
\ge \int_0^x g''(t)\dt = g'(x)-g'(0) = g'(x).
\end{align*}
Das selbe Argument angewandt auf $f$ liefert, $f(x) \ge g(x)$ und für $x\in
(-1,0]$ folgt $-f(x) \ge -g(x)$.\qedhere
\end{proof}

\begin{prop}[Bernsteins Ungleichung]
\index{Bernsteins Ungleichung}
\label{prop:3.3}
Sei $(\Omega,\AA,P)$ ein Wahrscheinlichkeitsraum und $B>0$, $\sigma > 0$ und
$n\ge 1$. Ferner seien $\xi_1,\ldots,\xi_n: \Omega\to\R$ i.i.d. (unabhängige und
identisch verteilte) Zufallsvariablen, so dass gilt
\begin{align*}
\begin{rcases}
\E_P \xi_i &= 0,\\
\norm{\xi_i}_\infty &\le B,\\
\E_P \xi^2 &\le \sigma^2
\end{rcases}
\text{ für alle } i = 1,\ldots,n.
\end{align*}
Dann gilt
\begin{align*}
P\left(\frac{1}{n} \sum_{i=1}^n \xi_i \ge \sqrt{\frac{2\sigma^2\tau}{n}} +
\frac{2 B \tau}{3n}\right) \le e^{-\tau},\quad \text{für alle } \tau >
0.\fishhere
\end{align*}
\end{prop}
\begin{proof}
Fixiere $t \ge 0$ und $\ep > 0$, dann gilt
\begin{align*}
P\left(\sum_{i=1}^n \xi_i \ge \ep n \right)
&= 
P\left(\exp\left(t\sum_{i=1}^n \xi_i\right) \ge \exp(t\ep n) \right)\\
&\overset{\text{Markov}}{\le}
e^{-t\ep n}\E_P \exp\left(t\sum_{i=1}^n \xi_i \right)\\
&\overset{\text{Unabh.}}{=}
e^{-t\ep n} \prod_{i=1}^n \E_P e^{t\xi_i}.
\end{align*}
Folglich ist
\begin{align*}
\E_P e^{t\xi_i} = \E_P \sum_{k=0}^\infty \frac{t^k}{k!}\xi_i^k
= \sum_{k=0}^\infty \frac{t^k}{k!}\E_P\xi^k
\end{align*}
und da $\E_P \xi_i = 0$ und $\E_P \xi^k \le \sigma^2 B^{k-2}$ folgt
\begin{align*}
\E_P e^{t\xi} &= 1 + \sum_{k=2}^\infty \frac{t^k}{k!}\E_P \xi_i^k
\le 1 + \sum_{k=2}^\infty \frac{t^k}{k!}\sigma^k B^{k-2}
= 1 + \frac{\sigma^2}{B^2}\sum_{k=2}^\infty \frac{t^k}{k!}B^k\\
&= 1 + \frac{\sigma^2}{B^2}\left(e^{tB} -tB-1\right).
\end{align*}
Damit gilt
\begin{align*}
P\left(\sum_{i=1}^n \xi_i \ge \ep n \right)
&\le e^{-tn \ep}
\left( 1+ \frac{\sigma^2}{B^2}\left(e^{tB}-tB-1\right) \right)^n\\
&\le
e^{-tn \ep}
\exp\left( n\frac{\sigma^2}{B^2}\left(e^{tB}-tB-1\right) \right)
\end{align*}
Sei $h(t)\defl - t n \ep + \frac{n\sigma^2}{B^2}\left(e^{tB}-tB-1\right)$, dann
suchen wir kritische Punkte
\begin{align*}
h'(t) = -n \ep + \frac{n\sigma^2}{B^2}\left(Be^{tB}-B\right)
= -n \ep + \frac{n\sigma^2}{B}\left(e^{tB}-1\right)
\overset{!}{=} 0.
\end{align*}
$t$ ist genau dann kritisch, wenn
\begin{align*}
\frac{\sigma^2n}{B} e^{tB} = \ep n + \frac{\sigma^2 n}{B}
\Leftrightarrow
e^{tB} = \frac{\ep B}{\sigma^2} + 1
\Leftrightarrow
t^* = \frac{1}{B}\log\left( 1+ \frac{\ep B}{\sigma^2}\right).
\end{align*}
$t^*$ ist der einzige Kandidat für ein Optimum. Weiterhin ist
\begin{align*}
\lim\limits_{t\to\pm\infty} h(t) = \infty
\end{align*}
und folglich hat $h$ ein globales Minimum bei $t^*$. Setzen wir $y=\frac{\ep
B}{\sigma^2}$, dann gilt $t^*=\frac{1}{B}\log(1+y)$ und somit
\begin{align*}
&-t^* \ep n + \frac{\sigma^2 n}{B^2}\left(e^{t^*B}-t^*B-1\right)\\
= &-\frac{\ep n}{B}\log(1+y) + \frac{\sigma^2 n}{B^2}(1+y-\log(1+y)-1)\\
= &\frac{\sigma^2 n}{B^2}
\left(-y\log(1+y) + y - \log(1+y) \right)\\
= &-\frac{\sigma^2n}{B^2}\left((1+y)\log(1+y)-y \right)\\
\overset{\ref{prop:3.2}}{\le}
&-\frac{\sigma^2 n}{B^2} \frac{3}{2}\frac{y^2}{y+3}
\overset{\text{Rechnung}}{=}
- \frac{3n\ep^2}{2\ep B + 6\sigma^2}.
\end{align*}
Damit ist
\begin{align*}
P\left(\frac{1}{n}\sum_{i=1}^n \xi_i \ge \ep \right)
\le \exp \left(-\frac{3n\ep^2}{2\ep B+6\sigma^2} \right).
\end{align*}
Wähle nun
\begin{align*}
\tau \defl \frac{3n \ep^2}{2\ep B + 6\sigma^2},
\end{align*}
dann folgt
\begin{align*}
\ep = \sqrt{\frac{2\sigma^2\tau}{n} + \frac{B^2\tau^2}{9n^2}} + \frac{B\tau}{3n}
\le \sqrt{\frac{2\sigma^2\tau}{n}} + \frac{2B\tau}{3n}.\qedhere
\end{align*}
\end{proof}

\begin{prop}[Hoeffdings Ungleichung]
\index{Hoeffdings Ungleichung}
\label{prop:3.4}
Sei $(\Omega,\AA,P)$ ein Wahrscheinlichkeitsraum, $a<b$, $1\le n\in\N$ und
$\xi_1,\ldots,\xi_n : \Omega\to[a,b]$ i.i.d. Dann gilt für $\tau >0$,
\begin{align*}
P \left(\frac{1}{n}\sum_{i=1}^n \left(\xi_i - \E_P \xi_i\right)\ge
(b-a)\sqrt{\frac{\tau}{2n}} \right) \le e^{-\tau}.\fishhere
\end{align*}
\end{prop}
\begin{bem*}[Beobachtung.]
Für $a=-b$ betrachte $\eta_i = \xi_i - \E_P\xi_i$.\\
Dann ist $\E_P\eta_i = 0$, $\E_P\eta_i^2 = \E_P\xi_i^2 - (\E_P\xi_i)^2 \le b^2
\defr \sigma^2$ und $\norm{\eta_i}_\infty \le 2b \defr B$. Mit Bernstein folgt,
\begin{align*}
P\left(\frac{1}{n}\sum_{i=1}^n \eta_i \ge \sqrt{\frac{2b^2\tau}{n}} +
\frac{4b\tau}{3n} \right) \le e^{-\tau},
\end{align*}
wobei $\sqrt{\frac{2b^2\tau}{n}} = 2b\sqrt{\frac{\tau}{2n}} =
(b-a)\sqrt{\frac{\tau}{2n}}$.\maphere
\end{bem*}

\begin{lem}[Lemma (Union bound)]
\index{Union bound}
\label{prop:3.5}
Sei $(\Omega,\AA,P)$ ein Wahrscheinlichkeitsraum, $f_1,\ldots,f_n: \Omega\to\R$
messbar. Dann gilt für alle $t\in\R$
\begin{align*}
\mu\left(\sup_{i=1,\ldots,n} f_i \ge t\right) \le
\sum_{i=1}^n \mu(f_i\ge t).\fishhere
\end{align*}
\end{lem}
\begin{proof}
$\setd{\sup_{i=1,\ldots,n} f_i \ge t} = \bigcup_{i=1}^n \setd{f_i\ge
t}$.\qedhere
\end{proof}

\begin{bem*}
Betrachte $\xi_1,\ldots,\xi_n,-\xi_1,\ldots,-\xi_n$ und das union bound für die
Summe über  1. Block $(\defr f_1)$ + 2. Block $(\defr f_2)$. Somit ist
\begin{align*}
\abs{\frac{1}{n}\sum_{i=1}^n \xi_i} = \sup\setd{f_1,f_2} 
\end{align*}
und folglich gilt
\begin{align*}
P \left(\abs{\frac{1}{n} \sum_{i=1}^n\xi_i } \ge \sqrt{\frac{2\sigma^2\tau}{n}}
+ \frac{2B\tau}{3 n} \right) \le 2e^{-\tau}.\maphere
\end{align*}
\end{bem*}
"`Zweiseitige Bernstein-Ungleichung"' (Hoeffdings analog!)